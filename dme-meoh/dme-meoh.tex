%%%%%%%%%%%%%%%%%%%%%%%%%%%%%%%%%%%%%%%%%%%%%%%%%%%%%%%%%%%%%%%%%%%%%%%
% This work is licensed under the Creative Commons Attribution 4.0
% International License. To view a copy of this license, visit
% http://creativecommons.org/licenses/by/4.0/.
%%%%%%%%%%%%%%%%%%%%%%%%%%%%%%%%%%%%%%%%%%%%%%%%%%%%%%%%%%%%%%%%%%%%%%%
\documentclass[12pt]{ussci}

\usepackage{xmpincl}
\includexmp{CC_Attribution_4.0_International}

\hypersetup{
    pdftitle={High-Pressure Autoignition of Binary Blends of Methanol and Dimethyl Ether},%
    pdfauthor={Bryan W. Weber},%
}

\usepackage{enumitem}
\setlist{noitemsep}

\usepackage[version=4]{mhchem}
\usepackage{siunitx}
\sisetup{separate-uncertainty=true}

\usepackage{graphicx}
\usepackage{pgf}
\usepgflibrary{fpu}

\usepackage[tableposition=top]{caption}
\usepackage{booktabs}
\usepackage{mathtools}
\usepackage{microtype}

\usepackage[capitalize]{cleveref}
%======================================================================
\addbibresource{dme-meoh.bib}
%======================================================================
\newcommand\papertopic{Reaction Kinetics}
%======================================================================
\DeclareSIUnit\torr{torr}

\title{ High-Pressure Autoignition of Binary Blends of Methanol and Dimethyl Ether }

\author[1,2]{Hongfu Wang}
\author[2*]{Bryan W.\ Weber}
\author[2]{Ruozhou Fang}
\author[2]{Chih-Jen Sung}
\affil[1]{School of Mechanical and Electrical Engineering, Nanchang University, Jiangxi Province, P.R. China}
\affil[2]{Department of Mechanical Engineering, University of Connecticut, Storrs,
CT, USA}
\affil[*]{Corresponding Author: \email{bryan.weber@uconn.edu}}

\begin{document}
\maketitle

%====================================================================
\begin{abstract} % not to exceed 200 words

    Reactivity Controlled Compression Ignition (RCCI) is a new advanced engine
    concept that uses a dual fuel mode of operation to achieve significant
    improvements in fuel economy and emissions output. The fuels that are
    typically used in this mode include a low- and a high-reactivity fuel in
    varying proportions to control ignition timing. As such, understanding the
    interaction effects during autoignition of binary fuel blends is critical
    to optimizing these RCCI engines. In this work, we measure the autoignition
    delays of binary blends of dimethyl ether (\ce{C2H6O}, DME) and methanol
    (\ce{CH4O}, MeOH) in a rapid compression machine. In these experiments,
    dimethyl ether and methanol function as the high- and low-reactivity fuels,
    respectively. We considered five fuel blends at varying blending ratios (by
    mole), including \SI{100}{\percent} DME-\SI{0}{\percent} MeOH,
    \SI{75}{\percent} DME-\SI{25}{\percent} MeOH, \SI{50}{\percent}
    DME-\SI{50}{\percent} MeOH, \SI{25}{\percent} DME-\SI{75}{\percent} MeOH,
    and \SI{0}{\percent} DME-\SI{100}{\percent} MeOH. Experiments are conducted
    at an engine-relevant pressure of \SI{30}{\bar}, for the stoichiometric
    equivalence ratio. In addition, the experimental results are compared with
    simulations using a chemical kinetic model for DME/MeOH combustion generated
    by merging independent, well-validated models for DME and MeOH.

\end{abstract}

% (Provide 2-4 keywords describing your research. Only abbreviations firmly
% established in the field may be used. These keywords will be used for
% sessioning/indexing purposes.)
\begin{keyword}
    chemical kinetics\sep rapid compression machine\sep binary fuel blends\sep advanced engines
\end{keyword}

\section{Introduction}\label{introduction}

To reduce the environmental impact of combustion, future combustion processes
must feature substantially reduced pollutant emissions while maintaining high
efficiency. A promising concept in this respect is low-temperature combustion
(LTC). As an outstanding representative of LTC technologies, the dual-fuel
Reactivity Controlled Compression Ignition (RCCI) operation has great potential
in terms of combustion controllability. The general principle of dual-fuel RCCI
combustion requires two fuels with different reactivities, that is using a high
reactivity fuel (such as diesel or dimethyl ether (DME)) to trigger the ignition
and combustion of low-reactivity fuels (such as gasoline, methanol, ethanol, or
butanol).

DME is considered an efficient alternative fuel for use in diesel engines
because it has excellent autoignition characteristics. The low boiling point
(\SI{248}{\K}), low critical point (\SI{400}{\K}), and high cetane number (\(>
55\)) of DME~\autocite{Arcoumanis2008,Teng2001} make it well suited for
compression ignition engines. In addition, the high oxygen content of DME
(\SI{34.8}{\percent} by mass) together with the absence of C-C bonds
contributes to ultra-low soot formation during DME
combustion~\autocite{Arcoumanis2008}.

The promoting effect of DME blending on fuels with poor autoignition qualities
such as methane and propane is a promising feature with respect to the RCCI
concept. Therefore, some research has been conducted to reveal the promoting
potential of DME. Several researchers~\autocite{Burke2015a,Tang2012a,Chen2007a}
have studied the ignition delays of DME/methane blends in shock tubes. The
results have shown that DME has a strong promoting effect on the autoignition of
methane, even when the concentration of methane is much higher than that of DME.
Further, \textcite{Dames2016} showed the promoting effect of DME in DME/propane
blends using a rapid compression machine. The results of the work of
\textcite{Dames2016} showed that propane combustion is promoted due to the large
amount of radicals produced by low-temperature DME oxidation. These studies
indicate the potential of DME as a combustion promoter together with fuels with
poor auto-ignition qualities.

For dual-fuel RCCI operation, a lower reactivity fuel should be studied in
combination with the high reactivity fuel. Methanol (MeOH) is well known as a
widely used alcoholic alternative fuel, but when operated in a single fuel mode
it tends to have poor autoignition quality~\autocite{Siebers1987}. Given the
great potential of DME as a combustion promoter, we expect that DME can
significantly improve the auto-ignition quality of methanol and the combination
will be effective for dual-fuel RCCI operation.

In this work, we explore the autoignition characteristics of DME/MeOH binary
blends. A set of ignition delay time data for DME/MeOH blends at different
blending ratios over a wide range of temperature at engine relevant pressure
conditions and the stoichiometric equivalence ratio is obtained in a rapid
compression machine (RCM). In addition, we compile a chemical kinetic model for
DME/MeOH combustion by merging independent models for the fuels. Simulations
utilizing this model are compared to experimental results and good agreement is
observed over the range of the experiments.

\section{Experimental Methods}\label{sec:experimental-methods}

The RCM used in this study is a single piston arrangement and is pneumatically
driven and hydraulically stopped. The device has been described in detail
previously~\autocite{Mittal2007a} and will be described here briefly for
reference. The end of compression (EOC) temperature and pressure (\(T_C\) and
\(P_C\) respectively), are independently changed by varying the overall
compression ratio, initial pressure, and initial temperature of the experiments.
The primary diagnostic on the RCM is the in-cylinder pressure. The pressure data
is processed by a Python package called UConnRCMPy~\autocite{uconnrcmpy}, which
calculates \(P_C\), \(T_C\), and the ignition delay(s). The definition of the
ignition delay is shown in \cref{fig:ign-delay-def}. The time of the EOC is
defined as the maximum of the pressure trace prior to the start of ignition and
the ignition delay is defined as the time from the EOC until local maxima in
the first time derivative of the pressure.

In addition to the reactive experiments, non-reactive experiments are conducted
to determine the influence of machine-specific behavior on the experimental
conditions and permit the calculation of the EOC temperature via the isentropic
relations between pressure and temperature~\autocite{Lee1998}. The EOC
temperature is calculated by the procedure described in
\cref{sec:rcm-modeling}.

\begin{figure}[htb]
    \centering
    \resizebox{0.6\textwidth}{!}{%% Creator: Matplotlib, PGF backend
%%
%% To include the figure in your LaTeX document, write
%%   \input{<filename>.pgf}
%%
%% Make sure the required packages are loaded in your preamble
%%   \usepackage{pgf}
%%
%% Figures using additional raster images can only be included by \input if
%% they are in the same directory as the main LaTeX file. For loading figures
%% from other directories you can use the `import` package
%%   \usepackage{import}
%% and then include the figures with
%%   \import{<path to file>}{<filename>.pgf}
%%
%% Matplotlib used the following preamble
%%   \usepackage[utf8x]{inputenc}
%%   \usepackage[T1]{fontenc}
%%   \usepackage{mathptmx}
%%   \usepackage{mathtools}
%%
\begingroup%
\makeatletter%
\begin{pgfpicture}%
\pgfpathrectangle{\pgfpointorigin}{\pgfqpoint{4.466769in}{3.983314in}}%
\pgfusepath{use as bounding box, clip}%
\begin{pgfscope}%
\pgfsetbuttcap%
\pgfsetmiterjoin%
\definecolor{currentfill}{rgb}{1.000000,1.000000,1.000000}%
\pgfsetfillcolor{currentfill}%
\pgfsetlinewidth{0.000000pt}%
\definecolor{currentstroke}{rgb}{1.000000,1.000000,1.000000}%
\pgfsetstrokecolor{currentstroke}%
\pgfsetdash{}{0pt}%
\pgfpathmoveto{\pgfqpoint{0.000000in}{0.000000in}}%
\pgfpathlineto{\pgfqpoint{4.466769in}{0.000000in}}%
\pgfpathlineto{\pgfqpoint{4.466769in}{3.983314in}}%
\pgfpathlineto{\pgfqpoint{0.000000in}{3.983314in}}%
\pgfpathclose%
\pgfusepath{fill}%
\end{pgfscope}%
\begin{pgfscope}%
\pgfsetbuttcap%
\pgfsetmiterjoin%
\definecolor{currentfill}{rgb}{1.000000,1.000000,1.000000}%
\pgfsetfillcolor{currentfill}%
\pgfsetlinewidth{0.000000pt}%
\definecolor{currentstroke}{rgb}{0.000000,0.000000,0.000000}%
\pgfsetstrokecolor{currentstroke}%
\pgfsetstrokeopacity{0.000000}%
\pgfsetdash{}{0pt}%
\pgfpathmoveto{\pgfqpoint{0.575469in}{0.560814in}}%
\pgfpathlineto{\pgfqpoint{3.675469in}{0.560814in}}%
\pgfpathlineto{\pgfqpoint{3.675469in}{3.833314in}}%
\pgfpathlineto{\pgfqpoint{0.575469in}{3.833314in}}%
\pgfpathclose%
\pgfusepath{fill}%
\end{pgfscope}%
\begin{pgfscope}%
\pgfsetbuttcap%
\pgfsetroundjoin%
\definecolor{currentfill}{rgb}{0.000000,0.000000,0.000000}%
\pgfsetfillcolor{currentfill}%
\pgfsetlinewidth{1.003750pt}%
\definecolor{currentstroke}{rgb}{0.000000,0.000000,0.000000}%
\pgfsetstrokecolor{currentstroke}%
\pgfsetdash{}{0pt}%
\pgfsys@defobject{currentmarker}{\pgfqpoint{0.000000in}{-0.069444in}}{\pgfqpoint{0.000000in}{0.000000in}}{%
\pgfpathmoveto{\pgfqpoint{0.000000in}{0.000000in}}%
\pgfpathlineto{\pgfqpoint{0.000000in}{-0.069444in}}%
\pgfusepath{stroke,fill}%
}%
\begin{pgfscope}%
\pgfsys@transformshift{0.885469in}{0.560814in}%
\pgfsys@useobject{currentmarker}{}%
\end{pgfscope}%
\end{pgfscope}%
\begin{pgfscope}%
\pgftext[x=0.885469in,y=0.442758in,,top]{\rmfamily\fontsize{10.000000}{12.000000}\selectfont \(\displaystyle -10\)}%
\end{pgfscope}%
\begin{pgfscope}%
\pgfsetbuttcap%
\pgfsetroundjoin%
\definecolor{currentfill}{rgb}{0.000000,0.000000,0.000000}%
\pgfsetfillcolor{currentfill}%
\pgfsetlinewidth{1.003750pt}%
\definecolor{currentstroke}{rgb}{0.000000,0.000000,0.000000}%
\pgfsetstrokecolor{currentstroke}%
\pgfsetdash{}{0pt}%
\pgfsys@defobject{currentmarker}{\pgfqpoint{0.000000in}{-0.069444in}}{\pgfqpoint{0.000000in}{0.000000in}}{%
\pgfpathmoveto{\pgfqpoint{0.000000in}{0.000000in}}%
\pgfpathlineto{\pgfqpoint{0.000000in}{-0.069444in}}%
\pgfusepath{stroke,fill}%
}%
\begin{pgfscope}%
\pgfsys@transformshift{1.505469in}{0.560814in}%
\pgfsys@useobject{currentmarker}{}%
\end{pgfscope}%
\end{pgfscope}%
\begin{pgfscope}%
\pgftext[x=1.505469in,y=0.442758in,,top]{\rmfamily\fontsize{10.000000}{12.000000}\selectfont \(\displaystyle 0\)}%
\end{pgfscope}%
\begin{pgfscope}%
\pgfsetbuttcap%
\pgfsetroundjoin%
\definecolor{currentfill}{rgb}{0.000000,0.000000,0.000000}%
\pgfsetfillcolor{currentfill}%
\pgfsetlinewidth{1.003750pt}%
\definecolor{currentstroke}{rgb}{0.000000,0.000000,0.000000}%
\pgfsetstrokecolor{currentstroke}%
\pgfsetdash{}{0pt}%
\pgfsys@defobject{currentmarker}{\pgfqpoint{0.000000in}{-0.069444in}}{\pgfqpoint{0.000000in}{0.000000in}}{%
\pgfpathmoveto{\pgfqpoint{0.000000in}{0.000000in}}%
\pgfpathlineto{\pgfqpoint{0.000000in}{-0.069444in}}%
\pgfusepath{stroke,fill}%
}%
\begin{pgfscope}%
\pgfsys@transformshift{2.125469in}{0.560814in}%
\pgfsys@useobject{currentmarker}{}%
\end{pgfscope}%
\end{pgfscope}%
\begin{pgfscope}%
\pgftext[x=2.125469in,y=0.442758in,,top]{\rmfamily\fontsize{10.000000}{12.000000}\selectfont \(\displaystyle 10\)}%
\end{pgfscope}%
\begin{pgfscope}%
\pgfsetbuttcap%
\pgfsetroundjoin%
\definecolor{currentfill}{rgb}{0.000000,0.000000,0.000000}%
\pgfsetfillcolor{currentfill}%
\pgfsetlinewidth{1.003750pt}%
\definecolor{currentstroke}{rgb}{0.000000,0.000000,0.000000}%
\pgfsetstrokecolor{currentstroke}%
\pgfsetdash{}{0pt}%
\pgfsys@defobject{currentmarker}{\pgfqpoint{0.000000in}{-0.069444in}}{\pgfqpoint{0.000000in}{0.000000in}}{%
\pgfpathmoveto{\pgfqpoint{0.000000in}{0.000000in}}%
\pgfpathlineto{\pgfqpoint{0.000000in}{-0.069444in}}%
\pgfusepath{stroke,fill}%
}%
\begin{pgfscope}%
\pgfsys@transformshift{2.745469in}{0.560814in}%
\pgfsys@useobject{currentmarker}{}%
\end{pgfscope}%
\end{pgfscope}%
\begin{pgfscope}%
\pgftext[x=2.745469in,y=0.442758in,,top]{\rmfamily\fontsize{10.000000}{12.000000}\selectfont \(\displaystyle 20\)}%
\end{pgfscope}%
\begin{pgfscope}%
\pgfsetbuttcap%
\pgfsetroundjoin%
\definecolor{currentfill}{rgb}{0.000000,0.000000,0.000000}%
\pgfsetfillcolor{currentfill}%
\pgfsetlinewidth{1.003750pt}%
\definecolor{currentstroke}{rgb}{0.000000,0.000000,0.000000}%
\pgfsetstrokecolor{currentstroke}%
\pgfsetdash{}{0pt}%
\pgfsys@defobject{currentmarker}{\pgfqpoint{0.000000in}{-0.069444in}}{\pgfqpoint{0.000000in}{0.000000in}}{%
\pgfpathmoveto{\pgfqpoint{0.000000in}{0.000000in}}%
\pgfpathlineto{\pgfqpoint{0.000000in}{-0.069444in}}%
\pgfusepath{stroke,fill}%
}%
\begin{pgfscope}%
\pgfsys@transformshift{3.365469in}{0.560814in}%
\pgfsys@useobject{currentmarker}{}%
\end{pgfscope}%
\end{pgfscope}%
\begin{pgfscope}%
\pgftext[x=3.365469in,y=0.442758in,,top]{\rmfamily\fontsize{10.000000}{12.000000}\selectfont \(\displaystyle 30\)}%
\end{pgfscope}%
\begin{pgfscope}%
\pgfsetbuttcap%
\pgfsetroundjoin%
\definecolor{currentfill}{rgb}{0.000000,0.000000,0.000000}%
\pgfsetfillcolor{currentfill}%
\pgfsetlinewidth{1.003750pt}%
\definecolor{currentstroke}{rgb}{0.000000,0.000000,0.000000}%
\pgfsetstrokecolor{currentstroke}%
\pgfsetdash{}{0pt}%
\pgfsys@defobject{currentmarker}{\pgfqpoint{0.000000in}{-0.034722in}}{\pgfqpoint{0.000000in}{0.000000in}}{%
\pgfpathmoveto{\pgfqpoint{0.000000in}{0.000000in}}%
\pgfpathlineto{\pgfqpoint{0.000000in}{-0.034722in}}%
\pgfusepath{stroke,fill}%
}%
\begin{pgfscope}%
\pgfsys@transformshift{0.730469in}{0.560814in}%
\pgfsys@useobject{currentmarker}{}%
\end{pgfscope}%
\end{pgfscope}%
\begin{pgfscope}%
\pgfsetbuttcap%
\pgfsetroundjoin%
\definecolor{currentfill}{rgb}{0.000000,0.000000,0.000000}%
\pgfsetfillcolor{currentfill}%
\pgfsetlinewidth{1.003750pt}%
\definecolor{currentstroke}{rgb}{0.000000,0.000000,0.000000}%
\pgfsetstrokecolor{currentstroke}%
\pgfsetdash{}{0pt}%
\pgfsys@defobject{currentmarker}{\pgfqpoint{0.000000in}{-0.034722in}}{\pgfqpoint{0.000000in}{0.000000in}}{%
\pgfpathmoveto{\pgfqpoint{0.000000in}{0.000000in}}%
\pgfpathlineto{\pgfqpoint{0.000000in}{-0.034722in}}%
\pgfusepath{stroke,fill}%
}%
\begin{pgfscope}%
\pgfsys@transformshift{1.040469in}{0.560814in}%
\pgfsys@useobject{currentmarker}{}%
\end{pgfscope}%
\end{pgfscope}%
\begin{pgfscope}%
\pgfsetbuttcap%
\pgfsetroundjoin%
\definecolor{currentfill}{rgb}{0.000000,0.000000,0.000000}%
\pgfsetfillcolor{currentfill}%
\pgfsetlinewidth{1.003750pt}%
\definecolor{currentstroke}{rgb}{0.000000,0.000000,0.000000}%
\pgfsetstrokecolor{currentstroke}%
\pgfsetdash{}{0pt}%
\pgfsys@defobject{currentmarker}{\pgfqpoint{0.000000in}{-0.034722in}}{\pgfqpoint{0.000000in}{0.000000in}}{%
\pgfpathmoveto{\pgfqpoint{0.000000in}{0.000000in}}%
\pgfpathlineto{\pgfqpoint{0.000000in}{-0.034722in}}%
\pgfusepath{stroke,fill}%
}%
\begin{pgfscope}%
\pgfsys@transformshift{1.195469in}{0.560814in}%
\pgfsys@useobject{currentmarker}{}%
\end{pgfscope}%
\end{pgfscope}%
\begin{pgfscope}%
\pgfsetbuttcap%
\pgfsetroundjoin%
\definecolor{currentfill}{rgb}{0.000000,0.000000,0.000000}%
\pgfsetfillcolor{currentfill}%
\pgfsetlinewidth{1.003750pt}%
\definecolor{currentstroke}{rgb}{0.000000,0.000000,0.000000}%
\pgfsetstrokecolor{currentstroke}%
\pgfsetdash{}{0pt}%
\pgfsys@defobject{currentmarker}{\pgfqpoint{0.000000in}{-0.034722in}}{\pgfqpoint{0.000000in}{0.000000in}}{%
\pgfpathmoveto{\pgfqpoint{0.000000in}{0.000000in}}%
\pgfpathlineto{\pgfqpoint{0.000000in}{-0.034722in}}%
\pgfusepath{stroke,fill}%
}%
\begin{pgfscope}%
\pgfsys@transformshift{1.350469in}{0.560814in}%
\pgfsys@useobject{currentmarker}{}%
\end{pgfscope}%
\end{pgfscope}%
\begin{pgfscope}%
\pgfsetbuttcap%
\pgfsetroundjoin%
\definecolor{currentfill}{rgb}{0.000000,0.000000,0.000000}%
\pgfsetfillcolor{currentfill}%
\pgfsetlinewidth{1.003750pt}%
\definecolor{currentstroke}{rgb}{0.000000,0.000000,0.000000}%
\pgfsetstrokecolor{currentstroke}%
\pgfsetdash{}{0pt}%
\pgfsys@defobject{currentmarker}{\pgfqpoint{0.000000in}{-0.034722in}}{\pgfqpoint{0.000000in}{0.000000in}}{%
\pgfpathmoveto{\pgfqpoint{0.000000in}{0.000000in}}%
\pgfpathlineto{\pgfqpoint{0.000000in}{-0.034722in}}%
\pgfusepath{stroke,fill}%
}%
\begin{pgfscope}%
\pgfsys@transformshift{1.660469in}{0.560814in}%
\pgfsys@useobject{currentmarker}{}%
\end{pgfscope}%
\end{pgfscope}%
\begin{pgfscope}%
\pgfsetbuttcap%
\pgfsetroundjoin%
\definecolor{currentfill}{rgb}{0.000000,0.000000,0.000000}%
\pgfsetfillcolor{currentfill}%
\pgfsetlinewidth{1.003750pt}%
\definecolor{currentstroke}{rgb}{0.000000,0.000000,0.000000}%
\pgfsetstrokecolor{currentstroke}%
\pgfsetdash{}{0pt}%
\pgfsys@defobject{currentmarker}{\pgfqpoint{0.000000in}{-0.034722in}}{\pgfqpoint{0.000000in}{0.000000in}}{%
\pgfpathmoveto{\pgfqpoint{0.000000in}{0.000000in}}%
\pgfpathlineto{\pgfqpoint{0.000000in}{-0.034722in}}%
\pgfusepath{stroke,fill}%
}%
\begin{pgfscope}%
\pgfsys@transformshift{1.815469in}{0.560814in}%
\pgfsys@useobject{currentmarker}{}%
\end{pgfscope}%
\end{pgfscope}%
\begin{pgfscope}%
\pgfsetbuttcap%
\pgfsetroundjoin%
\definecolor{currentfill}{rgb}{0.000000,0.000000,0.000000}%
\pgfsetfillcolor{currentfill}%
\pgfsetlinewidth{1.003750pt}%
\definecolor{currentstroke}{rgb}{0.000000,0.000000,0.000000}%
\pgfsetstrokecolor{currentstroke}%
\pgfsetdash{}{0pt}%
\pgfsys@defobject{currentmarker}{\pgfqpoint{0.000000in}{-0.034722in}}{\pgfqpoint{0.000000in}{0.000000in}}{%
\pgfpathmoveto{\pgfqpoint{0.000000in}{0.000000in}}%
\pgfpathlineto{\pgfqpoint{0.000000in}{-0.034722in}}%
\pgfusepath{stroke,fill}%
}%
\begin{pgfscope}%
\pgfsys@transformshift{1.970469in}{0.560814in}%
\pgfsys@useobject{currentmarker}{}%
\end{pgfscope}%
\end{pgfscope}%
\begin{pgfscope}%
\pgfsetbuttcap%
\pgfsetroundjoin%
\definecolor{currentfill}{rgb}{0.000000,0.000000,0.000000}%
\pgfsetfillcolor{currentfill}%
\pgfsetlinewidth{1.003750pt}%
\definecolor{currentstroke}{rgb}{0.000000,0.000000,0.000000}%
\pgfsetstrokecolor{currentstroke}%
\pgfsetdash{}{0pt}%
\pgfsys@defobject{currentmarker}{\pgfqpoint{0.000000in}{-0.034722in}}{\pgfqpoint{0.000000in}{0.000000in}}{%
\pgfpathmoveto{\pgfqpoint{0.000000in}{0.000000in}}%
\pgfpathlineto{\pgfqpoint{0.000000in}{-0.034722in}}%
\pgfusepath{stroke,fill}%
}%
\begin{pgfscope}%
\pgfsys@transformshift{2.280469in}{0.560814in}%
\pgfsys@useobject{currentmarker}{}%
\end{pgfscope}%
\end{pgfscope}%
\begin{pgfscope}%
\pgfsetbuttcap%
\pgfsetroundjoin%
\definecolor{currentfill}{rgb}{0.000000,0.000000,0.000000}%
\pgfsetfillcolor{currentfill}%
\pgfsetlinewidth{1.003750pt}%
\definecolor{currentstroke}{rgb}{0.000000,0.000000,0.000000}%
\pgfsetstrokecolor{currentstroke}%
\pgfsetdash{}{0pt}%
\pgfsys@defobject{currentmarker}{\pgfqpoint{0.000000in}{-0.034722in}}{\pgfqpoint{0.000000in}{0.000000in}}{%
\pgfpathmoveto{\pgfqpoint{0.000000in}{0.000000in}}%
\pgfpathlineto{\pgfqpoint{0.000000in}{-0.034722in}}%
\pgfusepath{stroke,fill}%
}%
\begin{pgfscope}%
\pgfsys@transformshift{2.435469in}{0.560814in}%
\pgfsys@useobject{currentmarker}{}%
\end{pgfscope}%
\end{pgfscope}%
\begin{pgfscope}%
\pgfsetbuttcap%
\pgfsetroundjoin%
\definecolor{currentfill}{rgb}{0.000000,0.000000,0.000000}%
\pgfsetfillcolor{currentfill}%
\pgfsetlinewidth{1.003750pt}%
\definecolor{currentstroke}{rgb}{0.000000,0.000000,0.000000}%
\pgfsetstrokecolor{currentstroke}%
\pgfsetdash{}{0pt}%
\pgfsys@defobject{currentmarker}{\pgfqpoint{0.000000in}{-0.034722in}}{\pgfqpoint{0.000000in}{0.000000in}}{%
\pgfpathmoveto{\pgfqpoint{0.000000in}{0.000000in}}%
\pgfpathlineto{\pgfqpoint{0.000000in}{-0.034722in}}%
\pgfusepath{stroke,fill}%
}%
\begin{pgfscope}%
\pgfsys@transformshift{2.590469in}{0.560814in}%
\pgfsys@useobject{currentmarker}{}%
\end{pgfscope}%
\end{pgfscope}%
\begin{pgfscope}%
\pgfsetbuttcap%
\pgfsetroundjoin%
\definecolor{currentfill}{rgb}{0.000000,0.000000,0.000000}%
\pgfsetfillcolor{currentfill}%
\pgfsetlinewidth{1.003750pt}%
\definecolor{currentstroke}{rgb}{0.000000,0.000000,0.000000}%
\pgfsetstrokecolor{currentstroke}%
\pgfsetdash{}{0pt}%
\pgfsys@defobject{currentmarker}{\pgfqpoint{0.000000in}{-0.034722in}}{\pgfqpoint{0.000000in}{0.000000in}}{%
\pgfpathmoveto{\pgfqpoint{0.000000in}{0.000000in}}%
\pgfpathlineto{\pgfqpoint{0.000000in}{-0.034722in}}%
\pgfusepath{stroke,fill}%
}%
\begin{pgfscope}%
\pgfsys@transformshift{2.900469in}{0.560814in}%
\pgfsys@useobject{currentmarker}{}%
\end{pgfscope}%
\end{pgfscope}%
\begin{pgfscope}%
\pgfsetbuttcap%
\pgfsetroundjoin%
\definecolor{currentfill}{rgb}{0.000000,0.000000,0.000000}%
\pgfsetfillcolor{currentfill}%
\pgfsetlinewidth{1.003750pt}%
\definecolor{currentstroke}{rgb}{0.000000,0.000000,0.000000}%
\pgfsetstrokecolor{currentstroke}%
\pgfsetdash{}{0pt}%
\pgfsys@defobject{currentmarker}{\pgfqpoint{0.000000in}{-0.034722in}}{\pgfqpoint{0.000000in}{0.000000in}}{%
\pgfpathmoveto{\pgfqpoint{0.000000in}{0.000000in}}%
\pgfpathlineto{\pgfqpoint{0.000000in}{-0.034722in}}%
\pgfusepath{stroke,fill}%
}%
\begin{pgfscope}%
\pgfsys@transformshift{3.055469in}{0.560814in}%
\pgfsys@useobject{currentmarker}{}%
\end{pgfscope}%
\end{pgfscope}%
\begin{pgfscope}%
\pgfsetbuttcap%
\pgfsetroundjoin%
\definecolor{currentfill}{rgb}{0.000000,0.000000,0.000000}%
\pgfsetfillcolor{currentfill}%
\pgfsetlinewidth{1.003750pt}%
\definecolor{currentstroke}{rgb}{0.000000,0.000000,0.000000}%
\pgfsetstrokecolor{currentstroke}%
\pgfsetdash{}{0pt}%
\pgfsys@defobject{currentmarker}{\pgfqpoint{0.000000in}{-0.034722in}}{\pgfqpoint{0.000000in}{0.000000in}}{%
\pgfpathmoveto{\pgfqpoint{0.000000in}{0.000000in}}%
\pgfpathlineto{\pgfqpoint{0.000000in}{-0.034722in}}%
\pgfusepath{stroke,fill}%
}%
\begin{pgfscope}%
\pgfsys@transformshift{3.210469in}{0.560814in}%
\pgfsys@useobject{currentmarker}{}%
\end{pgfscope}%
\end{pgfscope}%
\begin{pgfscope}%
\pgfsetbuttcap%
\pgfsetroundjoin%
\definecolor{currentfill}{rgb}{0.000000,0.000000,0.000000}%
\pgfsetfillcolor{currentfill}%
\pgfsetlinewidth{1.003750pt}%
\definecolor{currentstroke}{rgb}{0.000000,0.000000,0.000000}%
\pgfsetstrokecolor{currentstroke}%
\pgfsetdash{}{0pt}%
\pgfsys@defobject{currentmarker}{\pgfqpoint{0.000000in}{-0.034722in}}{\pgfqpoint{0.000000in}{0.000000in}}{%
\pgfpathmoveto{\pgfqpoint{0.000000in}{0.000000in}}%
\pgfpathlineto{\pgfqpoint{0.000000in}{-0.034722in}}%
\pgfusepath{stroke,fill}%
}%
\begin{pgfscope}%
\pgfsys@transformshift{3.520469in}{0.560814in}%
\pgfsys@useobject{currentmarker}{}%
\end{pgfscope}%
\end{pgfscope}%
\begin{pgfscope}%
\pgfsetbuttcap%
\pgfsetroundjoin%
\definecolor{currentfill}{rgb}{0.000000,0.000000,0.000000}%
\pgfsetfillcolor{currentfill}%
\pgfsetlinewidth{1.003750pt}%
\definecolor{currentstroke}{rgb}{0.000000,0.000000,0.000000}%
\pgfsetstrokecolor{currentstroke}%
\pgfsetdash{}{0pt}%
\pgfsys@defobject{currentmarker}{\pgfqpoint{0.000000in}{-0.034722in}}{\pgfqpoint{0.000000in}{0.000000in}}{%
\pgfpathmoveto{\pgfqpoint{0.000000in}{0.000000in}}%
\pgfpathlineto{\pgfqpoint{0.000000in}{-0.034722in}}%
\pgfusepath{stroke,fill}%
}%
\begin{pgfscope}%
\pgfsys@transformshift{3.675469in}{0.560814in}%
\pgfsys@useobject{currentmarker}{}%
\end{pgfscope}%
\end{pgfscope}%
\begin{pgfscope}%
\pgftext[x=2.125469in,y=0.249080in,,top]{\rmfamily\fontsize{12.000000}{14.400000}\selectfont Time, ms}%
\end{pgfscope}%
\begin{pgfscope}%
\pgfsetbuttcap%
\pgfsetroundjoin%
\definecolor{currentfill}{rgb}{0.000000,0.000000,0.000000}%
\pgfsetfillcolor{currentfill}%
\pgfsetlinewidth{1.003750pt}%
\definecolor{currentstroke}{rgb}{0.000000,0.000000,0.000000}%
\pgfsetstrokecolor{currentstroke}%
\pgfsetdash{}{0pt}%
\pgfsys@defobject{currentmarker}{\pgfqpoint{-0.069444in}{0.000000in}}{\pgfqpoint{0.000000in}{0.000000in}}{%
\pgfpathmoveto{\pgfqpoint{0.000000in}{0.000000in}}%
\pgfpathlineto{\pgfqpoint{-0.069444in}{0.000000in}}%
\pgfusepath{stroke,fill}%
}%
\begin{pgfscope}%
\pgfsys@transformshift{0.575469in}{0.560814in}%
\pgfsys@useobject{currentmarker}{}%
\end{pgfscope}%
\end{pgfscope}%
\begin{pgfscope}%
\pgftext[x=0.387969in,y=0.513731in,left,base]{\rmfamily\fontsize{10.000000}{12.000000}\selectfont \(\displaystyle 0\)}%
\end{pgfscope}%
\begin{pgfscope}%
\pgfsetbuttcap%
\pgfsetroundjoin%
\definecolor{currentfill}{rgb}{0.000000,0.000000,0.000000}%
\pgfsetfillcolor{currentfill}%
\pgfsetlinewidth{1.003750pt}%
\definecolor{currentstroke}{rgb}{0.000000,0.000000,0.000000}%
\pgfsetstrokecolor{currentstroke}%
\pgfsetdash{}{0pt}%
\pgfsys@defobject{currentmarker}{\pgfqpoint{-0.069444in}{0.000000in}}{\pgfqpoint{0.000000in}{0.000000in}}{%
\pgfpathmoveto{\pgfqpoint{0.000000in}{0.000000in}}%
\pgfpathlineto{\pgfqpoint{-0.069444in}{0.000000in}}%
\pgfusepath{stroke,fill}%
}%
\begin{pgfscope}%
\pgfsys@transformshift{0.575469in}{1.106231in}%
\pgfsys@useobject{currentmarker}{}%
\end{pgfscope}%
\end{pgfscope}%
\begin{pgfscope}%
\pgftext[x=0.318525in,y=1.059148in,left,base]{\rmfamily\fontsize{10.000000}{12.000000}\selectfont \(\displaystyle 10\)}%
\end{pgfscope}%
\begin{pgfscope}%
\pgfsetbuttcap%
\pgfsetroundjoin%
\definecolor{currentfill}{rgb}{0.000000,0.000000,0.000000}%
\pgfsetfillcolor{currentfill}%
\pgfsetlinewidth{1.003750pt}%
\definecolor{currentstroke}{rgb}{0.000000,0.000000,0.000000}%
\pgfsetstrokecolor{currentstroke}%
\pgfsetdash{}{0pt}%
\pgfsys@defobject{currentmarker}{\pgfqpoint{-0.069444in}{0.000000in}}{\pgfqpoint{0.000000in}{0.000000in}}{%
\pgfpathmoveto{\pgfqpoint{0.000000in}{0.000000in}}%
\pgfpathlineto{\pgfqpoint{-0.069444in}{0.000000in}}%
\pgfusepath{stroke,fill}%
}%
\begin{pgfscope}%
\pgfsys@transformshift{0.575469in}{1.651647in}%
\pgfsys@useobject{currentmarker}{}%
\end{pgfscope}%
\end{pgfscope}%
\begin{pgfscope}%
\pgftext[x=0.318525in,y=1.604565in,left,base]{\rmfamily\fontsize{10.000000}{12.000000}\selectfont \(\displaystyle 20\)}%
\end{pgfscope}%
\begin{pgfscope}%
\pgfsetbuttcap%
\pgfsetroundjoin%
\definecolor{currentfill}{rgb}{0.000000,0.000000,0.000000}%
\pgfsetfillcolor{currentfill}%
\pgfsetlinewidth{1.003750pt}%
\definecolor{currentstroke}{rgb}{0.000000,0.000000,0.000000}%
\pgfsetstrokecolor{currentstroke}%
\pgfsetdash{}{0pt}%
\pgfsys@defobject{currentmarker}{\pgfqpoint{-0.069444in}{0.000000in}}{\pgfqpoint{0.000000in}{0.000000in}}{%
\pgfpathmoveto{\pgfqpoint{0.000000in}{0.000000in}}%
\pgfpathlineto{\pgfqpoint{-0.069444in}{0.000000in}}%
\pgfusepath{stroke,fill}%
}%
\begin{pgfscope}%
\pgfsys@transformshift{0.575469in}{2.197064in}%
\pgfsys@useobject{currentmarker}{}%
\end{pgfscope}%
\end{pgfscope}%
\begin{pgfscope}%
\pgftext[x=0.318525in,y=2.149981in,left,base]{\rmfamily\fontsize{10.000000}{12.000000}\selectfont \(\displaystyle 30\)}%
\end{pgfscope}%
\begin{pgfscope}%
\pgfsetbuttcap%
\pgfsetroundjoin%
\definecolor{currentfill}{rgb}{0.000000,0.000000,0.000000}%
\pgfsetfillcolor{currentfill}%
\pgfsetlinewidth{1.003750pt}%
\definecolor{currentstroke}{rgb}{0.000000,0.000000,0.000000}%
\pgfsetstrokecolor{currentstroke}%
\pgfsetdash{}{0pt}%
\pgfsys@defobject{currentmarker}{\pgfqpoint{-0.069444in}{0.000000in}}{\pgfqpoint{0.000000in}{0.000000in}}{%
\pgfpathmoveto{\pgfqpoint{0.000000in}{0.000000in}}%
\pgfpathlineto{\pgfqpoint{-0.069444in}{0.000000in}}%
\pgfusepath{stroke,fill}%
}%
\begin{pgfscope}%
\pgfsys@transformshift{0.575469in}{2.742481in}%
\pgfsys@useobject{currentmarker}{}%
\end{pgfscope}%
\end{pgfscope}%
\begin{pgfscope}%
\pgftext[x=0.318525in,y=2.695398in,left,base]{\rmfamily\fontsize{10.000000}{12.000000}\selectfont \(\displaystyle 40\)}%
\end{pgfscope}%
\begin{pgfscope}%
\pgfsetbuttcap%
\pgfsetroundjoin%
\definecolor{currentfill}{rgb}{0.000000,0.000000,0.000000}%
\pgfsetfillcolor{currentfill}%
\pgfsetlinewidth{1.003750pt}%
\definecolor{currentstroke}{rgb}{0.000000,0.000000,0.000000}%
\pgfsetstrokecolor{currentstroke}%
\pgfsetdash{}{0pt}%
\pgfsys@defobject{currentmarker}{\pgfqpoint{-0.069444in}{0.000000in}}{\pgfqpoint{0.000000in}{0.000000in}}{%
\pgfpathmoveto{\pgfqpoint{0.000000in}{0.000000in}}%
\pgfpathlineto{\pgfqpoint{-0.069444in}{0.000000in}}%
\pgfusepath{stroke,fill}%
}%
\begin{pgfscope}%
\pgfsys@transformshift{0.575469in}{3.287897in}%
\pgfsys@useobject{currentmarker}{}%
\end{pgfscope}%
\end{pgfscope}%
\begin{pgfscope}%
\pgftext[x=0.318525in,y=3.240815in,left,base]{\rmfamily\fontsize{10.000000}{12.000000}\selectfont \(\displaystyle 50\)}%
\end{pgfscope}%
\begin{pgfscope}%
\pgfsetbuttcap%
\pgfsetroundjoin%
\definecolor{currentfill}{rgb}{0.000000,0.000000,0.000000}%
\pgfsetfillcolor{currentfill}%
\pgfsetlinewidth{1.003750pt}%
\definecolor{currentstroke}{rgb}{0.000000,0.000000,0.000000}%
\pgfsetstrokecolor{currentstroke}%
\pgfsetdash{}{0pt}%
\pgfsys@defobject{currentmarker}{\pgfqpoint{-0.069444in}{0.000000in}}{\pgfqpoint{0.000000in}{0.000000in}}{%
\pgfpathmoveto{\pgfqpoint{0.000000in}{0.000000in}}%
\pgfpathlineto{\pgfqpoint{-0.069444in}{0.000000in}}%
\pgfusepath{stroke,fill}%
}%
\begin{pgfscope}%
\pgfsys@transformshift{0.575469in}{3.833314in}%
\pgfsys@useobject{currentmarker}{}%
\end{pgfscope}%
\end{pgfscope}%
\begin{pgfscope}%
\pgftext[x=0.318525in,y=3.786231in,left,base]{\rmfamily\fontsize{10.000000}{12.000000}\selectfont \(\displaystyle 60\)}%
\end{pgfscope}%
\begin{pgfscope}%
\pgfsetbuttcap%
\pgfsetroundjoin%
\definecolor{currentfill}{rgb}{0.000000,0.000000,0.000000}%
\pgfsetfillcolor{currentfill}%
\pgfsetlinewidth{1.003750pt}%
\definecolor{currentstroke}{rgb}{0.000000,0.000000,0.000000}%
\pgfsetstrokecolor{currentstroke}%
\pgfsetdash{}{0pt}%
\pgfsys@defobject{currentmarker}{\pgfqpoint{-0.034722in}{0.000000in}}{\pgfqpoint{0.000000in}{0.000000in}}{%
\pgfpathmoveto{\pgfqpoint{0.000000in}{0.000000in}}%
\pgfpathlineto{\pgfqpoint{-0.034722in}{0.000000in}}%
\pgfusepath{stroke,fill}%
}%
\begin{pgfscope}%
\pgfsys@transformshift{0.575469in}{0.833522in}%
\pgfsys@useobject{currentmarker}{}%
\end{pgfscope}%
\end{pgfscope}%
\begin{pgfscope}%
\pgfsetbuttcap%
\pgfsetroundjoin%
\definecolor{currentfill}{rgb}{0.000000,0.000000,0.000000}%
\pgfsetfillcolor{currentfill}%
\pgfsetlinewidth{1.003750pt}%
\definecolor{currentstroke}{rgb}{0.000000,0.000000,0.000000}%
\pgfsetstrokecolor{currentstroke}%
\pgfsetdash{}{0pt}%
\pgfsys@defobject{currentmarker}{\pgfqpoint{-0.034722in}{0.000000in}}{\pgfqpoint{0.000000in}{0.000000in}}{%
\pgfpathmoveto{\pgfqpoint{0.000000in}{0.000000in}}%
\pgfpathlineto{\pgfqpoint{-0.034722in}{0.000000in}}%
\pgfusepath{stroke,fill}%
}%
\begin{pgfscope}%
\pgfsys@transformshift{0.575469in}{1.378939in}%
\pgfsys@useobject{currentmarker}{}%
\end{pgfscope}%
\end{pgfscope}%
\begin{pgfscope}%
\pgfsetbuttcap%
\pgfsetroundjoin%
\definecolor{currentfill}{rgb}{0.000000,0.000000,0.000000}%
\pgfsetfillcolor{currentfill}%
\pgfsetlinewidth{1.003750pt}%
\definecolor{currentstroke}{rgb}{0.000000,0.000000,0.000000}%
\pgfsetstrokecolor{currentstroke}%
\pgfsetdash{}{0pt}%
\pgfsys@defobject{currentmarker}{\pgfqpoint{-0.034722in}{0.000000in}}{\pgfqpoint{0.000000in}{0.000000in}}{%
\pgfpathmoveto{\pgfqpoint{0.000000in}{0.000000in}}%
\pgfpathlineto{\pgfqpoint{-0.034722in}{0.000000in}}%
\pgfusepath{stroke,fill}%
}%
\begin{pgfscope}%
\pgfsys@transformshift{0.575469in}{1.924356in}%
\pgfsys@useobject{currentmarker}{}%
\end{pgfscope}%
\end{pgfscope}%
\begin{pgfscope}%
\pgfsetbuttcap%
\pgfsetroundjoin%
\definecolor{currentfill}{rgb}{0.000000,0.000000,0.000000}%
\pgfsetfillcolor{currentfill}%
\pgfsetlinewidth{1.003750pt}%
\definecolor{currentstroke}{rgb}{0.000000,0.000000,0.000000}%
\pgfsetstrokecolor{currentstroke}%
\pgfsetdash{}{0pt}%
\pgfsys@defobject{currentmarker}{\pgfqpoint{-0.034722in}{0.000000in}}{\pgfqpoint{0.000000in}{0.000000in}}{%
\pgfpathmoveto{\pgfqpoint{0.000000in}{0.000000in}}%
\pgfpathlineto{\pgfqpoint{-0.034722in}{0.000000in}}%
\pgfusepath{stroke,fill}%
}%
\begin{pgfscope}%
\pgfsys@transformshift{0.575469in}{2.469772in}%
\pgfsys@useobject{currentmarker}{}%
\end{pgfscope}%
\end{pgfscope}%
\begin{pgfscope}%
\pgfsetbuttcap%
\pgfsetroundjoin%
\definecolor{currentfill}{rgb}{0.000000,0.000000,0.000000}%
\pgfsetfillcolor{currentfill}%
\pgfsetlinewidth{1.003750pt}%
\definecolor{currentstroke}{rgb}{0.000000,0.000000,0.000000}%
\pgfsetstrokecolor{currentstroke}%
\pgfsetdash{}{0pt}%
\pgfsys@defobject{currentmarker}{\pgfqpoint{-0.034722in}{0.000000in}}{\pgfqpoint{0.000000in}{0.000000in}}{%
\pgfpathmoveto{\pgfqpoint{0.000000in}{0.000000in}}%
\pgfpathlineto{\pgfqpoint{-0.034722in}{0.000000in}}%
\pgfusepath{stroke,fill}%
}%
\begin{pgfscope}%
\pgfsys@transformshift{0.575469in}{3.015189in}%
\pgfsys@useobject{currentmarker}{}%
\end{pgfscope}%
\end{pgfscope}%
\begin{pgfscope}%
\pgfsetbuttcap%
\pgfsetroundjoin%
\definecolor{currentfill}{rgb}{0.000000,0.000000,0.000000}%
\pgfsetfillcolor{currentfill}%
\pgfsetlinewidth{1.003750pt}%
\definecolor{currentstroke}{rgb}{0.000000,0.000000,0.000000}%
\pgfsetstrokecolor{currentstroke}%
\pgfsetdash{}{0pt}%
\pgfsys@defobject{currentmarker}{\pgfqpoint{-0.034722in}{0.000000in}}{\pgfqpoint{0.000000in}{0.000000in}}{%
\pgfpathmoveto{\pgfqpoint{0.000000in}{0.000000in}}%
\pgfpathlineto{\pgfqpoint{-0.034722in}{0.000000in}}%
\pgfusepath{stroke,fill}%
}%
\begin{pgfscope}%
\pgfsys@transformshift{0.575469in}{3.560606in}%
\pgfsys@useobject{currentmarker}{}%
\end{pgfscope}%
\end{pgfscope}%
\begin{pgfscope}%
\pgftext[x=0.249080in,y=2.197064in,,bottom,rotate=90.000000]{\rmfamily\fontsize{12.000000}{14.400000}\selectfont Pressure, bar}%
\end{pgfscope}%
\begin{pgfscope}%
\pgfpathrectangle{\pgfqpoint{0.575469in}{0.560814in}}{\pgfqpoint{3.100000in}{3.272500in}} %
\pgfusepath{clip}%
\pgfsetrectcap%
\pgfsetroundjoin%
\pgfsetlinewidth{1.505625pt}%
\definecolor{currentstroke}{rgb}{0.121569,0.466667,0.705882}%
\pgfsetstrokecolor{currentstroke}%
\pgfsetdash{}{0pt}%
\pgfpathmoveto{\pgfqpoint{0.574849in}{0.625132in}}%
\pgfpathlineto{\pgfqpoint{0.619489in}{0.629104in}}%
\pgfpathlineto{\pgfqpoint{0.711869in}{0.638810in}}%
\pgfpathlineto{\pgfqpoint{0.727989in}{0.640852in}}%
\pgfpathlineto{\pgfqpoint{0.786269in}{0.649816in}}%
\pgfpathlineto{\pgfqpoint{0.844549in}{0.660586in}}%
\pgfpathlineto{\pgfqpoint{0.864389in}{0.664803in}}%
\pgfpathlineto{\pgfqpoint{0.879889in}{0.668368in}}%
\pgfpathlineto{\pgfqpoint{0.894769in}{0.672539in}}%
\pgfpathlineto{\pgfqpoint{0.936929in}{0.682950in}}%
\pgfpathlineto{\pgfqpoint{1.002029in}{0.704622in}}%
\pgfpathlineto{\pgfqpoint{1.049769in}{0.724681in}}%
\pgfpathlineto{\pgfqpoint{1.063409in}{0.731760in}}%
\pgfpathlineto{\pgfqpoint{1.097509in}{0.750797in}}%
\pgfpathlineto{\pgfqpoint{1.111149in}{0.759537in}}%
\pgfpathlineto{\pgfqpoint{1.151449in}{0.791984in}}%
\pgfpathlineto{\pgfqpoint{1.168189in}{0.806619in}}%
\pgfpathlineto{\pgfqpoint{1.178729in}{0.817169in}}%
\pgfpathlineto{\pgfqpoint{1.194229in}{0.834838in}}%
\pgfpathlineto{\pgfqpoint{1.214069in}{0.860850in}}%
\pgfpathlineto{\pgfqpoint{1.237009in}{0.896321in}}%
\pgfpathlineto{\pgfqpoint{1.258089in}{0.934141in}}%
\pgfpathlineto{\pgfqpoint{1.278549in}{0.977300in}}%
\pgfpathlineto{\pgfqpoint{1.293429in}{1.011372in}}%
\pgfpathlineto{\pgfqpoint{1.333729in}{1.116159in}}%
\pgfpathlineto{\pgfqpoint{1.355429in}{1.180428in}}%
\pgfpathlineto{\pgfqpoint{1.362869in}{1.203992in}}%
\pgfpathlineto{\pgfqpoint{1.376509in}{1.238945in}}%
\pgfpathlineto{\pgfqpoint{1.404409in}{1.306007in}}%
\pgfpathlineto{\pgfqpoint{1.417429in}{1.330796in}}%
\pgfpathlineto{\pgfqpoint{1.428589in}{1.344492in}}%
\pgfpathlineto{\pgfqpoint{1.433549in}{1.349705in}}%
\pgfpathlineto{\pgfqpoint{1.439749in}{1.356049in}}%
\pgfpathlineto{\pgfqpoint{1.447189in}{1.359067in}}%
\pgfpathlineto{\pgfqpoint{1.455249in}{1.361262in}}%
\pgfpathlineto{\pgfqpoint{1.459589in}{1.362375in}}%
\pgfpathlineto{\pgfqpoint{1.481909in}{1.366316in}}%
\pgfpathlineto{\pgfqpoint{1.498649in}{1.373781in}}%
\pgfpathlineto{\pgfqpoint{1.507329in}{1.374099in}}%
\pgfpathlineto{\pgfqpoint{1.523449in}{1.372051in}}%
\pgfpathlineto{\pgfqpoint{1.536469in}{1.372722in}}%
\pgfpathlineto{\pgfqpoint{1.547009in}{1.372898in}}%
\pgfpathlineto{\pgfqpoint{1.553209in}{1.372277in}}%
\pgfpathlineto{\pgfqpoint{1.558789in}{1.371389in}}%
\pgfpathlineto{\pgfqpoint{1.566229in}{1.371135in}}%
\pgfpathlineto{\pgfqpoint{1.574289in}{1.371608in}}%
\pgfpathlineto{\pgfqpoint{1.581109in}{1.370669in}}%
\pgfpathlineto{\pgfqpoint{1.595989in}{1.366031in}}%
\pgfpathlineto{\pgfqpoint{1.604049in}{1.363212in}}%
\pgfpathlineto{\pgfqpoint{1.610869in}{1.362797in}}%
\pgfpathlineto{\pgfqpoint{1.617069in}{1.363695in}}%
\pgfpathlineto{\pgfqpoint{1.633809in}{1.365474in}}%
\pgfpathlineto{\pgfqpoint{1.646209in}{1.361928in}}%
\pgfpathlineto{\pgfqpoint{1.663569in}{1.358130in}}%
\pgfpathlineto{\pgfqpoint{1.726189in}{1.352472in}}%
\pgfpathlineto{\pgfqpoint{1.740449in}{1.350531in}}%
\pgfpathlineto{\pgfqpoint{1.754089in}{1.350796in}}%
\pgfpathlineto{\pgfqpoint{1.770209in}{1.349989in}}%
\pgfpathlineto{\pgfqpoint{1.812989in}{1.345901in}}%
\pgfpathlineto{\pgfqpoint{1.831589in}{1.344841in}}%
\pgfpathlineto{\pgfqpoint{1.865069in}{1.341215in}}%
\pgfpathlineto{\pgfqpoint{1.878709in}{1.340668in}}%
\pgfpathlineto{\pgfqpoint{1.915289in}{1.338039in}}%
\pgfpathlineto{\pgfqpoint{1.935129in}{1.337092in}}%
\pgfpathlineto{\pgfqpoint{1.952489in}{1.337717in}}%
\pgfpathlineto{\pgfqpoint{1.965509in}{1.336679in}}%
\pgfpathlineto{\pgfqpoint{2.000849in}{1.334431in}}%
\pgfpathlineto{\pgfqpoint{2.013869in}{1.333638in}}%
\pgfpathlineto{\pgfqpoint{2.063469in}{1.329109in}}%
\pgfpathlineto{\pgfqpoint{2.077729in}{1.328541in}}%
\pgfpathlineto{\pgfqpoint{2.098189in}{1.327415in}}%
\pgfpathlineto{\pgfqpoint{2.130429in}{1.325184in}}%
\pgfpathlineto{\pgfqpoint{2.140969in}{1.325025in}}%
\pgfpathlineto{\pgfqpoint{2.170109in}{1.324660in}}%
\pgfpathlineto{\pgfqpoint{2.182509in}{1.323662in}}%
\pgfpathlineto{\pgfqpoint{2.199249in}{1.322965in}}%
\pgfpathlineto{\pgfqpoint{2.218469in}{1.322903in}}%
\pgfpathlineto{\pgfqpoint{2.246369in}{1.320649in}}%
\pgfpathlineto{\pgfqpoint{2.279229in}{1.318997in}}%
\pgfpathlineto{\pgfqpoint{2.300929in}{1.318183in}}%
\pgfpathlineto{\pgfqpoint{2.328829in}{1.317421in}}%
\pgfpathlineto{\pgfqpoint{2.348049in}{1.317636in}}%
\pgfpathlineto{\pgfqpoint{2.370989in}{1.317961in}}%
\pgfpathlineto{\pgfqpoint{2.381529in}{1.318004in}}%
\pgfpathlineto{\pgfqpoint{2.398269in}{1.316911in}}%
\pgfpathlineto{\pgfqpoint{2.413769in}{1.318480in}}%
\pgfpathlineto{\pgfqpoint{2.426169in}{1.319630in}}%
\pgfpathlineto{\pgfqpoint{2.480109in}{1.322397in}}%
\pgfpathlineto{\pgfqpoint{2.540249in}{1.330502in}}%
\pgfpathlineto{\pgfqpoint{2.561949in}{1.334071in}}%
\pgfpathlineto{\pgfqpoint{2.622089in}{1.340859in}}%
\pgfpathlineto{\pgfqpoint{2.638829in}{1.342766in}}%
\pgfpathlineto{\pgfqpoint{2.669829in}{1.347068in}}%
\pgfpathlineto{\pgfqpoint{2.745469in}{1.354220in}}%
\pgfpathlineto{\pgfqpoint{2.760349in}{1.354738in}}%
\pgfpathlineto{\pgfqpoint{2.778329in}{1.356175in}}%
\pgfpathlineto{\pgfqpoint{2.801269in}{1.356776in}}%
\pgfpathlineto{\pgfqpoint{2.831029in}{1.358982in}}%
\pgfpathlineto{\pgfqpoint{2.883729in}{1.362323in}}%
\pgfpathlineto{\pgfqpoint{2.901709in}{1.364524in}}%
\pgfpathlineto{\pgfqpoint{2.940769in}{1.367700in}}%
\pgfpathlineto{\pgfqpoint{3.036869in}{1.377799in}}%
\pgfpathlineto{\pgfqpoint{3.090809in}{1.385708in}}%
\pgfpathlineto{\pgfqpoint{3.134209in}{1.395042in}}%
\pgfpathlineto{\pgfqpoint{3.162729in}{1.404160in}}%
\pgfpathlineto{\pgfqpoint{3.180089in}{1.412205in}}%
\pgfpathlineto{\pgfqpoint{3.191249in}{1.418873in}}%
\pgfpathlineto{\pgfqpoint{3.199309in}{1.425537in}}%
\pgfpathlineto{\pgfqpoint{3.204889in}{1.432903in}}%
\pgfpathlineto{\pgfqpoint{3.209849in}{1.442928in}}%
\pgfpathlineto{\pgfqpoint{3.214189in}{1.456786in}}%
\pgfpathlineto{\pgfqpoint{3.218529in}{1.478684in}}%
\pgfpathlineto{\pgfqpoint{3.222249in}{1.508738in}}%
\pgfpathlineto{\pgfqpoint{3.225969in}{1.556837in}}%
\pgfpathlineto{\pgfqpoint{3.229689in}{1.634367in}}%
\pgfpathlineto{\pgfqpoint{3.233409in}{1.760366in}}%
\pgfpathlineto{\pgfqpoint{3.237129in}{1.966894in}}%
\pgfpathlineto{\pgfqpoint{3.240849in}{2.306568in}}%
\pgfpathlineto{\pgfqpoint{3.248909in}{3.186784in}}%
\pgfpathlineto{\pgfqpoint{3.252009in}{3.284945in}}%
\pgfpathlineto{\pgfqpoint{3.256969in}{3.364793in}}%
\pgfpathlineto{\pgfqpoint{3.259449in}{3.386682in}}%
\pgfpathlineto{\pgfqpoint{3.260689in}{3.395146in}}%
\pgfpathlineto{\pgfqpoint{3.261309in}{3.393402in}}%
\pgfpathlineto{\pgfqpoint{3.269989in}{3.317281in}}%
\pgfpathlineto{\pgfqpoint{3.270609in}{3.314696in}}%
\pgfpathlineto{\pgfqpoint{3.277429in}{3.234831in}}%
\pgfpathlineto{\pgfqpoint{3.284249in}{3.172989in}}%
\pgfpathlineto{\pgfqpoint{3.285489in}{3.165598in}}%
\pgfpathlineto{\pgfqpoint{3.288589in}{3.137791in}}%
\pgfpathlineto{\pgfqpoint{3.290449in}{3.120914in}}%
\pgfpathlineto{\pgfqpoint{3.292309in}{3.107623in}}%
\pgfpathlineto{\pgfqpoint{3.296029in}{3.071500in}}%
\pgfpathlineto{\pgfqpoint{3.300369in}{3.039189in}}%
\pgfpathlineto{\pgfqpoint{3.309669in}{2.975198in}}%
\pgfpathlineto{\pgfqpoint{3.314629in}{2.935192in}}%
\pgfpathlineto{\pgfqpoint{3.318349in}{2.910805in}}%
\pgfpathlineto{\pgfqpoint{3.334469in}{2.830995in}}%
\pgfpathlineto{\pgfqpoint{3.336329in}{2.823789in}}%
\pgfpathlineto{\pgfqpoint{3.359889in}{2.725812in}}%
\pgfpathlineto{\pgfqpoint{3.361749in}{2.717859in}}%
\pgfpathlineto{\pgfqpoint{3.365469in}{2.702614in}}%
\pgfpathlineto{\pgfqpoint{3.380969in}{2.648185in}}%
\pgfpathlineto{\pgfqpoint{3.384689in}{2.637574in}}%
\pgfpathlineto{\pgfqpoint{3.407009in}{2.570903in}}%
\pgfpathlineto{\pgfqpoint{3.434909in}{2.502686in}}%
\pgfpathlineto{\pgfqpoint{3.437389in}{2.498087in}}%
\pgfpathlineto{\pgfqpoint{3.452269in}{2.466506in}}%
\pgfpathlineto{\pgfqpoint{3.455369in}{2.459668in}}%
\pgfpathlineto{\pgfqpoint{3.465289in}{2.440797in}}%
\pgfpathlineto{\pgfqpoint{3.470249in}{2.433706in}}%
\pgfpathlineto{\pgfqpoint{3.475829in}{2.424951in}}%
\pgfpathlineto{\pgfqpoint{3.481409in}{2.416544in}}%
\pgfpathlineto{\pgfqpoint{3.490089in}{2.400601in}}%
\pgfpathlineto{\pgfqpoint{3.496289in}{2.391082in}}%
\pgfpathlineto{\pgfqpoint{3.505589in}{2.380920in}}%
\pgfpathlineto{\pgfqpoint{3.517989in}{2.363966in}}%
\pgfpathlineto{\pgfqpoint{3.528529in}{2.352075in}}%
\pgfpathlineto{\pgfqpoint{3.532249in}{2.348443in}}%
\pgfpathlineto{\pgfqpoint{3.540309in}{2.339803in}}%
\pgfpathlineto{\pgfqpoint{3.545269in}{2.334765in}}%
\pgfpathlineto{\pgfqpoint{3.552709in}{2.327914in}}%
\pgfpathlineto{\pgfqpoint{3.560769in}{2.322285in}}%
\pgfpathlineto{\pgfqpoint{3.566969in}{2.317847in}}%
\pgfpathlineto{\pgfqpoint{3.575029in}{2.311155in}}%
\pgfpathlineto{\pgfqpoint{3.581849in}{2.306970in}}%
\pgfpathlineto{\pgfqpoint{3.586809in}{2.304953in}}%
\pgfpathlineto{\pgfqpoint{3.597349in}{2.299478in}}%
\pgfpathlineto{\pgfqpoint{3.622769in}{2.289710in}}%
\pgfpathlineto{\pgfqpoint{3.627729in}{2.288080in}}%
\pgfpathlineto{\pgfqpoint{3.637649in}{2.285056in}}%
\pgfpathlineto{\pgfqpoint{3.643849in}{2.284243in}}%
\pgfpathlineto{\pgfqpoint{3.654389in}{2.282317in}}%
\pgfpathlineto{\pgfqpoint{3.676089in}{2.274372in}}%
\pgfpathlineto{\pgfqpoint{3.676089in}{2.274372in}}%
\pgfusepath{stroke}%
\end{pgfscope}%
\begin{pgfscope}%
\pgfpathrectangle{\pgfqpoint{0.575469in}{0.560814in}}{\pgfqpoint{3.100000in}{3.272500in}} %
\pgfusepath{clip}%
\pgfsetrectcap%
\pgfsetroundjoin%
\pgfsetlinewidth{1.505625pt}%
\definecolor{currentstroke}{rgb}{1.000000,0.498039,0.054902}%
\pgfsetstrokecolor{currentstroke}%
\pgfsetdash{}{0pt}%
\pgfpathmoveto{\pgfqpoint{0.575469in}{0.624976in}}%
\pgfpathlineto{\pgfqpoint{0.623209in}{0.629200in}}%
\pgfpathlineto{\pgfqpoint{0.730469in}{0.640579in}}%
\pgfpathlineto{\pgfqpoint{0.823469in}{0.655355in}}%
\pgfpathlineto{\pgfqpoint{0.901589in}{0.671167in}}%
\pgfpathlineto{\pgfqpoint{0.984669in}{0.695995in}}%
\pgfpathlineto{\pgfqpoint{1.009469in}{0.705838in}}%
\pgfpathlineto{\pgfqpoint{1.047909in}{0.722086in}}%
\pgfpathlineto{\pgfqpoint{1.070229in}{0.733451in}}%
\pgfpathlineto{\pgfqpoint{1.108049in}{0.754957in}}%
\pgfpathlineto{\pgfqpoint{1.127269in}{0.767729in}}%
\pgfpathlineto{\pgfqpoint{1.149589in}{0.785379in}}%
\pgfpathlineto{\pgfqpoint{1.170049in}{0.803655in}}%
\pgfpathlineto{\pgfqpoint{1.193609in}{0.828779in}}%
\pgfpathlineto{\pgfqpoint{1.207869in}{0.845925in}}%
\pgfpathlineto{\pgfqpoint{1.225849in}{0.871450in}}%
\pgfpathlineto{\pgfqpoint{1.245069in}{0.902930in}}%
\pgfpathlineto{\pgfqpoint{1.270489in}{0.952046in}}%
\pgfpathlineto{\pgfqpoint{1.288469in}{0.991431in}}%
\pgfpathlineto{\pgfqpoint{1.346749in}{1.141437in}}%
\pgfpathlineto{\pgfqpoint{1.370929in}{1.212772in}}%
\pgfpathlineto{\pgfqpoint{1.396969in}{1.279096in}}%
\pgfpathlineto{\pgfqpoint{1.417429in}{1.316380in}}%
\pgfpathlineto{\pgfqpoint{1.434789in}{1.340894in}}%
\pgfpathlineto{\pgfqpoint{1.442849in}{1.348891in}}%
\pgfpathlineto{\pgfqpoint{1.449669in}{1.352395in}}%
\pgfpathlineto{\pgfqpoint{1.458349in}{1.355517in}}%
\pgfpathlineto{\pgfqpoint{1.470129in}{1.358586in}}%
\pgfpathlineto{\pgfqpoint{1.483769in}{1.362500in}}%
\pgfpathlineto{\pgfqpoint{1.502369in}{1.369119in}}%
\pgfpathlineto{\pgfqpoint{1.511049in}{1.369315in}}%
\pgfpathlineto{\pgfqpoint{1.534609in}{1.369683in}}%
\pgfpathlineto{\pgfqpoint{1.539569in}{1.369563in}}%
\pgfpathlineto{\pgfqpoint{1.547009in}{1.369621in}}%
\pgfpathlineto{\pgfqpoint{1.556929in}{1.368623in}}%
\pgfpathlineto{\pgfqpoint{1.620169in}{1.358763in}}%
\pgfpathlineto{\pgfqpoint{1.633809in}{1.358445in}}%
\pgfpathlineto{\pgfqpoint{1.647449in}{1.355439in}}%
\pgfpathlineto{\pgfqpoint{1.658609in}{1.353274in}}%
\pgfpathlineto{\pgfqpoint{1.679069in}{1.352598in}}%
\pgfpathlineto{\pgfqpoint{1.695189in}{1.351399in}}%
\pgfpathlineto{\pgfqpoint{1.742929in}{1.346933in}}%
\pgfpathlineto{\pgfqpoint{1.780129in}{1.344246in}}%
\pgfpathlineto{\pgfqpoint{1.806789in}{1.341291in}}%
\pgfpathlineto{\pgfqpoint{1.840889in}{1.338897in}}%
\pgfpathlineto{\pgfqpoint{1.859489in}{1.337732in}}%
\pgfpathlineto{\pgfqpoint{1.915289in}{1.333541in}}%
\pgfpathlineto{\pgfqpoint{1.932649in}{1.331817in}}%
\pgfpathlineto{\pgfqpoint{1.972949in}{1.330092in}}%
\pgfpathlineto{\pgfqpoint{1.994029in}{1.328721in}}%
\pgfpathlineto{\pgfqpoint{2.025649in}{1.328105in}}%
\pgfpathlineto{\pgfqpoint{2.047349in}{1.326525in}}%
\pgfpathlineto{\pgfqpoint{2.107489in}{1.322011in}}%
\pgfpathlineto{\pgfqpoint{2.147169in}{1.319088in}}%
\pgfpathlineto{\pgfqpoint{2.163909in}{1.317607in}}%
\pgfpathlineto{\pgfqpoint{2.188709in}{1.315778in}}%
\pgfpathlineto{\pgfqpoint{2.208549in}{1.314642in}}%
\pgfpathlineto{\pgfqpoint{2.224049in}{1.313736in}}%
\pgfpathlineto{\pgfqpoint{2.251949in}{1.311749in}}%
\pgfpathlineto{\pgfqpoint{2.294729in}{1.308929in}}%
\pgfpathlineto{\pgfqpoint{2.497469in}{1.292023in}}%
\pgfpathlineto{\pgfqpoint{2.534669in}{1.290434in}}%
\pgfpathlineto{\pgfqpoint{2.586749in}{1.288025in}}%
\pgfpathlineto{\pgfqpoint{2.648749in}{1.285230in}}%
\pgfpathlineto{\pgfqpoint{2.672309in}{1.284364in}}%
\pgfpathlineto{\pgfqpoint{2.750429in}{1.281978in}}%
\pgfpathlineto{\pgfqpoint{2.826069in}{1.280685in}}%
\pgfpathlineto{\pgfqpoint{2.896129in}{1.279997in}}%
\pgfpathlineto{\pgfqpoint{2.912249in}{1.279811in}}%
\pgfpathlineto{\pgfqpoint{2.930229in}{1.279072in}}%
\pgfpathlineto{\pgfqpoint{2.960609in}{1.278886in}}%
\pgfpathlineto{\pgfqpoint{2.989129in}{1.278681in}}%
\pgfpathlineto{\pgfqpoint{3.136069in}{1.276220in}}%
\pgfpathlineto{\pgfqpoint{3.183809in}{1.276323in}}%
\pgfpathlineto{\pgfqpoint{3.225349in}{1.274496in}}%
\pgfpathlineto{\pgfqpoint{3.354309in}{1.270757in}}%
\pgfpathlineto{\pgfqpoint{3.413829in}{1.269901in}}%
\pgfpathlineto{\pgfqpoint{3.435529in}{1.268745in}}%
\pgfpathlineto{\pgfqpoint{3.473969in}{1.267143in}}%
\pgfpathlineto{\pgfqpoint{3.606649in}{1.264430in}}%
\pgfpathlineto{\pgfqpoint{3.630829in}{1.264477in}}%
\pgfpathlineto{\pgfqpoint{3.676089in}{1.263149in}}%
\pgfpathlineto{\pgfqpoint{3.676089in}{1.263149in}}%
\pgfusepath{stroke}%
\end{pgfscope}%
\begin{pgfscope}%
\pgfpathrectangle{\pgfqpoint{0.575469in}{0.560814in}}{\pgfqpoint{3.100000in}{3.272500in}} %
\pgfusepath{clip}%
\pgfsetrectcap%
\pgfsetroundjoin%
\pgfsetlinewidth{1.505625pt}%
\definecolor{currentstroke}{rgb}{0.000000,0.000000,0.000000}%
\pgfsetstrokecolor{currentstroke}%
\pgfsetdash{}{0pt}%
\pgfpathmoveto{\pgfqpoint{1.505469in}{0.560814in}}%
\pgfpathlineto{\pgfqpoint{1.505469in}{1.869814in}}%
\pgfusepath{stroke}%
\end{pgfscope}%
\begin{pgfscope}%
\pgfpathrectangle{\pgfqpoint{0.575469in}{0.560814in}}{\pgfqpoint{3.100000in}{3.272500in}} %
\pgfusepath{clip}%
\pgfsetrectcap%
\pgfsetroundjoin%
\pgfsetlinewidth{1.505625pt}%
\definecolor{currentstroke}{rgb}{0.000000,0.000000,0.000000}%
\pgfsetstrokecolor{currentstroke}%
\pgfsetdash{}{0pt}%
\pgfpathmoveto{\pgfqpoint{3.245189in}{0.560814in}}%
\pgfpathlineto{\pgfqpoint{3.245189in}{3.833314in}}%
\pgfusepath{stroke}%
\end{pgfscope}%
\begin{pgfscope}%
\pgfpathrectangle{\pgfqpoint{0.575469in}{0.560814in}}{\pgfqpoint{3.100000in}{3.272500in}} %
\pgfusepath{clip}%
\pgfsetrectcap%
\pgfsetroundjoin%
\pgfsetlinewidth{1.505625pt}%
\definecolor{currentstroke}{rgb}{0.000000,0.000000,0.000000}%
\pgfsetstrokecolor{currentstroke}%
\pgfsetdash{}{0pt}%
\pgfpathmoveto{\pgfqpoint{2.532809in}{0.560814in}}%
\pgfpathlineto{\pgfqpoint{2.532809in}{1.215314in}}%
\pgfusepath{stroke}%
\end{pgfscope}%
\begin{pgfscope}%
\pgfsetrectcap%
\pgfsetmiterjoin%
\pgfsetlinewidth{0.803000pt}%
\definecolor{currentstroke}{rgb}{0.000000,0.000000,0.000000}%
\pgfsetstrokecolor{currentstroke}%
\pgfsetdash{}{0pt}%
\pgfpathmoveto{\pgfqpoint{0.575469in}{0.560814in}}%
\pgfpathlineto{\pgfqpoint{0.575469in}{3.833314in}}%
\pgfusepath{stroke}%
\end{pgfscope}%
\begin{pgfscope}%
\pgfsetrectcap%
\pgfsetmiterjoin%
\pgfsetlinewidth{0.803000pt}%
\definecolor{currentstroke}{rgb}{0.000000,0.000000,0.000000}%
\pgfsetstrokecolor{currentstroke}%
\pgfsetdash{}{0pt}%
\pgfpathmoveto{\pgfqpoint{3.675469in}{0.560814in}}%
\pgfpathlineto{\pgfqpoint{3.675469in}{3.833314in}}%
\pgfusepath{stroke}%
\end{pgfscope}%
\begin{pgfscope}%
\pgfsetrectcap%
\pgfsetmiterjoin%
\pgfsetlinewidth{0.803000pt}%
\definecolor{currentstroke}{rgb}{0.000000,0.000000,0.000000}%
\pgfsetstrokecolor{currentstroke}%
\pgfsetdash{}{0pt}%
\pgfpathmoveto{\pgfqpoint{0.575469in}{0.560814in}}%
\pgfpathlineto{\pgfqpoint{3.675469in}{0.560814in}}%
\pgfusepath{stroke}%
\end{pgfscope}%
\begin{pgfscope}%
\pgfsetrectcap%
\pgfsetmiterjoin%
\pgfsetlinewidth{0.803000pt}%
\definecolor{currentstroke}{rgb}{0.000000,0.000000,0.000000}%
\pgfsetstrokecolor{currentstroke}%
\pgfsetdash{}{0pt}%
\pgfpathmoveto{\pgfqpoint{0.575469in}{3.833314in}}%
\pgfpathlineto{\pgfqpoint{3.675469in}{3.833314in}}%
\pgfusepath{stroke}%
\end{pgfscope}%
\begin{pgfscope}%
\pgfsetroundcap%
\pgfsetroundjoin%
\pgfsetlinewidth{1.003750pt}%
\definecolor{currentstroke}{rgb}{0.000000,0.000000,0.000000}%
\pgfsetstrokecolor{currentstroke}%
\pgfsetdash{}{0pt}%
\pgfpathmoveto{\pgfqpoint{2.517281in}{0.888064in}}%
\pgfpathquadraticcurveto{\pgfqpoint{2.019139in}{0.888064in}}{\pgfqpoint{1.520997in}{0.888064in}}%
\pgfusepath{stroke}%
\end{pgfscope}%
\begin{pgfscope}%
\pgfsetroundcap%
\pgfsetroundjoin%
\definecolor{currentfill}{rgb}{0.000000,0.000000,0.000000}%
\pgfsetfillcolor{currentfill}%
\pgfsetlinewidth{1.003750pt}%
\definecolor{currentstroke}{rgb}{0.000000,0.000000,0.000000}%
\pgfsetstrokecolor{currentstroke}%
\pgfsetdash{}{0pt}%
\pgfpathmoveto{\pgfqpoint{2.450614in}{0.921397in}}%
\pgfpathlineto{\pgfqpoint{2.517281in}{0.888064in}}%
\pgfpathlineto{\pgfqpoint{2.450614in}{0.854731in}}%
\pgfpathlineto{\pgfqpoint{2.450614in}{0.921397in}}%
\pgfpathclose%
\pgfusepath{stroke,fill}%
\end{pgfscope}%
\begin{pgfscope}%
\pgfsetroundcap%
\pgfsetroundjoin%
\definecolor{currentfill}{rgb}{0.000000,0.000000,0.000000}%
\pgfsetfillcolor{currentfill}%
\pgfsetlinewidth{1.003750pt}%
\definecolor{currentstroke}{rgb}{0.000000,0.000000,0.000000}%
\pgfsetstrokecolor{currentstroke}%
\pgfsetdash{}{0pt}%
\pgfpathmoveto{\pgfqpoint{1.587664in}{0.854731in}}%
\pgfpathlineto{\pgfqpoint{1.520997in}{0.888064in}}%
\pgfpathlineto{\pgfqpoint{1.587664in}{0.921397in}}%
\pgfpathlineto{\pgfqpoint{1.587664in}{0.854731in}}%
\pgfpathclose%
\pgfusepath{stroke,fill}%
\end{pgfscope}%
\begin{pgfscope}%
\pgftext[x=2.019139in,y=0.971397in,left,base]{\rmfamily\fontsize{12.000000}{14.400000}\selectfont \(\displaystyle \tau_1\)}%
\end{pgfscope}%
\begin{pgfscope}%
\pgftext[x=2.375329in,y=1.790536in,left,base]{\rmfamily\fontsize{12.000000}{14.400000}\selectfont \(\displaystyle \tau\)}%
\end{pgfscope}%
\begin{pgfscope}%
\pgfsetroundcap%
\pgfsetroundjoin%
\pgfsetlinewidth{1.003750pt}%
\definecolor{currentstroke}{rgb}{0.000000,0.000000,0.000000}%
\pgfsetstrokecolor{currentstroke}%
\pgfsetdash{}{0pt}%
\pgfpathmoveto{\pgfqpoint{3.229661in}{1.651647in}}%
\pgfpathquadraticcurveto{\pgfqpoint{2.375329in}{1.651647in}}{\pgfqpoint{1.520997in}{1.651647in}}%
\pgfusepath{stroke}%
\end{pgfscope}%
\begin{pgfscope}%
\pgfsetroundcap%
\pgfsetroundjoin%
\definecolor{currentfill}{rgb}{0.000000,0.000000,0.000000}%
\pgfsetfillcolor{currentfill}%
\pgfsetlinewidth{1.003750pt}%
\definecolor{currentstroke}{rgb}{0.000000,0.000000,0.000000}%
\pgfsetstrokecolor{currentstroke}%
\pgfsetdash{}{0pt}%
\pgfpathmoveto{\pgfqpoint{3.162994in}{1.684981in}}%
\pgfpathlineto{\pgfqpoint{3.229661in}{1.651647in}}%
\pgfpathlineto{\pgfqpoint{3.162994in}{1.618314in}}%
\pgfpathlineto{\pgfqpoint{3.162994in}{1.684981in}}%
\pgfpathclose%
\pgfusepath{stroke,fill}%
\end{pgfscope}%
\begin{pgfscope}%
\pgfsetroundcap%
\pgfsetroundjoin%
\definecolor{currentfill}{rgb}{0.000000,0.000000,0.000000}%
\pgfsetfillcolor{currentfill}%
\pgfsetlinewidth{1.003750pt}%
\definecolor{currentstroke}{rgb}{0.000000,0.000000,0.000000}%
\pgfsetstrokecolor{currentstroke}%
\pgfsetdash{}{0pt}%
\pgfpathmoveto{\pgfqpoint{1.587664in}{1.618314in}}%
\pgfpathlineto{\pgfqpoint{1.520997in}{1.651647in}}%
\pgfpathlineto{\pgfqpoint{1.587664in}{1.684981in}}%
\pgfpathlineto{\pgfqpoint{1.587664in}{1.618314in}}%
\pgfpathclose%
\pgfusepath{stroke,fill}%
\end{pgfscope}%
\begin{pgfscope}%
\pgftext[x=1.505469in,y=1.924356in,left,base]{\rmfamily\fontsize{12.000000}{14.400000}\selectfont EOC}%
\end{pgfscope}%
\begin{pgfscope}%
\pgfsetroundcap%
\pgfsetroundjoin%
\pgfsetlinewidth{1.003750pt}%
\definecolor{currentstroke}{rgb}{0.000000,0.000000,0.000000}%
\pgfsetstrokecolor{currentstroke}%
\pgfsetdash{}{0pt}%
\pgfpathmoveto{\pgfqpoint{0.900997in}{1.106231in}}%
\pgfpathquadraticcurveto{\pgfqpoint{1.102469in}{1.106231in}}{\pgfqpoint{1.319469in}{1.106231in}}%
\pgfusepath{stroke}%
\end{pgfscope}%
\begin{pgfscope}%
\pgfsetroundcap%
\pgfsetroundjoin%
\definecolor{currentfill}{rgb}{0.000000,0.000000,0.000000}%
\pgfsetfillcolor{currentfill}%
\pgfsetlinewidth{1.003750pt}%
\definecolor{currentstroke}{rgb}{0.000000,0.000000,0.000000}%
\pgfsetstrokecolor{currentstroke}%
\pgfsetdash{}{0pt}%
\pgfpathmoveto{\pgfqpoint{0.967664in}{1.072897in}}%
\pgfpathlineto{\pgfqpoint{0.900997in}{1.106231in}}%
\pgfpathlineto{\pgfqpoint{0.967664in}{1.139564in}}%
\pgfpathlineto{\pgfqpoint{0.967664in}{1.072897in}}%
\pgfpathclose%
\pgfusepath{stroke,fill}%
\end{pgfscope}%
\begin{pgfscope}%
\pgfsetbuttcap%
\pgfsetroundjoin%
\definecolor{currentfill}{rgb}{0.000000,0.000000,0.000000}%
\pgfsetfillcolor{currentfill}%
\pgfsetlinewidth{1.003750pt}%
\definecolor{currentstroke}{rgb}{0.000000,0.000000,0.000000}%
\pgfsetstrokecolor{currentstroke}%
\pgfsetdash{}{0pt}%
\pgfsys@defobject{currentmarker}{\pgfqpoint{0.000000in}{0.000000in}}{\pgfqpoint{0.069444in}{0.000000in}}{%
\pgfpathmoveto{\pgfqpoint{0.000000in}{0.000000in}}%
\pgfpathlineto{\pgfqpoint{0.069444in}{0.000000in}}%
\pgfusepath{stroke,fill}%
}%
\begin{pgfscope}%
\pgfsys@transformshift{3.675469in}{0.560814in}%
\pgfsys@useobject{currentmarker}{}%
\end{pgfscope}%
\end{pgfscope}%
\begin{pgfscope}%
\pgftext[x=3.793525in,y=0.513731in,left,base]{\rmfamily\fontsize{10.000000}{12.000000}\selectfont \(\displaystyle -0.25\)}%
\end{pgfscope}%
\begin{pgfscope}%
\pgfsetbuttcap%
\pgfsetroundjoin%
\definecolor{currentfill}{rgb}{0.000000,0.000000,0.000000}%
\pgfsetfillcolor{currentfill}%
\pgfsetlinewidth{1.003750pt}%
\definecolor{currentstroke}{rgb}{0.000000,0.000000,0.000000}%
\pgfsetstrokecolor{currentstroke}%
\pgfsetdash{}{0pt}%
\pgfsys@defobject{currentmarker}{\pgfqpoint{0.000000in}{0.000000in}}{\pgfqpoint{0.069444in}{0.000000in}}{%
\pgfpathmoveto{\pgfqpoint{0.000000in}{0.000000in}}%
\pgfpathlineto{\pgfqpoint{0.069444in}{0.000000in}}%
\pgfusepath{stroke,fill}%
}%
\begin{pgfscope}%
\pgfsys@transformshift{3.675469in}{0.924425in}%
\pgfsys@useobject{currentmarker}{}%
\end{pgfscope}%
\end{pgfscope}%
\begin{pgfscope}%
\pgftext[x=3.793525in,y=0.877342in,left,base]{\rmfamily\fontsize{10.000000}{12.000000}\selectfont \(\displaystyle 0.00\)}%
\end{pgfscope}%
\begin{pgfscope}%
\pgfsetbuttcap%
\pgfsetroundjoin%
\definecolor{currentfill}{rgb}{0.000000,0.000000,0.000000}%
\pgfsetfillcolor{currentfill}%
\pgfsetlinewidth{1.003750pt}%
\definecolor{currentstroke}{rgb}{0.000000,0.000000,0.000000}%
\pgfsetstrokecolor{currentstroke}%
\pgfsetdash{}{0pt}%
\pgfsys@defobject{currentmarker}{\pgfqpoint{0.000000in}{0.000000in}}{\pgfqpoint{0.069444in}{0.000000in}}{%
\pgfpathmoveto{\pgfqpoint{0.000000in}{0.000000in}}%
\pgfpathlineto{\pgfqpoint{0.069444in}{0.000000in}}%
\pgfusepath{stroke,fill}%
}%
\begin{pgfscope}%
\pgfsys@transformshift{3.675469in}{1.288036in}%
\pgfsys@useobject{currentmarker}{}%
\end{pgfscope}%
\end{pgfscope}%
\begin{pgfscope}%
\pgftext[x=3.793525in,y=1.240953in,left,base]{\rmfamily\fontsize{10.000000}{12.000000}\selectfont \(\displaystyle 0.25\)}%
\end{pgfscope}%
\begin{pgfscope}%
\pgfsetbuttcap%
\pgfsetroundjoin%
\definecolor{currentfill}{rgb}{0.000000,0.000000,0.000000}%
\pgfsetfillcolor{currentfill}%
\pgfsetlinewidth{1.003750pt}%
\definecolor{currentstroke}{rgb}{0.000000,0.000000,0.000000}%
\pgfsetstrokecolor{currentstroke}%
\pgfsetdash{}{0pt}%
\pgfsys@defobject{currentmarker}{\pgfqpoint{0.000000in}{0.000000in}}{\pgfqpoint{0.069444in}{0.000000in}}{%
\pgfpathmoveto{\pgfqpoint{0.000000in}{0.000000in}}%
\pgfpathlineto{\pgfqpoint{0.069444in}{0.000000in}}%
\pgfusepath{stroke,fill}%
}%
\begin{pgfscope}%
\pgfsys@transformshift{3.675469in}{1.651647in}%
\pgfsys@useobject{currentmarker}{}%
\end{pgfscope}%
\end{pgfscope}%
\begin{pgfscope}%
\pgftext[x=3.793525in,y=1.604565in,left,base]{\rmfamily\fontsize{10.000000}{12.000000}\selectfont \(\displaystyle 0.50\)}%
\end{pgfscope}%
\begin{pgfscope}%
\pgfsetbuttcap%
\pgfsetroundjoin%
\definecolor{currentfill}{rgb}{0.000000,0.000000,0.000000}%
\pgfsetfillcolor{currentfill}%
\pgfsetlinewidth{1.003750pt}%
\definecolor{currentstroke}{rgb}{0.000000,0.000000,0.000000}%
\pgfsetstrokecolor{currentstroke}%
\pgfsetdash{}{0pt}%
\pgfsys@defobject{currentmarker}{\pgfqpoint{0.000000in}{0.000000in}}{\pgfqpoint{0.069444in}{0.000000in}}{%
\pgfpathmoveto{\pgfqpoint{0.000000in}{0.000000in}}%
\pgfpathlineto{\pgfqpoint{0.069444in}{0.000000in}}%
\pgfusepath{stroke,fill}%
}%
\begin{pgfscope}%
\pgfsys@transformshift{3.675469in}{2.015258in}%
\pgfsys@useobject{currentmarker}{}%
\end{pgfscope}%
\end{pgfscope}%
\begin{pgfscope}%
\pgftext[x=3.793525in,y=1.968176in,left,base]{\rmfamily\fontsize{10.000000}{12.000000}\selectfont \(\displaystyle 0.75\)}%
\end{pgfscope}%
\begin{pgfscope}%
\pgfsetbuttcap%
\pgfsetroundjoin%
\definecolor{currentfill}{rgb}{0.000000,0.000000,0.000000}%
\pgfsetfillcolor{currentfill}%
\pgfsetlinewidth{1.003750pt}%
\definecolor{currentstroke}{rgb}{0.000000,0.000000,0.000000}%
\pgfsetstrokecolor{currentstroke}%
\pgfsetdash{}{0pt}%
\pgfsys@defobject{currentmarker}{\pgfqpoint{0.000000in}{0.000000in}}{\pgfqpoint{0.069444in}{0.000000in}}{%
\pgfpathmoveto{\pgfqpoint{0.000000in}{0.000000in}}%
\pgfpathlineto{\pgfqpoint{0.069444in}{0.000000in}}%
\pgfusepath{stroke,fill}%
}%
\begin{pgfscope}%
\pgfsys@transformshift{3.675469in}{2.378869in}%
\pgfsys@useobject{currentmarker}{}%
\end{pgfscope}%
\end{pgfscope}%
\begin{pgfscope}%
\pgftext[x=3.793525in,y=2.331787in,left,base]{\rmfamily\fontsize{10.000000}{12.000000}\selectfont \(\displaystyle 1.00\)}%
\end{pgfscope}%
\begin{pgfscope}%
\pgfsetbuttcap%
\pgfsetroundjoin%
\definecolor{currentfill}{rgb}{0.000000,0.000000,0.000000}%
\pgfsetfillcolor{currentfill}%
\pgfsetlinewidth{1.003750pt}%
\definecolor{currentstroke}{rgb}{0.000000,0.000000,0.000000}%
\pgfsetstrokecolor{currentstroke}%
\pgfsetdash{}{0pt}%
\pgfsys@defobject{currentmarker}{\pgfqpoint{0.000000in}{0.000000in}}{\pgfqpoint{0.069444in}{0.000000in}}{%
\pgfpathmoveto{\pgfqpoint{0.000000in}{0.000000in}}%
\pgfpathlineto{\pgfqpoint{0.069444in}{0.000000in}}%
\pgfusepath{stroke,fill}%
}%
\begin{pgfscope}%
\pgfsys@transformshift{3.675469in}{2.742481in}%
\pgfsys@useobject{currentmarker}{}%
\end{pgfscope}%
\end{pgfscope}%
\begin{pgfscope}%
\pgftext[x=3.793525in,y=2.695398in,left,base]{\rmfamily\fontsize{10.000000}{12.000000}\selectfont \(\displaystyle 1.25\)}%
\end{pgfscope}%
\begin{pgfscope}%
\pgfsetbuttcap%
\pgfsetroundjoin%
\definecolor{currentfill}{rgb}{0.000000,0.000000,0.000000}%
\pgfsetfillcolor{currentfill}%
\pgfsetlinewidth{1.003750pt}%
\definecolor{currentstroke}{rgb}{0.000000,0.000000,0.000000}%
\pgfsetstrokecolor{currentstroke}%
\pgfsetdash{}{0pt}%
\pgfsys@defobject{currentmarker}{\pgfqpoint{0.000000in}{0.000000in}}{\pgfqpoint{0.069444in}{0.000000in}}{%
\pgfpathmoveto{\pgfqpoint{0.000000in}{0.000000in}}%
\pgfpathlineto{\pgfqpoint{0.069444in}{0.000000in}}%
\pgfusepath{stroke,fill}%
}%
\begin{pgfscope}%
\pgfsys@transformshift{3.675469in}{3.106092in}%
\pgfsys@useobject{currentmarker}{}%
\end{pgfscope}%
\end{pgfscope}%
\begin{pgfscope}%
\pgftext[x=3.793525in,y=3.059009in,left,base]{\rmfamily\fontsize{10.000000}{12.000000}\selectfont \(\displaystyle 1.50\)}%
\end{pgfscope}%
\begin{pgfscope}%
\pgfsetbuttcap%
\pgfsetroundjoin%
\definecolor{currentfill}{rgb}{0.000000,0.000000,0.000000}%
\pgfsetfillcolor{currentfill}%
\pgfsetlinewidth{1.003750pt}%
\definecolor{currentstroke}{rgb}{0.000000,0.000000,0.000000}%
\pgfsetstrokecolor{currentstroke}%
\pgfsetdash{}{0pt}%
\pgfsys@defobject{currentmarker}{\pgfqpoint{0.000000in}{0.000000in}}{\pgfqpoint{0.069444in}{0.000000in}}{%
\pgfpathmoveto{\pgfqpoint{0.000000in}{0.000000in}}%
\pgfpathlineto{\pgfqpoint{0.069444in}{0.000000in}}%
\pgfusepath{stroke,fill}%
}%
\begin{pgfscope}%
\pgfsys@transformshift{3.675469in}{3.469703in}%
\pgfsys@useobject{currentmarker}{}%
\end{pgfscope}%
\end{pgfscope}%
\begin{pgfscope}%
\pgftext[x=3.793525in,y=3.422620in,left,base]{\rmfamily\fontsize{10.000000}{12.000000}\selectfont \(\displaystyle 1.75\)}%
\end{pgfscope}%
\begin{pgfscope}%
\pgfsetbuttcap%
\pgfsetroundjoin%
\definecolor{currentfill}{rgb}{0.000000,0.000000,0.000000}%
\pgfsetfillcolor{currentfill}%
\pgfsetlinewidth{1.003750pt}%
\definecolor{currentstroke}{rgb}{0.000000,0.000000,0.000000}%
\pgfsetstrokecolor{currentstroke}%
\pgfsetdash{}{0pt}%
\pgfsys@defobject{currentmarker}{\pgfqpoint{0.000000in}{0.000000in}}{\pgfqpoint{0.069444in}{0.000000in}}{%
\pgfpathmoveto{\pgfqpoint{0.000000in}{0.000000in}}%
\pgfpathlineto{\pgfqpoint{0.069444in}{0.000000in}}%
\pgfusepath{stroke,fill}%
}%
\begin{pgfscope}%
\pgfsys@transformshift{3.675469in}{3.833314in}%
\pgfsys@useobject{currentmarker}{}%
\end{pgfscope}%
\end{pgfscope}%
\begin{pgfscope}%
\pgftext[x=3.793525in,y=3.786231in,left,base]{\rmfamily\fontsize{10.000000}{12.000000}\selectfont \(\displaystyle 2.00\)}%
\end{pgfscope}%
\begin{pgfscope}%
\pgfsetbuttcap%
\pgfsetroundjoin%
\definecolor{currentfill}{rgb}{0.000000,0.000000,0.000000}%
\pgfsetfillcolor{currentfill}%
\pgfsetlinewidth{1.003750pt}%
\definecolor{currentstroke}{rgb}{0.000000,0.000000,0.000000}%
\pgfsetstrokecolor{currentstroke}%
\pgfsetdash{}{0pt}%
\pgfsys@defobject{currentmarker}{\pgfqpoint{0.000000in}{0.000000in}}{\pgfqpoint{0.034722in}{0.000000in}}{%
\pgfpathmoveto{\pgfqpoint{0.000000in}{0.000000in}}%
\pgfpathlineto{\pgfqpoint{0.034722in}{0.000000in}}%
\pgfusepath{stroke,fill}%
}%
\begin{pgfscope}%
\pgfsys@transformshift{3.675469in}{0.742619in}%
\pgfsys@useobject{currentmarker}{}%
\end{pgfscope}%
\end{pgfscope}%
\begin{pgfscope}%
\pgfsetbuttcap%
\pgfsetroundjoin%
\definecolor{currentfill}{rgb}{0.000000,0.000000,0.000000}%
\pgfsetfillcolor{currentfill}%
\pgfsetlinewidth{1.003750pt}%
\definecolor{currentstroke}{rgb}{0.000000,0.000000,0.000000}%
\pgfsetstrokecolor{currentstroke}%
\pgfsetdash{}{0pt}%
\pgfsys@defobject{currentmarker}{\pgfqpoint{0.000000in}{0.000000in}}{\pgfqpoint{0.034722in}{0.000000in}}{%
\pgfpathmoveto{\pgfqpoint{0.000000in}{0.000000in}}%
\pgfpathlineto{\pgfqpoint{0.034722in}{0.000000in}}%
\pgfusepath{stroke,fill}%
}%
\begin{pgfscope}%
\pgfsys@transformshift{3.675469in}{1.106231in}%
\pgfsys@useobject{currentmarker}{}%
\end{pgfscope}%
\end{pgfscope}%
\begin{pgfscope}%
\pgfsetbuttcap%
\pgfsetroundjoin%
\definecolor{currentfill}{rgb}{0.000000,0.000000,0.000000}%
\pgfsetfillcolor{currentfill}%
\pgfsetlinewidth{1.003750pt}%
\definecolor{currentstroke}{rgb}{0.000000,0.000000,0.000000}%
\pgfsetstrokecolor{currentstroke}%
\pgfsetdash{}{0pt}%
\pgfsys@defobject{currentmarker}{\pgfqpoint{0.000000in}{0.000000in}}{\pgfqpoint{0.034722in}{0.000000in}}{%
\pgfpathmoveto{\pgfqpoint{0.000000in}{0.000000in}}%
\pgfpathlineto{\pgfqpoint{0.034722in}{0.000000in}}%
\pgfusepath{stroke,fill}%
}%
\begin{pgfscope}%
\pgfsys@transformshift{3.675469in}{1.469842in}%
\pgfsys@useobject{currentmarker}{}%
\end{pgfscope}%
\end{pgfscope}%
\begin{pgfscope}%
\pgfsetbuttcap%
\pgfsetroundjoin%
\definecolor{currentfill}{rgb}{0.000000,0.000000,0.000000}%
\pgfsetfillcolor{currentfill}%
\pgfsetlinewidth{1.003750pt}%
\definecolor{currentstroke}{rgb}{0.000000,0.000000,0.000000}%
\pgfsetstrokecolor{currentstroke}%
\pgfsetdash{}{0pt}%
\pgfsys@defobject{currentmarker}{\pgfqpoint{0.000000in}{0.000000in}}{\pgfqpoint{0.034722in}{0.000000in}}{%
\pgfpathmoveto{\pgfqpoint{0.000000in}{0.000000in}}%
\pgfpathlineto{\pgfqpoint{0.034722in}{0.000000in}}%
\pgfusepath{stroke,fill}%
}%
\begin{pgfscope}%
\pgfsys@transformshift{3.675469in}{1.833453in}%
\pgfsys@useobject{currentmarker}{}%
\end{pgfscope}%
\end{pgfscope}%
\begin{pgfscope}%
\pgfsetbuttcap%
\pgfsetroundjoin%
\definecolor{currentfill}{rgb}{0.000000,0.000000,0.000000}%
\pgfsetfillcolor{currentfill}%
\pgfsetlinewidth{1.003750pt}%
\definecolor{currentstroke}{rgb}{0.000000,0.000000,0.000000}%
\pgfsetstrokecolor{currentstroke}%
\pgfsetdash{}{0pt}%
\pgfsys@defobject{currentmarker}{\pgfqpoint{0.000000in}{0.000000in}}{\pgfqpoint{0.034722in}{0.000000in}}{%
\pgfpathmoveto{\pgfqpoint{0.000000in}{0.000000in}}%
\pgfpathlineto{\pgfqpoint{0.034722in}{0.000000in}}%
\pgfusepath{stroke,fill}%
}%
\begin{pgfscope}%
\pgfsys@transformshift{3.675469in}{2.197064in}%
\pgfsys@useobject{currentmarker}{}%
\end{pgfscope}%
\end{pgfscope}%
\begin{pgfscope}%
\pgfsetbuttcap%
\pgfsetroundjoin%
\definecolor{currentfill}{rgb}{0.000000,0.000000,0.000000}%
\pgfsetfillcolor{currentfill}%
\pgfsetlinewidth{1.003750pt}%
\definecolor{currentstroke}{rgb}{0.000000,0.000000,0.000000}%
\pgfsetstrokecolor{currentstroke}%
\pgfsetdash{}{0pt}%
\pgfsys@defobject{currentmarker}{\pgfqpoint{0.000000in}{0.000000in}}{\pgfqpoint{0.034722in}{0.000000in}}{%
\pgfpathmoveto{\pgfqpoint{0.000000in}{0.000000in}}%
\pgfpathlineto{\pgfqpoint{0.034722in}{0.000000in}}%
\pgfusepath{stroke,fill}%
}%
\begin{pgfscope}%
\pgfsys@transformshift{3.675469in}{2.560675in}%
\pgfsys@useobject{currentmarker}{}%
\end{pgfscope}%
\end{pgfscope}%
\begin{pgfscope}%
\pgfsetbuttcap%
\pgfsetroundjoin%
\definecolor{currentfill}{rgb}{0.000000,0.000000,0.000000}%
\pgfsetfillcolor{currentfill}%
\pgfsetlinewidth{1.003750pt}%
\definecolor{currentstroke}{rgb}{0.000000,0.000000,0.000000}%
\pgfsetstrokecolor{currentstroke}%
\pgfsetdash{}{0pt}%
\pgfsys@defobject{currentmarker}{\pgfqpoint{0.000000in}{0.000000in}}{\pgfqpoint{0.034722in}{0.000000in}}{%
\pgfpathmoveto{\pgfqpoint{0.000000in}{0.000000in}}%
\pgfpathlineto{\pgfqpoint{0.034722in}{0.000000in}}%
\pgfusepath{stroke,fill}%
}%
\begin{pgfscope}%
\pgfsys@transformshift{3.675469in}{2.924286in}%
\pgfsys@useobject{currentmarker}{}%
\end{pgfscope}%
\end{pgfscope}%
\begin{pgfscope}%
\pgfsetbuttcap%
\pgfsetroundjoin%
\definecolor{currentfill}{rgb}{0.000000,0.000000,0.000000}%
\pgfsetfillcolor{currentfill}%
\pgfsetlinewidth{1.003750pt}%
\definecolor{currentstroke}{rgb}{0.000000,0.000000,0.000000}%
\pgfsetstrokecolor{currentstroke}%
\pgfsetdash{}{0pt}%
\pgfsys@defobject{currentmarker}{\pgfqpoint{0.000000in}{0.000000in}}{\pgfqpoint{0.034722in}{0.000000in}}{%
\pgfpathmoveto{\pgfqpoint{0.000000in}{0.000000in}}%
\pgfpathlineto{\pgfqpoint{0.034722in}{0.000000in}}%
\pgfusepath{stroke,fill}%
}%
\begin{pgfscope}%
\pgfsys@transformshift{3.675469in}{3.287897in}%
\pgfsys@useobject{currentmarker}{}%
\end{pgfscope}%
\end{pgfscope}%
\begin{pgfscope}%
\pgfsetbuttcap%
\pgfsetroundjoin%
\definecolor{currentfill}{rgb}{0.000000,0.000000,0.000000}%
\pgfsetfillcolor{currentfill}%
\pgfsetlinewidth{1.003750pt}%
\definecolor{currentstroke}{rgb}{0.000000,0.000000,0.000000}%
\pgfsetstrokecolor{currentstroke}%
\pgfsetdash{}{0pt}%
\pgfsys@defobject{currentmarker}{\pgfqpoint{0.000000in}{0.000000in}}{\pgfqpoint{0.034722in}{0.000000in}}{%
\pgfpathmoveto{\pgfqpoint{0.000000in}{0.000000in}}%
\pgfpathlineto{\pgfqpoint{0.034722in}{0.000000in}}%
\pgfusepath{stroke,fill}%
}%
\begin{pgfscope}%
\pgfsys@transformshift{3.675469in}{3.651508in}%
\pgfsys@useobject{currentmarker}{}%
\end{pgfscope}%
\end{pgfscope}%
\begin{pgfscope}%
\pgftext[x=4.217689in,y=2.197064in,,top,rotate=90.000000]{\rmfamily\fontsize{12.000000}{14.400000}\selectfont Time Derivative of Pressure, bar/ms}%
\end{pgfscope}%
\begin{pgfscope}%
\pgfpathrectangle{\pgfqpoint{0.575469in}{0.560814in}}{\pgfqpoint{3.100000in}{3.272500in}} %
\pgfusepath{clip}%
\pgfsetrectcap%
\pgfsetroundjoin%
\pgfsetlinewidth{1.505625pt}%
\definecolor{currentstroke}{rgb}{0.172549,0.627451,0.172549}%
\pgfsetstrokecolor{currentstroke}%
\pgfsetdash{}{0pt}%
\pgfpathmoveto{\pgfqpoint{0.574849in}{1.053175in}}%
\pgfpathlineto{\pgfqpoint{0.577329in}{1.055578in}}%
\pgfpathlineto{\pgfqpoint{0.579189in}{1.056613in}}%
\pgfpathlineto{\pgfqpoint{0.581669in}{1.060121in}}%
\pgfpathlineto{\pgfqpoint{0.584149in}{1.063322in}}%
\pgfpathlineto{\pgfqpoint{0.586629in}{1.063828in}}%
\pgfpathlineto{\pgfqpoint{0.593449in}{1.072914in}}%
\pgfpathlineto{\pgfqpoint{0.596549in}{1.079889in}}%
\pgfpathlineto{\pgfqpoint{0.600889in}{1.082627in}}%
\pgfpathlineto{\pgfqpoint{0.603369in}{1.084302in}}%
\pgfpathlineto{\pgfqpoint{0.604609in}{1.084770in}}%
\pgfpathlineto{\pgfqpoint{0.607089in}{1.087810in}}%
\pgfpathlineto{\pgfqpoint{0.610809in}{1.085621in}}%
\pgfpathlineto{\pgfqpoint{0.618869in}{1.089929in}}%
\pgfpathlineto{\pgfqpoint{0.620109in}{1.089463in}}%
\pgfpathlineto{\pgfqpoint{0.623209in}{1.093431in}}%
\pgfpathlineto{\pgfqpoint{0.629409in}{1.113470in}}%
\pgfpathlineto{\pgfqpoint{0.635609in}{1.118333in}}%
\pgfpathlineto{\pgfqpoint{0.639949in}{1.122489in}}%
\pgfpathlineto{\pgfqpoint{0.642429in}{1.123674in}}%
\pgfpathlineto{\pgfqpoint{0.645529in}{1.123871in}}%
\pgfpathlineto{\pgfqpoint{0.649249in}{1.121317in}}%
\pgfpathlineto{\pgfqpoint{0.652349in}{1.116194in}}%
\pgfpathlineto{\pgfqpoint{0.654829in}{1.112028in}}%
\pgfpathlineto{\pgfqpoint{0.659169in}{1.103355in}}%
\pgfpathlineto{\pgfqpoint{0.661649in}{1.103193in}}%
\pgfpathlineto{\pgfqpoint{0.662889in}{1.102895in}}%
\pgfpathlineto{\pgfqpoint{0.667849in}{1.095990in}}%
\pgfpathlineto{\pgfqpoint{0.670949in}{1.094752in}}%
\pgfpathlineto{\pgfqpoint{0.675909in}{1.101273in}}%
\pgfpathlineto{\pgfqpoint{0.679009in}{1.101475in}}%
\pgfpathlineto{\pgfqpoint{0.684589in}{1.102846in}}%
\pgfpathlineto{\pgfqpoint{0.692029in}{1.095252in}}%
\pgfpathlineto{\pgfqpoint{0.696989in}{1.097635in}}%
\pgfpathlineto{\pgfqpoint{0.699469in}{1.096699in}}%
\pgfpathlineto{\pgfqpoint{0.701949in}{1.100736in}}%
\pgfpathlineto{\pgfqpoint{0.703809in}{1.102505in}}%
\pgfpathlineto{\pgfqpoint{0.706289in}{1.103964in}}%
\pgfpathlineto{\pgfqpoint{0.711869in}{1.113964in}}%
\pgfpathlineto{\pgfqpoint{0.714349in}{1.121245in}}%
\pgfpathlineto{\pgfqpoint{0.717449in}{1.125146in}}%
\pgfpathlineto{\pgfqpoint{0.722409in}{1.127367in}}%
\pgfpathlineto{\pgfqpoint{0.724889in}{1.130220in}}%
\pgfpathlineto{\pgfqpoint{0.729229in}{1.129535in}}%
\pgfpathlineto{\pgfqpoint{0.734809in}{1.132969in}}%
\pgfpathlineto{\pgfqpoint{0.739769in}{1.141341in}}%
\pgfpathlineto{\pgfqpoint{0.741629in}{1.143732in}}%
\pgfpathlineto{\pgfqpoint{0.750309in}{1.172713in}}%
\pgfpathlineto{\pgfqpoint{0.754029in}{1.185340in}}%
\pgfpathlineto{\pgfqpoint{0.762089in}{1.197016in}}%
\pgfpathlineto{\pgfqpoint{0.769529in}{1.199123in}}%
\pgfpathlineto{\pgfqpoint{0.771389in}{1.199887in}}%
\pgfpathlineto{\pgfqpoint{0.773249in}{1.199737in}}%
\pgfpathlineto{\pgfqpoint{0.775729in}{1.201972in}}%
\pgfpathlineto{\pgfqpoint{0.778829in}{1.202422in}}%
\pgfpathlineto{\pgfqpoint{0.781309in}{1.202393in}}%
\pgfpathlineto{\pgfqpoint{0.783789in}{1.202834in}}%
\pgfpathlineto{\pgfqpoint{0.786889in}{1.204465in}}%
\pgfpathlineto{\pgfqpoint{0.791849in}{1.209335in}}%
\pgfpathlineto{\pgfqpoint{0.796809in}{1.215670in}}%
\pgfpathlineto{\pgfqpoint{0.804249in}{1.228867in}}%
\pgfpathlineto{\pgfqpoint{0.806729in}{1.230028in}}%
\pgfpathlineto{\pgfqpoint{0.808589in}{1.231015in}}%
\pgfpathlineto{\pgfqpoint{0.810449in}{1.230164in}}%
\pgfpathlineto{\pgfqpoint{0.815409in}{1.232063in}}%
\pgfpathlineto{\pgfqpoint{0.817889in}{1.230793in}}%
\pgfpathlineto{\pgfqpoint{0.820989in}{1.235899in}}%
\pgfpathlineto{\pgfqpoint{0.823469in}{1.239253in}}%
\pgfpathlineto{\pgfqpoint{0.828429in}{1.243161in}}%
\pgfpathlineto{\pgfqpoint{0.836489in}{1.259986in}}%
\pgfpathlineto{\pgfqpoint{0.843309in}{1.269338in}}%
\pgfpathlineto{\pgfqpoint{0.845789in}{1.268619in}}%
\pgfpathlineto{\pgfqpoint{0.848889in}{1.265823in}}%
\pgfpathlineto{\pgfqpoint{0.853229in}{1.268003in}}%
\pgfpathlineto{\pgfqpoint{0.855709in}{1.270433in}}%
\pgfpathlineto{\pgfqpoint{0.857569in}{1.271435in}}%
\pgfpathlineto{\pgfqpoint{0.860669in}{1.269973in}}%
\pgfpathlineto{\pgfqpoint{0.863149in}{1.270471in}}%
\pgfpathlineto{\pgfqpoint{0.865009in}{1.274289in}}%
\pgfpathlineto{\pgfqpoint{0.867489in}{1.278171in}}%
\pgfpathlineto{\pgfqpoint{0.869349in}{1.279344in}}%
\pgfpathlineto{\pgfqpoint{0.873069in}{1.287962in}}%
\pgfpathlineto{\pgfqpoint{0.881129in}{1.312189in}}%
\pgfpathlineto{\pgfqpoint{0.886709in}{1.324183in}}%
\pgfpathlineto{\pgfqpoint{0.890429in}{1.325293in}}%
\pgfpathlineto{\pgfqpoint{0.893529in}{1.328208in}}%
\pgfpathlineto{\pgfqpoint{0.896009in}{1.330657in}}%
\pgfpathlineto{\pgfqpoint{0.897869in}{1.330568in}}%
\pgfpathlineto{\pgfqpoint{0.899729in}{1.331565in}}%
\pgfpathlineto{\pgfqpoint{0.902209in}{1.331795in}}%
\pgfpathlineto{\pgfqpoint{0.905309in}{1.334239in}}%
\pgfpathlineto{\pgfqpoint{0.907789in}{1.337153in}}%
\pgfpathlineto{\pgfqpoint{0.918329in}{1.364111in}}%
\pgfpathlineto{\pgfqpoint{0.923289in}{1.377200in}}%
\pgfpathlineto{\pgfqpoint{0.926389in}{1.380823in}}%
\pgfpathlineto{\pgfqpoint{0.927009in}{1.380041in}}%
\pgfpathlineto{\pgfqpoint{0.930109in}{1.380139in}}%
\pgfpathlineto{\pgfqpoint{0.931969in}{1.381307in}}%
\pgfpathlineto{\pgfqpoint{0.933829in}{1.381417in}}%
\pgfpathlineto{\pgfqpoint{0.938169in}{1.390389in}}%
\pgfpathlineto{\pgfqpoint{0.954289in}{1.443432in}}%
\pgfpathlineto{\pgfqpoint{0.957389in}{1.452746in}}%
\pgfpathlineto{\pgfqpoint{0.959869in}{1.456989in}}%
\pgfpathlineto{\pgfqpoint{0.972889in}{1.492007in}}%
\pgfpathlineto{\pgfqpoint{0.975369in}{1.492915in}}%
\pgfpathlineto{\pgfqpoint{0.980949in}{1.505169in}}%
\pgfpathlineto{\pgfqpoint{0.985289in}{1.516308in}}%
\pgfpathlineto{\pgfqpoint{0.991489in}{1.531575in}}%
\pgfpathlineto{\pgfqpoint{1.005129in}{1.575852in}}%
\pgfpathlineto{\pgfqpoint{1.007609in}{1.585158in}}%
\pgfpathlineto{\pgfqpoint{1.011949in}{1.600808in}}%
\pgfpathlineto{\pgfqpoint{1.018769in}{1.618531in}}%
\pgfpathlineto{\pgfqpoint{1.039229in}{1.696544in}}%
\pgfpathlineto{\pgfqpoint{1.053489in}{1.737047in}}%
\pgfpathlineto{\pgfqpoint{1.057209in}{1.752071in}}%
\pgfpathlineto{\pgfqpoint{1.068369in}{1.809445in}}%
\pgfpathlineto{\pgfqpoint{1.072709in}{1.827661in}}%
\pgfpathlineto{\pgfqpoint{1.077669in}{1.848007in}}%
\pgfpathlineto{\pgfqpoint{1.082009in}{1.870883in}}%
\pgfpathlineto{\pgfqpoint{1.086349in}{1.901524in}}%
\pgfpathlineto{\pgfqpoint{1.099369in}{2.002683in}}%
\pgfpathlineto{\pgfqpoint{1.104329in}{2.039523in}}%
\pgfpathlineto{\pgfqpoint{1.110529in}{2.075905in}}%
\pgfpathlineto{\pgfqpoint{1.119209in}{2.124655in}}%
\pgfpathlineto{\pgfqpoint{1.126649in}{2.167876in}}%
\pgfpathlineto{\pgfqpoint{1.134709in}{2.236067in}}%
\pgfpathlineto{\pgfqpoint{1.142149in}{2.315035in}}%
\pgfpathlineto{\pgfqpoint{1.166949in}{2.594589in}}%
\pgfpathlineto{\pgfqpoint{1.178729in}{2.751079in}}%
\pgfpathlineto{\pgfqpoint{1.202289in}{3.070956in}}%
\pgfpathlineto{\pgfqpoint{1.211589in}{3.240909in}}%
\pgfpathlineto{\pgfqpoint{1.242590in}{3.843314in}}%
\pgfpathmoveto{\pgfqpoint{1.406568in}{3.843314in}}%
\pgfpathlineto{\pgfqpoint{1.413709in}{3.533920in}}%
\pgfpathlineto{\pgfqpoint{1.425489in}{3.031251in}}%
\pgfpathlineto{\pgfqpoint{1.441609in}{2.411806in}}%
\pgfpathlineto{\pgfqpoint{1.470129in}{1.527296in}}%
\pgfpathlineto{\pgfqpoint{1.475089in}{1.418937in}}%
\pgfpathlineto{\pgfqpoint{1.478809in}{1.347084in}}%
\pgfpathlineto{\pgfqpoint{1.488729in}{1.195285in}}%
\pgfpathlineto{\pgfqpoint{1.493069in}{1.171587in}}%
\pgfpathlineto{\pgfqpoint{1.494309in}{1.168570in}}%
\pgfpathlineto{\pgfqpoint{1.494929in}{1.169179in}}%
\pgfpathlineto{\pgfqpoint{1.495549in}{1.170634in}}%
\pgfpathlineto{\pgfqpoint{1.496789in}{1.160751in}}%
\pgfpathlineto{\pgfqpoint{1.500509in}{1.139731in}}%
\pgfpathlineto{\pgfqpoint{1.501749in}{1.133760in}}%
\pgfpathlineto{\pgfqpoint{1.508569in}{1.080970in}}%
\pgfpathlineto{\pgfqpoint{1.509189in}{1.078321in}}%
\pgfpathlineto{\pgfqpoint{1.509809in}{1.078588in}}%
\pgfpathlineto{\pgfqpoint{1.511049in}{1.079789in}}%
\pgfpathlineto{\pgfqpoint{1.512289in}{1.075024in}}%
\pgfpathlineto{\pgfqpoint{1.512909in}{1.072059in}}%
\pgfpathlineto{\pgfqpoint{1.513529in}{1.072256in}}%
\pgfpathlineto{\pgfqpoint{1.515389in}{1.076267in}}%
\pgfpathlineto{\pgfqpoint{1.516009in}{1.077070in}}%
\pgfpathlineto{\pgfqpoint{1.517249in}{1.072076in}}%
\pgfpathlineto{\pgfqpoint{1.521589in}{1.052199in}}%
\pgfpathlineto{\pgfqpoint{1.522829in}{1.050444in}}%
\pgfpathlineto{\pgfqpoint{1.525929in}{1.043113in}}%
\pgfpathlineto{\pgfqpoint{1.532129in}{0.978172in}}%
\pgfpathlineto{\pgfqpoint{1.539569in}{0.881290in}}%
\pgfpathlineto{\pgfqpoint{1.553209in}{0.749342in}}%
\pgfpathlineto{\pgfqpoint{1.556309in}{0.742561in}}%
\pgfpathlineto{\pgfqpoint{1.558169in}{0.744634in}}%
\pgfpathlineto{\pgfqpoint{1.559409in}{0.750456in}}%
\pgfpathlineto{\pgfqpoint{1.561889in}{0.765895in}}%
\pgfpathlineto{\pgfqpoint{1.564369in}{0.764229in}}%
\pgfpathlineto{\pgfqpoint{1.570569in}{0.777514in}}%
\pgfpathlineto{\pgfqpoint{1.574909in}{0.792078in}}%
\pgfpathlineto{\pgfqpoint{1.581109in}{0.808036in}}%
\pgfpathlineto{\pgfqpoint{1.581729in}{0.807423in}}%
\pgfpathlineto{\pgfqpoint{1.587929in}{0.788787in}}%
\pgfpathlineto{\pgfqpoint{1.597229in}{0.744454in}}%
\pgfpathlineto{\pgfqpoint{1.599709in}{0.738062in}}%
\pgfpathlineto{\pgfqpoint{1.600949in}{0.740353in}}%
\pgfpathlineto{\pgfqpoint{1.602809in}{0.735119in}}%
\pgfpathlineto{\pgfqpoint{1.605289in}{0.722662in}}%
\pgfpathlineto{\pgfqpoint{1.607769in}{0.716170in}}%
\pgfpathlineto{\pgfqpoint{1.611489in}{0.707258in}}%
\pgfpathlineto{\pgfqpoint{1.613969in}{0.705115in}}%
\pgfpathlineto{\pgfqpoint{1.617689in}{0.688011in}}%
\pgfpathlineto{\pgfqpoint{1.623889in}{0.679604in}}%
\pgfpathlineto{\pgfqpoint{1.625129in}{0.681636in}}%
\pgfpathlineto{\pgfqpoint{1.629469in}{0.695177in}}%
\pgfpathlineto{\pgfqpoint{1.631949in}{0.700701in}}%
\pgfpathlineto{\pgfqpoint{1.636289in}{0.718263in}}%
\pgfpathlineto{\pgfqpoint{1.643729in}{0.758764in}}%
\pgfpathlineto{\pgfqpoint{1.648069in}{0.785262in}}%
\pgfpathlineto{\pgfqpoint{1.651789in}{0.798064in}}%
\pgfpathlineto{\pgfqpoint{1.653029in}{0.797380in}}%
\pgfpathlineto{\pgfqpoint{1.654889in}{0.795170in}}%
\pgfpathlineto{\pgfqpoint{1.656749in}{0.798218in}}%
\pgfpathlineto{\pgfqpoint{1.657989in}{0.800583in}}%
\pgfpathlineto{\pgfqpoint{1.658609in}{0.800309in}}%
\pgfpathlineto{\pgfqpoint{1.659849in}{0.797430in}}%
\pgfpathlineto{\pgfqpoint{1.663569in}{0.778817in}}%
\pgfpathlineto{\pgfqpoint{1.667909in}{0.751331in}}%
\pgfpathlineto{\pgfqpoint{1.674109in}{0.708857in}}%
\pgfpathlineto{\pgfqpoint{1.678449in}{0.692424in}}%
\pgfpathlineto{\pgfqpoint{1.680929in}{0.689742in}}%
\pgfpathlineto{\pgfqpoint{1.682169in}{0.690375in}}%
\pgfpathlineto{\pgfqpoint{1.688369in}{0.705284in}}%
\pgfpathlineto{\pgfqpoint{1.696429in}{0.739933in}}%
\pgfpathlineto{\pgfqpoint{1.703249in}{0.776441in}}%
\pgfpathlineto{\pgfqpoint{1.706349in}{0.782095in}}%
\pgfpathlineto{\pgfqpoint{1.711309in}{0.794919in}}%
\pgfpathlineto{\pgfqpoint{1.711929in}{0.794642in}}%
\pgfpathlineto{\pgfqpoint{1.713789in}{0.795178in}}%
\pgfpathlineto{\pgfqpoint{1.717509in}{0.796720in}}%
\pgfpathlineto{\pgfqpoint{1.720609in}{0.795381in}}%
\pgfpathlineto{\pgfqpoint{1.723089in}{0.795919in}}%
\pgfpathlineto{\pgfqpoint{1.726189in}{0.793504in}}%
\pgfpathlineto{\pgfqpoint{1.729289in}{0.787639in}}%
\pgfpathlineto{\pgfqpoint{1.742929in}{0.759050in}}%
\pgfpathlineto{\pgfqpoint{1.749749in}{0.754563in}}%
\pgfpathlineto{\pgfqpoint{1.753469in}{0.751382in}}%
\pgfpathlineto{\pgfqpoint{1.756569in}{0.754498in}}%
\pgfpathlineto{\pgfqpoint{1.778889in}{0.818861in}}%
\pgfpathlineto{\pgfqpoint{1.781369in}{0.821316in}}%
\pgfpathlineto{\pgfqpoint{1.787569in}{0.817151in}}%
\pgfpathlineto{\pgfqpoint{1.790669in}{0.811487in}}%
\pgfpathlineto{\pgfqpoint{1.799349in}{0.782413in}}%
\pgfpathlineto{\pgfqpoint{1.806789in}{0.773166in}}%
\pgfpathlineto{\pgfqpoint{1.810509in}{0.765181in}}%
\pgfpathlineto{\pgfqpoint{1.812369in}{0.763236in}}%
\pgfpathlineto{\pgfqpoint{1.815469in}{0.764679in}}%
\pgfpathlineto{\pgfqpoint{1.817329in}{0.769370in}}%
\pgfpathlineto{\pgfqpoint{1.827249in}{0.796658in}}%
\pgfpathlineto{\pgfqpoint{1.832829in}{0.807629in}}%
\pgfpathlineto{\pgfqpoint{1.836549in}{0.814866in}}%
\pgfpathlineto{\pgfqpoint{1.843369in}{0.821883in}}%
\pgfpathlineto{\pgfqpoint{1.847089in}{0.826566in}}%
\pgfpathlineto{\pgfqpoint{1.850809in}{0.826091in}}%
\pgfpathlineto{\pgfqpoint{1.860109in}{0.805753in}}%
\pgfpathlineto{\pgfqpoint{1.865689in}{0.795102in}}%
\pgfpathlineto{\pgfqpoint{1.868169in}{0.790535in}}%
\pgfpathlineto{\pgfqpoint{1.873749in}{0.784729in}}%
\pgfpathlineto{\pgfqpoint{1.874989in}{0.784350in}}%
\pgfpathlineto{\pgfqpoint{1.880569in}{0.790802in}}%
\pgfpathlineto{\pgfqpoint{1.883669in}{0.798367in}}%
\pgfpathlineto{\pgfqpoint{1.888009in}{0.812115in}}%
\pgfpathlineto{\pgfqpoint{1.892969in}{0.831799in}}%
\pgfpathlineto{\pgfqpoint{1.896689in}{0.839470in}}%
\pgfpathlineto{\pgfqpoint{1.907229in}{0.863371in}}%
\pgfpathlineto{\pgfqpoint{1.909089in}{0.862379in}}%
\pgfpathlineto{\pgfqpoint{1.919009in}{0.842738in}}%
\pgfpathlineto{\pgfqpoint{1.923349in}{0.844307in}}%
\pgfpathlineto{\pgfqpoint{1.927689in}{0.838830in}}%
\pgfpathlineto{\pgfqpoint{1.933269in}{0.839692in}}%
\pgfpathlineto{\pgfqpoint{1.936369in}{0.832746in}}%
\pgfpathlineto{\pgfqpoint{1.938849in}{0.830028in}}%
\pgfpathlineto{\pgfqpoint{1.940709in}{0.828595in}}%
\pgfpathlineto{\pgfqpoint{1.946909in}{0.823958in}}%
\pgfpathlineto{\pgfqpoint{1.950009in}{0.827693in}}%
\pgfpathlineto{\pgfqpoint{1.958689in}{0.852911in}}%
\pgfpathlineto{\pgfqpoint{1.963029in}{0.855317in}}%
\pgfpathlineto{\pgfqpoint{1.964889in}{0.854946in}}%
\pgfpathlineto{\pgfqpoint{1.967989in}{0.856064in}}%
\pgfpathlineto{\pgfqpoint{1.972329in}{0.854375in}}%
\pgfpathlineto{\pgfqpoint{1.974189in}{0.853998in}}%
\pgfpathlineto{\pgfqpoint{1.976669in}{0.850835in}}%
\pgfpathlineto{\pgfqpoint{1.980389in}{0.842373in}}%
\pgfpathlineto{\pgfqpoint{1.985349in}{0.822563in}}%
\pgfpathlineto{\pgfqpoint{1.990929in}{0.804201in}}%
\pgfpathlineto{\pgfqpoint{1.994649in}{0.795147in}}%
\pgfpathlineto{\pgfqpoint{2.002089in}{0.783823in}}%
\pgfpathlineto{\pgfqpoint{2.003949in}{0.783774in}}%
\pgfpathlineto{\pgfqpoint{2.007669in}{0.785563in}}%
\pgfpathlineto{\pgfqpoint{2.011389in}{0.789415in}}%
\pgfpathlineto{\pgfqpoint{2.013249in}{0.791750in}}%
\pgfpathlineto{\pgfqpoint{2.018209in}{0.802938in}}%
\pgfpathlineto{\pgfqpoint{2.020689in}{0.808210in}}%
\pgfpathlineto{\pgfqpoint{2.023789in}{0.808167in}}%
\pgfpathlineto{\pgfqpoint{2.028749in}{0.806891in}}%
\pgfpathlineto{\pgfqpoint{2.032469in}{0.810353in}}%
\pgfpathlineto{\pgfqpoint{2.034329in}{0.810793in}}%
\pgfpathlineto{\pgfqpoint{2.036189in}{0.812906in}}%
\pgfpathlineto{\pgfqpoint{2.040529in}{0.820065in}}%
\pgfpathlineto{\pgfqpoint{2.043009in}{0.819014in}}%
\pgfpathlineto{\pgfqpoint{2.046109in}{0.811940in}}%
\pgfpathlineto{\pgfqpoint{2.052929in}{0.798574in}}%
\pgfpathlineto{\pgfqpoint{2.056029in}{0.797121in}}%
\pgfpathlineto{\pgfqpoint{2.057269in}{0.796835in}}%
\pgfpathlineto{\pgfqpoint{2.060989in}{0.793583in}}%
\pgfpathlineto{\pgfqpoint{2.065949in}{0.789855in}}%
\pgfpathlineto{\pgfqpoint{2.069049in}{0.787596in}}%
\pgfpathlineto{\pgfqpoint{2.071529in}{0.788476in}}%
\pgfpathlineto{\pgfqpoint{2.074009in}{0.793635in}}%
\pgfpathlineto{\pgfqpoint{2.084549in}{0.825487in}}%
\pgfpathlineto{\pgfqpoint{2.087029in}{0.831060in}}%
\pgfpathlineto{\pgfqpoint{2.093229in}{0.838938in}}%
\pgfpathlineto{\pgfqpoint{2.100669in}{0.842779in}}%
\pgfpathlineto{\pgfqpoint{2.111209in}{0.833745in}}%
\pgfpathlineto{\pgfqpoint{2.113069in}{0.834191in}}%
\pgfpathlineto{\pgfqpoint{2.115549in}{0.837900in}}%
\pgfpathlineto{\pgfqpoint{2.120509in}{0.849532in}}%
\pgfpathlineto{\pgfqpoint{2.123609in}{0.855904in}}%
\pgfpathlineto{\pgfqpoint{2.125469in}{0.855287in}}%
\pgfpathlineto{\pgfqpoint{2.130429in}{0.850183in}}%
\pgfpathlineto{\pgfqpoint{2.135389in}{0.840063in}}%
\pgfpathlineto{\pgfqpoint{2.138489in}{0.839357in}}%
\pgfpathlineto{\pgfqpoint{2.145309in}{0.845873in}}%
\pgfpathlineto{\pgfqpoint{2.148409in}{0.850360in}}%
\pgfpathlineto{\pgfqpoint{2.151509in}{0.858779in}}%
\pgfpathlineto{\pgfqpoint{2.163289in}{0.893217in}}%
\pgfpathlineto{\pgfqpoint{2.165769in}{0.894063in}}%
\pgfpathlineto{\pgfqpoint{2.167629in}{0.892086in}}%
\pgfpathlineto{\pgfqpoint{2.170729in}{0.888831in}}%
\pgfpathlineto{\pgfqpoint{2.172589in}{0.887865in}}%
\pgfpathlineto{\pgfqpoint{2.176309in}{0.880821in}}%
\pgfpathlineto{\pgfqpoint{2.181269in}{0.865969in}}%
\pgfpathlineto{\pgfqpoint{2.184989in}{0.861647in}}%
\pgfpathlineto{\pgfqpoint{2.187469in}{0.862503in}}%
\pgfpathlineto{\pgfqpoint{2.189329in}{0.862690in}}%
\pgfpathlineto{\pgfqpoint{2.191189in}{0.864855in}}%
\pgfpathlineto{\pgfqpoint{2.199249in}{0.861328in}}%
\pgfpathlineto{\pgfqpoint{2.207309in}{0.859301in}}%
\pgfpathlineto{\pgfqpoint{2.215989in}{0.843220in}}%
\pgfpathlineto{\pgfqpoint{2.218469in}{0.840073in}}%
\pgfpathlineto{\pgfqpoint{2.224669in}{0.838843in}}%
\pgfpathlineto{\pgfqpoint{2.236449in}{0.852685in}}%
\pgfpathlineto{\pgfqpoint{2.242649in}{0.853737in}}%
\pgfpathlineto{\pgfqpoint{2.245749in}{0.849967in}}%
\pgfpathlineto{\pgfqpoint{2.248229in}{0.844772in}}%
\pgfpathlineto{\pgfqpoint{2.253189in}{0.834668in}}%
\pgfpathlineto{\pgfqpoint{2.256909in}{0.831273in}}%
\pgfpathlineto{\pgfqpoint{2.261249in}{0.831554in}}%
\pgfpathlineto{\pgfqpoint{2.263729in}{0.832354in}}%
\pgfpathlineto{\pgfqpoint{2.267449in}{0.831878in}}%
\pgfpathlineto{\pgfqpoint{2.271789in}{0.832589in}}%
\pgfpathlineto{\pgfqpoint{2.281089in}{0.849932in}}%
\pgfpathlineto{\pgfqpoint{2.283569in}{0.855090in}}%
\pgfpathlineto{\pgfqpoint{2.289149in}{0.860905in}}%
\pgfpathlineto{\pgfqpoint{2.291629in}{0.865084in}}%
\pgfpathlineto{\pgfqpoint{2.296589in}{0.875515in}}%
\pgfpathlineto{\pgfqpoint{2.301549in}{0.875210in}}%
\pgfpathlineto{\pgfqpoint{2.305889in}{0.875147in}}%
\pgfpathlineto{\pgfqpoint{2.307749in}{0.876521in}}%
\pgfpathlineto{\pgfqpoint{2.319529in}{0.897047in}}%
\pgfpathlineto{\pgfqpoint{2.326349in}{0.907796in}}%
\pgfpathlineto{\pgfqpoint{2.326969in}{0.907136in}}%
\pgfpathlineto{\pgfqpoint{2.328209in}{0.907153in}}%
\pgfpathlineto{\pgfqpoint{2.330069in}{0.908931in}}%
\pgfpathlineto{\pgfqpoint{2.331929in}{0.908949in}}%
\pgfpathlineto{\pgfqpoint{2.334409in}{0.909852in}}%
\pgfpathlineto{\pgfqpoint{2.338129in}{0.904148in}}%
\pgfpathlineto{\pgfqpoint{2.343089in}{0.899987in}}%
\pgfpathlineto{\pgfqpoint{2.345569in}{0.899907in}}%
\pgfpathlineto{\pgfqpoint{2.349289in}{0.901653in}}%
\pgfpathlineto{\pgfqpoint{2.357349in}{0.916130in}}%
\pgfpathlineto{\pgfqpoint{2.364789in}{0.940677in}}%
\pgfpathlineto{\pgfqpoint{2.370369in}{0.958329in}}%
\pgfpathlineto{\pgfqpoint{2.374089in}{0.960366in}}%
\pgfpathlineto{\pgfqpoint{2.375949in}{0.958773in}}%
\pgfpathlineto{\pgfqpoint{2.379669in}{0.961990in}}%
\pgfpathlineto{\pgfqpoint{2.381529in}{0.963442in}}%
\pgfpathlineto{\pgfqpoint{2.385869in}{0.959502in}}%
\pgfpathlineto{\pgfqpoint{2.388969in}{0.953378in}}%
\pgfpathlineto{\pgfqpoint{2.390829in}{0.953648in}}%
\pgfpathlineto{\pgfqpoint{2.395169in}{0.955524in}}%
\pgfpathlineto{\pgfqpoint{2.400129in}{0.966192in}}%
\pgfpathlineto{\pgfqpoint{2.405709in}{0.976073in}}%
\pgfpathlineto{\pgfqpoint{2.413149in}{0.985145in}}%
\pgfpathlineto{\pgfqpoint{2.415009in}{0.986093in}}%
\pgfpathlineto{\pgfqpoint{2.416869in}{0.986564in}}%
\pgfpathlineto{\pgfqpoint{2.418729in}{0.984467in}}%
\pgfpathlineto{\pgfqpoint{2.430509in}{1.003070in}}%
\pgfpathlineto{\pgfqpoint{2.432369in}{1.007797in}}%
\pgfpathlineto{\pgfqpoint{2.438569in}{1.029107in}}%
\pgfpathlineto{\pgfqpoint{2.453449in}{1.065260in}}%
\pgfpathlineto{\pgfqpoint{2.455309in}{1.066515in}}%
\pgfpathlineto{\pgfqpoint{2.459649in}{1.064689in}}%
\pgfpathlineto{\pgfqpoint{2.463989in}{1.065024in}}%
\pgfpathlineto{\pgfqpoint{2.467089in}{1.066320in}}%
\pgfpathlineto{\pgfqpoint{2.471429in}{1.069537in}}%
\pgfpathlineto{\pgfqpoint{2.474529in}{1.074065in}}%
\pgfpathlineto{\pgfqpoint{2.478249in}{1.085295in}}%
\pgfpathlineto{\pgfqpoint{2.485689in}{1.102804in}}%
\pgfpathlineto{\pgfqpoint{2.500569in}{1.128152in}}%
\pgfpathlineto{\pgfqpoint{2.503049in}{1.134191in}}%
\pgfpathlineto{\pgfqpoint{2.506149in}{1.135804in}}%
\pgfpathlineto{\pgfqpoint{2.508629in}{1.137001in}}%
\pgfpathlineto{\pgfqpoint{2.512349in}{1.142819in}}%
\pgfpathlineto{\pgfqpoint{2.514829in}{1.142104in}}%
\pgfpathlineto{\pgfqpoint{2.516689in}{1.143790in}}%
\pgfpathlineto{\pgfqpoint{2.527229in}{1.163171in}}%
\pgfpathlineto{\pgfqpoint{2.530949in}{1.162790in}}%
\pgfpathlineto{\pgfqpoint{2.533429in}{1.164354in}}%
\pgfpathlineto{\pgfqpoint{2.537149in}{1.163889in}}%
\pgfpathlineto{\pgfqpoint{2.540249in}{1.161110in}}%
\pgfpathlineto{\pgfqpoint{2.542729in}{1.157318in}}%
\pgfpathlineto{\pgfqpoint{2.550789in}{1.147587in}}%
\pgfpathlineto{\pgfqpoint{2.554509in}{1.148244in}}%
\pgfpathlineto{\pgfqpoint{2.558849in}{1.144785in}}%
\pgfpathlineto{\pgfqpoint{2.560089in}{1.144690in}}%
\pgfpathlineto{\pgfqpoint{2.563189in}{1.146175in}}%
\pgfpathlineto{\pgfqpoint{2.568769in}{1.142224in}}%
\pgfpathlineto{\pgfqpoint{2.573109in}{1.135625in}}%
\pgfpathlineto{\pgfqpoint{2.577449in}{1.130446in}}%
\pgfpathlineto{\pgfqpoint{2.583029in}{1.133329in}}%
\pgfpathlineto{\pgfqpoint{2.585509in}{1.130897in}}%
\pgfpathlineto{\pgfqpoint{2.595429in}{1.117780in}}%
\pgfpathlineto{\pgfqpoint{2.600389in}{1.118205in}}%
\pgfpathlineto{\pgfqpoint{2.604729in}{1.121632in}}%
\pgfpathlineto{\pgfqpoint{2.607209in}{1.123276in}}%
\pgfpathlineto{\pgfqpoint{2.612169in}{1.124406in}}%
\pgfpathlineto{\pgfqpoint{2.614029in}{1.124035in}}%
\pgfpathlineto{\pgfqpoint{2.622089in}{1.117333in}}%
\pgfpathlineto{\pgfqpoint{2.625189in}{1.117929in}}%
\pgfpathlineto{\pgfqpoint{2.634489in}{1.113206in}}%
\pgfpathlineto{\pgfqpoint{2.636969in}{1.114008in}}%
\pgfpathlineto{\pgfqpoint{2.639449in}{1.115307in}}%
\pgfpathlineto{\pgfqpoint{2.641309in}{1.114821in}}%
\pgfpathlineto{\pgfqpoint{2.646269in}{1.118171in}}%
\pgfpathlineto{\pgfqpoint{2.649369in}{1.117203in}}%
\pgfpathlineto{\pgfqpoint{2.651229in}{1.118036in}}%
\pgfpathlineto{\pgfqpoint{2.653089in}{1.118413in}}%
\pgfpathlineto{\pgfqpoint{2.658049in}{1.114031in}}%
\pgfpathlineto{\pgfqpoint{2.659909in}{1.115139in}}%
\pgfpathlineto{\pgfqpoint{2.661769in}{1.116244in}}%
\pgfpathlineto{\pgfqpoint{2.667349in}{1.120780in}}%
\pgfpathlineto{\pgfqpoint{2.670449in}{1.124166in}}%
\pgfpathlineto{\pgfqpoint{2.672309in}{1.124710in}}%
\pgfpathlineto{\pgfqpoint{2.676029in}{1.125325in}}%
\pgfpathlineto{\pgfqpoint{2.679749in}{1.128962in}}%
\pgfpathlineto{\pgfqpoint{2.684709in}{1.128304in}}%
\pgfpathlineto{\pgfqpoint{2.689049in}{1.120117in}}%
\pgfpathlineto{\pgfqpoint{2.707649in}{1.066203in}}%
\pgfpathlineto{\pgfqpoint{2.710129in}{1.063804in}}%
\pgfpathlineto{\pgfqpoint{2.713229in}{1.064637in}}%
\pgfpathlineto{\pgfqpoint{2.720669in}{1.069134in}}%
\pgfpathlineto{\pgfqpoint{2.723769in}{1.068245in}}%
\pgfpathlineto{\pgfqpoint{2.725009in}{1.067581in}}%
\pgfpathlineto{\pgfqpoint{2.732449in}{1.051924in}}%
\pgfpathlineto{\pgfqpoint{2.735549in}{1.045497in}}%
\pgfpathlineto{\pgfqpoint{2.737409in}{1.040550in}}%
\pgfpathlineto{\pgfqpoint{2.744229in}{1.022171in}}%
\pgfpathlineto{\pgfqpoint{2.746709in}{1.019752in}}%
\pgfpathlineto{\pgfqpoint{2.750429in}{1.022015in}}%
\pgfpathlineto{\pgfqpoint{2.751669in}{1.022308in}}%
\pgfpathlineto{\pgfqpoint{2.754149in}{1.022289in}}%
\pgfpathlineto{\pgfqpoint{2.759109in}{1.022127in}}%
\pgfpathlineto{\pgfqpoint{2.761589in}{1.021629in}}%
\pgfpathlineto{\pgfqpoint{2.770269in}{1.012971in}}%
\pgfpathlineto{\pgfqpoint{2.773989in}{1.006048in}}%
\pgfpathlineto{\pgfqpoint{2.777709in}{1.003548in}}%
\pgfpathlineto{\pgfqpoint{2.781429in}{1.001670in}}%
\pgfpathlineto{\pgfqpoint{2.782669in}{1.002677in}}%
\pgfpathlineto{\pgfqpoint{2.791349in}{1.018208in}}%
\pgfpathlineto{\pgfqpoint{2.800029in}{1.034691in}}%
\pgfpathlineto{\pgfqpoint{2.803749in}{1.037195in}}%
\pgfpathlineto{\pgfqpoint{2.806849in}{1.033393in}}%
\pgfpathlineto{\pgfqpoint{2.811809in}{1.020779in}}%
\pgfpathlineto{\pgfqpoint{2.815529in}{1.010524in}}%
\pgfpathlineto{\pgfqpoint{2.818629in}{1.006842in}}%
\pgfpathlineto{\pgfqpoint{2.820489in}{1.006877in}}%
\pgfpathlineto{\pgfqpoint{2.824829in}{1.006761in}}%
\pgfpathlineto{\pgfqpoint{2.827309in}{1.010746in}}%
\pgfpathlineto{\pgfqpoint{2.832889in}{1.023384in}}%
\pgfpathlineto{\pgfqpoint{2.837849in}{1.040976in}}%
\pgfpathlineto{\pgfqpoint{2.841569in}{1.045984in}}%
\pgfpathlineto{\pgfqpoint{2.844049in}{1.046644in}}%
\pgfpathlineto{\pgfqpoint{2.845909in}{1.048708in}}%
\pgfpathlineto{\pgfqpoint{2.847769in}{1.049970in}}%
\pgfpathlineto{\pgfqpoint{2.849629in}{1.050610in}}%
\pgfpathlineto{\pgfqpoint{2.852109in}{1.052118in}}%
\pgfpathlineto{\pgfqpoint{2.857069in}{1.049421in}}%
\pgfpathlineto{\pgfqpoint{2.859549in}{1.049644in}}%
\pgfpathlineto{\pgfqpoint{2.868229in}{1.054204in}}%
\pgfpathlineto{\pgfqpoint{2.873189in}{1.050603in}}%
\pgfpathlineto{\pgfqpoint{2.876909in}{1.051822in}}%
\pgfpathlineto{\pgfqpoint{2.883729in}{1.057367in}}%
\pgfpathlineto{\pgfqpoint{2.891169in}{1.054191in}}%
\pgfpathlineto{\pgfqpoint{2.894269in}{1.052937in}}%
\pgfpathlineto{\pgfqpoint{2.902949in}{1.053477in}}%
\pgfpathlineto{\pgfqpoint{2.906049in}{1.055655in}}%
\pgfpathlineto{\pgfqpoint{2.911629in}{1.064791in}}%
\pgfpathlineto{\pgfqpoint{2.914109in}{1.066957in}}%
\pgfpathlineto{\pgfqpoint{2.916589in}{1.073431in}}%
\pgfpathlineto{\pgfqpoint{2.920309in}{1.080923in}}%
\pgfpathlineto{\pgfqpoint{2.925269in}{1.087802in}}%
\pgfpathlineto{\pgfqpoint{2.927749in}{1.086353in}}%
\pgfpathlineto{\pgfqpoint{2.930229in}{1.087981in}}%
\pgfpathlineto{\pgfqpoint{2.933949in}{1.085430in}}%
\pgfpathlineto{\pgfqpoint{2.938289in}{1.082808in}}%
\pgfpathlineto{\pgfqpoint{2.946349in}{1.062136in}}%
\pgfpathlineto{\pgfqpoint{2.950689in}{1.062279in}}%
\pgfpathlineto{\pgfqpoint{2.954409in}{1.062453in}}%
\pgfpathlineto{\pgfqpoint{2.956889in}{1.063642in}}%
\pgfpathlineto{\pgfqpoint{2.961849in}{1.070352in}}%
\pgfpathlineto{\pgfqpoint{2.968049in}{1.084630in}}%
\pgfpathlineto{\pgfqpoint{2.970529in}{1.088250in}}%
\pgfpathlineto{\pgfqpoint{2.974869in}{1.090408in}}%
\pgfpathlineto{\pgfqpoint{2.977969in}{1.092787in}}%
\pgfpathlineto{\pgfqpoint{2.980449in}{1.092330in}}%
\pgfpathlineto{\pgfqpoint{2.982929in}{1.094917in}}%
\pgfpathlineto{\pgfqpoint{2.985409in}{1.096614in}}%
\pgfpathlineto{\pgfqpoint{2.993469in}{1.102786in}}%
\pgfpathlineto{\pgfqpoint{2.996569in}{1.106296in}}%
\pgfpathlineto{\pgfqpoint{3.000289in}{1.114158in}}%
\pgfpathlineto{\pgfqpoint{3.004629in}{1.120102in}}%
\pgfpathlineto{\pgfqpoint{3.008969in}{1.126062in}}%
\pgfpathlineto{\pgfqpoint{3.011449in}{1.124855in}}%
\pgfpathlineto{\pgfqpoint{3.013309in}{1.125780in}}%
\pgfpathlineto{\pgfqpoint{3.020129in}{1.121703in}}%
\pgfpathlineto{\pgfqpoint{3.025709in}{1.123761in}}%
\pgfpathlineto{\pgfqpoint{3.028809in}{1.125617in}}%
\pgfpathlineto{\pgfqpoint{3.033769in}{1.128985in}}%
\pgfpathlineto{\pgfqpoint{3.036249in}{1.135166in}}%
\pgfpathlineto{\pgfqpoint{3.043689in}{1.158565in}}%
\pgfpathlineto{\pgfqpoint{3.052369in}{1.172794in}}%
\pgfpathlineto{\pgfqpoint{3.054849in}{1.172941in}}%
\pgfpathlineto{\pgfqpoint{3.056709in}{1.171535in}}%
\pgfpathlineto{\pgfqpoint{3.062289in}{1.173106in}}%
\pgfpathlineto{\pgfqpoint{3.064149in}{1.175626in}}%
\pgfpathlineto{\pgfqpoint{3.069109in}{1.183326in}}%
\pgfpathlineto{\pgfqpoint{3.071589in}{1.186647in}}%
\pgfpathlineto{\pgfqpoint{3.080889in}{1.207939in}}%
\pgfpathlineto{\pgfqpoint{3.088329in}{1.224722in}}%
\pgfpathlineto{\pgfqpoint{3.092049in}{1.233136in}}%
\pgfpathlineto{\pgfqpoint{3.098249in}{1.250383in}}%
\pgfpathlineto{\pgfqpoint{3.107549in}{1.278970in}}%
\pgfpathlineto{\pgfqpoint{3.119949in}{1.326760in}}%
\pgfpathlineto{\pgfqpoint{3.132349in}{1.401289in}}%
\pgfpathlineto{\pgfqpoint{3.135449in}{1.418664in}}%
\pgfpathlineto{\pgfqpoint{3.144749in}{1.488055in}}%
\pgfpathlineto{\pgfqpoint{3.150329in}{1.546818in}}%
\pgfpathlineto{\pgfqpoint{3.155289in}{1.626470in}}%
\pgfpathlineto{\pgfqpoint{3.160249in}{1.745097in}}%
\pgfpathlineto{\pgfqpoint{3.164589in}{1.912015in}}%
\pgfpathlineto{\pgfqpoint{3.168929in}{2.178290in}}%
\pgfpathlineto{\pgfqpoint{3.172649in}{2.543770in}}%
\pgfpathlineto{\pgfqpoint{3.176369in}{3.124628in}}%
\pgfpathlineto{\pgfqpoint{3.179360in}{3.843314in}}%
\pgfpathlineto{\pgfqpoint{3.179360in}{3.843314in}}%
\pgfusepath{stroke}%
\end{pgfscope}%
\begin{pgfscope}%
\pgfsetrectcap%
\pgfsetmiterjoin%
\pgfsetlinewidth{0.803000pt}%
\definecolor{currentstroke}{rgb}{0.000000,0.000000,0.000000}%
\pgfsetstrokecolor{currentstroke}%
\pgfsetdash{}{0pt}%
\pgfpathmoveto{\pgfqpoint{0.575469in}{0.560814in}}%
\pgfpathlineto{\pgfqpoint{0.575469in}{3.833314in}}%
\pgfusepath{stroke}%
\end{pgfscope}%
\begin{pgfscope}%
\pgfsetrectcap%
\pgfsetmiterjoin%
\pgfsetlinewidth{0.803000pt}%
\definecolor{currentstroke}{rgb}{0.000000,0.000000,0.000000}%
\pgfsetstrokecolor{currentstroke}%
\pgfsetdash{}{0pt}%
\pgfpathmoveto{\pgfqpoint{3.675469in}{0.560814in}}%
\pgfpathlineto{\pgfqpoint{3.675469in}{3.833314in}}%
\pgfusepath{stroke}%
\end{pgfscope}%
\begin{pgfscope}%
\pgfsetrectcap%
\pgfsetmiterjoin%
\pgfsetlinewidth{0.803000pt}%
\definecolor{currentstroke}{rgb}{0.000000,0.000000,0.000000}%
\pgfsetstrokecolor{currentstroke}%
\pgfsetdash{}{0pt}%
\pgfpathmoveto{\pgfqpoint{0.575469in}{0.560814in}}%
\pgfpathlineto{\pgfqpoint{3.675469in}{0.560814in}}%
\pgfusepath{stroke}%
\end{pgfscope}%
\begin{pgfscope}%
\pgfsetrectcap%
\pgfsetmiterjoin%
\pgfsetlinewidth{0.803000pt}%
\definecolor{currentstroke}{rgb}{0.000000,0.000000,0.000000}%
\pgfsetstrokecolor{currentstroke}%
\pgfsetdash{}{0pt}%
\pgfpathmoveto{\pgfqpoint{0.575469in}{3.833314in}}%
\pgfpathlineto{\pgfqpoint{3.675469in}{3.833314in}}%
\pgfusepath{stroke}%
\end{pgfscope}%
\begin{pgfscope}%
\pgfsetroundcap%
\pgfsetroundjoin%
\pgfsetlinewidth{1.003750pt}%
\definecolor{currentstroke}{rgb}{0.000000,0.000000,0.000000}%
\pgfsetstrokecolor{currentstroke}%
\pgfsetdash{}{0pt}%
\pgfpathmoveto{\pgfqpoint{2.233941in}{2.378869in}}%
\pgfpathquadraticcurveto{\pgfqpoint{1.846469in}{2.378869in}}{\pgfqpoint{1.443469in}{2.378869in}}%
\pgfusepath{stroke}%
\end{pgfscope}%
\begin{pgfscope}%
\pgfsetroundcap%
\pgfsetroundjoin%
\definecolor{currentfill}{rgb}{0.000000,0.000000,0.000000}%
\pgfsetfillcolor{currentfill}%
\pgfsetlinewidth{1.003750pt}%
\definecolor{currentstroke}{rgb}{0.000000,0.000000,0.000000}%
\pgfsetstrokecolor{currentstroke}%
\pgfsetdash{}{0pt}%
\pgfpathmoveto{\pgfqpoint{2.167274in}{2.412203in}}%
\pgfpathlineto{\pgfqpoint{2.233941in}{2.378869in}}%
\pgfpathlineto{\pgfqpoint{2.167274in}{2.345536in}}%
\pgfpathlineto{\pgfqpoint{2.167274in}{2.412203in}}%
\pgfpathclose%
\pgfusepath{stroke,fill}%
\end{pgfscope}%
\begin{pgfscope}%
\pgfsetbuttcap%
\pgfsetmiterjoin%
\definecolor{currentfill}{rgb}{1.000000,1.000000,1.000000}%
\pgfsetfillcolor{currentfill}%
\pgfsetfillopacity{0.800000}%
\pgfsetlinewidth{1.003750pt}%
\definecolor{currentstroke}{rgb}{0.800000,0.800000,0.800000}%
\pgfsetstrokecolor{currentstroke}%
\pgfsetstrokeopacity{0.800000}%
\pgfsetdash{}{0pt}%
\pgfpathmoveto{\pgfqpoint{1.602691in}{2.828363in}}%
\pgfpathlineto{\pgfqpoint{2.943931in}{2.828363in}}%
\pgfpathquadraticcurveto{\pgfqpoint{2.971709in}{2.828363in}}{\pgfqpoint{2.971709in}{2.856141in}}%
\pgfpathlineto{\pgfqpoint{2.971709in}{3.572467in}}%
\pgfpathquadraticcurveto{\pgfqpoint{2.971709in}{3.600244in}}{\pgfqpoint{2.943931in}{3.600244in}}%
\pgfpathlineto{\pgfqpoint{1.602691in}{3.600244in}}%
\pgfpathquadraticcurveto{\pgfqpoint{1.574914in}{3.600244in}}{\pgfqpoint{1.574914in}{3.572467in}}%
\pgfpathlineto{\pgfqpoint{1.574914in}{2.856141in}}%
\pgfpathquadraticcurveto{\pgfqpoint{1.574914in}{2.828363in}}{\pgfqpoint{1.602691in}{2.828363in}}%
\pgfpathclose%
\pgfusepath{stroke,fill}%
\end{pgfscope}%
\begin{pgfscope}%
\pgfsetrectcap%
\pgfsetroundjoin%
\pgfsetlinewidth{2.007500pt}%
\definecolor{currentstroke}{rgb}{0.121569,0.466667,0.705882}%
\pgfsetstrokecolor{currentstroke}%
\pgfsetdash{}{0pt}%
\pgfpathmoveto{\pgfqpoint{1.630469in}{3.496078in}}%
\pgfpathlineto{\pgfqpoint{1.908247in}{3.496078in}}%
\pgfusepath{stroke}%
\end{pgfscope}%
\begin{pgfscope}%
\pgftext[x=2.019358in,y=3.447467in,left,base]{\rmfamily\fontsize{10.000000}{12.000000}\selectfont Pressure}%
\end{pgfscope}%
\begin{pgfscope}%
\pgfsetrectcap%
\pgfsetroundjoin%
\pgfsetlinewidth{2.007500pt}%
\definecolor{currentstroke}{rgb}{1.000000,0.498039,0.054902}%
\pgfsetstrokecolor{currentstroke}%
\pgfsetdash{}{0pt}%
\pgfpathmoveto{\pgfqpoint{1.630469in}{3.215832in}}%
\pgfpathlineto{\pgfqpoint{1.908247in}{3.215832in}}%
\pgfusepath{stroke}%
\end{pgfscope}%
\begin{pgfscope}%
\pgftext[x=2.019358in,y=3.253789in,left,base]{\rmfamily\fontsize{10.000000}{12.000000}\selectfont Non-Reactive}%
\end{pgfscope}%
\begin{pgfscope}%
\pgftext[x=2.019358in,y=3.110722in,left,base]{\rmfamily\fontsize{10.000000}{12.000000}\selectfont Pressure}%
\end{pgfscope}%
\begin{pgfscope}%
\pgfsetrectcap%
\pgfsetroundjoin%
\pgfsetlinewidth{2.007500pt}%
\definecolor{currentstroke}{rgb}{0.172549,0.627451,0.172549}%
\pgfsetstrokecolor{currentstroke}%
\pgfsetdash{}{0pt}%
\pgfpathmoveto{\pgfqpoint{1.630469in}{2.962598in}}%
\pgfpathlineto{\pgfqpoint{1.908247in}{2.962598in}}%
\pgfusepath{stroke}%
\end{pgfscope}%
\begin{pgfscope}%
\pgftext[x=2.019358in,y=2.913987in,left,base]{\rmfamily\fontsize{10.000000}{12.000000}\selectfont Time Derivative}%
\end{pgfscope}%
\end{pgfpicture}%
\makeatother%
\endgroup%
}
    \caption{Definition of the ignition delay used in this work. The
    experiment in this figure was conducted for a \SI{25}{\percent} DME blend
    with \(P_0=\SI{1.1528}{\bar}\), \(T_0=\SI{338}{\K}\),
    \(P_C=\SI{29.98}{\bar}\), \(T_C=\SI{748}{\K}\), and
    \(\tau=\SI{35.39\pm1.93}{\ms}\).}
    \label{fig:ign-delay-def}
\end{figure}

The RCM is equipped with heaters to control the initial temperature of the
mixture. After filling in the components to the mixing tanks, the heaters are
switched on and the system is allowed \SI{1.5}{\hour} to come to steady state.
The mixing tanks are also equipped with magnetic stir bars so the reactants are
well mixed for the duration of the experiments.

The mixtures considered in this study are shown in \cref{tab:mixtures}. The
``\si{\percent} DME'' and ``\si{\percent} MeOH'' columns indicate the molar
percent of each component in the fuel blend. Mixtures are prepared in stainless
steel mixing tanks. The proportions of reactants in the mixture are determined
by specifying the absolute mass of the methanol in the mixture (if present), the
equivalence ratio, the oxidizer composition (in this study, \ce{O2} and \ce{N2}
in the ratio of $1:3.76$ are used throughout), and the molar ratio of DME/MeOH
in the fuel blend. Since MeOH is a liquid at room temperature and pressure, it
is injected into the mixing tank through a septum. Proportions of DME, \ce{O2},
and \ce{N2} are added manometrically at room temperature.

\begin{table}[htb]
    \centering
    \caption{Mixtures considered in this work}
    \begin{tabular}{SScccc}
        \toprule
        && \multicolumn{4}{c}{Mole Fraction (purity)} \\
        \cmidrule{3-6}
        {\si{\percent} DME}& {\si{\percent} MeOH} & DME (\SI{99.7}{\percent}) & MeOH (\SI{100.00}{\percent}) & \ce{O2} (\SI{99.994}{\percent}) & \ce{N2} (\SI{99.999}{\percent})  \\
        \midrule
        100 & 0 & 0.0654 & 0.0000 & 0.1963 & 0.7383 \\
        75 & 25 & 0.0556 & 0.0185 & 0.1945 & 0.7314 \\
        50 & 50 & 0.0427 & 0.0427 & 0.1921 & 0.7225 \\
        25 & 75 & 0.0252 & 0.0756 & 0.1889 & 0.7103 \\
        0 & 100 & 0.0000 & 0.1229 & 0.1843 & 0.6928 \\
        \bottomrule
    \end{tabular}
    \label{tab:mixtures}
\end{table}

\section{Computational Methods}\label{sec:computational-methods}

To the best of our knowledge, there are no chemical kinetic models for the
combustion of binary blends of DME and MeOH available in the literature.
Therefore, we compile a kinetic model in this work by combining two independent
models. The kinetics for DME are taken from the work of \textcite{Burke2015a}
while the kinetics for MeOH are taken from the work of \textcite{Burke2016}.

To combine the two models, duplicate reactions and species were taken from the
\textcite{Burke2015a} model; however, the models were produced by the same
research group approximately one year apart, so we do not expect many
differences in the common chemistry. In the work of \textcite{Dames2016}, it was
found that for combined models of high-reactivity fuels such as DME and
low-reactivity fuels such as propane, cross-reactions between the fuels do not
strongly affect the ignition delay and the fuels instead interact through
radical species such as \ce{OH}. Therefore, we do not consider any
cross-reactions between the high- and low-reactivity fuels in this study (DME
and MeOH, respectively).

\subsection{RCM Modeling}\label{sec:rcm-modeling}

All of the simulations in this work use the Python interface for Cantera 2.3.0
\autocite{cantera}. Two types of simulations are considered. The first is used
to calculate \(T_C\). Detailed descriptions of the use of Cantera for these
simulations can be found in the work of \textcite{Weber2016a} and
\textcite{Dames2016}; a brief overview is given here. As mentioned in
\cref{sec:experimental-methods}, non-reactive experiments are conducted to
characterize the machine-specific effects on the experimental conditions in the
RCM. This pressure trace is used to compute a volume trace by assuming that the
reactants undergo a reversible, adiabatic, constant composition (i.e.,
isentropic) compression during the compression stroke and an isentropic
expansion after the EOC. The volume trace is applied to a non-reactive
simulation conducted in an \verb|IdealGasReactor| in Cantera \autocite{cantera}
and the temperature at the end of compression is reported as \(T_C\).

The second type of simulation uses a constant-volume, adiabatic reactor. This
method does not consider the effect of the compression stroke and
post-compression heat loss present in the experiments and the initial conditions
in the simulation are set equal to the EOC conditions in the experiment. The
ignition delay is defined as the time required for the simulated temperature to
increase by \SI{400}{\K} over the initial temperature in the simulation.

\section{Results and Discussion}\label{sec:results-and-discussion}

Ignition delay results for the mixtures listed in \cref{tab:mixtures} are shown
in \cref{fig:ign-delays} for the stoichiometric equivalence ratio and \(P_C =
\SI[number-unit-product={\ }]{30}{\bar}\). In the following, we use the
shorthand of specifying the molar percent of DME in the fuel blend to indicate
the blending condition.

It can be seen in \cref{fig:ign-delays} that the \SI{100}{\percent} DME case
(\SI{0}{\percent} MeOH) is the most reactive while the \SI{0}{\percent} DME case
(\SI{100}{\percent} MeOH) is the least reactive. Interestingly, the change in
reactivity as MeOH is added to DME appears to be non-linear with respect to the
molar percent of MeOH added. In other words, the change in ignition delay at a
fixed \(T_C\) is smaller going from \SI{100}{\percent} DME to \SI{50}{\percent}
DME than going from \SI{50}{\percent} DME to \SI{0}{\percent} DME.

This is also demonstrated by \cref{fig:temp-comp}, which shows the \(T_C\)
values for ignition delays near \SI{20}{\ms} at the range of mixtures considered
in this study. As the \si{\percent} DME decreases in the blend, the temperature
required to achieve the same ignition delay increases. However, the temperature
increase from \SI{100}{\percent} to \SI{50}{\percent} DME is much smaller than
the increase from \SI{50}{\percent} to \SI{0}{\percent} DME.

\begin{figure}[htb]
    \begin{minipage}[t]{0.48\textwidth}
        \centering
        \resizebox{\linewidth}{!}{%% Creator: Matplotlib, PGF backend
%%
%% To include the figure in your LaTeX document, write
%%   \input{<filename>.pgf}
%%
%% Make sure the required packages are loaded in your preamble
%%   \usepackage{pgf}
%%
%% Figures using additional raster images can only be included by \input if
%% they are in the same directory as the main LaTeX file. For loading figures
%% from other directories you can use the `import` package
%%   \usepackage{import}
%% and then include the figures with
%%   \import{<path to file>}{<filename>.pgf}
%%
%% Matplotlib used the following preamble
%%   \usepackage[utf8x]{inputenc}
%%   \usepackage[T1]{fontenc}
%%   \usepackage{mathptmx}
%%   \usepackage{mathtools}
%%
\begingroup%
\makeatletter%
\begin{pgfpicture}%
\pgfpathrectangle{\pgfpointorigin}{\pgfqpoint{3.894914in}{3.614772in}}%
\pgfusepath{use as bounding box, clip}%
\begin{pgfscope}%
\pgfsetbuttcap%
\pgfsetmiterjoin%
\definecolor{currentfill}{rgb}{1.000000,1.000000,1.000000}%
\pgfsetfillcolor{currentfill}%
\pgfsetlinewidth{0.000000pt}%
\definecolor{currentstroke}{rgb}{1.000000,1.000000,1.000000}%
\pgfsetstrokecolor{currentstroke}%
\pgfsetdash{}{0pt}%
\pgfpathmoveto{\pgfqpoint{0.000000in}{0.000000in}}%
\pgfpathlineto{\pgfqpoint{3.894914in}{0.000000in}}%
\pgfpathlineto{\pgfqpoint{3.894914in}{3.614772in}}%
\pgfpathlineto{\pgfqpoint{0.000000in}{3.614772in}}%
\pgfpathclose%
\pgfusepath{fill}%
\end{pgfscope}%
\begin{pgfscope}%
\pgfsetbuttcap%
\pgfsetmiterjoin%
\definecolor{currentfill}{rgb}{1.000000,1.000000,1.000000}%
\pgfsetfillcolor{currentfill}%
\pgfsetlinewidth{0.000000pt}%
\definecolor{currentstroke}{rgb}{0.000000,0.000000,0.000000}%
\pgfsetstrokecolor{currentstroke}%
\pgfsetstrokeopacity{0.000000}%
\pgfsetdash{}{0pt}%
\pgfpathmoveto{\pgfqpoint{0.644914in}{0.577483in}}%
\pgfpathlineto{\pgfqpoint{3.744914in}{0.577483in}}%
\pgfpathlineto{\pgfqpoint{3.744914in}{3.272483in}}%
\pgfpathlineto{\pgfqpoint{0.644914in}{3.272483in}}%
\pgfpathclose%
\pgfusepath{fill}%
\end{pgfscope}%
\begin{pgfscope}%
\pgfsetbuttcap%
\pgfsetroundjoin%
\definecolor{currentfill}{rgb}{0.000000,0.000000,0.000000}%
\pgfsetfillcolor{currentfill}%
\pgfsetlinewidth{1.003750pt}%
\definecolor{currentstroke}{rgb}{0.000000,0.000000,0.000000}%
\pgfsetstrokecolor{currentstroke}%
\pgfsetdash{}{0pt}%
\pgfsys@defobject{currentmarker}{\pgfqpoint{0.000000in}{-0.069444in}}{\pgfqpoint{0.000000in}{0.000000in}}{%
\pgfpathmoveto{\pgfqpoint{0.000000in}{0.000000in}}%
\pgfpathlineto{\pgfqpoint{0.000000in}{-0.069444in}}%
\pgfusepath{stroke,fill}%
}%
\begin{pgfscope}%
\pgfsys@transformshift{1.034920in}{0.577483in}%
\pgfsys@useobject{currentmarker}{}%
\end{pgfscope}%
\end{pgfscope}%
\begin{pgfscope}%
\pgftext[x=1.034920in,y=0.459427in,,top]{\rmfamily\fontsize{10.000000}{12.000000}\selectfont \(\displaystyle 1.1\)}%
\end{pgfscope}%
\begin{pgfscope}%
\pgfsetbuttcap%
\pgfsetroundjoin%
\definecolor{currentfill}{rgb}{0.000000,0.000000,0.000000}%
\pgfsetfillcolor{currentfill}%
\pgfsetlinewidth{1.003750pt}%
\definecolor{currentstroke}{rgb}{0.000000,0.000000,0.000000}%
\pgfsetstrokecolor{currentstroke}%
\pgfsetdash{}{0pt}%
\pgfsys@defobject{currentmarker}{\pgfqpoint{0.000000in}{-0.069444in}}{\pgfqpoint{0.000000in}{0.000000in}}{%
\pgfpathmoveto{\pgfqpoint{0.000000in}{0.000000in}}%
\pgfpathlineto{\pgfqpoint{0.000000in}{-0.069444in}}%
\pgfusepath{stroke,fill}%
}%
\begin{pgfscope}%
\pgfsys@transformshift{1.738586in}{0.577483in}%
\pgfsys@useobject{currentmarker}{}%
\end{pgfscope}%
\end{pgfscope}%
\begin{pgfscope}%
\pgftext[x=1.738586in,y=0.459427in,,top]{\rmfamily\fontsize{10.000000}{12.000000}\selectfont \(\displaystyle 1.2\)}%
\end{pgfscope}%
\begin{pgfscope}%
\pgfsetbuttcap%
\pgfsetroundjoin%
\definecolor{currentfill}{rgb}{0.000000,0.000000,0.000000}%
\pgfsetfillcolor{currentfill}%
\pgfsetlinewidth{1.003750pt}%
\definecolor{currentstroke}{rgb}{0.000000,0.000000,0.000000}%
\pgfsetstrokecolor{currentstroke}%
\pgfsetdash{}{0pt}%
\pgfsys@defobject{currentmarker}{\pgfqpoint{0.000000in}{-0.069444in}}{\pgfqpoint{0.000000in}{0.000000in}}{%
\pgfpathmoveto{\pgfqpoint{0.000000in}{0.000000in}}%
\pgfpathlineto{\pgfqpoint{0.000000in}{-0.069444in}}%
\pgfusepath{stroke,fill}%
}%
\begin{pgfscope}%
\pgfsys@transformshift{2.442252in}{0.577483in}%
\pgfsys@useobject{currentmarker}{}%
\end{pgfscope}%
\end{pgfscope}%
\begin{pgfscope}%
\pgftext[x=2.442252in,y=0.459427in,,top]{\rmfamily\fontsize{10.000000}{12.000000}\selectfont \(\displaystyle 1.3\)}%
\end{pgfscope}%
\begin{pgfscope}%
\pgfsetbuttcap%
\pgfsetroundjoin%
\definecolor{currentfill}{rgb}{0.000000,0.000000,0.000000}%
\pgfsetfillcolor{currentfill}%
\pgfsetlinewidth{1.003750pt}%
\definecolor{currentstroke}{rgb}{0.000000,0.000000,0.000000}%
\pgfsetstrokecolor{currentstroke}%
\pgfsetdash{}{0pt}%
\pgfsys@defobject{currentmarker}{\pgfqpoint{0.000000in}{-0.069444in}}{\pgfqpoint{0.000000in}{0.000000in}}{%
\pgfpathmoveto{\pgfqpoint{0.000000in}{0.000000in}}%
\pgfpathlineto{\pgfqpoint{0.000000in}{-0.069444in}}%
\pgfusepath{stroke,fill}%
}%
\begin{pgfscope}%
\pgfsys@transformshift{3.145918in}{0.577483in}%
\pgfsys@useobject{currentmarker}{}%
\end{pgfscope}%
\end{pgfscope}%
\begin{pgfscope}%
\pgftext[x=3.145918in,y=0.459427in,,top]{\rmfamily\fontsize{10.000000}{12.000000}\selectfont \(\displaystyle 1.4\)}%
\end{pgfscope}%
\begin{pgfscope}%
\pgfsetbuttcap%
\pgfsetroundjoin%
\definecolor{currentfill}{rgb}{0.000000,0.000000,0.000000}%
\pgfsetfillcolor{currentfill}%
\pgfsetlinewidth{1.003750pt}%
\definecolor{currentstroke}{rgb}{0.000000,0.000000,0.000000}%
\pgfsetstrokecolor{currentstroke}%
\pgfsetdash{}{0pt}%
\pgfsys@defobject{currentmarker}{\pgfqpoint{0.000000in}{-0.034722in}}{\pgfqpoint{0.000000in}{0.000000in}}{%
\pgfpathmoveto{\pgfqpoint{0.000000in}{0.000000in}}%
\pgfpathlineto{\pgfqpoint{0.000000in}{-0.034722in}}%
\pgfusepath{stroke,fill}%
}%
\begin{pgfscope}%
\pgfsys@transformshift{0.683087in}{0.577483in}%
\pgfsys@useobject{currentmarker}{}%
\end{pgfscope}%
\end{pgfscope}%
\begin{pgfscope}%
\pgfsetbuttcap%
\pgfsetroundjoin%
\definecolor{currentfill}{rgb}{0.000000,0.000000,0.000000}%
\pgfsetfillcolor{currentfill}%
\pgfsetlinewidth{1.003750pt}%
\definecolor{currentstroke}{rgb}{0.000000,0.000000,0.000000}%
\pgfsetstrokecolor{currentstroke}%
\pgfsetdash{}{0pt}%
\pgfsys@defobject{currentmarker}{\pgfqpoint{0.000000in}{-0.034722in}}{\pgfqpoint{0.000000in}{0.000000in}}{%
\pgfpathmoveto{\pgfqpoint{0.000000in}{0.000000in}}%
\pgfpathlineto{\pgfqpoint{0.000000in}{-0.034722in}}%
\pgfusepath{stroke,fill}%
}%
\begin{pgfscope}%
\pgfsys@transformshift{0.859004in}{0.577483in}%
\pgfsys@useobject{currentmarker}{}%
\end{pgfscope}%
\end{pgfscope}%
\begin{pgfscope}%
\pgfsetbuttcap%
\pgfsetroundjoin%
\definecolor{currentfill}{rgb}{0.000000,0.000000,0.000000}%
\pgfsetfillcolor{currentfill}%
\pgfsetlinewidth{1.003750pt}%
\definecolor{currentstroke}{rgb}{0.000000,0.000000,0.000000}%
\pgfsetstrokecolor{currentstroke}%
\pgfsetdash{}{0pt}%
\pgfsys@defobject{currentmarker}{\pgfqpoint{0.000000in}{-0.034722in}}{\pgfqpoint{0.000000in}{0.000000in}}{%
\pgfpathmoveto{\pgfqpoint{0.000000in}{0.000000in}}%
\pgfpathlineto{\pgfqpoint{0.000000in}{-0.034722in}}%
\pgfusepath{stroke,fill}%
}%
\begin{pgfscope}%
\pgfsys@transformshift{1.210837in}{0.577483in}%
\pgfsys@useobject{currentmarker}{}%
\end{pgfscope}%
\end{pgfscope}%
\begin{pgfscope}%
\pgfsetbuttcap%
\pgfsetroundjoin%
\definecolor{currentfill}{rgb}{0.000000,0.000000,0.000000}%
\pgfsetfillcolor{currentfill}%
\pgfsetlinewidth{1.003750pt}%
\definecolor{currentstroke}{rgb}{0.000000,0.000000,0.000000}%
\pgfsetstrokecolor{currentstroke}%
\pgfsetdash{}{0pt}%
\pgfsys@defobject{currentmarker}{\pgfqpoint{0.000000in}{-0.034722in}}{\pgfqpoint{0.000000in}{0.000000in}}{%
\pgfpathmoveto{\pgfqpoint{0.000000in}{0.000000in}}%
\pgfpathlineto{\pgfqpoint{0.000000in}{-0.034722in}}%
\pgfusepath{stroke,fill}%
}%
\begin{pgfscope}%
\pgfsys@transformshift{1.386753in}{0.577483in}%
\pgfsys@useobject{currentmarker}{}%
\end{pgfscope}%
\end{pgfscope}%
\begin{pgfscope}%
\pgfsetbuttcap%
\pgfsetroundjoin%
\definecolor{currentfill}{rgb}{0.000000,0.000000,0.000000}%
\pgfsetfillcolor{currentfill}%
\pgfsetlinewidth{1.003750pt}%
\definecolor{currentstroke}{rgb}{0.000000,0.000000,0.000000}%
\pgfsetstrokecolor{currentstroke}%
\pgfsetdash{}{0pt}%
\pgfsys@defobject{currentmarker}{\pgfqpoint{0.000000in}{-0.034722in}}{\pgfqpoint{0.000000in}{0.000000in}}{%
\pgfpathmoveto{\pgfqpoint{0.000000in}{0.000000in}}%
\pgfpathlineto{\pgfqpoint{0.000000in}{-0.034722in}}%
\pgfusepath{stroke,fill}%
}%
\begin{pgfscope}%
\pgfsys@transformshift{1.562670in}{0.577483in}%
\pgfsys@useobject{currentmarker}{}%
\end{pgfscope}%
\end{pgfscope}%
\begin{pgfscope}%
\pgfsetbuttcap%
\pgfsetroundjoin%
\definecolor{currentfill}{rgb}{0.000000,0.000000,0.000000}%
\pgfsetfillcolor{currentfill}%
\pgfsetlinewidth{1.003750pt}%
\definecolor{currentstroke}{rgb}{0.000000,0.000000,0.000000}%
\pgfsetstrokecolor{currentstroke}%
\pgfsetdash{}{0pt}%
\pgfsys@defobject{currentmarker}{\pgfqpoint{0.000000in}{-0.034722in}}{\pgfqpoint{0.000000in}{0.000000in}}{%
\pgfpathmoveto{\pgfqpoint{0.000000in}{0.000000in}}%
\pgfpathlineto{\pgfqpoint{0.000000in}{-0.034722in}}%
\pgfusepath{stroke,fill}%
}%
\begin{pgfscope}%
\pgfsys@transformshift{1.914503in}{0.577483in}%
\pgfsys@useobject{currentmarker}{}%
\end{pgfscope}%
\end{pgfscope}%
\begin{pgfscope}%
\pgfsetbuttcap%
\pgfsetroundjoin%
\definecolor{currentfill}{rgb}{0.000000,0.000000,0.000000}%
\pgfsetfillcolor{currentfill}%
\pgfsetlinewidth{1.003750pt}%
\definecolor{currentstroke}{rgb}{0.000000,0.000000,0.000000}%
\pgfsetstrokecolor{currentstroke}%
\pgfsetdash{}{0pt}%
\pgfsys@defobject{currentmarker}{\pgfqpoint{0.000000in}{-0.034722in}}{\pgfqpoint{0.000000in}{0.000000in}}{%
\pgfpathmoveto{\pgfqpoint{0.000000in}{0.000000in}}%
\pgfpathlineto{\pgfqpoint{0.000000in}{-0.034722in}}%
\pgfusepath{stroke,fill}%
}%
\begin{pgfscope}%
\pgfsys@transformshift{2.090419in}{0.577483in}%
\pgfsys@useobject{currentmarker}{}%
\end{pgfscope}%
\end{pgfscope}%
\begin{pgfscope}%
\pgfsetbuttcap%
\pgfsetroundjoin%
\definecolor{currentfill}{rgb}{0.000000,0.000000,0.000000}%
\pgfsetfillcolor{currentfill}%
\pgfsetlinewidth{1.003750pt}%
\definecolor{currentstroke}{rgb}{0.000000,0.000000,0.000000}%
\pgfsetstrokecolor{currentstroke}%
\pgfsetdash{}{0pt}%
\pgfsys@defobject{currentmarker}{\pgfqpoint{0.000000in}{-0.034722in}}{\pgfqpoint{0.000000in}{0.000000in}}{%
\pgfpathmoveto{\pgfqpoint{0.000000in}{0.000000in}}%
\pgfpathlineto{\pgfqpoint{0.000000in}{-0.034722in}}%
\pgfusepath{stroke,fill}%
}%
\begin{pgfscope}%
\pgfsys@transformshift{2.266336in}{0.577483in}%
\pgfsys@useobject{currentmarker}{}%
\end{pgfscope}%
\end{pgfscope}%
\begin{pgfscope}%
\pgfsetbuttcap%
\pgfsetroundjoin%
\definecolor{currentfill}{rgb}{0.000000,0.000000,0.000000}%
\pgfsetfillcolor{currentfill}%
\pgfsetlinewidth{1.003750pt}%
\definecolor{currentstroke}{rgb}{0.000000,0.000000,0.000000}%
\pgfsetstrokecolor{currentstroke}%
\pgfsetdash{}{0pt}%
\pgfsys@defobject{currentmarker}{\pgfqpoint{0.000000in}{-0.034722in}}{\pgfqpoint{0.000000in}{0.000000in}}{%
\pgfpathmoveto{\pgfqpoint{0.000000in}{0.000000in}}%
\pgfpathlineto{\pgfqpoint{0.000000in}{-0.034722in}}%
\pgfusepath{stroke,fill}%
}%
\begin{pgfscope}%
\pgfsys@transformshift{2.618169in}{0.577483in}%
\pgfsys@useobject{currentmarker}{}%
\end{pgfscope}%
\end{pgfscope}%
\begin{pgfscope}%
\pgfsetbuttcap%
\pgfsetroundjoin%
\definecolor{currentfill}{rgb}{0.000000,0.000000,0.000000}%
\pgfsetfillcolor{currentfill}%
\pgfsetlinewidth{1.003750pt}%
\definecolor{currentstroke}{rgb}{0.000000,0.000000,0.000000}%
\pgfsetstrokecolor{currentstroke}%
\pgfsetdash{}{0pt}%
\pgfsys@defobject{currentmarker}{\pgfqpoint{0.000000in}{-0.034722in}}{\pgfqpoint{0.000000in}{0.000000in}}{%
\pgfpathmoveto{\pgfqpoint{0.000000in}{0.000000in}}%
\pgfpathlineto{\pgfqpoint{0.000000in}{-0.034722in}}%
\pgfusepath{stroke,fill}%
}%
\begin{pgfscope}%
\pgfsys@transformshift{2.794085in}{0.577483in}%
\pgfsys@useobject{currentmarker}{}%
\end{pgfscope}%
\end{pgfscope}%
\begin{pgfscope}%
\pgfsetbuttcap%
\pgfsetroundjoin%
\definecolor{currentfill}{rgb}{0.000000,0.000000,0.000000}%
\pgfsetfillcolor{currentfill}%
\pgfsetlinewidth{1.003750pt}%
\definecolor{currentstroke}{rgb}{0.000000,0.000000,0.000000}%
\pgfsetstrokecolor{currentstroke}%
\pgfsetdash{}{0pt}%
\pgfsys@defobject{currentmarker}{\pgfqpoint{0.000000in}{-0.034722in}}{\pgfqpoint{0.000000in}{0.000000in}}{%
\pgfpathmoveto{\pgfqpoint{0.000000in}{0.000000in}}%
\pgfpathlineto{\pgfqpoint{0.000000in}{-0.034722in}}%
\pgfusepath{stroke,fill}%
}%
\begin{pgfscope}%
\pgfsys@transformshift{2.970002in}{0.577483in}%
\pgfsys@useobject{currentmarker}{}%
\end{pgfscope}%
\end{pgfscope}%
\begin{pgfscope}%
\pgfsetbuttcap%
\pgfsetroundjoin%
\definecolor{currentfill}{rgb}{0.000000,0.000000,0.000000}%
\pgfsetfillcolor{currentfill}%
\pgfsetlinewidth{1.003750pt}%
\definecolor{currentstroke}{rgb}{0.000000,0.000000,0.000000}%
\pgfsetstrokecolor{currentstroke}%
\pgfsetdash{}{0pt}%
\pgfsys@defobject{currentmarker}{\pgfqpoint{0.000000in}{-0.034722in}}{\pgfqpoint{0.000000in}{0.000000in}}{%
\pgfpathmoveto{\pgfqpoint{0.000000in}{0.000000in}}%
\pgfpathlineto{\pgfqpoint{0.000000in}{-0.034722in}}%
\pgfusepath{stroke,fill}%
}%
\begin{pgfscope}%
\pgfsys@transformshift{3.321835in}{0.577483in}%
\pgfsys@useobject{currentmarker}{}%
\end{pgfscope}%
\end{pgfscope}%
\begin{pgfscope}%
\pgfsetbuttcap%
\pgfsetroundjoin%
\definecolor{currentfill}{rgb}{0.000000,0.000000,0.000000}%
\pgfsetfillcolor{currentfill}%
\pgfsetlinewidth{1.003750pt}%
\definecolor{currentstroke}{rgb}{0.000000,0.000000,0.000000}%
\pgfsetstrokecolor{currentstroke}%
\pgfsetdash{}{0pt}%
\pgfsys@defobject{currentmarker}{\pgfqpoint{0.000000in}{-0.034722in}}{\pgfqpoint{0.000000in}{0.000000in}}{%
\pgfpathmoveto{\pgfqpoint{0.000000in}{0.000000in}}%
\pgfpathlineto{\pgfqpoint{0.000000in}{-0.034722in}}%
\pgfusepath{stroke,fill}%
}%
\begin{pgfscope}%
\pgfsys@transformshift{3.497751in}{0.577483in}%
\pgfsys@useobject{currentmarker}{}%
\end{pgfscope}%
\end{pgfscope}%
\begin{pgfscope}%
\pgfsetbuttcap%
\pgfsetroundjoin%
\definecolor{currentfill}{rgb}{0.000000,0.000000,0.000000}%
\pgfsetfillcolor{currentfill}%
\pgfsetlinewidth{1.003750pt}%
\definecolor{currentstroke}{rgb}{0.000000,0.000000,0.000000}%
\pgfsetstrokecolor{currentstroke}%
\pgfsetdash{}{0pt}%
\pgfsys@defobject{currentmarker}{\pgfqpoint{0.000000in}{-0.034722in}}{\pgfqpoint{0.000000in}{0.000000in}}{%
\pgfpathmoveto{\pgfqpoint{0.000000in}{0.000000in}}%
\pgfpathlineto{\pgfqpoint{0.000000in}{-0.034722in}}%
\pgfusepath{stroke,fill}%
}%
\begin{pgfscope}%
\pgfsys@transformshift{3.673667in}{0.577483in}%
\pgfsys@useobject{currentmarker}{}%
\end{pgfscope}%
\end{pgfscope}%
\begin{pgfscope}%
\pgftext[x=2.194914in,y=0.265749in,,top]{\rmfamily\fontsize{12.000000}{14.400000}\selectfont \(\displaystyle 1000/T_C\), 1/K}%
\end{pgfscope}%
\begin{pgfscope}%
\pgfsetbuttcap%
\pgfsetroundjoin%
\definecolor{currentfill}{rgb}{0.000000,0.000000,0.000000}%
\pgfsetfillcolor{currentfill}%
\pgfsetlinewidth{1.003750pt}%
\definecolor{currentstroke}{rgb}{0.000000,0.000000,0.000000}%
\pgfsetstrokecolor{currentstroke}%
\pgfsetdash{}{0pt}%
\pgfsys@defobject{currentmarker}{\pgfqpoint{-0.069444in}{0.000000in}}{\pgfqpoint{0.000000in}{0.000000in}}{%
\pgfpathmoveto{\pgfqpoint{0.000000in}{0.000000in}}%
\pgfpathlineto{\pgfqpoint{-0.069444in}{0.000000in}}%
\pgfusepath{stroke,fill}%
}%
\begin{pgfscope}%
\pgfsys@transformshift{0.644914in}{0.577483in}%
\pgfsys@useobject{currentmarker}{}%
\end{pgfscope}%
\end{pgfscope}%
\begin{pgfscope}%
\pgftext[x=0.457414in,y=0.530400in,left,base]{\rmfamily\fontsize{10.000000}{12.000000}\selectfont 1}%
\end{pgfscope}%
\begin{pgfscope}%
\pgfsetbuttcap%
\pgfsetroundjoin%
\definecolor{currentfill}{rgb}{0.000000,0.000000,0.000000}%
\pgfsetfillcolor{currentfill}%
\pgfsetlinewidth{1.003750pt}%
\definecolor{currentstroke}{rgb}{0.000000,0.000000,0.000000}%
\pgfsetstrokecolor{currentstroke}%
\pgfsetdash{}{0pt}%
\pgfsys@defobject{currentmarker}{\pgfqpoint{-0.069444in}{0.000000in}}{\pgfqpoint{0.000000in}{0.000000in}}{%
\pgfpathmoveto{\pgfqpoint{0.000000in}{0.000000in}}%
\pgfpathlineto{\pgfqpoint{-0.069444in}{0.000000in}}%
\pgfusepath{stroke,fill}%
}%
\begin{pgfscope}%
\pgfsys@transformshift{0.644914in}{1.815942in}%
\pgfsys@useobject{currentmarker}{}%
\end{pgfscope}%
\end{pgfscope}%
\begin{pgfscope}%
\pgftext[x=0.387969in,y=1.768859in,left,base]{\rmfamily\fontsize{10.000000}{12.000000}\selectfont 10}%
\end{pgfscope}%
\begin{pgfscope}%
\pgfsetbuttcap%
\pgfsetroundjoin%
\definecolor{currentfill}{rgb}{0.000000,0.000000,0.000000}%
\pgfsetfillcolor{currentfill}%
\pgfsetlinewidth{1.003750pt}%
\definecolor{currentstroke}{rgb}{0.000000,0.000000,0.000000}%
\pgfsetstrokecolor{currentstroke}%
\pgfsetdash{}{0pt}%
\pgfsys@defobject{currentmarker}{\pgfqpoint{-0.069444in}{0.000000in}}{\pgfqpoint{0.000000in}{0.000000in}}{%
\pgfpathmoveto{\pgfqpoint{0.000000in}{0.000000in}}%
\pgfpathlineto{\pgfqpoint{-0.069444in}{0.000000in}}%
\pgfusepath{stroke,fill}%
}%
\begin{pgfscope}%
\pgfsys@transformshift{0.644914in}{3.054401in}%
\pgfsys@useobject{currentmarker}{}%
\end{pgfscope}%
\end{pgfscope}%
\begin{pgfscope}%
\pgftext[x=0.318525in,y=3.007318in,left,base]{\rmfamily\fontsize{10.000000}{12.000000}\selectfont 100}%
\end{pgfscope}%
\begin{pgfscope}%
\pgfsetbuttcap%
\pgfsetroundjoin%
\definecolor{currentfill}{rgb}{0.000000,0.000000,0.000000}%
\pgfsetfillcolor{currentfill}%
\pgfsetlinewidth{1.003750pt}%
\definecolor{currentstroke}{rgb}{0.000000,0.000000,0.000000}%
\pgfsetstrokecolor{currentstroke}%
\pgfsetdash{}{0pt}%
\pgfsys@defobject{currentmarker}{\pgfqpoint{-0.034722in}{0.000000in}}{\pgfqpoint{0.000000in}{0.000000in}}{%
\pgfpathmoveto{\pgfqpoint{0.000000in}{0.000000in}}%
\pgfpathlineto{\pgfqpoint{-0.034722in}{0.000000in}}%
\pgfusepath{stroke,fill}%
}%
\begin{pgfscope}%
\pgfsys@transformshift{0.644914in}{0.950296in}%
\pgfsys@useobject{currentmarker}{}%
\end{pgfscope}%
\end{pgfscope}%
\begin{pgfscope}%
\pgfsetbuttcap%
\pgfsetroundjoin%
\definecolor{currentfill}{rgb}{0.000000,0.000000,0.000000}%
\pgfsetfillcolor{currentfill}%
\pgfsetlinewidth{1.003750pt}%
\definecolor{currentstroke}{rgb}{0.000000,0.000000,0.000000}%
\pgfsetstrokecolor{currentstroke}%
\pgfsetdash{}{0pt}%
\pgfsys@defobject{currentmarker}{\pgfqpoint{-0.034722in}{0.000000in}}{\pgfqpoint{0.000000in}{0.000000in}}{%
\pgfpathmoveto{\pgfqpoint{0.000000in}{0.000000in}}%
\pgfpathlineto{\pgfqpoint{-0.034722in}{0.000000in}}%
\pgfusepath{stroke,fill}%
}%
\begin{pgfscope}%
\pgfsys@transformshift{0.644914in}{1.168378in}%
\pgfsys@useobject{currentmarker}{}%
\end{pgfscope}%
\end{pgfscope}%
\begin{pgfscope}%
\pgfsetbuttcap%
\pgfsetroundjoin%
\definecolor{currentfill}{rgb}{0.000000,0.000000,0.000000}%
\pgfsetfillcolor{currentfill}%
\pgfsetlinewidth{1.003750pt}%
\definecolor{currentstroke}{rgb}{0.000000,0.000000,0.000000}%
\pgfsetstrokecolor{currentstroke}%
\pgfsetdash{}{0pt}%
\pgfsys@defobject{currentmarker}{\pgfqpoint{-0.034722in}{0.000000in}}{\pgfqpoint{0.000000in}{0.000000in}}{%
\pgfpathmoveto{\pgfqpoint{0.000000in}{0.000000in}}%
\pgfpathlineto{\pgfqpoint{-0.034722in}{0.000000in}}%
\pgfusepath{stroke,fill}%
}%
\begin{pgfscope}%
\pgfsys@transformshift{0.644914in}{1.323109in}%
\pgfsys@useobject{currentmarker}{}%
\end{pgfscope}%
\end{pgfscope}%
\begin{pgfscope}%
\pgfsetbuttcap%
\pgfsetroundjoin%
\definecolor{currentfill}{rgb}{0.000000,0.000000,0.000000}%
\pgfsetfillcolor{currentfill}%
\pgfsetlinewidth{1.003750pt}%
\definecolor{currentstroke}{rgb}{0.000000,0.000000,0.000000}%
\pgfsetstrokecolor{currentstroke}%
\pgfsetdash{}{0pt}%
\pgfsys@defobject{currentmarker}{\pgfqpoint{-0.034722in}{0.000000in}}{\pgfqpoint{0.000000in}{0.000000in}}{%
\pgfpathmoveto{\pgfqpoint{0.000000in}{0.000000in}}%
\pgfpathlineto{\pgfqpoint{-0.034722in}{0.000000in}}%
\pgfusepath{stroke,fill}%
}%
\begin{pgfscope}%
\pgfsys@transformshift{0.644914in}{1.443129in}%
\pgfsys@useobject{currentmarker}{}%
\end{pgfscope}%
\end{pgfscope}%
\begin{pgfscope}%
\pgfsetbuttcap%
\pgfsetroundjoin%
\definecolor{currentfill}{rgb}{0.000000,0.000000,0.000000}%
\pgfsetfillcolor{currentfill}%
\pgfsetlinewidth{1.003750pt}%
\definecolor{currentstroke}{rgb}{0.000000,0.000000,0.000000}%
\pgfsetstrokecolor{currentstroke}%
\pgfsetdash{}{0pt}%
\pgfsys@defobject{currentmarker}{\pgfqpoint{-0.034722in}{0.000000in}}{\pgfqpoint{0.000000in}{0.000000in}}{%
\pgfpathmoveto{\pgfqpoint{0.000000in}{0.000000in}}%
\pgfpathlineto{\pgfqpoint{-0.034722in}{0.000000in}}%
\pgfusepath{stroke,fill}%
}%
\begin{pgfscope}%
\pgfsys@transformshift{0.644914in}{1.541191in}%
\pgfsys@useobject{currentmarker}{}%
\end{pgfscope}%
\end{pgfscope}%
\begin{pgfscope}%
\pgfsetbuttcap%
\pgfsetroundjoin%
\definecolor{currentfill}{rgb}{0.000000,0.000000,0.000000}%
\pgfsetfillcolor{currentfill}%
\pgfsetlinewidth{1.003750pt}%
\definecolor{currentstroke}{rgb}{0.000000,0.000000,0.000000}%
\pgfsetstrokecolor{currentstroke}%
\pgfsetdash{}{0pt}%
\pgfsys@defobject{currentmarker}{\pgfqpoint{-0.034722in}{0.000000in}}{\pgfqpoint{0.000000in}{0.000000in}}{%
\pgfpathmoveto{\pgfqpoint{0.000000in}{0.000000in}}%
\pgfpathlineto{\pgfqpoint{-0.034722in}{0.000000in}}%
\pgfusepath{stroke,fill}%
}%
\begin{pgfscope}%
\pgfsys@transformshift{0.644914in}{1.624102in}%
\pgfsys@useobject{currentmarker}{}%
\end{pgfscope}%
\end{pgfscope}%
\begin{pgfscope}%
\pgfsetbuttcap%
\pgfsetroundjoin%
\definecolor{currentfill}{rgb}{0.000000,0.000000,0.000000}%
\pgfsetfillcolor{currentfill}%
\pgfsetlinewidth{1.003750pt}%
\definecolor{currentstroke}{rgb}{0.000000,0.000000,0.000000}%
\pgfsetstrokecolor{currentstroke}%
\pgfsetdash{}{0pt}%
\pgfsys@defobject{currentmarker}{\pgfqpoint{-0.034722in}{0.000000in}}{\pgfqpoint{0.000000in}{0.000000in}}{%
\pgfpathmoveto{\pgfqpoint{0.000000in}{0.000000in}}%
\pgfpathlineto{\pgfqpoint{-0.034722in}{0.000000in}}%
\pgfusepath{stroke,fill}%
}%
\begin{pgfscope}%
\pgfsys@transformshift{0.644914in}{1.695923in}%
\pgfsys@useobject{currentmarker}{}%
\end{pgfscope}%
\end{pgfscope}%
\begin{pgfscope}%
\pgfsetbuttcap%
\pgfsetroundjoin%
\definecolor{currentfill}{rgb}{0.000000,0.000000,0.000000}%
\pgfsetfillcolor{currentfill}%
\pgfsetlinewidth{1.003750pt}%
\definecolor{currentstroke}{rgb}{0.000000,0.000000,0.000000}%
\pgfsetstrokecolor{currentstroke}%
\pgfsetdash{}{0pt}%
\pgfsys@defobject{currentmarker}{\pgfqpoint{-0.034722in}{0.000000in}}{\pgfqpoint{0.000000in}{0.000000in}}{%
\pgfpathmoveto{\pgfqpoint{0.000000in}{0.000000in}}%
\pgfpathlineto{\pgfqpoint{-0.034722in}{0.000000in}}%
\pgfusepath{stroke,fill}%
}%
\begin{pgfscope}%
\pgfsys@transformshift{0.644914in}{1.759273in}%
\pgfsys@useobject{currentmarker}{}%
\end{pgfscope}%
\end{pgfscope}%
\begin{pgfscope}%
\pgfsetbuttcap%
\pgfsetroundjoin%
\definecolor{currentfill}{rgb}{0.000000,0.000000,0.000000}%
\pgfsetfillcolor{currentfill}%
\pgfsetlinewidth{1.003750pt}%
\definecolor{currentstroke}{rgb}{0.000000,0.000000,0.000000}%
\pgfsetstrokecolor{currentstroke}%
\pgfsetdash{}{0pt}%
\pgfsys@defobject{currentmarker}{\pgfqpoint{-0.034722in}{0.000000in}}{\pgfqpoint{0.000000in}{0.000000in}}{%
\pgfpathmoveto{\pgfqpoint{0.000000in}{0.000000in}}%
\pgfpathlineto{\pgfqpoint{-0.034722in}{0.000000in}}%
\pgfusepath{stroke,fill}%
}%
\begin{pgfscope}%
\pgfsys@transformshift{0.644914in}{2.188755in}%
\pgfsys@useobject{currentmarker}{}%
\end{pgfscope}%
\end{pgfscope}%
\begin{pgfscope}%
\pgfsetbuttcap%
\pgfsetroundjoin%
\definecolor{currentfill}{rgb}{0.000000,0.000000,0.000000}%
\pgfsetfillcolor{currentfill}%
\pgfsetlinewidth{1.003750pt}%
\definecolor{currentstroke}{rgb}{0.000000,0.000000,0.000000}%
\pgfsetstrokecolor{currentstroke}%
\pgfsetdash{}{0pt}%
\pgfsys@defobject{currentmarker}{\pgfqpoint{-0.034722in}{0.000000in}}{\pgfqpoint{0.000000in}{0.000000in}}{%
\pgfpathmoveto{\pgfqpoint{0.000000in}{0.000000in}}%
\pgfpathlineto{\pgfqpoint{-0.034722in}{0.000000in}}%
\pgfusepath{stroke,fill}%
}%
\begin{pgfscope}%
\pgfsys@transformshift{0.644914in}{2.406837in}%
\pgfsys@useobject{currentmarker}{}%
\end{pgfscope}%
\end{pgfscope}%
\begin{pgfscope}%
\pgfsetbuttcap%
\pgfsetroundjoin%
\definecolor{currentfill}{rgb}{0.000000,0.000000,0.000000}%
\pgfsetfillcolor{currentfill}%
\pgfsetlinewidth{1.003750pt}%
\definecolor{currentstroke}{rgb}{0.000000,0.000000,0.000000}%
\pgfsetstrokecolor{currentstroke}%
\pgfsetdash{}{0pt}%
\pgfsys@defobject{currentmarker}{\pgfqpoint{-0.034722in}{0.000000in}}{\pgfqpoint{0.000000in}{0.000000in}}{%
\pgfpathmoveto{\pgfqpoint{0.000000in}{0.000000in}}%
\pgfpathlineto{\pgfqpoint{-0.034722in}{0.000000in}}%
\pgfusepath{stroke,fill}%
}%
\begin{pgfscope}%
\pgfsys@transformshift{0.644914in}{2.561569in}%
\pgfsys@useobject{currentmarker}{}%
\end{pgfscope}%
\end{pgfscope}%
\begin{pgfscope}%
\pgfsetbuttcap%
\pgfsetroundjoin%
\definecolor{currentfill}{rgb}{0.000000,0.000000,0.000000}%
\pgfsetfillcolor{currentfill}%
\pgfsetlinewidth{1.003750pt}%
\definecolor{currentstroke}{rgb}{0.000000,0.000000,0.000000}%
\pgfsetstrokecolor{currentstroke}%
\pgfsetdash{}{0pt}%
\pgfsys@defobject{currentmarker}{\pgfqpoint{-0.034722in}{0.000000in}}{\pgfqpoint{0.000000in}{0.000000in}}{%
\pgfpathmoveto{\pgfqpoint{0.000000in}{0.000000in}}%
\pgfpathlineto{\pgfqpoint{-0.034722in}{0.000000in}}%
\pgfusepath{stroke,fill}%
}%
\begin{pgfscope}%
\pgfsys@transformshift{0.644914in}{2.681588in}%
\pgfsys@useobject{currentmarker}{}%
\end{pgfscope}%
\end{pgfscope}%
\begin{pgfscope}%
\pgfsetbuttcap%
\pgfsetroundjoin%
\definecolor{currentfill}{rgb}{0.000000,0.000000,0.000000}%
\pgfsetfillcolor{currentfill}%
\pgfsetlinewidth{1.003750pt}%
\definecolor{currentstroke}{rgb}{0.000000,0.000000,0.000000}%
\pgfsetstrokecolor{currentstroke}%
\pgfsetdash{}{0pt}%
\pgfsys@defobject{currentmarker}{\pgfqpoint{-0.034722in}{0.000000in}}{\pgfqpoint{0.000000in}{0.000000in}}{%
\pgfpathmoveto{\pgfqpoint{0.000000in}{0.000000in}}%
\pgfpathlineto{\pgfqpoint{-0.034722in}{0.000000in}}%
\pgfusepath{stroke,fill}%
}%
\begin{pgfscope}%
\pgfsys@transformshift{0.644914in}{2.779650in}%
\pgfsys@useobject{currentmarker}{}%
\end{pgfscope}%
\end{pgfscope}%
\begin{pgfscope}%
\pgfsetbuttcap%
\pgfsetroundjoin%
\definecolor{currentfill}{rgb}{0.000000,0.000000,0.000000}%
\pgfsetfillcolor{currentfill}%
\pgfsetlinewidth{1.003750pt}%
\definecolor{currentstroke}{rgb}{0.000000,0.000000,0.000000}%
\pgfsetstrokecolor{currentstroke}%
\pgfsetdash{}{0pt}%
\pgfsys@defobject{currentmarker}{\pgfqpoint{-0.034722in}{0.000000in}}{\pgfqpoint{0.000000in}{0.000000in}}{%
\pgfpathmoveto{\pgfqpoint{0.000000in}{0.000000in}}%
\pgfpathlineto{\pgfqpoint{-0.034722in}{0.000000in}}%
\pgfusepath{stroke,fill}%
}%
\begin{pgfscope}%
\pgfsys@transformshift{0.644914in}{2.862561in}%
\pgfsys@useobject{currentmarker}{}%
\end{pgfscope}%
\end{pgfscope}%
\begin{pgfscope}%
\pgfsetbuttcap%
\pgfsetroundjoin%
\definecolor{currentfill}{rgb}{0.000000,0.000000,0.000000}%
\pgfsetfillcolor{currentfill}%
\pgfsetlinewidth{1.003750pt}%
\definecolor{currentstroke}{rgb}{0.000000,0.000000,0.000000}%
\pgfsetstrokecolor{currentstroke}%
\pgfsetdash{}{0pt}%
\pgfsys@defobject{currentmarker}{\pgfqpoint{-0.034722in}{0.000000in}}{\pgfqpoint{0.000000in}{0.000000in}}{%
\pgfpathmoveto{\pgfqpoint{0.000000in}{0.000000in}}%
\pgfpathlineto{\pgfqpoint{-0.034722in}{0.000000in}}%
\pgfusepath{stroke,fill}%
}%
\begin{pgfscope}%
\pgfsys@transformshift{0.644914in}{2.934382in}%
\pgfsys@useobject{currentmarker}{}%
\end{pgfscope}%
\end{pgfscope}%
\begin{pgfscope}%
\pgfsetbuttcap%
\pgfsetroundjoin%
\definecolor{currentfill}{rgb}{0.000000,0.000000,0.000000}%
\pgfsetfillcolor{currentfill}%
\pgfsetlinewidth{1.003750pt}%
\definecolor{currentstroke}{rgb}{0.000000,0.000000,0.000000}%
\pgfsetstrokecolor{currentstroke}%
\pgfsetdash{}{0pt}%
\pgfsys@defobject{currentmarker}{\pgfqpoint{-0.034722in}{0.000000in}}{\pgfqpoint{0.000000in}{0.000000in}}{%
\pgfpathmoveto{\pgfqpoint{0.000000in}{0.000000in}}%
\pgfpathlineto{\pgfqpoint{-0.034722in}{0.000000in}}%
\pgfusepath{stroke,fill}%
}%
\begin{pgfscope}%
\pgfsys@transformshift{0.644914in}{2.997732in}%
\pgfsys@useobject{currentmarker}{}%
\end{pgfscope}%
\end{pgfscope}%
\begin{pgfscope}%
\pgftext[x=0.249080in,y=1.924983in,,bottom,rotate=90.000000]{\rmfamily\fontsize{12.000000}{14.400000}\selectfont Ignition Delay, ms}%
\end{pgfscope}%
\begin{pgfscope}%
\pgfpathrectangle{\pgfqpoint{0.644914in}{0.577483in}}{\pgfqpoint{3.100000in}{2.695000in}} %
\pgfusepath{clip}%
\pgfsetbuttcap%
\pgfsetroundjoin%
\pgfsetlinewidth{1.505625pt}%
\definecolor{currentstroke}{rgb}{0.121569,0.466667,0.705882}%
\pgfsetstrokecolor{currentstroke}%
\pgfsetdash{}{0pt}%
\pgfpathmoveto{\pgfqpoint{0.785823in}{1.472435in}}%
\pgfpathlineto{\pgfqpoint{0.785823in}{1.562286in}}%
\pgfusepath{stroke}%
\end{pgfscope}%
\begin{pgfscope}%
\pgfpathrectangle{\pgfqpoint{0.644914in}{0.577483in}}{\pgfqpoint{3.100000in}{2.695000in}} %
\pgfusepath{clip}%
\pgfsetbuttcap%
\pgfsetroundjoin%
\pgfsetlinewidth{1.505625pt}%
\definecolor{currentstroke}{rgb}{0.121569,0.466667,0.705882}%
\pgfsetstrokecolor{currentstroke}%
\pgfsetdash{}{0pt}%
\pgfpathmoveto{\pgfqpoint{0.882929in}{1.600543in}}%
\pgfpathlineto{\pgfqpoint{0.882929in}{1.669748in}}%
\pgfusepath{stroke}%
\end{pgfscope}%
\begin{pgfscope}%
\pgfpathrectangle{\pgfqpoint{0.644914in}{0.577483in}}{\pgfqpoint{3.100000in}{2.695000in}} %
\pgfusepath{clip}%
\pgfsetbuttcap%
\pgfsetroundjoin%
\pgfsetlinewidth{1.505625pt}%
\definecolor{currentstroke}{rgb}{0.121569,0.466667,0.705882}%
\pgfsetstrokecolor{currentstroke}%
\pgfsetdash{}{0pt}%
\pgfpathmoveto{\pgfqpoint{1.049697in}{1.973356in}}%
\pgfpathlineto{\pgfqpoint{1.049697in}{2.038309in}}%
\pgfusepath{stroke}%
\end{pgfscope}%
\begin{pgfscope}%
\pgfpathrectangle{\pgfqpoint{0.644914in}{0.577483in}}{\pgfqpoint{3.100000in}{2.695000in}} %
\pgfusepath{clip}%
\pgfsetbuttcap%
\pgfsetroundjoin%
\pgfsetlinewidth{1.505625pt}%
\definecolor{currentstroke}{rgb}{0.121569,0.466667,0.705882}%
\pgfsetstrokecolor{currentstroke}%
\pgfsetdash{}{0pt}%
\pgfpathmoveto{\pgfqpoint{1.246724in}{2.182806in}}%
\pgfpathlineto{\pgfqpoint{1.246724in}{2.252106in}}%
\pgfusepath{stroke}%
\end{pgfscope}%
\begin{pgfscope}%
\pgfpathrectangle{\pgfqpoint{0.644914in}{0.577483in}}{\pgfqpoint{3.100000in}{2.695000in}} %
\pgfusepath{clip}%
\pgfsetbuttcap%
\pgfsetroundjoin%
\pgfsetlinewidth{1.505625pt}%
\definecolor{currentstroke}{rgb}{0.121569,0.466667,0.705882}%
\pgfsetstrokecolor{currentstroke}%
\pgfsetdash{}{0pt}%
\pgfpathmoveto{\pgfqpoint{1.288240in}{2.218571in}}%
\pgfpathlineto{\pgfqpoint{1.288240in}{2.387489in}}%
\pgfusepath{stroke}%
\end{pgfscope}%
\begin{pgfscope}%
\pgfpathrectangle{\pgfqpoint{0.644914in}{0.577483in}}{\pgfqpoint{3.100000in}{2.695000in}} %
\pgfusepath{clip}%
\pgfsetbuttcap%
\pgfsetroundjoin%
\pgfsetlinewidth{1.505625pt}%
\definecolor{currentstroke}{rgb}{0.121569,0.466667,0.705882}%
\pgfsetstrokecolor{currentstroke}%
\pgfsetdash{}{0pt}%
\pgfpathmoveto{\pgfqpoint{1.366347in}{2.389344in}}%
\pgfpathlineto{\pgfqpoint{1.366347in}{2.432737in}}%
\pgfusepath{stroke}%
\end{pgfscope}%
\begin{pgfscope}%
\pgfpathrectangle{\pgfqpoint{0.644914in}{0.577483in}}{\pgfqpoint{3.100000in}{2.695000in}} %
\pgfusepath{clip}%
\pgfsetbuttcap%
\pgfsetroundjoin%
\pgfsetlinewidth{1.505625pt}%
\definecolor{currentstroke}{rgb}{0.121569,0.466667,0.705882}%
\pgfsetstrokecolor{currentstroke}%
\pgfsetdash{}{0pt}%
\pgfpathmoveto{\pgfqpoint{1.527487in}{2.708649in}}%
\pgfpathlineto{\pgfqpoint{1.527487in}{2.754510in}}%
\pgfusepath{stroke}%
\end{pgfscope}%
\begin{pgfscope}%
\pgfpathrectangle{\pgfqpoint{0.644914in}{0.577483in}}{\pgfqpoint{3.100000in}{2.695000in}} %
\pgfusepath{clip}%
\pgfsetbuttcap%
\pgfsetroundjoin%
\pgfsetlinewidth{1.505625pt}%
\definecolor{currentstroke}{rgb}{0.121569,0.466667,0.705882}%
\pgfsetstrokecolor{currentstroke}%
\pgfsetdash{}{0pt}%
\pgfpathmoveto{\pgfqpoint{1.583076in}{2.691183in}}%
\pgfpathlineto{\pgfqpoint{1.583076in}{2.730302in}}%
\pgfusepath{stroke}%
\end{pgfscope}%
\begin{pgfscope}%
\pgfpathrectangle{\pgfqpoint{0.644914in}{0.577483in}}{\pgfqpoint{3.100000in}{2.695000in}} %
\pgfusepath{clip}%
\pgfsetbuttcap%
\pgfsetroundjoin%
\pgfsetlinewidth{1.505625pt}%
\definecolor{currentstroke}{rgb}{0.121569,0.466667,0.705882}%
\pgfsetstrokecolor{currentstroke}%
\pgfsetdash{}{0pt}%
\pgfpathmoveto{\pgfqpoint{1.620370in}{2.798500in}}%
\pgfpathlineto{\pgfqpoint{1.620370in}{2.854042in}}%
\pgfusepath{stroke}%
\end{pgfscope}%
\begin{pgfscope}%
\pgfpathrectangle{\pgfqpoint{0.644914in}{0.577483in}}{\pgfqpoint{3.100000in}{2.695000in}} %
\pgfusepath{clip}%
\pgfsetbuttcap%
\pgfsetroundjoin%
\pgfsetlinewidth{1.505625pt}%
\definecolor{currentstroke}{rgb}{0.121569,0.466667,0.705882}%
\pgfsetstrokecolor{currentstroke}%
\pgfsetdash{}{0pt}%
\pgfpathmoveto{\pgfqpoint{1.772362in}{2.997373in}}%
\pgfpathlineto{\pgfqpoint{1.772362in}{3.029636in}}%
\pgfusepath{stroke}%
\end{pgfscope}%
\begin{pgfscope}%
\pgfpathrectangle{\pgfqpoint{0.644914in}{0.577483in}}{\pgfqpoint{3.100000in}{2.695000in}} %
\pgfusepath{clip}%
\pgfsetbuttcap%
\pgfsetroundjoin%
\pgfsetlinewidth{1.505625pt}%
\definecolor{currentstroke}{rgb}{0.121569,0.466667,0.705882}%
\pgfsetstrokecolor{currentstroke}%
\pgfsetdash{}{0pt}%
\pgfpathmoveto{\pgfqpoint{1.367051in}{1.430063in}}%
\pgfpathlineto{\pgfqpoint{1.367051in}{1.468345in}}%
\pgfusepath{stroke}%
\end{pgfscope}%
\begin{pgfscope}%
\pgfpathrectangle{\pgfqpoint{0.644914in}{0.577483in}}{\pgfqpoint{3.100000in}{2.695000in}} %
\pgfusepath{clip}%
\pgfsetbuttcap%
\pgfsetroundjoin%
\pgfsetlinewidth{1.505625pt}%
\definecolor{currentstroke}{rgb}{0.121569,0.466667,0.705882}%
\pgfsetstrokecolor{currentstroke}%
\pgfsetdash{}{0pt}%
\pgfpathmoveto{\pgfqpoint{1.635147in}{1.630214in}}%
\pgfpathlineto{\pgfqpoint{1.635147in}{1.645197in}}%
\pgfusepath{stroke}%
\end{pgfscope}%
\begin{pgfscope}%
\pgfpathrectangle{\pgfqpoint{0.644914in}{0.577483in}}{\pgfqpoint{3.100000in}{2.695000in}} %
\pgfusepath{clip}%
\pgfsetbuttcap%
\pgfsetroundjoin%
\pgfsetlinewidth{1.505625pt}%
\definecolor{currentstroke}{rgb}{0.121569,0.466667,0.705882}%
\pgfsetstrokecolor{currentstroke}%
\pgfsetdash{}{0pt}%
\pgfpathmoveto{\pgfqpoint{1.913799in}{1.873041in}}%
\pgfpathlineto{\pgfqpoint{1.913799in}{1.933373in}}%
\pgfusepath{stroke}%
\end{pgfscope}%
\begin{pgfscope}%
\pgfpathrectangle{\pgfqpoint{0.644914in}{0.577483in}}{\pgfqpoint{3.100000in}{2.695000in}} %
\pgfusepath{clip}%
\pgfsetbuttcap%
\pgfsetroundjoin%
\pgfsetlinewidth{1.505625pt}%
\definecolor{currentstroke}{rgb}{0.121569,0.466667,0.705882}%
\pgfsetstrokecolor{currentstroke}%
\pgfsetdash{}{0pt}%
\pgfpathmoveto{\pgfqpoint{2.108715in}{2.090477in}}%
\pgfpathlineto{\pgfqpoint{2.108715in}{2.131488in}}%
\pgfusepath{stroke}%
\end{pgfscope}%
\begin{pgfscope}%
\pgfpathrectangle{\pgfqpoint{0.644914in}{0.577483in}}{\pgfqpoint{3.100000in}{2.695000in}} %
\pgfusepath{clip}%
\pgfsetbuttcap%
\pgfsetroundjoin%
\pgfsetlinewidth{1.505625pt}%
\definecolor{currentstroke}{rgb}{0.121569,0.466667,0.705882}%
\pgfsetstrokecolor{currentstroke}%
\pgfsetdash{}{0pt}%
\pgfpathmoveto{\pgfqpoint{2.373997in}{1.825537in}}%
\pgfpathlineto{\pgfqpoint{2.373997in}{2.000744in}}%
\pgfusepath{stroke}%
\end{pgfscope}%
\begin{pgfscope}%
\pgfpathrectangle{\pgfqpoint{0.644914in}{0.577483in}}{\pgfqpoint{3.100000in}{2.695000in}} %
\pgfusepath{clip}%
\pgfsetbuttcap%
\pgfsetroundjoin%
\pgfsetlinewidth{1.505625pt}%
\definecolor{currentstroke}{rgb}{0.121569,0.466667,0.705882}%
\pgfsetstrokecolor{currentstroke}%
\pgfsetdash{}{0pt}%
\pgfpathmoveto{\pgfqpoint{2.590726in}{1.881677in}}%
\pgfpathlineto{\pgfqpoint{2.590726in}{1.944499in}}%
\pgfusepath{stroke}%
\end{pgfscope}%
\begin{pgfscope}%
\pgfpathrectangle{\pgfqpoint{0.644914in}{0.577483in}}{\pgfqpoint{3.100000in}{2.695000in}} %
\pgfusepath{clip}%
\pgfsetbuttcap%
\pgfsetroundjoin%
\pgfsetlinewidth{1.505625pt}%
\definecolor{currentstroke}{rgb}{0.121569,0.466667,0.705882}%
\pgfsetstrokecolor{currentstroke}%
\pgfsetdash{}{0pt}%
\pgfpathmoveto{\pgfqpoint{2.886265in}{1.828698in}}%
\pgfpathlineto{\pgfqpoint{2.886265in}{1.921129in}}%
\pgfusepath{stroke}%
\end{pgfscope}%
\begin{pgfscope}%
\pgfpathrectangle{\pgfqpoint{0.644914in}{0.577483in}}{\pgfqpoint{3.100000in}{2.695000in}} %
\pgfusepath{clip}%
\pgfsetbuttcap%
\pgfsetroundjoin%
\pgfsetlinewidth{1.505625pt}%
\definecolor{currentstroke}{rgb}{0.121569,0.466667,0.705882}%
\pgfsetstrokecolor{currentstroke}%
\pgfsetdash{}{0pt}%
\pgfpathmoveto{\pgfqpoint{3.126919in}{1.781516in}}%
\pgfpathlineto{\pgfqpoint{3.126919in}{1.846266in}}%
\pgfusepath{stroke}%
\end{pgfscope}%
\begin{pgfscope}%
\pgfpathrectangle{\pgfqpoint{0.644914in}{0.577483in}}{\pgfqpoint{3.100000in}{2.695000in}} %
\pgfusepath{clip}%
\pgfsetbuttcap%
\pgfsetroundjoin%
\pgfsetlinewidth{1.505625pt}%
\definecolor{currentstroke}{rgb}{0.121569,0.466667,0.705882}%
\pgfsetstrokecolor{currentstroke}%
\pgfsetdash{}{0pt}%
\pgfpathmoveto{\pgfqpoint{3.345759in}{1.970942in}}%
\pgfpathlineto{\pgfqpoint{3.345759in}{1.996147in}}%
\pgfusepath{stroke}%
\end{pgfscope}%
\begin{pgfscope}%
\pgfpathrectangle{\pgfqpoint{0.644914in}{0.577483in}}{\pgfqpoint{3.100000in}{2.695000in}} %
\pgfusepath{clip}%
\pgfsetbuttcap%
\pgfsetroundjoin%
\pgfsetlinewidth{1.505625pt}%
\definecolor{currentstroke}{rgb}{0.121569,0.466667,0.705882}%
\pgfsetstrokecolor{currentstroke}%
\pgfsetdash{}{0pt}%
\pgfpathmoveto{\pgfqpoint{3.604005in}{2.115087in}}%
\pgfpathlineto{\pgfqpoint{3.604005in}{2.175689in}}%
\pgfusepath{stroke}%
\end{pgfscope}%
\begin{pgfscope}%
\pgfpathrectangle{\pgfqpoint{0.644914in}{0.577483in}}{\pgfqpoint{3.100000in}{2.695000in}} %
\pgfusepath{clip}%
\pgfsetbuttcap%
\pgfsetroundjoin%
\pgfsetlinewidth{1.505625pt}%
\definecolor{currentstroke}{rgb}{0.121569,0.466667,0.705882}%
\pgfsetstrokecolor{currentstroke}%
\pgfsetdash{}{0pt}%
\pgfpathmoveto{\pgfqpoint{0.785823in}{1.472435in}}%
\pgfpathlineto{\pgfqpoint{0.785823in}{1.562286in}}%
\pgfusepath{stroke}%
\end{pgfscope}%
\begin{pgfscope}%
\pgfpathrectangle{\pgfqpoint{0.644914in}{0.577483in}}{\pgfqpoint{3.100000in}{2.695000in}} %
\pgfusepath{clip}%
\pgfsetbuttcap%
\pgfsetroundjoin%
\pgfsetlinewidth{1.505625pt}%
\definecolor{currentstroke}{rgb}{0.121569,0.466667,0.705882}%
\pgfsetstrokecolor{currentstroke}%
\pgfsetdash{}{0pt}%
\pgfpathmoveto{\pgfqpoint{0.882929in}{1.600543in}}%
\pgfpathlineto{\pgfqpoint{0.882929in}{1.669748in}}%
\pgfusepath{stroke}%
\end{pgfscope}%
\begin{pgfscope}%
\pgfpathrectangle{\pgfqpoint{0.644914in}{0.577483in}}{\pgfqpoint{3.100000in}{2.695000in}} %
\pgfusepath{clip}%
\pgfsetbuttcap%
\pgfsetroundjoin%
\pgfsetlinewidth{1.505625pt}%
\definecolor{currentstroke}{rgb}{0.121569,0.466667,0.705882}%
\pgfsetstrokecolor{currentstroke}%
\pgfsetdash{}{0pt}%
\pgfpathmoveto{\pgfqpoint{1.049697in}{1.973356in}}%
\pgfpathlineto{\pgfqpoint{1.049697in}{2.038309in}}%
\pgfusepath{stroke}%
\end{pgfscope}%
\begin{pgfscope}%
\pgfpathrectangle{\pgfqpoint{0.644914in}{0.577483in}}{\pgfqpoint{3.100000in}{2.695000in}} %
\pgfusepath{clip}%
\pgfsetbuttcap%
\pgfsetroundjoin%
\pgfsetlinewidth{1.505625pt}%
\definecolor{currentstroke}{rgb}{0.121569,0.466667,0.705882}%
\pgfsetstrokecolor{currentstroke}%
\pgfsetdash{}{0pt}%
\pgfpathmoveto{\pgfqpoint{1.246724in}{2.182806in}}%
\pgfpathlineto{\pgfqpoint{1.246724in}{2.252106in}}%
\pgfusepath{stroke}%
\end{pgfscope}%
\begin{pgfscope}%
\pgfpathrectangle{\pgfqpoint{0.644914in}{0.577483in}}{\pgfqpoint{3.100000in}{2.695000in}} %
\pgfusepath{clip}%
\pgfsetbuttcap%
\pgfsetroundjoin%
\pgfsetlinewidth{1.505625pt}%
\definecolor{currentstroke}{rgb}{0.121569,0.466667,0.705882}%
\pgfsetstrokecolor{currentstroke}%
\pgfsetdash{}{0pt}%
\pgfpathmoveto{\pgfqpoint{1.288240in}{2.218571in}}%
\pgfpathlineto{\pgfqpoint{1.288240in}{2.387489in}}%
\pgfusepath{stroke}%
\end{pgfscope}%
\begin{pgfscope}%
\pgfpathrectangle{\pgfqpoint{0.644914in}{0.577483in}}{\pgfqpoint{3.100000in}{2.695000in}} %
\pgfusepath{clip}%
\pgfsetbuttcap%
\pgfsetroundjoin%
\pgfsetlinewidth{1.505625pt}%
\definecolor{currentstroke}{rgb}{0.121569,0.466667,0.705882}%
\pgfsetstrokecolor{currentstroke}%
\pgfsetdash{}{0pt}%
\pgfpathmoveto{\pgfqpoint{1.366347in}{2.389344in}}%
\pgfpathlineto{\pgfqpoint{1.366347in}{2.432737in}}%
\pgfusepath{stroke}%
\end{pgfscope}%
\begin{pgfscope}%
\pgfpathrectangle{\pgfqpoint{0.644914in}{0.577483in}}{\pgfqpoint{3.100000in}{2.695000in}} %
\pgfusepath{clip}%
\pgfsetbuttcap%
\pgfsetroundjoin%
\pgfsetlinewidth{1.505625pt}%
\definecolor{currentstroke}{rgb}{0.121569,0.466667,0.705882}%
\pgfsetstrokecolor{currentstroke}%
\pgfsetdash{}{0pt}%
\pgfpathmoveto{\pgfqpoint{1.527487in}{2.708649in}}%
\pgfpathlineto{\pgfqpoint{1.527487in}{2.754510in}}%
\pgfusepath{stroke}%
\end{pgfscope}%
\begin{pgfscope}%
\pgfpathrectangle{\pgfqpoint{0.644914in}{0.577483in}}{\pgfqpoint{3.100000in}{2.695000in}} %
\pgfusepath{clip}%
\pgfsetbuttcap%
\pgfsetroundjoin%
\pgfsetlinewidth{1.505625pt}%
\definecolor{currentstroke}{rgb}{0.121569,0.466667,0.705882}%
\pgfsetstrokecolor{currentstroke}%
\pgfsetdash{}{0pt}%
\pgfpathmoveto{\pgfqpoint{1.583076in}{2.691183in}}%
\pgfpathlineto{\pgfqpoint{1.583076in}{2.730302in}}%
\pgfusepath{stroke}%
\end{pgfscope}%
\begin{pgfscope}%
\pgfpathrectangle{\pgfqpoint{0.644914in}{0.577483in}}{\pgfqpoint{3.100000in}{2.695000in}} %
\pgfusepath{clip}%
\pgfsetbuttcap%
\pgfsetroundjoin%
\pgfsetlinewidth{1.505625pt}%
\definecolor{currentstroke}{rgb}{0.121569,0.466667,0.705882}%
\pgfsetstrokecolor{currentstroke}%
\pgfsetdash{}{0pt}%
\pgfpathmoveto{\pgfqpoint{1.620370in}{2.798500in}}%
\pgfpathlineto{\pgfqpoint{1.620370in}{2.854042in}}%
\pgfusepath{stroke}%
\end{pgfscope}%
\begin{pgfscope}%
\pgfpathrectangle{\pgfqpoint{0.644914in}{0.577483in}}{\pgfqpoint{3.100000in}{2.695000in}} %
\pgfusepath{clip}%
\pgfsetbuttcap%
\pgfsetroundjoin%
\pgfsetlinewidth{1.505625pt}%
\definecolor{currentstroke}{rgb}{0.121569,0.466667,0.705882}%
\pgfsetstrokecolor{currentstroke}%
\pgfsetdash{}{0pt}%
\pgfpathmoveto{\pgfqpoint{1.772362in}{2.997373in}}%
\pgfpathlineto{\pgfqpoint{1.772362in}{3.029636in}}%
\pgfusepath{stroke}%
\end{pgfscope}%
\begin{pgfscope}%
\pgfpathrectangle{\pgfqpoint{0.644914in}{0.577483in}}{\pgfqpoint{3.100000in}{2.695000in}} %
\pgfusepath{clip}%
\pgfsetbuttcap%
\pgfsetroundjoin%
\pgfsetlinewidth{1.505625pt}%
\definecolor{currentstroke}{rgb}{0.121569,0.466667,0.705882}%
\pgfsetstrokecolor{currentstroke}%
\pgfsetdash{}{0pt}%
\pgfpathmoveto{\pgfqpoint{1.367051in}{1.430063in}}%
\pgfpathlineto{\pgfqpoint{1.367051in}{1.468345in}}%
\pgfusepath{stroke}%
\end{pgfscope}%
\begin{pgfscope}%
\pgfpathrectangle{\pgfqpoint{0.644914in}{0.577483in}}{\pgfqpoint{3.100000in}{2.695000in}} %
\pgfusepath{clip}%
\pgfsetbuttcap%
\pgfsetroundjoin%
\pgfsetlinewidth{1.505625pt}%
\definecolor{currentstroke}{rgb}{0.121569,0.466667,0.705882}%
\pgfsetstrokecolor{currentstroke}%
\pgfsetdash{}{0pt}%
\pgfpathmoveto{\pgfqpoint{1.635147in}{1.630214in}}%
\pgfpathlineto{\pgfqpoint{1.635147in}{1.645197in}}%
\pgfusepath{stroke}%
\end{pgfscope}%
\begin{pgfscope}%
\pgfpathrectangle{\pgfqpoint{0.644914in}{0.577483in}}{\pgfqpoint{3.100000in}{2.695000in}} %
\pgfusepath{clip}%
\pgfsetbuttcap%
\pgfsetroundjoin%
\pgfsetlinewidth{1.505625pt}%
\definecolor{currentstroke}{rgb}{0.121569,0.466667,0.705882}%
\pgfsetstrokecolor{currentstroke}%
\pgfsetdash{}{0pt}%
\pgfpathmoveto{\pgfqpoint{1.913799in}{1.873041in}}%
\pgfpathlineto{\pgfqpoint{1.913799in}{1.933373in}}%
\pgfusepath{stroke}%
\end{pgfscope}%
\begin{pgfscope}%
\pgfpathrectangle{\pgfqpoint{0.644914in}{0.577483in}}{\pgfqpoint{3.100000in}{2.695000in}} %
\pgfusepath{clip}%
\pgfsetbuttcap%
\pgfsetroundjoin%
\pgfsetlinewidth{1.505625pt}%
\definecolor{currentstroke}{rgb}{0.121569,0.466667,0.705882}%
\pgfsetstrokecolor{currentstroke}%
\pgfsetdash{}{0pt}%
\pgfpathmoveto{\pgfqpoint{2.108715in}{2.090477in}}%
\pgfpathlineto{\pgfqpoint{2.108715in}{2.131488in}}%
\pgfusepath{stroke}%
\end{pgfscope}%
\begin{pgfscope}%
\pgfpathrectangle{\pgfqpoint{0.644914in}{0.577483in}}{\pgfqpoint{3.100000in}{2.695000in}} %
\pgfusepath{clip}%
\pgfsetbuttcap%
\pgfsetroundjoin%
\pgfsetlinewidth{1.505625pt}%
\definecolor{currentstroke}{rgb}{0.121569,0.466667,0.705882}%
\pgfsetstrokecolor{currentstroke}%
\pgfsetdash{}{0pt}%
\pgfpathmoveto{\pgfqpoint{2.373997in}{1.825537in}}%
\pgfpathlineto{\pgfqpoint{2.373997in}{2.000744in}}%
\pgfusepath{stroke}%
\end{pgfscope}%
\begin{pgfscope}%
\pgfpathrectangle{\pgfqpoint{0.644914in}{0.577483in}}{\pgfqpoint{3.100000in}{2.695000in}} %
\pgfusepath{clip}%
\pgfsetbuttcap%
\pgfsetroundjoin%
\pgfsetlinewidth{1.505625pt}%
\definecolor{currentstroke}{rgb}{0.121569,0.466667,0.705882}%
\pgfsetstrokecolor{currentstroke}%
\pgfsetdash{}{0pt}%
\pgfpathmoveto{\pgfqpoint{2.590726in}{1.881677in}}%
\pgfpathlineto{\pgfqpoint{2.590726in}{1.944499in}}%
\pgfusepath{stroke}%
\end{pgfscope}%
\begin{pgfscope}%
\pgfpathrectangle{\pgfqpoint{0.644914in}{0.577483in}}{\pgfqpoint{3.100000in}{2.695000in}} %
\pgfusepath{clip}%
\pgfsetbuttcap%
\pgfsetroundjoin%
\pgfsetlinewidth{1.505625pt}%
\definecolor{currentstroke}{rgb}{0.121569,0.466667,0.705882}%
\pgfsetstrokecolor{currentstroke}%
\pgfsetdash{}{0pt}%
\pgfpathmoveto{\pgfqpoint{2.886265in}{1.828698in}}%
\pgfpathlineto{\pgfqpoint{2.886265in}{1.921129in}}%
\pgfusepath{stroke}%
\end{pgfscope}%
\begin{pgfscope}%
\pgfpathrectangle{\pgfqpoint{0.644914in}{0.577483in}}{\pgfqpoint{3.100000in}{2.695000in}} %
\pgfusepath{clip}%
\pgfsetbuttcap%
\pgfsetroundjoin%
\pgfsetlinewidth{1.505625pt}%
\definecolor{currentstroke}{rgb}{0.121569,0.466667,0.705882}%
\pgfsetstrokecolor{currentstroke}%
\pgfsetdash{}{0pt}%
\pgfpathmoveto{\pgfqpoint{3.126919in}{1.781516in}}%
\pgfpathlineto{\pgfqpoint{3.126919in}{1.846266in}}%
\pgfusepath{stroke}%
\end{pgfscope}%
\begin{pgfscope}%
\pgfpathrectangle{\pgfqpoint{0.644914in}{0.577483in}}{\pgfqpoint{3.100000in}{2.695000in}} %
\pgfusepath{clip}%
\pgfsetbuttcap%
\pgfsetroundjoin%
\pgfsetlinewidth{1.505625pt}%
\definecolor{currentstroke}{rgb}{0.121569,0.466667,0.705882}%
\pgfsetstrokecolor{currentstroke}%
\pgfsetdash{}{0pt}%
\pgfpathmoveto{\pgfqpoint{3.345759in}{1.970942in}}%
\pgfpathlineto{\pgfqpoint{3.345759in}{1.996147in}}%
\pgfusepath{stroke}%
\end{pgfscope}%
\begin{pgfscope}%
\pgfpathrectangle{\pgfqpoint{0.644914in}{0.577483in}}{\pgfqpoint{3.100000in}{2.695000in}} %
\pgfusepath{clip}%
\pgfsetbuttcap%
\pgfsetroundjoin%
\pgfsetlinewidth{1.505625pt}%
\definecolor{currentstroke}{rgb}{0.121569,0.466667,0.705882}%
\pgfsetstrokecolor{currentstroke}%
\pgfsetdash{}{0pt}%
\pgfpathmoveto{\pgfqpoint{3.604005in}{2.115087in}}%
\pgfpathlineto{\pgfqpoint{3.604005in}{2.175689in}}%
\pgfusepath{stroke}%
\end{pgfscope}%
\begin{pgfscope}%
\pgfpathrectangle{\pgfqpoint{0.644914in}{0.577483in}}{\pgfqpoint{3.100000in}{2.695000in}} %
\pgfusepath{clip}%
\pgfsetbuttcap%
\pgfsetroundjoin%
\pgfsetlinewidth{1.505625pt}%
\definecolor{currentstroke}{rgb}{0.121569,0.466667,0.705882}%
\pgfsetstrokecolor{currentstroke}%
\pgfsetdash{}{0pt}%
\pgfpathmoveto{\pgfqpoint{0.785823in}{1.472435in}}%
\pgfpathlineto{\pgfqpoint{0.785823in}{1.562286in}}%
\pgfusepath{stroke}%
\end{pgfscope}%
\begin{pgfscope}%
\pgfpathrectangle{\pgfqpoint{0.644914in}{0.577483in}}{\pgfqpoint{3.100000in}{2.695000in}} %
\pgfusepath{clip}%
\pgfsetbuttcap%
\pgfsetroundjoin%
\pgfsetlinewidth{1.505625pt}%
\definecolor{currentstroke}{rgb}{0.121569,0.466667,0.705882}%
\pgfsetstrokecolor{currentstroke}%
\pgfsetdash{}{0pt}%
\pgfpathmoveto{\pgfqpoint{0.882929in}{1.600543in}}%
\pgfpathlineto{\pgfqpoint{0.882929in}{1.669748in}}%
\pgfusepath{stroke}%
\end{pgfscope}%
\begin{pgfscope}%
\pgfpathrectangle{\pgfqpoint{0.644914in}{0.577483in}}{\pgfqpoint{3.100000in}{2.695000in}} %
\pgfusepath{clip}%
\pgfsetbuttcap%
\pgfsetroundjoin%
\pgfsetlinewidth{1.505625pt}%
\definecolor{currentstroke}{rgb}{0.121569,0.466667,0.705882}%
\pgfsetstrokecolor{currentstroke}%
\pgfsetdash{}{0pt}%
\pgfpathmoveto{\pgfqpoint{1.049697in}{1.973356in}}%
\pgfpathlineto{\pgfqpoint{1.049697in}{2.038309in}}%
\pgfusepath{stroke}%
\end{pgfscope}%
\begin{pgfscope}%
\pgfpathrectangle{\pgfqpoint{0.644914in}{0.577483in}}{\pgfqpoint{3.100000in}{2.695000in}} %
\pgfusepath{clip}%
\pgfsetbuttcap%
\pgfsetroundjoin%
\pgfsetlinewidth{1.505625pt}%
\definecolor{currentstroke}{rgb}{0.121569,0.466667,0.705882}%
\pgfsetstrokecolor{currentstroke}%
\pgfsetdash{}{0pt}%
\pgfpathmoveto{\pgfqpoint{1.246724in}{2.182806in}}%
\pgfpathlineto{\pgfqpoint{1.246724in}{2.252106in}}%
\pgfusepath{stroke}%
\end{pgfscope}%
\begin{pgfscope}%
\pgfpathrectangle{\pgfqpoint{0.644914in}{0.577483in}}{\pgfqpoint{3.100000in}{2.695000in}} %
\pgfusepath{clip}%
\pgfsetbuttcap%
\pgfsetroundjoin%
\pgfsetlinewidth{1.505625pt}%
\definecolor{currentstroke}{rgb}{0.121569,0.466667,0.705882}%
\pgfsetstrokecolor{currentstroke}%
\pgfsetdash{}{0pt}%
\pgfpathmoveto{\pgfqpoint{1.288240in}{2.218571in}}%
\pgfpathlineto{\pgfqpoint{1.288240in}{2.387489in}}%
\pgfusepath{stroke}%
\end{pgfscope}%
\begin{pgfscope}%
\pgfpathrectangle{\pgfqpoint{0.644914in}{0.577483in}}{\pgfqpoint{3.100000in}{2.695000in}} %
\pgfusepath{clip}%
\pgfsetbuttcap%
\pgfsetroundjoin%
\pgfsetlinewidth{1.505625pt}%
\definecolor{currentstroke}{rgb}{0.121569,0.466667,0.705882}%
\pgfsetstrokecolor{currentstroke}%
\pgfsetdash{}{0pt}%
\pgfpathmoveto{\pgfqpoint{1.366347in}{2.389344in}}%
\pgfpathlineto{\pgfqpoint{1.366347in}{2.432737in}}%
\pgfusepath{stroke}%
\end{pgfscope}%
\begin{pgfscope}%
\pgfpathrectangle{\pgfqpoint{0.644914in}{0.577483in}}{\pgfqpoint{3.100000in}{2.695000in}} %
\pgfusepath{clip}%
\pgfsetbuttcap%
\pgfsetroundjoin%
\pgfsetlinewidth{1.505625pt}%
\definecolor{currentstroke}{rgb}{0.121569,0.466667,0.705882}%
\pgfsetstrokecolor{currentstroke}%
\pgfsetdash{}{0pt}%
\pgfpathmoveto{\pgfqpoint{1.527487in}{2.708649in}}%
\pgfpathlineto{\pgfqpoint{1.527487in}{2.754510in}}%
\pgfusepath{stroke}%
\end{pgfscope}%
\begin{pgfscope}%
\pgfpathrectangle{\pgfqpoint{0.644914in}{0.577483in}}{\pgfqpoint{3.100000in}{2.695000in}} %
\pgfusepath{clip}%
\pgfsetbuttcap%
\pgfsetroundjoin%
\pgfsetlinewidth{1.505625pt}%
\definecolor{currentstroke}{rgb}{0.121569,0.466667,0.705882}%
\pgfsetstrokecolor{currentstroke}%
\pgfsetdash{}{0pt}%
\pgfpathmoveto{\pgfqpoint{1.583076in}{2.691183in}}%
\pgfpathlineto{\pgfqpoint{1.583076in}{2.730302in}}%
\pgfusepath{stroke}%
\end{pgfscope}%
\begin{pgfscope}%
\pgfpathrectangle{\pgfqpoint{0.644914in}{0.577483in}}{\pgfqpoint{3.100000in}{2.695000in}} %
\pgfusepath{clip}%
\pgfsetbuttcap%
\pgfsetroundjoin%
\pgfsetlinewidth{1.505625pt}%
\definecolor{currentstroke}{rgb}{0.121569,0.466667,0.705882}%
\pgfsetstrokecolor{currentstroke}%
\pgfsetdash{}{0pt}%
\pgfpathmoveto{\pgfqpoint{1.620370in}{2.798500in}}%
\pgfpathlineto{\pgfqpoint{1.620370in}{2.854042in}}%
\pgfusepath{stroke}%
\end{pgfscope}%
\begin{pgfscope}%
\pgfpathrectangle{\pgfqpoint{0.644914in}{0.577483in}}{\pgfqpoint{3.100000in}{2.695000in}} %
\pgfusepath{clip}%
\pgfsetbuttcap%
\pgfsetroundjoin%
\pgfsetlinewidth{1.505625pt}%
\definecolor{currentstroke}{rgb}{0.121569,0.466667,0.705882}%
\pgfsetstrokecolor{currentstroke}%
\pgfsetdash{}{0pt}%
\pgfpathmoveto{\pgfqpoint{1.772362in}{2.997373in}}%
\pgfpathlineto{\pgfqpoint{1.772362in}{3.029636in}}%
\pgfusepath{stroke}%
\end{pgfscope}%
\begin{pgfscope}%
\pgfpathrectangle{\pgfqpoint{0.644914in}{0.577483in}}{\pgfqpoint{3.100000in}{2.695000in}} %
\pgfusepath{clip}%
\pgfsetbuttcap%
\pgfsetroundjoin%
\pgfsetlinewidth{1.505625pt}%
\definecolor{currentstroke}{rgb}{0.121569,0.466667,0.705882}%
\pgfsetstrokecolor{currentstroke}%
\pgfsetdash{}{0pt}%
\pgfpathmoveto{\pgfqpoint{1.367051in}{1.430063in}}%
\pgfpathlineto{\pgfqpoint{1.367051in}{1.468345in}}%
\pgfusepath{stroke}%
\end{pgfscope}%
\begin{pgfscope}%
\pgfpathrectangle{\pgfqpoint{0.644914in}{0.577483in}}{\pgfqpoint{3.100000in}{2.695000in}} %
\pgfusepath{clip}%
\pgfsetbuttcap%
\pgfsetroundjoin%
\pgfsetlinewidth{1.505625pt}%
\definecolor{currentstroke}{rgb}{0.121569,0.466667,0.705882}%
\pgfsetstrokecolor{currentstroke}%
\pgfsetdash{}{0pt}%
\pgfpathmoveto{\pgfqpoint{1.635147in}{1.630214in}}%
\pgfpathlineto{\pgfqpoint{1.635147in}{1.645197in}}%
\pgfusepath{stroke}%
\end{pgfscope}%
\begin{pgfscope}%
\pgfpathrectangle{\pgfqpoint{0.644914in}{0.577483in}}{\pgfqpoint{3.100000in}{2.695000in}} %
\pgfusepath{clip}%
\pgfsetbuttcap%
\pgfsetroundjoin%
\pgfsetlinewidth{1.505625pt}%
\definecolor{currentstroke}{rgb}{0.121569,0.466667,0.705882}%
\pgfsetstrokecolor{currentstroke}%
\pgfsetdash{}{0pt}%
\pgfpathmoveto{\pgfqpoint{1.913799in}{1.873041in}}%
\pgfpathlineto{\pgfqpoint{1.913799in}{1.933373in}}%
\pgfusepath{stroke}%
\end{pgfscope}%
\begin{pgfscope}%
\pgfpathrectangle{\pgfqpoint{0.644914in}{0.577483in}}{\pgfqpoint{3.100000in}{2.695000in}} %
\pgfusepath{clip}%
\pgfsetbuttcap%
\pgfsetroundjoin%
\pgfsetlinewidth{1.505625pt}%
\definecolor{currentstroke}{rgb}{0.121569,0.466667,0.705882}%
\pgfsetstrokecolor{currentstroke}%
\pgfsetdash{}{0pt}%
\pgfpathmoveto{\pgfqpoint{2.108715in}{2.090477in}}%
\pgfpathlineto{\pgfqpoint{2.108715in}{2.131488in}}%
\pgfusepath{stroke}%
\end{pgfscope}%
\begin{pgfscope}%
\pgfpathrectangle{\pgfqpoint{0.644914in}{0.577483in}}{\pgfqpoint{3.100000in}{2.695000in}} %
\pgfusepath{clip}%
\pgfsetbuttcap%
\pgfsetroundjoin%
\pgfsetlinewidth{1.505625pt}%
\definecolor{currentstroke}{rgb}{0.121569,0.466667,0.705882}%
\pgfsetstrokecolor{currentstroke}%
\pgfsetdash{}{0pt}%
\pgfpathmoveto{\pgfqpoint{2.373997in}{1.825537in}}%
\pgfpathlineto{\pgfqpoint{2.373997in}{2.000744in}}%
\pgfusepath{stroke}%
\end{pgfscope}%
\begin{pgfscope}%
\pgfpathrectangle{\pgfqpoint{0.644914in}{0.577483in}}{\pgfqpoint{3.100000in}{2.695000in}} %
\pgfusepath{clip}%
\pgfsetbuttcap%
\pgfsetroundjoin%
\pgfsetlinewidth{1.505625pt}%
\definecolor{currentstroke}{rgb}{0.121569,0.466667,0.705882}%
\pgfsetstrokecolor{currentstroke}%
\pgfsetdash{}{0pt}%
\pgfpathmoveto{\pgfqpoint{2.590726in}{1.881677in}}%
\pgfpathlineto{\pgfqpoint{2.590726in}{1.944499in}}%
\pgfusepath{stroke}%
\end{pgfscope}%
\begin{pgfscope}%
\pgfpathrectangle{\pgfqpoint{0.644914in}{0.577483in}}{\pgfqpoint{3.100000in}{2.695000in}} %
\pgfusepath{clip}%
\pgfsetbuttcap%
\pgfsetroundjoin%
\pgfsetlinewidth{1.505625pt}%
\definecolor{currentstroke}{rgb}{0.121569,0.466667,0.705882}%
\pgfsetstrokecolor{currentstroke}%
\pgfsetdash{}{0pt}%
\pgfpathmoveto{\pgfqpoint{2.886265in}{1.828698in}}%
\pgfpathlineto{\pgfqpoint{2.886265in}{1.921129in}}%
\pgfusepath{stroke}%
\end{pgfscope}%
\begin{pgfscope}%
\pgfpathrectangle{\pgfqpoint{0.644914in}{0.577483in}}{\pgfqpoint{3.100000in}{2.695000in}} %
\pgfusepath{clip}%
\pgfsetbuttcap%
\pgfsetroundjoin%
\pgfsetlinewidth{1.505625pt}%
\definecolor{currentstroke}{rgb}{0.121569,0.466667,0.705882}%
\pgfsetstrokecolor{currentstroke}%
\pgfsetdash{}{0pt}%
\pgfpathmoveto{\pgfqpoint{3.126919in}{1.781516in}}%
\pgfpathlineto{\pgfqpoint{3.126919in}{1.846266in}}%
\pgfusepath{stroke}%
\end{pgfscope}%
\begin{pgfscope}%
\pgfpathrectangle{\pgfqpoint{0.644914in}{0.577483in}}{\pgfqpoint{3.100000in}{2.695000in}} %
\pgfusepath{clip}%
\pgfsetbuttcap%
\pgfsetroundjoin%
\pgfsetlinewidth{1.505625pt}%
\definecolor{currentstroke}{rgb}{0.121569,0.466667,0.705882}%
\pgfsetstrokecolor{currentstroke}%
\pgfsetdash{}{0pt}%
\pgfpathmoveto{\pgfqpoint{3.345759in}{1.970942in}}%
\pgfpathlineto{\pgfqpoint{3.345759in}{1.996147in}}%
\pgfusepath{stroke}%
\end{pgfscope}%
\begin{pgfscope}%
\pgfpathrectangle{\pgfqpoint{0.644914in}{0.577483in}}{\pgfqpoint{3.100000in}{2.695000in}} %
\pgfusepath{clip}%
\pgfsetbuttcap%
\pgfsetroundjoin%
\pgfsetlinewidth{1.505625pt}%
\definecolor{currentstroke}{rgb}{0.121569,0.466667,0.705882}%
\pgfsetstrokecolor{currentstroke}%
\pgfsetdash{}{0pt}%
\pgfpathmoveto{\pgfqpoint{3.604005in}{2.115087in}}%
\pgfpathlineto{\pgfqpoint{3.604005in}{2.175689in}}%
\pgfusepath{stroke}%
\end{pgfscope}%
\begin{pgfscope}%
\pgfpathrectangle{\pgfqpoint{0.644914in}{0.577483in}}{\pgfqpoint{3.100000in}{2.695000in}} %
\pgfusepath{clip}%
\pgfsetbuttcap%
\pgfsetroundjoin%
\definecolor{currentfill}{rgb}{1.000000,0.498039,0.054902}%
\pgfsetfillcolor{currentfill}%
\pgfsetlinewidth{1.003750pt}%
\definecolor{currentstroke}{rgb}{1.000000,0.498039,0.054902}%
\pgfsetstrokecolor{currentstroke}%
\pgfsetdash{}{0pt}%
\pgfsys@defobject{currentmarker}{\pgfqpoint{-0.034722in}{-0.034722in}}{\pgfqpoint{0.034722in}{0.034722in}}{%
\pgfpathmoveto{\pgfqpoint{0.000000in}{-0.034722in}}%
\pgfpathcurveto{\pgfqpoint{0.009208in}{-0.034722in}}{\pgfqpoint{0.018041in}{-0.031064in}}{\pgfqpoint{0.024552in}{-0.024552in}}%
\pgfpathcurveto{\pgfqpoint{0.031064in}{-0.018041in}}{\pgfqpoint{0.034722in}{-0.009208in}}{\pgfqpoint{0.034722in}{0.000000in}}%
\pgfpathcurveto{\pgfqpoint{0.034722in}{0.009208in}}{\pgfqpoint{0.031064in}{0.018041in}}{\pgfqpoint{0.024552in}{0.024552in}}%
\pgfpathcurveto{\pgfqpoint{0.018041in}{0.031064in}}{\pgfqpoint{0.009208in}{0.034722in}}{\pgfqpoint{0.000000in}{0.034722in}}%
\pgfpathcurveto{\pgfqpoint{-0.009208in}{0.034722in}}{\pgfqpoint{-0.018041in}{0.031064in}}{\pgfqpoint{-0.024552in}{0.024552in}}%
\pgfpathcurveto{\pgfqpoint{-0.031064in}{0.018041in}}{\pgfqpoint{-0.034722in}{0.009208in}}{\pgfqpoint{-0.034722in}{0.000000in}}%
\pgfpathcurveto{\pgfqpoint{-0.034722in}{-0.009208in}}{\pgfqpoint{-0.031064in}{-0.018041in}}{\pgfqpoint{-0.024552in}{-0.024552in}}%
\pgfpathcurveto{\pgfqpoint{-0.018041in}{-0.031064in}}{\pgfqpoint{-0.009208in}{-0.034722in}}{\pgfqpoint{0.000000in}{-0.034722in}}%
\pgfpathclose%
\pgfusepath{stroke,fill}%
}%
\begin{pgfscope}%
\pgfsys@transformshift{0.785823in}{1.869637in}%
\pgfsys@useobject{currentmarker}{}%
\end{pgfscope}%
\begin{pgfscope}%
\pgfsys@transformshift{0.882929in}{2.103299in}%
\pgfsys@useobject{currentmarker}{}%
\end{pgfscope}%
\begin{pgfscope}%
\pgfsys@transformshift{1.049697in}{2.274688in}%
\pgfsys@useobject{currentmarker}{}%
\end{pgfscope}%
\begin{pgfscope}%
\pgfsys@transformshift{1.246724in}{2.363102in}%
\pgfsys@useobject{currentmarker}{}%
\end{pgfscope}%
\begin{pgfscope}%
\pgfsys@transformshift{1.288240in}{2.369391in}%
\pgfsys@useobject{currentmarker}{}%
\end{pgfscope}%
\begin{pgfscope}%
\pgfsys@transformshift{1.366347in}{2.309265in}%
\pgfsys@useobject{currentmarker}{}%
\end{pgfscope}%
\begin{pgfscope}%
\pgfsys@transformshift{1.527487in}{2.139618in}%
\pgfsys@useobject{currentmarker}{}%
\end{pgfscope}%
\begin{pgfscope}%
\pgfsys@transformshift{1.583076in}{2.050194in}%
\pgfsys@useobject{currentmarker}{}%
\end{pgfscope}%
\begin{pgfscope}%
\pgfsys@transformshift{1.620370in}{2.005958in}%
\pgfsys@useobject{currentmarker}{}%
\end{pgfscope}%
\begin{pgfscope}%
\pgfsys@transformshift{1.772362in}{1.741973in}%
\pgfsys@useobject{currentmarker}{}%
\end{pgfscope}%
\end{pgfscope}%
\begin{pgfscope}%
\pgfpathrectangle{\pgfqpoint{0.644914in}{0.577483in}}{\pgfqpoint{3.100000in}{2.695000in}} %
\pgfusepath{clip}%
\pgfsetrectcap%
\pgfsetroundjoin%
\pgfsetlinewidth{1.505625pt}%
\definecolor{currentstroke}{rgb}{1.000000,0.498039,0.054902}%
\pgfsetstrokecolor{currentstroke}%
\pgfsetdash{}{0pt}%
\pgfpathmoveto{\pgfqpoint{0.785823in}{1.875062in}}%
\pgfpathlineto{\pgfqpoint{0.805956in}{1.923716in}}%
\pgfpathlineto{\pgfqpoint{0.826090in}{1.969578in}}%
\pgfpathlineto{\pgfqpoint{0.846223in}{2.012689in}}%
\pgfpathlineto{\pgfqpoint{0.866357in}{2.053090in}}%
\pgfpathlineto{\pgfqpoint{0.886490in}{2.090822in}}%
\pgfpathlineto{\pgfqpoint{0.906623in}{2.125927in}}%
\pgfpathlineto{\pgfqpoint{0.926757in}{2.158447in}}%
\pgfpathlineto{\pgfqpoint{0.946890in}{2.188422in}}%
\pgfpathlineto{\pgfqpoint{0.967024in}{2.215894in}}%
\pgfpathlineto{\pgfqpoint{0.987157in}{2.240904in}}%
\pgfpathlineto{\pgfqpoint{1.007291in}{2.263494in}}%
\pgfpathlineto{\pgfqpoint{1.027424in}{2.283705in}}%
\pgfpathlineto{\pgfqpoint{1.047558in}{2.301578in}}%
\pgfpathlineto{\pgfqpoint{1.067691in}{2.317155in}}%
\pgfpathlineto{\pgfqpoint{1.087825in}{2.330477in}}%
\pgfpathlineto{\pgfqpoint{1.107958in}{2.341586in}}%
\pgfpathlineto{\pgfqpoint{1.128092in}{2.350522in}}%
\pgfpathlineto{\pgfqpoint{1.148225in}{2.357327in}}%
\pgfpathlineto{\pgfqpoint{1.168358in}{2.362042in}}%
\pgfpathlineto{\pgfqpoint{1.188492in}{2.364710in}}%
\pgfpathlineto{\pgfqpoint{1.208625in}{2.365370in}}%
\pgfpathlineto{\pgfqpoint{1.228759in}{2.364065in}}%
\pgfpathlineto{\pgfqpoint{1.248892in}{2.360836in}}%
\pgfpathlineto{\pgfqpoint{1.269026in}{2.355724in}}%
\pgfpathlineto{\pgfqpoint{1.289159in}{2.348771in}}%
\pgfpathlineto{\pgfqpoint{1.309293in}{2.340017in}}%
\pgfpathlineto{\pgfqpoint{1.329426in}{2.329505in}}%
\pgfpathlineto{\pgfqpoint{1.349560in}{2.317275in}}%
\pgfpathlineto{\pgfqpoint{1.369693in}{2.303369in}}%
\pgfpathlineto{\pgfqpoint{1.389827in}{2.287828in}}%
\pgfpathlineto{\pgfqpoint{1.409960in}{2.270694in}}%
\pgfpathlineto{\pgfqpoint{1.430093in}{2.252008in}}%
\pgfpathlineto{\pgfqpoint{1.450227in}{2.231811in}}%
\pgfpathlineto{\pgfqpoint{1.470360in}{2.210145in}}%
\pgfpathlineto{\pgfqpoint{1.490494in}{2.187051in}}%
\pgfpathlineto{\pgfqpoint{1.510627in}{2.162570in}}%
\pgfpathlineto{\pgfqpoint{1.530761in}{2.136744in}}%
\pgfpathlineto{\pgfqpoint{1.550894in}{2.109614in}}%
\pgfpathlineto{\pgfqpoint{1.571028in}{2.081222in}}%
\pgfpathlineto{\pgfqpoint{1.591161in}{2.051608in}}%
\pgfpathlineto{\pgfqpoint{1.611295in}{2.020814in}}%
\pgfpathlineto{\pgfqpoint{1.631428in}{1.988882in}}%
\pgfpathlineto{\pgfqpoint{1.651561in}{1.955852in}}%
\pgfpathlineto{\pgfqpoint{1.671695in}{1.921767in}}%
\pgfpathlineto{\pgfqpoint{1.691828in}{1.886667in}}%
\pgfpathlineto{\pgfqpoint{1.711962in}{1.850594in}}%
\pgfpathlineto{\pgfqpoint{1.732095in}{1.813590in}}%
\pgfpathlineto{\pgfqpoint{1.752229in}{1.775694in}}%
\pgfpathlineto{\pgfqpoint{1.772362in}{1.736950in}}%
\pgfusepath{stroke}%
\end{pgfscope}%
\begin{pgfscope}%
\pgfpathrectangle{\pgfqpoint{0.644914in}{0.577483in}}{\pgfqpoint{3.100000in}{2.695000in}} %
\pgfusepath{clip}%
\pgfsetbuttcap%
\pgfsetroundjoin%
\definecolor{currentfill}{rgb}{0.172549,0.627451,0.172549}%
\pgfsetfillcolor{currentfill}%
\pgfsetlinewidth{1.003750pt}%
\definecolor{currentstroke}{rgb}{0.172549,0.627451,0.172549}%
\pgfsetstrokecolor{currentstroke}%
\pgfsetdash{}{0pt}%
\pgfsys@defobject{currentmarker}{\pgfqpoint{-0.034722in}{-0.034722in}}{\pgfqpoint{0.034722in}{0.034722in}}{%
\pgfpathmoveto{\pgfqpoint{0.000000in}{-0.034722in}}%
\pgfpathcurveto{\pgfqpoint{0.009208in}{-0.034722in}}{\pgfqpoint{0.018041in}{-0.031064in}}{\pgfqpoint{0.024552in}{-0.024552in}}%
\pgfpathcurveto{\pgfqpoint{0.031064in}{-0.018041in}}{\pgfqpoint{0.034722in}{-0.009208in}}{\pgfqpoint{0.034722in}{0.000000in}}%
\pgfpathcurveto{\pgfqpoint{0.034722in}{0.009208in}}{\pgfqpoint{0.031064in}{0.018041in}}{\pgfqpoint{0.024552in}{0.024552in}}%
\pgfpathcurveto{\pgfqpoint{0.018041in}{0.031064in}}{\pgfqpoint{0.009208in}{0.034722in}}{\pgfqpoint{0.000000in}{0.034722in}}%
\pgfpathcurveto{\pgfqpoint{-0.009208in}{0.034722in}}{\pgfqpoint{-0.018041in}{0.031064in}}{\pgfqpoint{-0.024552in}{0.024552in}}%
\pgfpathcurveto{\pgfqpoint{-0.031064in}{0.018041in}}{\pgfqpoint{-0.034722in}{0.009208in}}{\pgfqpoint{-0.034722in}{0.000000in}}%
\pgfpathcurveto{\pgfqpoint{-0.034722in}{-0.009208in}}{\pgfqpoint{-0.031064in}{-0.018041in}}{\pgfqpoint{-0.024552in}{-0.024552in}}%
\pgfpathcurveto{\pgfqpoint{-0.018041in}{-0.031064in}}{\pgfqpoint{-0.009208in}{-0.034722in}}{\pgfqpoint{0.000000in}{-0.034722in}}%
\pgfpathclose%
\pgfusepath{stroke,fill}%
}%
\begin{pgfscope}%
\pgfsys@transformshift{0.785823in}{1.185619in}%
\pgfsys@useobject{currentmarker}{}%
\end{pgfscope}%
\begin{pgfscope}%
\pgfsys@transformshift{0.882929in}{1.245698in}%
\pgfsys@useobject{currentmarker}{}%
\end{pgfscope}%
\begin{pgfscope}%
\pgfsys@transformshift{1.049697in}{1.240567in}%
\pgfsys@useobject{currentmarker}{}%
\end{pgfscope}%
\begin{pgfscope}%
\pgfsys@transformshift{1.246724in}{1.177108in}%
\pgfsys@useobject{currentmarker}{}%
\end{pgfscope}%
\begin{pgfscope}%
\pgfsys@transformshift{1.288240in}{1.152609in}%
\pgfsys@useobject{currentmarker}{}%
\end{pgfscope}%
\begin{pgfscope}%
\pgfsys@transformshift{1.366347in}{1.082841in}%
\pgfsys@useobject{currentmarker}{}%
\end{pgfscope}%
\begin{pgfscope}%
\pgfsys@transformshift{1.527487in}{0.941083in}%
\pgfsys@useobject{currentmarker}{}%
\end{pgfscope}%
\begin{pgfscope}%
\pgfsys@transformshift{1.583076in}{0.928343in}%
\pgfsys@useobject{currentmarker}{}%
\end{pgfscope}%
\begin{pgfscope}%
\pgfsys@transformshift{1.620370in}{1.076747in}%
\pgfsys@useobject{currentmarker}{}%
\end{pgfscope}%
\begin{pgfscope}%
\pgfsys@transformshift{1.772362in}{0.787467in}%
\pgfsys@useobject{currentmarker}{}%
\end{pgfscope}%
\end{pgfscope}%
\begin{pgfscope}%
\pgfpathrectangle{\pgfqpoint{0.644914in}{0.577483in}}{\pgfqpoint{3.100000in}{2.695000in}} %
\pgfusepath{clip}%
\pgfsetrectcap%
\pgfsetroundjoin%
\pgfsetlinewidth{1.505625pt}%
\definecolor{currentstroke}{rgb}{0.172549,0.627451,0.172549}%
\pgfsetstrokecolor{currentstroke}%
\pgfsetdash{}{0pt}%
\pgfpathmoveto{\pgfqpoint{0.785823in}{1.200211in}}%
\pgfpathlineto{\pgfqpoint{0.805956in}{1.208022in}}%
\pgfpathlineto{\pgfqpoint{0.826090in}{1.214854in}}%
\pgfpathlineto{\pgfqpoint{0.846223in}{1.220728in}}%
\pgfpathlineto{\pgfqpoint{0.866357in}{1.225664in}}%
\pgfpathlineto{\pgfqpoint{0.886490in}{1.229686in}}%
\pgfpathlineto{\pgfqpoint{0.906623in}{1.232812in}}%
\pgfpathlineto{\pgfqpoint{0.926757in}{1.235065in}}%
\pgfpathlineto{\pgfqpoint{0.946890in}{1.236466in}}%
\pgfpathlineto{\pgfqpoint{0.967024in}{1.237037in}}%
\pgfpathlineto{\pgfqpoint{0.987157in}{1.236797in}}%
\pgfpathlineto{\pgfqpoint{1.007291in}{1.235770in}}%
\pgfpathlineto{\pgfqpoint{1.027424in}{1.233975in}}%
\pgfpathlineto{\pgfqpoint{1.047558in}{1.231434in}}%
\pgfpathlineto{\pgfqpoint{1.067691in}{1.228168in}}%
\pgfpathlineto{\pgfqpoint{1.087825in}{1.224199in}}%
\pgfpathlineto{\pgfqpoint{1.107958in}{1.219548in}}%
\pgfpathlineto{\pgfqpoint{1.128092in}{1.214235in}}%
\pgfpathlineto{\pgfqpoint{1.148225in}{1.208283in}}%
\pgfpathlineto{\pgfqpoint{1.168358in}{1.201712in}}%
\pgfpathlineto{\pgfqpoint{1.188492in}{1.194544in}}%
\pgfpathlineto{\pgfqpoint{1.208625in}{1.186799in}}%
\pgfpathlineto{\pgfqpoint{1.228759in}{1.178500in}}%
\pgfpathlineto{\pgfqpoint{1.248892in}{1.169667in}}%
\pgfpathlineto{\pgfqpoint{1.269026in}{1.160321in}}%
\pgfpathlineto{\pgfqpoint{1.289159in}{1.150484in}}%
\pgfpathlineto{\pgfqpoint{1.309293in}{1.140177in}}%
\pgfpathlineto{\pgfqpoint{1.329426in}{1.129421in}}%
\pgfpathlineto{\pgfqpoint{1.349560in}{1.118237in}}%
\pgfpathlineto{\pgfqpoint{1.369693in}{1.106648in}}%
\pgfpathlineto{\pgfqpoint{1.389827in}{1.094672in}}%
\pgfpathlineto{\pgfqpoint{1.409960in}{1.082333in}}%
\pgfpathlineto{\pgfqpoint{1.430093in}{1.069652in}}%
\pgfpathlineto{\pgfqpoint{1.450227in}{1.056648in}}%
\pgfpathlineto{\pgfqpoint{1.470360in}{1.043345in}}%
\pgfpathlineto{\pgfqpoint{1.490494in}{1.029762in}}%
\pgfpathlineto{\pgfqpoint{1.510627in}{1.015922in}}%
\pgfpathlineto{\pgfqpoint{1.530761in}{1.001844in}}%
\pgfpathlineto{\pgfqpoint{1.550894in}{0.987552in}}%
\pgfpathlineto{\pgfqpoint{1.571028in}{0.973065in}}%
\pgfpathlineto{\pgfqpoint{1.591161in}{0.958406in}}%
\pgfpathlineto{\pgfqpoint{1.611295in}{0.943594in}}%
\pgfpathlineto{\pgfqpoint{1.631428in}{0.928652in}}%
\pgfpathlineto{\pgfqpoint{1.651561in}{0.913601in}}%
\pgfpathlineto{\pgfqpoint{1.671695in}{0.898462in}}%
\pgfpathlineto{\pgfqpoint{1.691828in}{0.883256in}}%
\pgfpathlineto{\pgfqpoint{1.711962in}{0.868004in}}%
\pgfpathlineto{\pgfqpoint{1.732095in}{0.852728in}}%
\pgfpathlineto{\pgfqpoint{1.752229in}{0.837449in}}%
\pgfpathlineto{\pgfqpoint{1.772362in}{0.822188in}}%
\pgfusepath{stroke}%
\end{pgfscope}%
\begin{pgfscope}%
\pgfpathrectangle{\pgfqpoint{0.644914in}{0.577483in}}{\pgfqpoint{3.100000in}{2.695000in}} %
\pgfusepath{clip}%
\pgfsetbuttcap%
\pgfsetroundjoin%
\pgfsetlinewidth{1.003750pt}%
\definecolor{currentstroke}{rgb}{1.000000,0.498039,0.054902}%
\pgfsetstrokecolor{currentstroke}%
\pgfsetdash{}{0pt}%
\pgfsys@defobject{currentmarker}{\pgfqpoint{-0.034722in}{-0.034722in}}{\pgfqpoint{0.034722in}{0.034722in}}{%
\pgfpathmoveto{\pgfqpoint{0.000000in}{-0.034722in}}%
\pgfpathcurveto{\pgfqpoint{0.009208in}{-0.034722in}}{\pgfqpoint{0.018041in}{-0.031064in}}{\pgfqpoint{0.024552in}{-0.024552in}}%
\pgfpathcurveto{\pgfqpoint{0.031064in}{-0.018041in}}{\pgfqpoint{0.034722in}{-0.009208in}}{\pgfqpoint{0.034722in}{0.000000in}}%
\pgfpathcurveto{\pgfqpoint{0.034722in}{0.009208in}}{\pgfqpoint{0.031064in}{0.018041in}}{\pgfqpoint{0.024552in}{0.024552in}}%
\pgfpathcurveto{\pgfqpoint{0.018041in}{0.031064in}}{\pgfqpoint{0.009208in}{0.034722in}}{\pgfqpoint{0.000000in}{0.034722in}}%
\pgfpathcurveto{\pgfqpoint{-0.009208in}{0.034722in}}{\pgfqpoint{-0.018041in}{0.031064in}}{\pgfqpoint{-0.024552in}{0.024552in}}%
\pgfpathcurveto{\pgfqpoint{-0.031064in}{0.018041in}}{\pgfqpoint{-0.034722in}{0.009208in}}{\pgfqpoint{-0.034722in}{0.000000in}}%
\pgfpathcurveto{\pgfqpoint{-0.034722in}{-0.009208in}}{\pgfqpoint{-0.031064in}{-0.018041in}}{\pgfqpoint{-0.024552in}{-0.024552in}}%
\pgfpathcurveto{\pgfqpoint{-0.018041in}{-0.031064in}}{\pgfqpoint{-0.009208in}{-0.034722in}}{\pgfqpoint{0.000000in}{-0.034722in}}%
\pgfpathclose%
\pgfusepath{stroke}%
}%
\begin{pgfscope}%
\pgfsys@transformshift{2.590726in}{0.941627in}%
\pgfsys@useobject{currentmarker}{}%
\end{pgfscope}%
\begin{pgfscope}%
\pgfsys@transformshift{2.886265in}{1.422246in}%
\pgfsys@useobject{currentmarker}{}%
\end{pgfscope}%
\begin{pgfscope}%
\pgfsys@transformshift{3.126919in}{1.659681in}%
\pgfsys@useobject{currentmarker}{}%
\end{pgfscope}%
\begin{pgfscope}%
\pgfsys@transformshift{3.345759in}{2.010057in}%
\pgfsys@useobject{currentmarker}{}%
\end{pgfscope}%
\begin{pgfscope}%
\pgfsys@transformshift{3.604005in}{2.297600in}%
\pgfsys@useobject{currentmarker}{}%
\end{pgfscope}%
\end{pgfscope}%
\begin{pgfscope}%
\pgfpathrectangle{\pgfqpoint{0.644914in}{0.577483in}}{\pgfqpoint{3.100000in}{2.695000in}} %
\pgfusepath{clip}%
\pgfsetbuttcap%
\pgfsetroundjoin%
\pgfsetlinewidth{1.505625pt}%
\definecolor{currentstroke}{rgb}{1.000000,0.498039,0.054902}%
\pgfsetstrokecolor{currentstroke}%
\pgfsetdash{{5.600000pt}{2.400000pt}}{0.000000pt}%
\pgfpathmoveto{\pgfqpoint{2.590726in}{0.946966in}}%
\pgfpathlineto{\pgfqpoint{2.611405in}{0.981512in}}%
\pgfpathlineto{\pgfqpoint{2.632084in}{1.015503in}}%
\pgfpathlineto{\pgfqpoint{2.652763in}{1.048956in}}%
\pgfpathlineto{\pgfqpoint{2.673442in}{1.081888in}}%
\pgfpathlineto{\pgfqpoint{2.694121in}{1.114316in}}%
\pgfpathlineto{\pgfqpoint{2.714801in}{1.146257in}}%
\pgfpathlineto{\pgfqpoint{2.735480in}{1.177730in}}%
\pgfpathlineto{\pgfqpoint{2.756159in}{1.208750in}}%
\pgfpathlineto{\pgfqpoint{2.776838in}{1.239336in}}%
\pgfpathlineto{\pgfqpoint{2.797517in}{1.269505in}}%
\pgfpathlineto{\pgfqpoint{2.818196in}{1.299273in}}%
\pgfpathlineto{\pgfqpoint{2.838876in}{1.328659in}}%
\pgfpathlineto{\pgfqpoint{2.859555in}{1.357679in}}%
\pgfpathlineto{\pgfqpoint{2.880234in}{1.386352in}}%
\pgfpathlineto{\pgfqpoint{2.900913in}{1.414693in}}%
\pgfpathlineto{\pgfqpoint{2.921592in}{1.442721in}}%
\pgfpathlineto{\pgfqpoint{2.942271in}{1.470452in}}%
\pgfpathlineto{\pgfqpoint{2.962951in}{1.497904in}}%
\pgfpathlineto{\pgfqpoint{2.983630in}{1.525095in}}%
\pgfpathlineto{\pgfqpoint{3.004309in}{1.552041in}}%
\pgfpathlineto{\pgfqpoint{3.024988in}{1.578760in}}%
\pgfpathlineto{\pgfqpoint{3.045667in}{1.605269in}}%
\pgfpathlineto{\pgfqpoint{3.066346in}{1.631585in}}%
\pgfpathlineto{\pgfqpoint{3.087026in}{1.657727in}}%
\pgfpathlineto{\pgfqpoint{3.107705in}{1.683710in}}%
\pgfpathlineto{\pgfqpoint{3.128384in}{1.709552in}}%
\pgfpathlineto{\pgfqpoint{3.149063in}{1.735271in}}%
\pgfpathlineto{\pgfqpoint{3.169742in}{1.760883in}}%
\pgfpathlineto{\pgfqpoint{3.190421in}{1.786407in}}%
\pgfpathlineto{\pgfqpoint{3.211100in}{1.811859in}}%
\pgfpathlineto{\pgfqpoint{3.231780in}{1.837256in}}%
\pgfpathlineto{\pgfqpoint{3.252459in}{1.862617in}}%
\pgfpathlineto{\pgfqpoint{3.273138in}{1.887958in}}%
\pgfpathlineto{\pgfqpoint{3.293817in}{1.913296in}}%
\pgfpathlineto{\pgfqpoint{3.314496in}{1.938649in}}%
\pgfpathlineto{\pgfqpoint{3.335175in}{1.964034in}}%
\pgfpathlineto{\pgfqpoint{3.355855in}{1.989469in}}%
\pgfpathlineto{\pgfqpoint{3.376534in}{2.014970in}}%
\pgfpathlineto{\pgfqpoint{3.397213in}{2.040555in}}%
\pgfpathlineto{\pgfqpoint{3.417892in}{2.066241in}}%
\pgfpathlineto{\pgfqpoint{3.438571in}{2.092046in}}%
\pgfpathlineto{\pgfqpoint{3.459250in}{2.117987in}}%
\pgfpathlineto{\pgfqpoint{3.479930in}{2.144080in}}%
\pgfpathlineto{\pgfqpoint{3.500609in}{2.170344in}}%
\pgfpathlineto{\pgfqpoint{3.521288in}{2.196796in}}%
\pgfpathlineto{\pgfqpoint{3.541967in}{2.223452in}}%
\pgfpathlineto{\pgfqpoint{3.562646in}{2.250331in}}%
\pgfpathlineto{\pgfqpoint{3.583325in}{2.277449in}}%
\pgfpathlineto{\pgfqpoint{3.604005in}{2.304824in}}%
\pgfusepath{stroke}%
\end{pgfscope}%
\begin{pgfscope}%
\pgfpathrectangle{\pgfqpoint{0.644914in}{0.577483in}}{\pgfqpoint{3.100000in}{2.695000in}} %
\pgfusepath{clip}%
\pgfsetbuttcap%
\pgfsetroundjoin%
\pgfsetlinewidth{1.003750pt}%
\definecolor{currentstroke}{rgb}{0.172549,0.627451,0.172549}%
\pgfsetstrokecolor{currentstroke}%
\pgfsetdash{}{0pt}%
\pgfsys@defobject{currentmarker}{\pgfqpoint{-0.034722in}{-0.034722in}}{\pgfqpoint{0.034722in}{0.034722in}}{%
\pgfpathmoveto{\pgfqpoint{0.000000in}{-0.034722in}}%
\pgfpathcurveto{\pgfqpoint{0.009208in}{-0.034722in}}{\pgfqpoint{0.018041in}{-0.031064in}}{\pgfqpoint{0.024552in}{-0.024552in}}%
\pgfpathcurveto{\pgfqpoint{0.031064in}{-0.018041in}}{\pgfqpoint{0.034722in}{-0.009208in}}{\pgfqpoint{0.034722in}{0.000000in}}%
\pgfpathcurveto{\pgfqpoint{0.034722in}{0.009208in}}{\pgfqpoint{0.031064in}{0.018041in}}{\pgfqpoint{0.024552in}{0.024552in}}%
\pgfpathcurveto{\pgfqpoint{0.018041in}{0.031064in}}{\pgfqpoint{0.009208in}{0.034722in}}{\pgfqpoint{0.000000in}{0.034722in}}%
\pgfpathcurveto{\pgfqpoint{-0.009208in}{0.034722in}}{\pgfqpoint{-0.018041in}{0.031064in}}{\pgfqpoint{-0.024552in}{0.024552in}}%
\pgfpathcurveto{\pgfqpoint{-0.031064in}{0.018041in}}{\pgfqpoint{-0.034722in}{0.009208in}}{\pgfqpoint{-0.034722in}{0.000000in}}%
\pgfpathcurveto{\pgfqpoint{-0.034722in}{-0.009208in}}{\pgfqpoint{-0.031064in}{-0.018041in}}{\pgfqpoint{-0.024552in}{-0.024552in}}%
\pgfpathcurveto{\pgfqpoint{-0.018041in}{-0.031064in}}{\pgfqpoint{-0.009208in}{-0.034722in}}{\pgfqpoint{0.000000in}{-0.034722in}}%
\pgfpathclose%
\pgfusepath{stroke}%
}%
\begin{pgfscope}%
\pgfsys@transformshift{2.590726in}{0.665906in}%
\pgfsys@useobject{currentmarker}{}%
\end{pgfscope}%
\begin{pgfscope}%
\pgfsys@transformshift{2.886265in}{1.234885in}%
\pgfsys@useobject{currentmarker}{}%
\end{pgfscope}%
\begin{pgfscope}%
\pgfsys@transformshift{3.126919in}{1.482780in}%
\pgfsys@useobject{currentmarker}{}%
\end{pgfscope}%
\begin{pgfscope}%
\pgfsys@transformshift{3.345759in}{1.883826in}%
\pgfsys@useobject{currentmarker}{}%
\end{pgfscope}%
\begin{pgfscope}%
\pgfsys@transformshift{3.604005in}{2.199908in}%
\pgfsys@useobject{currentmarker}{}%
\end{pgfscope}%
\end{pgfscope}%
\begin{pgfscope}%
\pgfpathrectangle{\pgfqpoint{0.644914in}{0.577483in}}{\pgfqpoint{3.100000in}{2.695000in}} %
\pgfusepath{clip}%
\pgfsetbuttcap%
\pgfsetroundjoin%
\pgfsetlinewidth{1.505625pt}%
\definecolor{currentstroke}{rgb}{0.172549,0.627451,0.172549}%
\pgfsetstrokecolor{currentstroke}%
\pgfsetdash{{5.600000pt}{2.400000pt}}{0.000000pt}%
\pgfpathmoveto{\pgfqpoint{2.590726in}{0.672931in}}%
\pgfpathlineto{\pgfqpoint{2.611405in}{0.714853in}}%
\pgfpathlineto{\pgfqpoint{2.632084in}{0.755883in}}%
\pgfpathlineto{\pgfqpoint{2.652763in}{0.796050in}}%
\pgfpathlineto{\pgfqpoint{2.673442in}{0.835382in}}%
\pgfpathlineto{\pgfqpoint{2.694121in}{0.873910in}}%
\pgfpathlineto{\pgfqpoint{2.714801in}{0.911660in}}%
\pgfpathlineto{\pgfqpoint{2.735480in}{0.948663in}}%
\pgfpathlineto{\pgfqpoint{2.756159in}{0.984948in}}%
\pgfpathlineto{\pgfqpoint{2.776838in}{1.020542in}}%
\pgfpathlineto{\pgfqpoint{2.797517in}{1.055475in}}%
\pgfpathlineto{\pgfqpoint{2.818196in}{1.089775in}}%
\pgfpathlineto{\pgfqpoint{2.838876in}{1.123472in}}%
\pgfpathlineto{\pgfqpoint{2.859555in}{1.156594in}}%
\pgfpathlineto{\pgfqpoint{2.880234in}{1.189170in}}%
\pgfpathlineto{\pgfqpoint{2.900913in}{1.221230in}}%
\pgfpathlineto{\pgfqpoint{2.921592in}{1.252801in}}%
\pgfpathlineto{\pgfqpoint{2.942271in}{1.283912in}}%
\pgfpathlineto{\pgfqpoint{2.962951in}{1.314593in}}%
\pgfpathlineto{\pgfqpoint{2.983630in}{1.344872in}}%
\pgfpathlineto{\pgfqpoint{3.004309in}{1.374779in}}%
\pgfpathlineto{\pgfqpoint{3.024988in}{1.404341in}}%
\pgfpathlineto{\pgfqpoint{3.045667in}{1.433587in}}%
\pgfpathlineto{\pgfqpoint{3.066346in}{1.462548in}}%
\pgfpathlineto{\pgfqpoint{3.087026in}{1.491251in}}%
\pgfpathlineto{\pgfqpoint{3.107705in}{1.519725in}}%
\pgfpathlineto{\pgfqpoint{3.128384in}{1.547999in}}%
\pgfpathlineto{\pgfqpoint{3.149063in}{1.576101in}}%
\pgfpathlineto{\pgfqpoint{3.169742in}{1.604062in}}%
\pgfpathlineto{\pgfqpoint{3.190421in}{1.631909in}}%
\pgfpathlineto{\pgfqpoint{3.211100in}{1.659671in}}%
\pgfpathlineto{\pgfqpoint{3.231780in}{1.687378in}}%
\pgfpathlineto{\pgfqpoint{3.252459in}{1.715058in}}%
\pgfpathlineto{\pgfqpoint{3.273138in}{1.742739in}}%
\pgfpathlineto{\pgfqpoint{3.293817in}{1.770451in}}%
\pgfpathlineto{\pgfqpoint{3.314496in}{1.798223in}}%
\pgfpathlineto{\pgfqpoint{3.335175in}{1.826083in}}%
\pgfpathlineto{\pgfqpoint{3.355855in}{1.854060in}}%
\pgfpathlineto{\pgfqpoint{3.376534in}{1.882182in}}%
\pgfpathlineto{\pgfqpoint{3.397213in}{1.910480in}}%
\pgfpathlineto{\pgfqpoint{3.417892in}{1.938982in}}%
\pgfpathlineto{\pgfqpoint{3.438571in}{1.967715in}}%
\pgfpathlineto{\pgfqpoint{3.459250in}{1.996711in}}%
\pgfpathlineto{\pgfqpoint{3.479930in}{2.025996in}}%
\pgfpathlineto{\pgfqpoint{3.500609in}{2.055600in}}%
\pgfpathlineto{\pgfqpoint{3.521288in}{2.085552in}}%
\pgfpathlineto{\pgfqpoint{3.541967in}{2.115881in}}%
\pgfpathlineto{\pgfqpoint{3.562646in}{2.146615in}}%
\pgfpathlineto{\pgfqpoint{3.583325in}{2.177783in}}%
\pgfpathlineto{\pgfqpoint{3.604005in}{2.209414in}}%
\pgfusepath{stroke}%
\end{pgfscope}%
\begin{pgfscope}%
\pgfpathrectangle{\pgfqpoint{0.644914in}{0.577483in}}{\pgfqpoint{3.100000in}{2.695000in}} %
\pgfusepath{clip}%
\pgfsetbuttcap%
\pgfsetroundjoin%
\definecolor{currentfill}{rgb}{1.000000,0.498039,0.054902}%
\pgfsetfillcolor{currentfill}%
\pgfsetlinewidth{1.003750pt}%
\definecolor{currentstroke}{rgb}{1.000000,0.498039,0.054902}%
\pgfsetstrokecolor{currentstroke}%
\pgfsetdash{}{0pt}%
\pgfsys@defobject{currentmarker}{\pgfqpoint{-0.034722in}{-0.034722in}}{\pgfqpoint{0.034722in}{0.034722in}}{%
\pgfpathmoveto{\pgfqpoint{0.000000in}{-0.034722in}}%
\pgfpathcurveto{\pgfqpoint{0.009208in}{-0.034722in}}{\pgfqpoint{0.018041in}{-0.031064in}}{\pgfqpoint{0.024552in}{-0.024552in}}%
\pgfpathcurveto{\pgfqpoint{0.031064in}{-0.018041in}}{\pgfqpoint{0.034722in}{-0.009208in}}{\pgfqpoint{0.034722in}{0.000000in}}%
\pgfpathcurveto{\pgfqpoint{0.034722in}{0.009208in}}{\pgfqpoint{0.031064in}{0.018041in}}{\pgfqpoint{0.024552in}{0.024552in}}%
\pgfpathcurveto{\pgfqpoint{0.018041in}{0.031064in}}{\pgfqpoint{0.009208in}{0.034722in}}{\pgfqpoint{0.000000in}{0.034722in}}%
\pgfpathcurveto{\pgfqpoint{-0.009208in}{0.034722in}}{\pgfqpoint{-0.018041in}{0.031064in}}{\pgfqpoint{-0.024552in}{0.024552in}}%
\pgfpathcurveto{\pgfqpoint{-0.031064in}{0.018041in}}{\pgfqpoint{-0.034722in}{0.009208in}}{\pgfqpoint{-0.034722in}{0.000000in}}%
\pgfpathcurveto{\pgfqpoint{-0.034722in}{-0.009208in}}{\pgfqpoint{-0.031064in}{-0.018041in}}{\pgfqpoint{-0.024552in}{-0.024552in}}%
\pgfpathcurveto{\pgfqpoint{-0.018041in}{-0.031064in}}{\pgfqpoint{-0.009208in}{-0.034722in}}{\pgfqpoint{0.000000in}{-0.034722in}}%
\pgfpathclose%
\pgfusepath{stroke,fill}%
}%
\begin{pgfscope}%
\pgfsys@transformshift{0.785823in}{1.869637in}%
\pgfsys@useobject{currentmarker}{}%
\end{pgfscope}%
\begin{pgfscope}%
\pgfsys@transformshift{0.882929in}{2.103299in}%
\pgfsys@useobject{currentmarker}{}%
\end{pgfscope}%
\begin{pgfscope}%
\pgfsys@transformshift{1.049697in}{2.274688in}%
\pgfsys@useobject{currentmarker}{}%
\end{pgfscope}%
\begin{pgfscope}%
\pgfsys@transformshift{1.246724in}{2.363102in}%
\pgfsys@useobject{currentmarker}{}%
\end{pgfscope}%
\begin{pgfscope}%
\pgfsys@transformshift{1.288240in}{2.369391in}%
\pgfsys@useobject{currentmarker}{}%
\end{pgfscope}%
\begin{pgfscope}%
\pgfsys@transformshift{1.366347in}{2.309265in}%
\pgfsys@useobject{currentmarker}{}%
\end{pgfscope}%
\begin{pgfscope}%
\pgfsys@transformshift{1.527487in}{2.139618in}%
\pgfsys@useobject{currentmarker}{}%
\end{pgfscope}%
\begin{pgfscope}%
\pgfsys@transformshift{1.583076in}{2.050194in}%
\pgfsys@useobject{currentmarker}{}%
\end{pgfscope}%
\begin{pgfscope}%
\pgfsys@transformshift{1.620370in}{2.005958in}%
\pgfsys@useobject{currentmarker}{}%
\end{pgfscope}%
\begin{pgfscope}%
\pgfsys@transformshift{1.772362in}{1.741973in}%
\pgfsys@useobject{currentmarker}{}%
\end{pgfscope}%
\end{pgfscope}%
\begin{pgfscope}%
\pgfpathrectangle{\pgfqpoint{0.644914in}{0.577483in}}{\pgfqpoint{3.100000in}{2.695000in}} %
\pgfusepath{clip}%
\pgfsetrectcap%
\pgfsetroundjoin%
\pgfsetlinewidth{1.505625pt}%
\definecolor{currentstroke}{rgb}{1.000000,0.498039,0.054902}%
\pgfsetstrokecolor{currentstroke}%
\pgfsetdash{}{0pt}%
\pgfpathmoveto{\pgfqpoint{0.785823in}{1.875062in}}%
\pgfpathlineto{\pgfqpoint{0.805956in}{1.923716in}}%
\pgfpathlineto{\pgfqpoint{0.826090in}{1.969578in}}%
\pgfpathlineto{\pgfqpoint{0.846223in}{2.012689in}}%
\pgfpathlineto{\pgfqpoint{0.866357in}{2.053090in}}%
\pgfpathlineto{\pgfqpoint{0.886490in}{2.090822in}}%
\pgfpathlineto{\pgfqpoint{0.906623in}{2.125927in}}%
\pgfpathlineto{\pgfqpoint{0.926757in}{2.158447in}}%
\pgfpathlineto{\pgfqpoint{0.946890in}{2.188422in}}%
\pgfpathlineto{\pgfqpoint{0.967024in}{2.215894in}}%
\pgfpathlineto{\pgfqpoint{0.987157in}{2.240904in}}%
\pgfpathlineto{\pgfqpoint{1.007291in}{2.263494in}}%
\pgfpathlineto{\pgfqpoint{1.027424in}{2.283705in}}%
\pgfpathlineto{\pgfqpoint{1.047558in}{2.301578in}}%
\pgfpathlineto{\pgfqpoint{1.067691in}{2.317155in}}%
\pgfpathlineto{\pgfqpoint{1.087825in}{2.330477in}}%
\pgfpathlineto{\pgfqpoint{1.107958in}{2.341586in}}%
\pgfpathlineto{\pgfqpoint{1.128092in}{2.350522in}}%
\pgfpathlineto{\pgfqpoint{1.148225in}{2.357327in}}%
\pgfpathlineto{\pgfqpoint{1.168358in}{2.362042in}}%
\pgfpathlineto{\pgfqpoint{1.188492in}{2.364710in}}%
\pgfpathlineto{\pgfqpoint{1.208625in}{2.365370in}}%
\pgfpathlineto{\pgfqpoint{1.228759in}{2.364065in}}%
\pgfpathlineto{\pgfqpoint{1.248892in}{2.360836in}}%
\pgfpathlineto{\pgfqpoint{1.269026in}{2.355724in}}%
\pgfpathlineto{\pgfqpoint{1.289159in}{2.348771in}}%
\pgfpathlineto{\pgfqpoint{1.309293in}{2.340017in}}%
\pgfpathlineto{\pgfqpoint{1.329426in}{2.329505in}}%
\pgfpathlineto{\pgfqpoint{1.349560in}{2.317275in}}%
\pgfpathlineto{\pgfqpoint{1.369693in}{2.303369in}}%
\pgfpathlineto{\pgfqpoint{1.389827in}{2.287828in}}%
\pgfpathlineto{\pgfqpoint{1.409960in}{2.270694in}}%
\pgfpathlineto{\pgfqpoint{1.430093in}{2.252008in}}%
\pgfpathlineto{\pgfqpoint{1.450227in}{2.231811in}}%
\pgfpathlineto{\pgfqpoint{1.470360in}{2.210145in}}%
\pgfpathlineto{\pgfqpoint{1.490494in}{2.187051in}}%
\pgfpathlineto{\pgfqpoint{1.510627in}{2.162570in}}%
\pgfpathlineto{\pgfqpoint{1.530761in}{2.136744in}}%
\pgfpathlineto{\pgfqpoint{1.550894in}{2.109614in}}%
\pgfpathlineto{\pgfqpoint{1.571028in}{2.081222in}}%
\pgfpathlineto{\pgfqpoint{1.591161in}{2.051608in}}%
\pgfpathlineto{\pgfqpoint{1.611295in}{2.020814in}}%
\pgfpathlineto{\pgfqpoint{1.631428in}{1.988882in}}%
\pgfpathlineto{\pgfqpoint{1.651561in}{1.955852in}}%
\pgfpathlineto{\pgfqpoint{1.671695in}{1.921767in}}%
\pgfpathlineto{\pgfqpoint{1.691828in}{1.886667in}}%
\pgfpathlineto{\pgfqpoint{1.711962in}{1.850594in}}%
\pgfpathlineto{\pgfqpoint{1.732095in}{1.813590in}}%
\pgfpathlineto{\pgfqpoint{1.752229in}{1.775694in}}%
\pgfpathlineto{\pgfqpoint{1.772362in}{1.736950in}}%
\pgfusepath{stroke}%
\end{pgfscope}%
\begin{pgfscope}%
\pgfpathrectangle{\pgfqpoint{0.644914in}{0.577483in}}{\pgfqpoint{3.100000in}{2.695000in}} %
\pgfusepath{clip}%
\pgfsetbuttcap%
\pgfsetroundjoin%
\definecolor{currentfill}{rgb}{0.172549,0.627451,0.172549}%
\pgfsetfillcolor{currentfill}%
\pgfsetlinewidth{1.003750pt}%
\definecolor{currentstroke}{rgb}{0.172549,0.627451,0.172549}%
\pgfsetstrokecolor{currentstroke}%
\pgfsetdash{}{0pt}%
\pgfsys@defobject{currentmarker}{\pgfqpoint{-0.034722in}{-0.034722in}}{\pgfqpoint{0.034722in}{0.034722in}}{%
\pgfpathmoveto{\pgfqpoint{0.000000in}{-0.034722in}}%
\pgfpathcurveto{\pgfqpoint{0.009208in}{-0.034722in}}{\pgfqpoint{0.018041in}{-0.031064in}}{\pgfqpoint{0.024552in}{-0.024552in}}%
\pgfpathcurveto{\pgfqpoint{0.031064in}{-0.018041in}}{\pgfqpoint{0.034722in}{-0.009208in}}{\pgfqpoint{0.034722in}{0.000000in}}%
\pgfpathcurveto{\pgfqpoint{0.034722in}{0.009208in}}{\pgfqpoint{0.031064in}{0.018041in}}{\pgfqpoint{0.024552in}{0.024552in}}%
\pgfpathcurveto{\pgfqpoint{0.018041in}{0.031064in}}{\pgfqpoint{0.009208in}{0.034722in}}{\pgfqpoint{0.000000in}{0.034722in}}%
\pgfpathcurveto{\pgfqpoint{-0.009208in}{0.034722in}}{\pgfqpoint{-0.018041in}{0.031064in}}{\pgfqpoint{-0.024552in}{0.024552in}}%
\pgfpathcurveto{\pgfqpoint{-0.031064in}{0.018041in}}{\pgfqpoint{-0.034722in}{0.009208in}}{\pgfqpoint{-0.034722in}{0.000000in}}%
\pgfpathcurveto{\pgfqpoint{-0.034722in}{-0.009208in}}{\pgfqpoint{-0.031064in}{-0.018041in}}{\pgfqpoint{-0.024552in}{-0.024552in}}%
\pgfpathcurveto{\pgfqpoint{-0.018041in}{-0.031064in}}{\pgfqpoint{-0.009208in}{-0.034722in}}{\pgfqpoint{0.000000in}{-0.034722in}}%
\pgfpathclose%
\pgfusepath{stroke,fill}%
}%
\begin{pgfscope}%
\pgfsys@transformshift{0.785823in}{1.185619in}%
\pgfsys@useobject{currentmarker}{}%
\end{pgfscope}%
\begin{pgfscope}%
\pgfsys@transformshift{0.882929in}{1.245698in}%
\pgfsys@useobject{currentmarker}{}%
\end{pgfscope}%
\begin{pgfscope}%
\pgfsys@transformshift{1.049697in}{1.240567in}%
\pgfsys@useobject{currentmarker}{}%
\end{pgfscope}%
\begin{pgfscope}%
\pgfsys@transformshift{1.246724in}{1.177108in}%
\pgfsys@useobject{currentmarker}{}%
\end{pgfscope}%
\begin{pgfscope}%
\pgfsys@transformshift{1.288240in}{1.152609in}%
\pgfsys@useobject{currentmarker}{}%
\end{pgfscope}%
\begin{pgfscope}%
\pgfsys@transformshift{1.366347in}{1.082841in}%
\pgfsys@useobject{currentmarker}{}%
\end{pgfscope}%
\begin{pgfscope}%
\pgfsys@transformshift{1.527487in}{0.941083in}%
\pgfsys@useobject{currentmarker}{}%
\end{pgfscope}%
\begin{pgfscope}%
\pgfsys@transformshift{1.583076in}{0.928343in}%
\pgfsys@useobject{currentmarker}{}%
\end{pgfscope}%
\begin{pgfscope}%
\pgfsys@transformshift{1.620370in}{1.076747in}%
\pgfsys@useobject{currentmarker}{}%
\end{pgfscope}%
\begin{pgfscope}%
\pgfsys@transformshift{1.772362in}{0.787467in}%
\pgfsys@useobject{currentmarker}{}%
\end{pgfscope}%
\end{pgfscope}%
\begin{pgfscope}%
\pgfpathrectangle{\pgfqpoint{0.644914in}{0.577483in}}{\pgfqpoint{3.100000in}{2.695000in}} %
\pgfusepath{clip}%
\pgfsetrectcap%
\pgfsetroundjoin%
\pgfsetlinewidth{1.505625pt}%
\definecolor{currentstroke}{rgb}{0.172549,0.627451,0.172549}%
\pgfsetstrokecolor{currentstroke}%
\pgfsetdash{}{0pt}%
\pgfpathmoveto{\pgfqpoint{0.785823in}{1.200211in}}%
\pgfpathlineto{\pgfqpoint{0.805956in}{1.208022in}}%
\pgfpathlineto{\pgfqpoint{0.826090in}{1.214854in}}%
\pgfpathlineto{\pgfqpoint{0.846223in}{1.220728in}}%
\pgfpathlineto{\pgfqpoint{0.866357in}{1.225664in}}%
\pgfpathlineto{\pgfqpoint{0.886490in}{1.229686in}}%
\pgfpathlineto{\pgfqpoint{0.906623in}{1.232812in}}%
\pgfpathlineto{\pgfqpoint{0.926757in}{1.235065in}}%
\pgfpathlineto{\pgfqpoint{0.946890in}{1.236466in}}%
\pgfpathlineto{\pgfqpoint{0.967024in}{1.237037in}}%
\pgfpathlineto{\pgfqpoint{0.987157in}{1.236797in}}%
\pgfpathlineto{\pgfqpoint{1.007291in}{1.235770in}}%
\pgfpathlineto{\pgfqpoint{1.027424in}{1.233975in}}%
\pgfpathlineto{\pgfqpoint{1.047558in}{1.231434in}}%
\pgfpathlineto{\pgfqpoint{1.067691in}{1.228168in}}%
\pgfpathlineto{\pgfqpoint{1.087825in}{1.224199in}}%
\pgfpathlineto{\pgfqpoint{1.107958in}{1.219548in}}%
\pgfpathlineto{\pgfqpoint{1.128092in}{1.214235in}}%
\pgfpathlineto{\pgfqpoint{1.148225in}{1.208283in}}%
\pgfpathlineto{\pgfqpoint{1.168358in}{1.201712in}}%
\pgfpathlineto{\pgfqpoint{1.188492in}{1.194544in}}%
\pgfpathlineto{\pgfqpoint{1.208625in}{1.186799in}}%
\pgfpathlineto{\pgfqpoint{1.228759in}{1.178500in}}%
\pgfpathlineto{\pgfqpoint{1.248892in}{1.169667in}}%
\pgfpathlineto{\pgfqpoint{1.269026in}{1.160321in}}%
\pgfpathlineto{\pgfqpoint{1.289159in}{1.150484in}}%
\pgfpathlineto{\pgfqpoint{1.309293in}{1.140177in}}%
\pgfpathlineto{\pgfqpoint{1.329426in}{1.129421in}}%
\pgfpathlineto{\pgfqpoint{1.349560in}{1.118237in}}%
\pgfpathlineto{\pgfqpoint{1.369693in}{1.106648in}}%
\pgfpathlineto{\pgfqpoint{1.389827in}{1.094672in}}%
\pgfpathlineto{\pgfqpoint{1.409960in}{1.082333in}}%
\pgfpathlineto{\pgfqpoint{1.430093in}{1.069652in}}%
\pgfpathlineto{\pgfqpoint{1.450227in}{1.056648in}}%
\pgfpathlineto{\pgfqpoint{1.470360in}{1.043345in}}%
\pgfpathlineto{\pgfqpoint{1.490494in}{1.029762in}}%
\pgfpathlineto{\pgfqpoint{1.510627in}{1.015922in}}%
\pgfpathlineto{\pgfqpoint{1.530761in}{1.001844in}}%
\pgfpathlineto{\pgfqpoint{1.550894in}{0.987552in}}%
\pgfpathlineto{\pgfqpoint{1.571028in}{0.973065in}}%
\pgfpathlineto{\pgfqpoint{1.591161in}{0.958406in}}%
\pgfpathlineto{\pgfqpoint{1.611295in}{0.943594in}}%
\pgfpathlineto{\pgfqpoint{1.631428in}{0.928652in}}%
\pgfpathlineto{\pgfqpoint{1.651561in}{0.913601in}}%
\pgfpathlineto{\pgfqpoint{1.671695in}{0.898462in}}%
\pgfpathlineto{\pgfqpoint{1.691828in}{0.883256in}}%
\pgfpathlineto{\pgfqpoint{1.711962in}{0.868004in}}%
\pgfpathlineto{\pgfqpoint{1.732095in}{0.852728in}}%
\pgfpathlineto{\pgfqpoint{1.752229in}{0.837449in}}%
\pgfpathlineto{\pgfqpoint{1.772362in}{0.822188in}}%
\pgfusepath{stroke}%
\end{pgfscope}%
\begin{pgfscope}%
\pgfpathrectangle{\pgfqpoint{0.644914in}{0.577483in}}{\pgfqpoint{3.100000in}{2.695000in}} %
\pgfusepath{clip}%
\pgfsetbuttcap%
\pgfsetroundjoin%
\pgfsetlinewidth{1.003750pt}%
\definecolor{currentstroke}{rgb}{1.000000,0.498039,0.054902}%
\pgfsetstrokecolor{currentstroke}%
\pgfsetdash{}{0pt}%
\pgfsys@defobject{currentmarker}{\pgfqpoint{-0.034722in}{-0.034722in}}{\pgfqpoint{0.034722in}{0.034722in}}{%
\pgfpathmoveto{\pgfqpoint{0.000000in}{-0.034722in}}%
\pgfpathcurveto{\pgfqpoint{0.009208in}{-0.034722in}}{\pgfqpoint{0.018041in}{-0.031064in}}{\pgfqpoint{0.024552in}{-0.024552in}}%
\pgfpathcurveto{\pgfqpoint{0.031064in}{-0.018041in}}{\pgfqpoint{0.034722in}{-0.009208in}}{\pgfqpoint{0.034722in}{0.000000in}}%
\pgfpathcurveto{\pgfqpoint{0.034722in}{0.009208in}}{\pgfqpoint{0.031064in}{0.018041in}}{\pgfqpoint{0.024552in}{0.024552in}}%
\pgfpathcurveto{\pgfqpoint{0.018041in}{0.031064in}}{\pgfqpoint{0.009208in}{0.034722in}}{\pgfqpoint{0.000000in}{0.034722in}}%
\pgfpathcurveto{\pgfqpoint{-0.009208in}{0.034722in}}{\pgfqpoint{-0.018041in}{0.031064in}}{\pgfqpoint{-0.024552in}{0.024552in}}%
\pgfpathcurveto{\pgfqpoint{-0.031064in}{0.018041in}}{\pgfqpoint{-0.034722in}{0.009208in}}{\pgfqpoint{-0.034722in}{0.000000in}}%
\pgfpathcurveto{\pgfqpoint{-0.034722in}{-0.009208in}}{\pgfqpoint{-0.031064in}{-0.018041in}}{\pgfqpoint{-0.024552in}{-0.024552in}}%
\pgfpathcurveto{\pgfqpoint{-0.018041in}{-0.031064in}}{\pgfqpoint{-0.009208in}{-0.034722in}}{\pgfqpoint{0.000000in}{-0.034722in}}%
\pgfpathclose%
\pgfusepath{stroke}%
}%
\begin{pgfscope}%
\pgfsys@transformshift{2.590726in}{0.941627in}%
\pgfsys@useobject{currentmarker}{}%
\end{pgfscope}%
\begin{pgfscope}%
\pgfsys@transformshift{2.886265in}{1.422246in}%
\pgfsys@useobject{currentmarker}{}%
\end{pgfscope}%
\begin{pgfscope}%
\pgfsys@transformshift{3.126919in}{1.659681in}%
\pgfsys@useobject{currentmarker}{}%
\end{pgfscope}%
\begin{pgfscope}%
\pgfsys@transformshift{3.345759in}{2.010057in}%
\pgfsys@useobject{currentmarker}{}%
\end{pgfscope}%
\begin{pgfscope}%
\pgfsys@transformshift{3.604005in}{2.297600in}%
\pgfsys@useobject{currentmarker}{}%
\end{pgfscope}%
\end{pgfscope}%
\begin{pgfscope}%
\pgfpathrectangle{\pgfqpoint{0.644914in}{0.577483in}}{\pgfqpoint{3.100000in}{2.695000in}} %
\pgfusepath{clip}%
\pgfsetbuttcap%
\pgfsetroundjoin%
\pgfsetlinewidth{1.505625pt}%
\definecolor{currentstroke}{rgb}{1.000000,0.498039,0.054902}%
\pgfsetstrokecolor{currentstroke}%
\pgfsetdash{{5.600000pt}{2.400000pt}}{0.000000pt}%
\pgfpathmoveto{\pgfqpoint{2.590726in}{0.946966in}}%
\pgfpathlineto{\pgfqpoint{2.611405in}{0.981512in}}%
\pgfpathlineto{\pgfqpoint{2.632084in}{1.015503in}}%
\pgfpathlineto{\pgfqpoint{2.652763in}{1.048956in}}%
\pgfpathlineto{\pgfqpoint{2.673442in}{1.081888in}}%
\pgfpathlineto{\pgfqpoint{2.694121in}{1.114316in}}%
\pgfpathlineto{\pgfqpoint{2.714801in}{1.146257in}}%
\pgfpathlineto{\pgfqpoint{2.735480in}{1.177730in}}%
\pgfpathlineto{\pgfqpoint{2.756159in}{1.208750in}}%
\pgfpathlineto{\pgfqpoint{2.776838in}{1.239336in}}%
\pgfpathlineto{\pgfqpoint{2.797517in}{1.269505in}}%
\pgfpathlineto{\pgfqpoint{2.818196in}{1.299273in}}%
\pgfpathlineto{\pgfqpoint{2.838876in}{1.328659in}}%
\pgfpathlineto{\pgfqpoint{2.859555in}{1.357679in}}%
\pgfpathlineto{\pgfqpoint{2.880234in}{1.386352in}}%
\pgfpathlineto{\pgfqpoint{2.900913in}{1.414693in}}%
\pgfpathlineto{\pgfqpoint{2.921592in}{1.442721in}}%
\pgfpathlineto{\pgfqpoint{2.942271in}{1.470452in}}%
\pgfpathlineto{\pgfqpoint{2.962951in}{1.497904in}}%
\pgfpathlineto{\pgfqpoint{2.983630in}{1.525095in}}%
\pgfpathlineto{\pgfqpoint{3.004309in}{1.552041in}}%
\pgfpathlineto{\pgfqpoint{3.024988in}{1.578760in}}%
\pgfpathlineto{\pgfqpoint{3.045667in}{1.605269in}}%
\pgfpathlineto{\pgfqpoint{3.066346in}{1.631585in}}%
\pgfpathlineto{\pgfqpoint{3.087026in}{1.657727in}}%
\pgfpathlineto{\pgfqpoint{3.107705in}{1.683710in}}%
\pgfpathlineto{\pgfqpoint{3.128384in}{1.709552in}}%
\pgfpathlineto{\pgfqpoint{3.149063in}{1.735271in}}%
\pgfpathlineto{\pgfqpoint{3.169742in}{1.760883in}}%
\pgfpathlineto{\pgfqpoint{3.190421in}{1.786407in}}%
\pgfpathlineto{\pgfqpoint{3.211100in}{1.811859in}}%
\pgfpathlineto{\pgfqpoint{3.231780in}{1.837256in}}%
\pgfpathlineto{\pgfqpoint{3.252459in}{1.862617in}}%
\pgfpathlineto{\pgfqpoint{3.273138in}{1.887958in}}%
\pgfpathlineto{\pgfqpoint{3.293817in}{1.913296in}}%
\pgfpathlineto{\pgfqpoint{3.314496in}{1.938649in}}%
\pgfpathlineto{\pgfqpoint{3.335175in}{1.964034in}}%
\pgfpathlineto{\pgfqpoint{3.355855in}{1.989469in}}%
\pgfpathlineto{\pgfqpoint{3.376534in}{2.014970in}}%
\pgfpathlineto{\pgfqpoint{3.397213in}{2.040555in}}%
\pgfpathlineto{\pgfqpoint{3.417892in}{2.066241in}}%
\pgfpathlineto{\pgfqpoint{3.438571in}{2.092046in}}%
\pgfpathlineto{\pgfqpoint{3.459250in}{2.117987in}}%
\pgfpathlineto{\pgfqpoint{3.479930in}{2.144080in}}%
\pgfpathlineto{\pgfqpoint{3.500609in}{2.170344in}}%
\pgfpathlineto{\pgfqpoint{3.521288in}{2.196796in}}%
\pgfpathlineto{\pgfqpoint{3.541967in}{2.223452in}}%
\pgfpathlineto{\pgfqpoint{3.562646in}{2.250331in}}%
\pgfpathlineto{\pgfqpoint{3.583325in}{2.277449in}}%
\pgfpathlineto{\pgfqpoint{3.604005in}{2.304824in}}%
\pgfusepath{stroke}%
\end{pgfscope}%
\begin{pgfscope}%
\pgfpathrectangle{\pgfqpoint{0.644914in}{0.577483in}}{\pgfqpoint{3.100000in}{2.695000in}} %
\pgfusepath{clip}%
\pgfsetbuttcap%
\pgfsetroundjoin%
\pgfsetlinewidth{1.003750pt}%
\definecolor{currentstroke}{rgb}{0.172549,0.627451,0.172549}%
\pgfsetstrokecolor{currentstroke}%
\pgfsetdash{}{0pt}%
\pgfsys@defobject{currentmarker}{\pgfqpoint{-0.034722in}{-0.034722in}}{\pgfqpoint{0.034722in}{0.034722in}}{%
\pgfpathmoveto{\pgfqpoint{0.000000in}{-0.034722in}}%
\pgfpathcurveto{\pgfqpoint{0.009208in}{-0.034722in}}{\pgfqpoint{0.018041in}{-0.031064in}}{\pgfqpoint{0.024552in}{-0.024552in}}%
\pgfpathcurveto{\pgfqpoint{0.031064in}{-0.018041in}}{\pgfqpoint{0.034722in}{-0.009208in}}{\pgfqpoint{0.034722in}{0.000000in}}%
\pgfpathcurveto{\pgfqpoint{0.034722in}{0.009208in}}{\pgfqpoint{0.031064in}{0.018041in}}{\pgfqpoint{0.024552in}{0.024552in}}%
\pgfpathcurveto{\pgfqpoint{0.018041in}{0.031064in}}{\pgfqpoint{0.009208in}{0.034722in}}{\pgfqpoint{0.000000in}{0.034722in}}%
\pgfpathcurveto{\pgfqpoint{-0.009208in}{0.034722in}}{\pgfqpoint{-0.018041in}{0.031064in}}{\pgfqpoint{-0.024552in}{0.024552in}}%
\pgfpathcurveto{\pgfqpoint{-0.031064in}{0.018041in}}{\pgfqpoint{-0.034722in}{0.009208in}}{\pgfqpoint{-0.034722in}{0.000000in}}%
\pgfpathcurveto{\pgfqpoint{-0.034722in}{-0.009208in}}{\pgfqpoint{-0.031064in}{-0.018041in}}{\pgfqpoint{-0.024552in}{-0.024552in}}%
\pgfpathcurveto{\pgfqpoint{-0.018041in}{-0.031064in}}{\pgfqpoint{-0.009208in}{-0.034722in}}{\pgfqpoint{0.000000in}{-0.034722in}}%
\pgfpathclose%
\pgfusepath{stroke}%
}%
\begin{pgfscope}%
\pgfsys@transformshift{2.590726in}{0.665906in}%
\pgfsys@useobject{currentmarker}{}%
\end{pgfscope}%
\begin{pgfscope}%
\pgfsys@transformshift{2.886265in}{1.234885in}%
\pgfsys@useobject{currentmarker}{}%
\end{pgfscope}%
\begin{pgfscope}%
\pgfsys@transformshift{3.126919in}{1.482780in}%
\pgfsys@useobject{currentmarker}{}%
\end{pgfscope}%
\begin{pgfscope}%
\pgfsys@transformshift{3.345759in}{1.883826in}%
\pgfsys@useobject{currentmarker}{}%
\end{pgfscope}%
\begin{pgfscope}%
\pgfsys@transformshift{3.604005in}{2.199908in}%
\pgfsys@useobject{currentmarker}{}%
\end{pgfscope}%
\end{pgfscope}%
\begin{pgfscope}%
\pgfpathrectangle{\pgfqpoint{0.644914in}{0.577483in}}{\pgfqpoint{3.100000in}{2.695000in}} %
\pgfusepath{clip}%
\pgfsetbuttcap%
\pgfsetroundjoin%
\pgfsetlinewidth{1.505625pt}%
\definecolor{currentstroke}{rgb}{0.172549,0.627451,0.172549}%
\pgfsetstrokecolor{currentstroke}%
\pgfsetdash{{5.600000pt}{2.400000pt}}{0.000000pt}%
\pgfpathmoveto{\pgfqpoint{2.590726in}{0.672931in}}%
\pgfpathlineto{\pgfqpoint{2.611405in}{0.714853in}}%
\pgfpathlineto{\pgfqpoint{2.632084in}{0.755883in}}%
\pgfpathlineto{\pgfqpoint{2.652763in}{0.796050in}}%
\pgfpathlineto{\pgfqpoint{2.673442in}{0.835382in}}%
\pgfpathlineto{\pgfqpoint{2.694121in}{0.873910in}}%
\pgfpathlineto{\pgfqpoint{2.714801in}{0.911660in}}%
\pgfpathlineto{\pgfqpoint{2.735480in}{0.948663in}}%
\pgfpathlineto{\pgfqpoint{2.756159in}{0.984948in}}%
\pgfpathlineto{\pgfqpoint{2.776838in}{1.020542in}}%
\pgfpathlineto{\pgfqpoint{2.797517in}{1.055475in}}%
\pgfpathlineto{\pgfqpoint{2.818196in}{1.089775in}}%
\pgfpathlineto{\pgfqpoint{2.838876in}{1.123472in}}%
\pgfpathlineto{\pgfqpoint{2.859555in}{1.156594in}}%
\pgfpathlineto{\pgfqpoint{2.880234in}{1.189170in}}%
\pgfpathlineto{\pgfqpoint{2.900913in}{1.221230in}}%
\pgfpathlineto{\pgfqpoint{2.921592in}{1.252801in}}%
\pgfpathlineto{\pgfqpoint{2.942271in}{1.283912in}}%
\pgfpathlineto{\pgfqpoint{2.962951in}{1.314593in}}%
\pgfpathlineto{\pgfqpoint{2.983630in}{1.344872in}}%
\pgfpathlineto{\pgfqpoint{3.004309in}{1.374779in}}%
\pgfpathlineto{\pgfqpoint{3.024988in}{1.404341in}}%
\pgfpathlineto{\pgfqpoint{3.045667in}{1.433587in}}%
\pgfpathlineto{\pgfqpoint{3.066346in}{1.462548in}}%
\pgfpathlineto{\pgfqpoint{3.087026in}{1.491251in}}%
\pgfpathlineto{\pgfqpoint{3.107705in}{1.519725in}}%
\pgfpathlineto{\pgfqpoint{3.128384in}{1.547999in}}%
\pgfpathlineto{\pgfqpoint{3.149063in}{1.576101in}}%
\pgfpathlineto{\pgfqpoint{3.169742in}{1.604062in}}%
\pgfpathlineto{\pgfqpoint{3.190421in}{1.631909in}}%
\pgfpathlineto{\pgfqpoint{3.211100in}{1.659671in}}%
\pgfpathlineto{\pgfqpoint{3.231780in}{1.687378in}}%
\pgfpathlineto{\pgfqpoint{3.252459in}{1.715058in}}%
\pgfpathlineto{\pgfqpoint{3.273138in}{1.742739in}}%
\pgfpathlineto{\pgfqpoint{3.293817in}{1.770451in}}%
\pgfpathlineto{\pgfqpoint{3.314496in}{1.798223in}}%
\pgfpathlineto{\pgfqpoint{3.335175in}{1.826083in}}%
\pgfpathlineto{\pgfqpoint{3.355855in}{1.854060in}}%
\pgfpathlineto{\pgfqpoint{3.376534in}{1.882182in}}%
\pgfpathlineto{\pgfqpoint{3.397213in}{1.910480in}}%
\pgfpathlineto{\pgfqpoint{3.417892in}{1.938982in}}%
\pgfpathlineto{\pgfqpoint{3.438571in}{1.967715in}}%
\pgfpathlineto{\pgfqpoint{3.459250in}{1.996711in}}%
\pgfpathlineto{\pgfqpoint{3.479930in}{2.025996in}}%
\pgfpathlineto{\pgfqpoint{3.500609in}{2.055600in}}%
\pgfpathlineto{\pgfqpoint{3.521288in}{2.085552in}}%
\pgfpathlineto{\pgfqpoint{3.541967in}{2.115881in}}%
\pgfpathlineto{\pgfqpoint{3.562646in}{2.146615in}}%
\pgfpathlineto{\pgfqpoint{3.583325in}{2.177783in}}%
\pgfpathlineto{\pgfqpoint{3.604005in}{2.209414in}}%
\pgfusepath{stroke}%
\end{pgfscope}%
\begin{pgfscope}%
\pgfpathrectangle{\pgfqpoint{0.644914in}{0.577483in}}{\pgfqpoint{3.100000in}{2.695000in}} %
\pgfusepath{clip}%
\pgfsetbuttcap%
\pgfsetroundjoin%
\definecolor{currentfill}{rgb}{1.000000,0.498039,0.054902}%
\pgfsetfillcolor{currentfill}%
\pgfsetlinewidth{1.003750pt}%
\definecolor{currentstroke}{rgb}{1.000000,0.498039,0.054902}%
\pgfsetstrokecolor{currentstroke}%
\pgfsetdash{}{0pt}%
\pgfsys@defobject{currentmarker}{\pgfqpoint{-0.034722in}{-0.034722in}}{\pgfqpoint{0.034722in}{0.034722in}}{%
\pgfpathmoveto{\pgfqpoint{0.000000in}{-0.034722in}}%
\pgfpathcurveto{\pgfqpoint{0.009208in}{-0.034722in}}{\pgfqpoint{0.018041in}{-0.031064in}}{\pgfqpoint{0.024552in}{-0.024552in}}%
\pgfpathcurveto{\pgfqpoint{0.031064in}{-0.018041in}}{\pgfqpoint{0.034722in}{-0.009208in}}{\pgfqpoint{0.034722in}{0.000000in}}%
\pgfpathcurveto{\pgfqpoint{0.034722in}{0.009208in}}{\pgfqpoint{0.031064in}{0.018041in}}{\pgfqpoint{0.024552in}{0.024552in}}%
\pgfpathcurveto{\pgfqpoint{0.018041in}{0.031064in}}{\pgfqpoint{0.009208in}{0.034722in}}{\pgfqpoint{0.000000in}{0.034722in}}%
\pgfpathcurveto{\pgfqpoint{-0.009208in}{0.034722in}}{\pgfqpoint{-0.018041in}{0.031064in}}{\pgfqpoint{-0.024552in}{0.024552in}}%
\pgfpathcurveto{\pgfqpoint{-0.031064in}{0.018041in}}{\pgfqpoint{-0.034722in}{0.009208in}}{\pgfqpoint{-0.034722in}{0.000000in}}%
\pgfpathcurveto{\pgfqpoint{-0.034722in}{-0.009208in}}{\pgfqpoint{-0.031064in}{-0.018041in}}{\pgfqpoint{-0.024552in}{-0.024552in}}%
\pgfpathcurveto{\pgfqpoint{-0.018041in}{-0.031064in}}{\pgfqpoint{-0.009208in}{-0.034722in}}{\pgfqpoint{0.000000in}{-0.034722in}}%
\pgfpathclose%
\pgfusepath{stroke,fill}%
}%
\begin{pgfscope}%
\pgfsys@transformshift{0.785823in}{1.869637in}%
\pgfsys@useobject{currentmarker}{}%
\end{pgfscope}%
\begin{pgfscope}%
\pgfsys@transformshift{0.882929in}{2.103299in}%
\pgfsys@useobject{currentmarker}{}%
\end{pgfscope}%
\begin{pgfscope}%
\pgfsys@transformshift{1.049697in}{2.274688in}%
\pgfsys@useobject{currentmarker}{}%
\end{pgfscope}%
\begin{pgfscope}%
\pgfsys@transformshift{1.246724in}{2.363102in}%
\pgfsys@useobject{currentmarker}{}%
\end{pgfscope}%
\begin{pgfscope}%
\pgfsys@transformshift{1.288240in}{2.369391in}%
\pgfsys@useobject{currentmarker}{}%
\end{pgfscope}%
\begin{pgfscope}%
\pgfsys@transformshift{1.366347in}{2.309265in}%
\pgfsys@useobject{currentmarker}{}%
\end{pgfscope}%
\begin{pgfscope}%
\pgfsys@transformshift{1.527487in}{2.139618in}%
\pgfsys@useobject{currentmarker}{}%
\end{pgfscope}%
\begin{pgfscope}%
\pgfsys@transformshift{1.583076in}{2.050194in}%
\pgfsys@useobject{currentmarker}{}%
\end{pgfscope}%
\begin{pgfscope}%
\pgfsys@transformshift{1.620370in}{2.005958in}%
\pgfsys@useobject{currentmarker}{}%
\end{pgfscope}%
\begin{pgfscope}%
\pgfsys@transformshift{1.772362in}{1.741973in}%
\pgfsys@useobject{currentmarker}{}%
\end{pgfscope}%
\end{pgfscope}%
\begin{pgfscope}%
\pgfpathrectangle{\pgfqpoint{0.644914in}{0.577483in}}{\pgfqpoint{3.100000in}{2.695000in}} %
\pgfusepath{clip}%
\pgfsetrectcap%
\pgfsetroundjoin%
\pgfsetlinewidth{1.505625pt}%
\definecolor{currentstroke}{rgb}{1.000000,0.498039,0.054902}%
\pgfsetstrokecolor{currentstroke}%
\pgfsetdash{}{0pt}%
\pgfpathmoveto{\pgfqpoint{0.785823in}{1.875062in}}%
\pgfpathlineto{\pgfqpoint{0.805956in}{1.923716in}}%
\pgfpathlineto{\pgfqpoint{0.826090in}{1.969578in}}%
\pgfpathlineto{\pgfqpoint{0.846223in}{2.012689in}}%
\pgfpathlineto{\pgfqpoint{0.866357in}{2.053090in}}%
\pgfpathlineto{\pgfqpoint{0.886490in}{2.090822in}}%
\pgfpathlineto{\pgfqpoint{0.906623in}{2.125927in}}%
\pgfpathlineto{\pgfqpoint{0.926757in}{2.158447in}}%
\pgfpathlineto{\pgfqpoint{0.946890in}{2.188422in}}%
\pgfpathlineto{\pgfqpoint{0.967024in}{2.215894in}}%
\pgfpathlineto{\pgfqpoint{0.987157in}{2.240904in}}%
\pgfpathlineto{\pgfqpoint{1.007291in}{2.263494in}}%
\pgfpathlineto{\pgfqpoint{1.027424in}{2.283705in}}%
\pgfpathlineto{\pgfqpoint{1.047558in}{2.301578in}}%
\pgfpathlineto{\pgfqpoint{1.067691in}{2.317155in}}%
\pgfpathlineto{\pgfqpoint{1.087825in}{2.330477in}}%
\pgfpathlineto{\pgfqpoint{1.107958in}{2.341586in}}%
\pgfpathlineto{\pgfqpoint{1.128092in}{2.350522in}}%
\pgfpathlineto{\pgfqpoint{1.148225in}{2.357327in}}%
\pgfpathlineto{\pgfqpoint{1.168358in}{2.362042in}}%
\pgfpathlineto{\pgfqpoint{1.188492in}{2.364710in}}%
\pgfpathlineto{\pgfqpoint{1.208625in}{2.365370in}}%
\pgfpathlineto{\pgfqpoint{1.228759in}{2.364065in}}%
\pgfpathlineto{\pgfqpoint{1.248892in}{2.360836in}}%
\pgfpathlineto{\pgfqpoint{1.269026in}{2.355724in}}%
\pgfpathlineto{\pgfqpoint{1.289159in}{2.348771in}}%
\pgfpathlineto{\pgfqpoint{1.309293in}{2.340017in}}%
\pgfpathlineto{\pgfqpoint{1.329426in}{2.329505in}}%
\pgfpathlineto{\pgfqpoint{1.349560in}{2.317275in}}%
\pgfpathlineto{\pgfqpoint{1.369693in}{2.303369in}}%
\pgfpathlineto{\pgfqpoint{1.389827in}{2.287828in}}%
\pgfpathlineto{\pgfqpoint{1.409960in}{2.270694in}}%
\pgfpathlineto{\pgfqpoint{1.430093in}{2.252008in}}%
\pgfpathlineto{\pgfqpoint{1.450227in}{2.231811in}}%
\pgfpathlineto{\pgfqpoint{1.470360in}{2.210145in}}%
\pgfpathlineto{\pgfqpoint{1.490494in}{2.187051in}}%
\pgfpathlineto{\pgfqpoint{1.510627in}{2.162570in}}%
\pgfpathlineto{\pgfqpoint{1.530761in}{2.136744in}}%
\pgfpathlineto{\pgfqpoint{1.550894in}{2.109614in}}%
\pgfpathlineto{\pgfqpoint{1.571028in}{2.081222in}}%
\pgfpathlineto{\pgfqpoint{1.591161in}{2.051608in}}%
\pgfpathlineto{\pgfqpoint{1.611295in}{2.020814in}}%
\pgfpathlineto{\pgfqpoint{1.631428in}{1.988882in}}%
\pgfpathlineto{\pgfqpoint{1.651561in}{1.955852in}}%
\pgfpathlineto{\pgfqpoint{1.671695in}{1.921767in}}%
\pgfpathlineto{\pgfqpoint{1.691828in}{1.886667in}}%
\pgfpathlineto{\pgfqpoint{1.711962in}{1.850594in}}%
\pgfpathlineto{\pgfqpoint{1.732095in}{1.813590in}}%
\pgfpathlineto{\pgfqpoint{1.752229in}{1.775694in}}%
\pgfpathlineto{\pgfqpoint{1.772362in}{1.736950in}}%
\pgfusepath{stroke}%
\end{pgfscope}%
\begin{pgfscope}%
\pgfpathrectangle{\pgfqpoint{0.644914in}{0.577483in}}{\pgfqpoint{3.100000in}{2.695000in}} %
\pgfusepath{clip}%
\pgfsetbuttcap%
\pgfsetroundjoin%
\definecolor{currentfill}{rgb}{0.172549,0.627451,0.172549}%
\pgfsetfillcolor{currentfill}%
\pgfsetlinewidth{1.003750pt}%
\definecolor{currentstroke}{rgb}{0.172549,0.627451,0.172549}%
\pgfsetstrokecolor{currentstroke}%
\pgfsetdash{}{0pt}%
\pgfsys@defobject{currentmarker}{\pgfqpoint{-0.034722in}{-0.034722in}}{\pgfqpoint{0.034722in}{0.034722in}}{%
\pgfpathmoveto{\pgfqpoint{0.000000in}{-0.034722in}}%
\pgfpathcurveto{\pgfqpoint{0.009208in}{-0.034722in}}{\pgfqpoint{0.018041in}{-0.031064in}}{\pgfqpoint{0.024552in}{-0.024552in}}%
\pgfpathcurveto{\pgfqpoint{0.031064in}{-0.018041in}}{\pgfqpoint{0.034722in}{-0.009208in}}{\pgfqpoint{0.034722in}{0.000000in}}%
\pgfpathcurveto{\pgfqpoint{0.034722in}{0.009208in}}{\pgfqpoint{0.031064in}{0.018041in}}{\pgfqpoint{0.024552in}{0.024552in}}%
\pgfpathcurveto{\pgfqpoint{0.018041in}{0.031064in}}{\pgfqpoint{0.009208in}{0.034722in}}{\pgfqpoint{0.000000in}{0.034722in}}%
\pgfpathcurveto{\pgfqpoint{-0.009208in}{0.034722in}}{\pgfqpoint{-0.018041in}{0.031064in}}{\pgfqpoint{-0.024552in}{0.024552in}}%
\pgfpathcurveto{\pgfqpoint{-0.031064in}{0.018041in}}{\pgfqpoint{-0.034722in}{0.009208in}}{\pgfqpoint{-0.034722in}{0.000000in}}%
\pgfpathcurveto{\pgfqpoint{-0.034722in}{-0.009208in}}{\pgfqpoint{-0.031064in}{-0.018041in}}{\pgfqpoint{-0.024552in}{-0.024552in}}%
\pgfpathcurveto{\pgfqpoint{-0.018041in}{-0.031064in}}{\pgfqpoint{-0.009208in}{-0.034722in}}{\pgfqpoint{0.000000in}{-0.034722in}}%
\pgfpathclose%
\pgfusepath{stroke,fill}%
}%
\begin{pgfscope}%
\pgfsys@transformshift{0.785823in}{1.185619in}%
\pgfsys@useobject{currentmarker}{}%
\end{pgfscope}%
\begin{pgfscope}%
\pgfsys@transformshift{0.882929in}{1.245698in}%
\pgfsys@useobject{currentmarker}{}%
\end{pgfscope}%
\begin{pgfscope}%
\pgfsys@transformshift{1.049697in}{1.240567in}%
\pgfsys@useobject{currentmarker}{}%
\end{pgfscope}%
\begin{pgfscope}%
\pgfsys@transformshift{1.246724in}{1.177108in}%
\pgfsys@useobject{currentmarker}{}%
\end{pgfscope}%
\begin{pgfscope}%
\pgfsys@transformshift{1.288240in}{1.152609in}%
\pgfsys@useobject{currentmarker}{}%
\end{pgfscope}%
\begin{pgfscope}%
\pgfsys@transformshift{1.366347in}{1.082841in}%
\pgfsys@useobject{currentmarker}{}%
\end{pgfscope}%
\begin{pgfscope}%
\pgfsys@transformshift{1.527487in}{0.941083in}%
\pgfsys@useobject{currentmarker}{}%
\end{pgfscope}%
\begin{pgfscope}%
\pgfsys@transformshift{1.583076in}{0.928343in}%
\pgfsys@useobject{currentmarker}{}%
\end{pgfscope}%
\begin{pgfscope}%
\pgfsys@transformshift{1.620370in}{1.076747in}%
\pgfsys@useobject{currentmarker}{}%
\end{pgfscope}%
\begin{pgfscope}%
\pgfsys@transformshift{1.772362in}{0.787467in}%
\pgfsys@useobject{currentmarker}{}%
\end{pgfscope}%
\end{pgfscope}%
\begin{pgfscope}%
\pgfpathrectangle{\pgfqpoint{0.644914in}{0.577483in}}{\pgfqpoint{3.100000in}{2.695000in}} %
\pgfusepath{clip}%
\pgfsetrectcap%
\pgfsetroundjoin%
\pgfsetlinewidth{1.505625pt}%
\definecolor{currentstroke}{rgb}{0.172549,0.627451,0.172549}%
\pgfsetstrokecolor{currentstroke}%
\pgfsetdash{}{0pt}%
\pgfpathmoveto{\pgfqpoint{0.785823in}{1.200211in}}%
\pgfpathlineto{\pgfqpoint{0.805956in}{1.208022in}}%
\pgfpathlineto{\pgfqpoint{0.826090in}{1.214854in}}%
\pgfpathlineto{\pgfqpoint{0.846223in}{1.220728in}}%
\pgfpathlineto{\pgfqpoint{0.866357in}{1.225664in}}%
\pgfpathlineto{\pgfqpoint{0.886490in}{1.229686in}}%
\pgfpathlineto{\pgfqpoint{0.906623in}{1.232812in}}%
\pgfpathlineto{\pgfqpoint{0.926757in}{1.235065in}}%
\pgfpathlineto{\pgfqpoint{0.946890in}{1.236466in}}%
\pgfpathlineto{\pgfqpoint{0.967024in}{1.237037in}}%
\pgfpathlineto{\pgfqpoint{0.987157in}{1.236797in}}%
\pgfpathlineto{\pgfqpoint{1.007291in}{1.235770in}}%
\pgfpathlineto{\pgfqpoint{1.027424in}{1.233975in}}%
\pgfpathlineto{\pgfqpoint{1.047558in}{1.231434in}}%
\pgfpathlineto{\pgfqpoint{1.067691in}{1.228168in}}%
\pgfpathlineto{\pgfqpoint{1.087825in}{1.224199in}}%
\pgfpathlineto{\pgfqpoint{1.107958in}{1.219548in}}%
\pgfpathlineto{\pgfqpoint{1.128092in}{1.214235in}}%
\pgfpathlineto{\pgfqpoint{1.148225in}{1.208283in}}%
\pgfpathlineto{\pgfqpoint{1.168358in}{1.201712in}}%
\pgfpathlineto{\pgfqpoint{1.188492in}{1.194544in}}%
\pgfpathlineto{\pgfqpoint{1.208625in}{1.186799in}}%
\pgfpathlineto{\pgfqpoint{1.228759in}{1.178500in}}%
\pgfpathlineto{\pgfqpoint{1.248892in}{1.169667in}}%
\pgfpathlineto{\pgfqpoint{1.269026in}{1.160321in}}%
\pgfpathlineto{\pgfqpoint{1.289159in}{1.150484in}}%
\pgfpathlineto{\pgfqpoint{1.309293in}{1.140177in}}%
\pgfpathlineto{\pgfqpoint{1.329426in}{1.129421in}}%
\pgfpathlineto{\pgfqpoint{1.349560in}{1.118237in}}%
\pgfpathlineto{\pgfqpoint{1.369693in}{1.106648in}}%
\pgfpathlineto{\pgfqpoint{1.389827in}{1.094672in}}%
\pgfpathlineto{\pgfqpoint{1.409960in}{1.082333in}}%
\pgfpathlineto{\pgfqpoint{1.430093in}{1.069652in}}%
\pgfpathlineto{\pgfqpoint{1.450227in}{1.056648in}}%
\pgfpathlineto{\pgfqpoint{1.470360in}{1.043345in}}%
\pgfpathlineto{\pgfqpoint{1.490494in}{1.029762in}}%
\pgfpathlineto{\pgfqpoint{1.510627in}{1.015922in}}%
\pgfpathlineto{\pgfqpoint{1.530761in}{1.001844in}}%
\pgfpathlineto{\pgfqpoint{1.550894in}{0.987552in}}%
\pgfpathlineto{\pgfqpoint{1.571028in}{0.973065in}}%
\pgfpathlineto{\pgfqpoint{1.591161in}{0.958406in}}%
\pgfpathlineto{\pgfqpoint{1.611295in}{0.943594in}}%
\pgfpathlineto{\pgfqpoint{1.631428in}{0.928652in}}%
\pgfpathlineto{\pgfqpoint{1.651561in}{0.913601in}}%
\pgfpathlineto{\pgfqpoint{1.671695in}{0.898462in}}%
\pgfpathlineto{\pgfqpoint{1.691828in}{0.883256in}}%
\pgfpathlineto{\pgfqpoint{1.711962in}{0.868004in}}%
\pgfpathlineto{\pgfqpoint{1.732095in}{0.852728in}}%
\pgfpathlineto{\pgfqpoint{1.752229in}{0.837449in}}%
\pgfpathlineto{\pgfqpoint{1.772362in}{0.822188in}}%
\pgfusepath{stroke}%
\end{pgfscope}%
\begin{pgfscope}%
\pgfpathrectangle{\pgfqpoint{0.644914in}{0.577483in}}{\pgfqpoint{3.100000in}{2.695000in}} %
\pgfusepath{clip}%
\pgfsetbuttcap%
\pgfsetroundjoin%
\pgfsetlinewidth{1.003750pt}%
\definecolor{currentstroke}{rgb}{1.000000,0.498039,0.054902}%
\pgfsetstrokecolor{currentstroke}%
\pgfsetdash{}{0pt}%
\pgfsys@defobject{currentmarker}{\pgfqpoint{-0.034722in}{-0.034722in}}{\pgfqpoint{0.034722in}{0.034722in}}{%
\pgfpathmoveto{\pgfqpoint{0.000000in}{-0.034722in}}%
\pgfpathcurveto{\pgfqpoint{0.009208in}{-0.034722in}}{\pgfqpoint{0.018041in}{-0.031064in}}{\pgfqpoint{0.024552in}{-0.024552in}}%
\pgfpathcurveto{\pgfqpoint{0.031064in}{-0.018041in}}{\pgfqpoint{0.034722in}{-0.009208in}}{\pgfqpoint{0.034722in}{0.000000in}}%
\pgfpathcurveto{\pgfqpoint{0.034722in}{0.009208in}}{\pgfqpoint{0.031064in}{0.018041in}}{\pgfqpoint{0.024552in}{0.024552in}}%
\pgfpathcurveto{\pgfqpoint{0.018041in}{0.031064in}}{\pgfqpoint{0.009208in}{0.034722in}}{\pgfqpoint{0.000000in}{0.034722in}}%
\pgfpathcurveto{\pgfqpoint{-0.009208in}{0.034722in}}{\pgfqpoint{-0.018041in}{0.031064in}}{\pgfqpoint{-0.024552in}{0.024552in}}%
\pgfpathcurveto{\pgfqpoint{-0.031064in}{0.018041in}}{\pgfqpoint{-0.034722in}{0.009208in}}{\pgfqpoint{-0.034722in}{0.000000in}}%
\pgfpathcurveto{\pgfqpoint{-0.034722in}{-0.009208in}}{\pgfqpoint{-0.031064in}{-0.018041in}}{\pgfqpoint{-0.024552in}{-0.024552in}}%
\pgfpathcurveto{\pgfqpoint{-0.018041in}{-0.031064in}}{\pgfqpoint{-0.009208in}{-0.034722in}}{\pgfqpoint{0.000000in}{-0.034722in}}%
\pgfpathclose%
\pgfusepath{stroke}%
}%
\begin{pgfscope}%
\pgfsys@transformshift{2.590726in}{0.941627in}%
\pgfsys@useobject{currentmarker}{}%
\end{pgfscope}%
\begin{pgfscope}%
\pgfsys@transformshift{2.886265in}{1.422246in}%
\pgfsys@useobject{currentmarker}{}%
\end{pgfscope}%
\begin{pgfscope}%
\pgfsys@transformshift{3.126919in}{1.659681in}%
\pgfsys@useobject{currentmarker}{}%
\end{pgfscope}%
\begin{pgfscope}%
\pgfsys@transformshift{3.345759in}{2.010057in}%
\pgfsys@useobject{currentmarker}{}%
\end{pgfscope}%
\begin{pgfscope}%
\pgfsys@transformshift{3.604005in}{2.297600in}%
\pgfsys@useobject{currentmarker}{}%
\end{pgfscope}%
\end{pgfscope}%
\begin{pgfscope}%
\pgfpathrectangle{\pgfqpoint{0.644914in}{0.577483in}}{\pgfqpoint{3.100000in}{2.695000in}} %
\pgfusepath{clip}%
\pgfsetbuttcap%
\pgfsetroundjoin%
\pgfsetlinewidth{1.505625pt}%
\definecolor{currentstroke}{rgb}{1.000000,0.498039,0.054902}%
\pgfsetstrokecolor{currentstroke}%
\pgfsetdash{{5.600000pt}{2.400000pt}}{0.000000pt}%
\pgfpathmoveto{\pgfqpoint{2.590726in}{0.946966in}}%
\pgfpathlineto{\pgfqpoint{2.611405in}{0.981512in}}%
\pgfpathlineto{\pgfqpoint{2.632084in}{1.015503in}}%
\pgfpathlineto{\pgfqpoint{2.652763in}{1.048956in}}%
\pgfpathlineto{\pgfqpoint{2.673442in}{1.081888in}}%
\pgfpathlineto{\pgfqpoint{2.694121in}{1.114316in}}%
\pgfpathlineto{\pgfqpoint{2.714801in}{1.146257in}}%
\pgfpathlineto{\pgfqpoint{2.735480in}{1.177730in}}%
\pgfpathlineto{\pgfqpoint{2.756159in}{1.208750in}}%
\pgfpathlineto{\pgfqpoint{2.776838in}{1.239336in}}%
\pgfpathlineto{\pgfqpoint{2.797517in}{1.269505in}}%
\pgfpathlineto{\pgfqpoint{2.818196in}{1.299273in}}%
\pgfpathlineto{\pgfqpoint{2.838876in}{1.328659in}}%
\pgfpathlineto{\pgfqpoint{2.859555in}{1.357679in}}%
\pgfpathlineto{\pgfqpoint{2.880234in}{1.386352in}}%
\pgfpathlineto{\pgfqpoint{2.900913in}{1.414693in}}%
\pgfpathlineto{\pgfqpoint{2.921592in}{1.442721in}}%
\pgfpathlineto{\pgfqpoint{2.942271in}{1.470452in}}%
\pgfpathlineto{\pgfqpoint{2.962951in}{1.497904in}}%
\pgfpathlineto{\pgfqpoint{2.983630in}{1.525095in}}%
\pgfpathlineto{\pgfqpoint{3.004309in}{1.552041in}}%
\pgfpathlineto{\pgfqpoint{3.024988in}{1.578760in}}%
\pgfpathlineto{\pgfqpoint{3.045667in}{1.605269in}}%
\pgfpathlineto{\pgfqpoint{3.066346in}{1.631585in}}%
\pgfpathlineto{\pgfqpoint{3.087026in}{1.657727in}}%
\pgfpathlineto{\pgfqpoint{3.107705in}{1.683710in}}%
\pgfpathlineto{\pgfqpoint{3.128384in}{1.709552in}}%
\pgfpathlineto{\pgfqpoint{3.149063in}{1.735271in}}%
\pgfpathlineto{\pgfqpoint{3.169742in}{1.760883in}}%
\pgfpathlineto{\pgfqpoint{3.190421in}{1.786407in}}%
\pgfpathlineto{\pgfqpoint{3.211100in}{1.811859in}}%
\pgfpathlineto{\pgfqpoint{3.231780in}{1.837256in}}%
\pgfpathlineto{\pgfqpoint{3.252459in}{1.862617in}}%
\pgfpathlineto{\pgfqpoint{3.273138in}{1.887958in}}%
\pgfpathlineto{\pgfqpoint{3.293817in}{1.913296in}}%
\pgfpathlineto{\pgfqpoint{3.314496in}{1.938649in}}%
\pgfpathlineto{\pgfqpoint{3.335175in}{1.964034in}}%
\pgfpathlineto{\pgfqpoint{3.355855in}{1.989469in}}%
\pgfpathlineto{\pgfqpoint{3.376534in}{2.014970in}}%
\pgfpathlineto{\pgfqpoint{3.397213in}{2.040555in}}%
\pgfpathlineto{\pgfqpoint{3.417892in}{2.066241in}}%
\pgfpathlineto{\pgfqpoint{3.438571in}{2.092046in}}%
\pgfpathlineto{\pgfqpoint{3.459250in}{2.117987in}}%
\pgfpathlineto{\pgfqpoint{3.479930in}{2.144080in}}%
\pgfpathlineto{\pgfqpoint{3.500609in}{2.170344in}}%
\pgfpathlineto{\pgfqpoint{3.521288in}{2.196796in}}%
\pgfpathlineto{\pgfqpoint{3.541967in}{2.223452in}}%
\pgfpathlineto{\pgfqpoint{3.562646in}{2.250331in}}%
\pgfpathlineto{\pgfqpoint{3.583325in}{2.277449in}}%
\pgfpathlineto{\pgfqpoint{3.604005in}{2.304824in}}%
\pgfusepath{stroke}%
\end{pgfscope}%
\begin{pgfscope}%
\pgfpathrectangle{\pgfqpoint{0.644914in}{0.577483in}}{\pgfqpoint{3.100000in}{2.695000in}} %
\pgfusepath{clip}%
\pgfsetbuttcap%
\pgfsetroundjoin%
\pgfsetlinewidth{1.003750pt}%
\definecolor{currentstroke}{rgb}{0.172549,0.627451,0.172549}%
\pgfsetstrokecolor{currentstroke}%
\pgfsetdash{}{0pt}%
\pgfsys@defobject{currentmarker}{\pgfqpoint{-0.034722in}{-0.034722in}}{\pgfqpoint{0.034722in}{0.034722in}}{%
\pgfpathmoveto{\pgfqpoint{0.000000in}{-0.034722in}}%
\pgfpathcurveto{\pgfqpoint{0.009208in}{-0.034722in}}{\pgfqpoint{0.018041in}{-0.031064in}}{\pgfqpoint{0.024552in}{-0.024552in}}%
\pgfpathcurveto{\pgfqpoint{0.031064in}{-0.018041in}}{\pgfqpoint{0.034722in}{-0.009208in}}{\pgfqpoint{0.034722in}{0.000000in}}%
\pgfpathcurveto{\pgfqpoint{0.034722in}{0.009208in}}{\pgfqpoint{0.031064in}{0.018041in}}{\pgfqpoint{0.024552in}{0.024552in}}%
\pgfpathcurveto{\pgfqpoint{0.018041in}{0.031064in}}{\pgfqpoint{0.009208in}{0.034722in}}{\pgfqpoint{0.000000in}{0.034722in}}%
\pgfpathcurveto{\pgfqpoint{-0.009208in}{0.034722in}}{\pgfqpoint{-0.018041in}{0.031064in}}{\pgfqpoint{-0.024552in}{0.024552in}}%
\pgfpathcurveto{\pgfqpoint{-0.031064in}{0.018041in}}{\pgfqpoint{-0.034722in}{0.009208in}}{\pgfqpoint{-0.034722in}{0.000000in}}%
\pgfpathcurveto{\pgfqpoint{-0.034722in}{-0.009208in}}{\pgfqpoint{-0.031064in}{-0.018041in}}{\pgfqpoint{-0.024552in}{-0.024552in}}%
\pgfpathcurveto{\pgfqpoint{-0.018041in}{-0.031064in}}{\pgfqpoint{-0.009208in}{-0.034722in}}{\pgfqpoint{0.000000in}{-0.034722in}}%
\pgfpathclose%
\pgfusepath{stroke}%
}%
\begin{pgfscope}%
\pgfsys@transformshift{2.590726in}{0.665906in}%
\pgfsys@useobject{currentmarker}{}%
\end{pgfscope}%
\begin{pgfscope}%
\pgfsys@transformshift{2.886265in}{1.234885in}%
\pgfsys@useobject{currentmarker}{}%
\end{pgfscope}%
\begin{pgfscope}%
\pgfsys@transformshift{3.126919in}{1.482780in}%
\pgfsys@useobject{currentmarker}{}%
\end{pgfscope}%
\begin{pgfscope}%
\pgfsys@transformshift{3.345759in}{1.883826in}%
\pgfsys@useobject{currentmarker}{}%
\end{pgfscope}%
\begin{pgfscope}%
\pgfsys@transformshift{3.604005in}{2.199908in}%
\pgfsys@useobject{currentmarker}{}%
\end{pgfscope}%
\end{pgfscope}%
\begin{pgfscope}%
\pgfpathrectangle{\pgfqpoint{0.644914in}{0.577483in}}{\pgfqpoint{3.100000in}{2.695000in}} %
\pgfusepath{clip}%
\pgfsetbuttcap%
\pgfsetroundjoin%
\pgfsetlinewidth{1.505625pt}%
\definecolor{currentstroke}{rgb}{0.172549,0.627451,0.172549}%
\pgfsetstrokecolor{currentstroke}%
\pgfsetdash{{5.600000pt}{2.400000pt}}{0.000000pt}%
\pgfpathmoveto{\pgfqpoint{2.590726in}{0.672931in}}%
\pgfpathlineto{\pgfqpoint{2.611405in}{0.714853in}}%
\pgfpathlineto{\pgfqpoint{2.632084in}{0.755883in}}%
\pgfpathlineto{\pgfqpoint{2.652763in}{0.796050in}}%
\pgfpathlineto{\pgfqpoint{2.673442in}{0.835382in}}%
\pgfpathlineto{\pgfqpoint{2.694121in}{0.873910in}}%
\pgfpathlineto{\pgfqpoint{2.714801in}{0.911660in}}%
\pgfpathlineto{\pgfqpoint{2.735480in}{0.948663in}}%
\pgfpathlineto{\pgfqpoint{2.756159in}{0.984948in}}%
\pgfpathlineto{\pgfqpoint{2.776838in}{1.020542in}}%
\pgfpathlineto{\pgfqpoint{2.797517in}{1.055475in}}%
\pgfpathlineto{\pgfqpoint{2.818196in}{1.089775in}}%
\pgfpathlineto{\pgfqpoint{2.838876in}{1.123472in}}%
\pgfpathlineto{\pgfqpoint{2.859555in}{1.156594in}}%
\pgfpathlineto{\pgfqpoint{2.880234in}{1.189170in}}%
\pgfpathlineto{\pgfqpoint{2.900913in}{1.221230in}}%
\pgfpathlineto{\pgfqpoint{2.921592in}{1.252801in}}%
\pgfpathlineto{\pgfqpoint{2.942271in}{1.283912in}}%
\pgfpathlineto{\pgfqpoint{2.962951in}{1.314593in}}%
\pgfpathlineto{\pgfqpoint{2.983630in}{1.344872in}}%
\pgfpathlineto{\pgfqpoint{3.004309in}{1.374779in}}%
\pgfpathlineto{\pgfqpoint{3.024988in}{1.404341in}}%
\pgfpathlineto{\pgfqpoint{3.045667in}{1.433587in}}%
\pgfpathlineto{\pgfqpoint{3.066346in}{1.462548in}}%
\pgfpathlineto{\pgfqpoint{3.087026in}{1.491251in}}%
\pgfpathlineto{\pgfqpoint{3.107705in}{1.519725in}}%
\pgfpathlineto{\pgfqpoint{3.128384in}{1.547999in}}%
\pgfpathlineto{\pgfqpoint{3.149063in}{1.576101in}}%
\pgfpathlineto{\pgfqpoint{3.169742in}{1.604062in}}%
\pgfpathlineto{\pgfqpoint{3.190421in}{1.631909in}}%
\pgfpathlineto{\pgfqpoint{3.211100in}{1.659671in}}%
\pgfpathlineto{\pgfqpoint{3.231780in}{1.687378in}}%
\pgfpathlineto{\pgfqpoint{3.252459in}{1.715058in}}%
\pgfpathlineto{\pgfqpoint{3.273138in}{1.742739in}}%
\pgfpathlineto{\pgfqpoint{3.293817in}{1.770451in}}%
\pgfpathlineto{\pgfqpoint{3.314496in}{1.798223in}}%
\pgfpathlineto{\pgfqpoint{3.335175in}{1.826083in}}%
\pgfpathlineto{\pgfqpoint{3.355855in}{1.854060in}}%
\pgfpathlineto{\pgfqpoint{3.376534in}{1.882182in}}%
\pgfpathlineto{\pgfqpoint{3.397213in}{1.910480in}}%
\pgfpathlineto{\pgfqpoint{3.417892in}{1.938982in}}%
\pgfpathlineto{\pgfqpoint{3.438571in}{1.967715in}}%
\pgfpathlineto{\pgfqpoint{3.459250in}{1.996711in}}%
\pgfpathlineto{\pgfqpoint{3.479930in}{2.025996in}}%
\pgfpathlineto{\pgfqpoint{3.500609in}{2.055600in}}%
\pgfpathlineto{\pgfqpoint{3.521288in}{2.085552in}}%
\pgfpathlineto{\pgfqpoint{3.541967in}{2.115881in}}%
\pgfpathlineto{\pgfqpoint{3.562646in}{2.146615in}}%
\pgfpathlineto{\pgfqpoint{3.583325in}{2.177783in}}%
\pgfpathlineto{\pgfqpoint{3.604005in}{2.209414in}}%
\pgfusepath{stroke}%
\end{pgfscope}%
\begin{pgfscope}%
\pgfpathrectangle{\pgfqpoint{0.644914in}{0.577483in}}{\pgfqpoint{3.100000in}{2.695000in}} %
\pgfusepath{clip}%
\pgfsetbuttcap%
\pgfsetroundjoin%
\definecolor{currentfill}{rgb}{0.121569,0.466667,0.705882}%
\pgfsetfillcolor{currentfill}%
\pgfsetlinewidth{1.003750pt}%
\definecolor{currentstroke}{rgb}{0.121569,0.466667,0.705882}%
\pgfsetstrokecolor{currentstroke}%
\pgfsetdash{}{0pt}%
\pgfsys@defobject{currentmarker}{\pgfqpoint{-0.048611in}{-0.048611in}}{\pgfqpoint{0.048611in}{0.048611in}}{%
\pgfpathmoveto{\pgfqpoint{0.000000in}{-0.048611in}}%
\pgfpathcurveto{\pgfqpoint{0.012892in}{-0.048611in}}{\pgfqpoint{0.025257in}{-0.043489in}}{\pgfqpoint{0.034373in}{-0.034373in}}%
\pgfpathcurveto{\pgfqpoint{0.043489in}{-0.025257in}}{\pgfqpoint{0.048611in}{-0.012892in}}{\pgfqpoint{0.048611in}{0.000000in}}%
\pgfpathcurveto{\pgfqpoint{0.048611in}{0.012892in}}{\pgfqpoint{0.043489in}{0.025257in}}{\pgfqpoint{0.034373in}{0.034373in}}%
\pgfpathcurveto{\pgfqpoint{0.025257in}{0.043489in}}{\pgfqpoint{0.012892in}{0.048611in}}{\pgfqpoint{0.000000in}{0.048611in}}%
\pgfpathcurveto{\pgfqpoint{-0.012892in}{0.048611in}}{\pgfqpoint{-0.025257in}{0.043489in}}{\pgfqpoint{-0.034373in}{0.034373in}}%
\pgfpathcurveto{\pgfqpoint{-0.043489in}{0.025257in}}{\pgfqpoint{-0.048611in}{0.012892in}}{\pgfqpoint{-0.048611in}{0.000000in}}%
\pgfpathcurveto{\pgfqpoint{-0.048611in}{-0.012892in}}{\pgfqpoint{-0.043489in}{-0.025257in}}{\pgfqpoint{-0.034373in}{-0.034373in}}%
\pgfpathcurveto{\pgfqpoint{-0.025257in}{-0.043489in}}{\pgfqpoint{-0.012892in}{-0.048611in}}{\pgfqpoint{0.000000in}{-0.048611in}}%
\pgfpathclose%
\pgfusepath{stroke,fill}%
}%
\begin{pgfscope}%
\pgfsys@transformshift{0.785823in}{1.519235in}%
\pgfsys@useobject{currentmarker}{}%
\end{pgfscope}%
\begin{pgfscope}%
\pgfsys@transformshift{0.882929in}{1.636258in}%
\pgfsys@useobject{currentmarker}{}%
\end{pgfscope}%
\begin{pgfscope}%
\pgfsys@transformshift{1.049697in}{2.006813in}%
\pgfsys@useobject{currentmarker}{}%
\end{pgfscope}%
\begin{pgfscope}%
\pgfsys@transformshift{1.246724in}{2.218571in}%
\pgfsys@useobject{currentmarker}{}%
\end{pgfscope}%
\begin{pgfscope}%
\pgfsys@transformshift{1.288240in}{2.309634in}%
\pgfsys@useobject{currentmarker}{}%
\end{pgfscope}%
\begin{pgfscope}%
\pgfsys@transformshift{1.366347in}{2.411478in}%
\pgfsys@useobject{currentmarker}{}%
\end{pgfscope}%
\begin{pgfscope}%
\pgfsys@transformshift{1.527487in}{2.732068in}%
\pgfsys@useobject{currentmarker}{}%
\end{pgfscope}%
\begin{pgfscope}%
\pgfsys@transformshift{1.583076in}{2.711098in}%
\pgfsys@useobject{currentmarker}{}%
\end{pgfscope}%
\begin{pgfscope}%
\pgfsys@transformshift{1.620370in}{2.826988in}%
\pgfsys@useobject{currentmarker}{}%
\end{pgfscope}%
\begin{pgfscope}%
\pgfsys@transformshift{1.772362in}{3.013747in}%
\pgfsys@useobject{currentmarker}{}%
\end{pgfscope}%
\end{pgfscope}%
\begin{pgfscope}%
\pgfpathrectangle{\pgfqpoint{0.644914in}{0.577483in}}{\pgfqpoint{3.100000in}{2.695000in}} %
\pgfusepath{clip}%
\pgfsetbuttcap%
\pgfsetroundjoin%
\pgfsetlinewidth{1.003750pt}%
\definecolor{currentstroke}{rgb}{0.121569,0.466667,0.705882}%
\pgfsetstrokecolor{currentstroke}%
\pgfsetdash{}{0pt}%
\pgfsys@defobject{currentmarker}{\pgfqpoint{-0.048611in}{-0.048611in}}{\pgfqpoint{0.048611in}{0.048611in}}{%
\pgfpathmoveto{\pgfqpoint{0.000000in}{-0.048611in}}%
\pgfpathcurveto{\pgfqpoint{0.012892in}{-0.048611in}}{\pgfqpoint{0.025257in}{-0.043489in}}{\pgfqpoint{0.034373in}{-0.034373in}}%
\pgfpathcurveto{\pgfqpoint{0.043489in}{-0.025257in}}{\pgfqpoint{0.048611in}{-0.012892in}}{\pgfqpoint{0.048611in}{0.000000in}}%
\pgfpathcurveto{\pgfqpoint{0.048611in}{0.012892in}}{\pgfqpoint{0.043489in}{0.025257in}}{\pgfqpoint{0.034373in}{0.034373in}}%
\pgfpathcurveto{\pgfqpoint{0.025257in}{0.043489in}}{\pgfqpoint{0.012892in}{0.048611in}}{\pgfqpoint{0.000000in}{0.048611in}}%
\pgfpathcurveto{\pgfqpoint{-0.012892in}{0.048611in}}{\pgfqpoint{-0.025257in}{0.043489in}}{\pgfqpoint{-0.034373in}{0.034373in}}%
\pgfpathcurveto{\pgfqpoint{-0.043489in}{0.025257in}}{\pgfqpoint{-0.048611in}{0.012892in}}{\pgfqpoint{-0.048611in}{0.000000in}}%
\pgfpathcurveto{\pgfqpoint{-0.048611in}{-0.012892in}}{\pgfqpoint{-0.043489in}{-0.025257in}}{\pgfqpoint{-0.034373in}{-0.034373in}}%
\pgfpathcurveto{\pgfqpoint{-0.025257in}{-0.043489in}}{\pgfqpoint{-0.012892in}{-0.048611in}}{\pgfqpoint{0.000000in}{-0.048611in}}%
\pgfpathclose%
\pgfusepath{stroke}%
}%
\begin{pgfscope}%
\pgfsys@transformshift{1.367051in}{1.449544in}%
\pgfsys@useobject{currentmarker}{}%
\end{pgfscope}%
\begin{pgfscope}%
\pgfsys@transformshift{1.635147in}{1.637758in}%
\pgfsys@useobject{currentmarker}{}%
\end{pgfscope}%
\begin{pgfscope}%
\pgfsys@transformshift{1.913799in}{1.904052in}%
\pgfsys@useobject{currentmarker}{}%
\end{pgfscope}%
\begin{pgfscope}%
\pgfsys@transformshift{2.108715in}{2.111374in}%
\pgfsys@useobject{currentmarker}{}%
\end{pgfscope}%
\begin{pgfscope}%
\pgfsys@transformshift{2.373997in}{1.920243in}%
\pgfsys@useobject{currentmarker}{}%
\end{pgfscope}%
\begin{pgfscope}%
\pgfsys@transformshift{2.590726in}{1.914005in}%
\pgfsys@useobject{currentmarker}{}%
\end{pgfscope}%
\begin{pgfscope}%
\pgfsys@transformshift{2.886265in}{1.876896in}%
\pgfsys@useobject{currentmarker}{}%
\end{pgfscope}%
\begin{pgfscope}%
\pgfsys@transformshift{3.126919in}{1.814865in}%
\pgfsys@useobject{currentmarker}{}%
\end{pgfscope}%
\begin{pgfscope}%
\pgfsys@transformshift{3.345759in}{1.983692in}%
\pgfsys@useobject{currentmarker}{}%
\end{pgfscope}%
\begin{pgfscope}%
\pgfsys@transformshift{3.604005in}{2.146241in}%
\pgfsys@useobject{currentmarker}{}%
\end{pgfscope}%
\end{pgfscope}%
\begin{pgfscope}%
\pgfpathrectangle{\pgfqpoint{0.644914in}{0.577483in}}{\pgfqpoint{3.100000in}{2.695000in}} %
\pgfusepath{clip}%
\pgfsetbuttcap%
\pgfsetroundjoin%
\definecolor{currentfill}{rgb}{0.121569,0.466667,0.705882}%
\pgfsetfillcolor{currentfill}%
\pgfsetlinewidth{1.003750pt}%
\definecolor{currentstroke}{rgb}{0.121569,0.466667,0.705882}%
\pgfsetstrokecolor{currentstroke}%
\pgfsetdash{}{0pt}%
\pgfsys@defobject{currentmarker}{\pgfqpoint{-0.048611in}{-0.048611in}}{\pgfqpoint{0.048611in}{0.048611in}}{%
\pgfpathmoveto{\pgfqpoint{0.000000in}{-0.048611in}}%
\pgfpathcurveto{\pgfqpoint{0.012892in}{-0.048611in}}{\pgfqpoint{0.025257in}{-0.043489in}}{\pgfqpoint{0.034373in}{-0.034373in}}%
\pgfpathcurveto{\pgfqpoint{0.043489in}{-0.025257in}}{\pgfqpoint{0.048611in}{-0.012892in}}{\pgfqpoint{0.048611in}{0.000000in}}%
\pgfpathcurveto{\pgfqpoint{0.048611in}{0.012892in}}{\pgfqpoint{0.043489in}{0.025257in}}{\pgfqpoint{0.034373in}{0.034373in}}%
\pgfpathcurveto{\pgfqpoint{0.025257in}{0.043489in}}{\pgfqpoint{0.012892in}{0.048611in}}{\pgfqpoint{0.000000in}{0.048611in}}%
\pgfpathcurveto{\pgfqpoint{-0.012892in}{0.048611in}}{\pgfqpoint{-0.025257in}{0.043489in}}{\pgfqpoint{-0.034373in}{0.034373in}}%
\pgfpathcurveto{\pgfqpoint{-0.043489in}{0.025257in}}{\pgfqpoint{-0.048611in}{0.012892in}}{\pgfqpoint{-0.048611in}{0.000000in}}%
\pgfpathcurveto{\pgfqpoint{-0.048611in}{-0.012892in}}{\pgfqpoint{-0.043489in}{-0.025257in}}{\pgfqpoint{-0.034373in}{-0.034373in}}%
\pgfpathcurveto{\pgfqpoint{-0.025257in}{-0.043489in}}{\pgfqpoint{-0.012892in}{-0.048611in}}{\pgfqpoint{0.000000in}{-0.048611in}}%
\pgfpathclose%
\pgfusepath{stroke,fill}%
}%
\begin{pgfscope}%
\pgfsys@transformshift{0.785823in}{1.519235in}%
\pgfsys@useobject{currentmarker}{}%
\end{pgfscope}%
\begin{pgfscope}%
\pgfsys@transformshift{0.882929in}{1.636258in}%
\pgfsys@useobject{currentmarker}{}%
\end{pgfscope}%
\begin{pgfscope}%
\pgfsys@transformshift{1.049697in}{2.006813in}%
\pgfsys@useobject{currentmarker}{}%
\end{pgfscope}%
\begin{pgfscope}%
\pgfsys@transformshift{1.246724in}{2.218571in}%
\pgfsys@useobject{currentmarker}{}%
\end{pgfscope}%
\begin{pgfscope}%
\pgfsys@transformshift{1.288240in}{2.309634in}%
\pgfsys@useobject{currentmarker}{}%
\end{pgfscope}%
\begin{pgfscope}%
\pgfsys@transformshift{1.366347in}{2.411478in}%
\pgfsys@useobject{currentmarker}{}%
\end{pgfscope}%
\begin{pgfscope}%
\pgfsys@transformshift{1.527487in}{2.732068in}%
\pgfsys@useobject{currentmarker}{}%
\end{pgfscope}%
\begin{pgfscope}%
\pgfsys@transformshift{1.583076in}{2.711098in}%
\pgfsys@useobject{currentmarker}{}%
\end{pgfscope}%
\begin{pgfscope}%
\pgfsys@transformshift{1.620370in}{2.826988in}%
\pgfsys@useobject{currentmarker}{}%
\end{pgfscope}%
\begin{pgfscope}%
\pgfsys@transformshift{1.772362in}{3.013747in}%
\pgfsys@useobject{currentmarker}{}%
\end{pgfscope}%
\end{pgfscope}%
\begin{pgfscope}%
\pgfpathrectangle{\pgfqpoint{0.644914in}{0.577483in}}{\pgfqpoint{3.100000in}{2.695000in}} %
\pgfusepath{clip}%
\pgfsetbuttcap%
\pgfsetroundjoin%
\pgfsetlinewidth{1.003750pt}%
\definecolor{currentstroke}{rgb}{0.121569,0.466667,0.705882}%
\pgfsetstrokecolor{currentstroke}%
\pgfsetdash{}{0pt}%
\pgfsys@defobject{currentmarker}{\pgfqpoint{-0.048611in}{-0.048611in}}{\pgfqpoint{0.048611in}{0.048611in}}{%
\pgfpathmoveto{\pgfqpoint{0.000000in}{-0.048611in}}%
\pgfpathcurveto{\pgfqpoint{0.012892in}{-0.048611in}}{\pgfqpoint{0.025257in}{-0.043489in}}{\pgfqpoint{0.034373in}{-0.034373in}}%
\pgfpathcurveto{\pgfqpoint{0.043489in}{-0.025257in}}{\pgfqpoint{0.048611in}{-0.012892in}}{\pgfqpoint{0.048611in}{0.000000in}}%
\pgfpathcurveto{\pgfqpoint{0.048611in}{0.012892in}}{\pgfqpoint{0.043489in}{0.025257in}}{\pgfqpoint{0.034373in}{0.034373in}}%
\pgfpathcurveto{\pgfqpoint{0.025257in}{0.043489in}}{\pgfqpoint{0.012892in}{0.048611in}}{\pgfqpoint{0.000000in}{0.048611in}}%
\pgfpathcurveto{\pgfqpoint{-0.012892in}{0.048611in}}{\pgfqpoint{-0.025257in}{0.043489in}}{\pgfqpoint{-0.034373in}{0.034373in}}%
\pgfpathcurveto{\pgfqpoint{-0.043489in}{0.025257in}}{\pgfqpoint{-0.048611in}{0.012892in}}{\pgfqpoint{-0.048611in}{0.000000in}}%
\pgfpathcurveto{\pgfqpoint{-0.048611in}{-0.012892in}}{\pgfqpoint{-0.043489in}{-0.025257in}}{\pgfqpoint{-0.034373in}{-0.034373in}}%
\pgfpathcurveto{\pgfqpoint{-0.025257in}{-0.043489in}}{\pgfqpoint{-0.012892in}{-0.048611in}}{\pgfqpoint{0.000000in}{-0.048611in}}%
\pgfpathclose%
\pgfusepath{stroke}%
}%
\begin{pgfscope}%
\pgfsys@transformshift{1.367051in}{1.449544in}%
\pgfsys@useobject{currentmarker}{}%
\end{pgfscope}%
\begin{pgfscope}%
\pgfsys@transformshift{1.635147in}{1.637758in}%
\pgfsys@useobject{currentmarker}{}%
\end{pgfscope}%
\begin{pgfscope}%
\pgfsys@transformshift{1.913799in}{1.904052in}%
\pgfsys@useobject{currentmarker}{}%
\end{pgfscope}%
\begin{pgfscope}%
\pgfsys@transformshift{2.108715in}{2.111374in}%
\pgfsys@useobject{currentmarker}{}%
\end{pgfscope}%
\begin{pgfscope}%
\pgfsys@transformshift{2.373997in}{1.920243in}%
\pgfsys@useobject{currentmarker}{}%
\end{pgfscope}%
\begin{pgfscope}%
\pgfsys@transformshift{2.590726in}{1.914005in}%
\pgfsys@useobject{currentmarker}{}%
\end{pgfscope}%
\begin{pgfscope}%
\pgfsys@transformshift{2.886265in}{1.876896in}%
\pgfsys@useobject{currentmarker}{}%
\end{pgfscope}%
\begin{pgfscope}%
\pgfsys@transformshift{3.126919in}{1.814865in}%
\pgfsys@useobject{currentmarker}{}%
\end{pgfscope}%
\begin{pgfscope}%
\pgfsys@transformshift{3.345759in}{1.983692in}%
\pgfsys@useobject{currentmarker}{}%
\end{pgfscope}%
\begin{pgfscope}%
\pgfsys@transformshift{3.604005in}{2.146241in}%
\pgfsys@useobject{currentmarker}{}%
\end{pgfscope}%
\end{pgfscope}%
\begin{pgfscope}%
\pgfpathrectangle{\pgfqpoint{0.644914in}{0.577483in}}{\pgfqpoint{3.100000in}{2.695000in}} %
\pgfusepath{clip}%
\pgfsetbuttcap%
\pgfsetroundjoin%
\definecolor{currentfill}{rgb}{0.121569,0.466667,0.705882}%
\pgfsetfillcolor{currentfill}%
\pgfsetlinewidth{1.003750pt}%
\definecolor{currentstroke}{rgb}{0.121569,0.466667,0.705882}%
\pgfsetstrokecolor{currentstroke}%
\pgfsetdash{}{0pt}%
\pgfsys@defobject{currentmarker}{\pgfqpoint{-0.048611in}{-0.048611in}}{\pgfqpoint{0.048611in}{0.048611in}}{%
\pgfpathmoveto{\pgfqpoint{0.000000in}{-0.048611in}}%
\pgfpathcurveto{\pgfqpoint{0.012892in}{-0.048611in}}{\pgfqpoint{0.025257in}{-0.043489in}}{\pgfqpoint{0.034373in}{-0.034373in}}%
\pgfpathcurveto{\pgfqpoint{0.043489in}{-0.025257in}}{\pgfqpoint{0.048611in}{-0.012892in}}{\pgfqpoint{0.048611in}{0.000000in}}%
\pgfpathcurveto{\pgfqpoint{0.048611in}{0.012892in}}{\pgfqpoint{0.043489in}{0.025257in}}{\pgfqpoint{0.034373in}{0.034373in}}%
\pgfpathcurveto{\pgfqpoint{0.025257in}{0.043489in}}{\pgfqpoint{0.012892in}{0.048611in}}{\pgfqpoint{0.000000in}{0.048611in}}%
\pgfpathcurveto{\pgfqpoint{-0.012892in}{0.048611in}}{\pgfqpoint{-0.025257in}{0.043489in}}{\pgfqpoint{-0.034373in}{0.034373in}}%
\pgfpathcurveto{\pgfqpoint{-0.043489in}{0.025257in}}{\pgfqpoint{-0.048611in}{0.012892in}}{\pgfqpoint{-0.048611in}{0.000000in}}%
\pgfpathcurveto{\pgfqpoint{-0.048611in}{-0.012892in}}{\pgfqpoint{-0.043489in}{-0.025257in}}{\pgfqpoint{-0.034373in}{-0.034373in}}%
\pgfpathcurveto{\pgfqpoint{-0.025257in}{-0.043489in}}{\pgfqpoint{-0.012892in}{-0.048611in}}{\pgfqpoint{0.000000in}{-0.048611in}}%
\pgfpathclose%
\pgfusepath{stroke,fill}%
}%
\begin{pgfscope}%
\pgfsys@transformshift{0.785823in}{1.519235in}%
\pgfsys@useobject{currentmarker}{}%
\end{pgfscope}%
\begin{pgfscope}%
\pgfsys@transformshift{0.882929in}{1.636258in}%
\pgfsys@useobject{currentmarker}{}%
\end{pgfscope}%
\begin{pgfscope}%
\pgfsys@transformshift{1.049697in}{2.006813in}%
\pgfsys@useobject{currentmarker}{}%
\end{pgfscope}%
\begin{pgfscope}%
\pgfsys@transformshift{1.246724in}{2.218571in}%
\pgfsys@useobject{currentmarker}{}%
\end{pgfscope}%
\begin{pgfscope}%
\pgfsys@transformshift{1.288240in}{2.309634in}%
\pgfsys@useobject{currentmarker}{}%
\end{pgfscope}%
\begin{pgfscope}%
\pgfsys@transformshift{1.366347in}{2.411478in}%
\pgfsys@useobject{currentmarker}{}%
\end{pgfscope}%
\begin{pgfscope}%
\pgfsys@transformshift{1.527487in}{2.732068in}%
\pgfsys@useobject{currentmarker}{}%
\end{pgfscope}%
\begin{pgfscope}%
\pgfsys@transformshift{1.583076in}{2.711098in}%
\pgfsys@useobject{currentmarker}{}%
\end{pgfscope}%
\begin{pgfscope}%
\pgfsys@transformshift{1.620370in}{2.826988in}%
\pgfsys@useobject{currentmarker}{}%
\end{pgfscope}%
\begin{pgfscope}%
\pgfsys@transformshift{1.772362in}{3.013747in}%
\pgfsys@useobject{currentmarker}{}%
\end{pgfscope}%
\end{pgfscope}%
\begin{pgfscope}%
\pgfpathrectangle{\pgfqpoint{0.644914in}{0.577483in}}{\pgfqpoint{3.100000in}{2.695000in}} %
\pgfusepath{clip}%
\pgfsetbuttcap%
\pgfsetroundjoin%
\pgfsetlinewidth{1.003750pt}%
\definecolor{currentstroke}{rgb}{0.121569,0.466667,0.705882}%
\pgfsetstrokecolor{currentstroke}%
\pgfsetdash{}{0pt}%
\pgfsys@defobject{currentmarker}{\pgfqpoint{-0.048611in}{-0.048611in}}{\pgfqpoint{0.048611in}{0.048611in}}{%
\pgfpathmoveto{\pgfqpoint{0.000000in}{-0.048611in}}%
\pgfpathcurveto{\pgfqpoint{0.012892in}{-0.048611in}}{\pgfqpoint{0.025257in}{-0.043489in}}{\pgfqpoint{0.034373in}{-0.034373in}}%
\pgfpathcurveto{\pgfqpoint{0.043489in}{-0.025257in}}{\pgfqpoint{0.048611in}{-0.012892in}}{\pgfqpoint{0.048611in}{0.000000in}}%
\pgfpathcurveto{\pgfqpoint{0.048611in}{0.012892in}}{\pgfqpoint{0.043489in}{0.025257in}}{\pgfqpoint{0.034373in}{0.034373in}}%
\pgfpathcurveto{\pgfqpoint{0.025257in}{0.043489in}}{\pgfqpoint{0.012892in}{0.048611in}}{\pgfqpoint{0.000000in}{0.048611in}}%
\pgfpathcurveto{\pgfqpoint{-0.012892in}{0.048611in}}{\pgfqpoint{-0.025257in}{0.043489in}}{\pgfqpoint{-0.034373in}{0.034373in}}%
\pgfpathcurveto{\pgfqpoint{-0.043489in}{0.025257in}}{\pgfqpoint{-0.048611in}{0.012892in}}{\pgfqpoint{-0.048611in}{0.000000in}}%
\pgfpathcurveto{\pgfqpoint{-0.048611in}{-0.012892in}}{\pgfqpoint{-0.043489in}{-0.025257in}}{\pgfqpoint{-0.034373in}{-0.034373in}}%
\pgfpathcurveto{\pgfqpoint{-0.025257in}{-0.043489in}}{\pgfqpoint{-0.012892in}{-0.048611in}}{\pgfqpoint{0.000000in}{-0.048611in}}%
\pgfpathclose%
\pgfusepath{stroke}%
}%
\begin{pgfscope}%
\pgfsys@transformshift{1.367051in}{1.449544in}%
\pgfsys@useobject{currentmarker}{}%
\end{pgfscope}%
\begin{pgfscope}%
\pgfsys@transformshift{1.635147in}{1.637758in}%
\pgfsys@useobject{currentmarker}{}%
\end{pgfscope}%
\begin{pgfscope}%
\pgfsys@transformshift{1.913799in}{1.904052in}%
\pgfsys@useobject{currentmarker}{}%
\end{pgfscope}%
\begin{pgfscope}%
\pgfsys@transformshift{2.108715in}{2.111374in}%
\pgfsys@useobject{currentmarker}{}%
\end{pgfscope}%
\begin{pgfscope}%
\pgfsys@transformshift{2.373997in}{1.920243in}%
\pgfsys@useobject{currentmarker}{}%
\end{pgfscope}%
\begin{pgfscope}%
\pgfsys@transformshift{2.590726in}{1.914005in}%
\pgfsys@useobject{currentmarker}{}%
\end{pgfscope}%
\begin{pgfscope}%
\pgfsys@transformshift{2.886265in}{1.876896in}%
\pgfsys@useobject{currentmarker}{}%
\end{pgfscope}%
\begin{pgfscope}%
\pgfsys@transformshift{3.126919in}{1.814865in}%
\pgfsys@useobject{currentmarker}{}%
\end{pgfscope}%
\begin{pgfscope}%
\pgfsys@transformshift{3.345759in}{1.983692in}%
\pgfsys@useobject{currentmarker}{}%
\end{pgfscope}%
\begin{pgfscope}%
\pgfsys@transformshift{3.604005in}{2.146241in}%
\pgfsys@useobject{currentmarker}{}%
\end{pgfscope}%
\end{pgfscope}%
\begin{pgfscope}%
\pgfsetrectcap%
\pgfsetmiterjoin%
\pgfsetlinewidth{0.803000pt}%
\definecolor{currentstroke}{rgb}{0.000000,0.000000,0.000000}%
\pgfsetstrokecolor{currentstroke}%
\pgfsetdash{}{0pt}%
\pgfpathmoveto{\pgfqpoint{0.644914in}{0.577483in}}%
\pgfpathlineto{\pgfqpoint{0.644914in}{3.272483in}}%
\pgfusepath{stroke}%
\end{pgfscope}%
\begin{pgfscope}%
\pgfsetrectcap%
\pgfsetmiterjoin%
\pgfsetlinewidth{0.803000pt}%
\definecolor{currentstroke}{rgb}{0.000000,0.000000,0.000000}%
\pgfsetstrokecolor{currentstroke}%
\pgfsetdash{}{0pt}%
\pgfpathmoveto{\pgfqpoint{3.744914in}{0.577483in}}%
\pgfpathlineto{\pgfqpoint{3.744914in}{3.272483in}}%
\pgfusepath{stroke}%
\end{pgfscope}%
\begin{pgfscope}%
\pgfsetrectcap%
\pgfsetmiterjoin%
\pgfsetlinewidth{0.803000pt}%
\definecolor{currentstroke}{rgb}{0.000000,0.000000,0.000000}%
\pgfsetstrokecolor{currentstroke}%
\pgfsetdash{}{0pt}%
\pgfpathmoveto{\pgfqpoint{0.644914in}{0.577483in}}%
\pgfpathlineto{\pgfqpoint{3.744914in}{0.577483in}}%
\pgfusepath{stroke}%
\end{pgfscope}%
\begin{pgfscope}%
\pgfsetrectcap%
\pgfsetmiterjoin%
\pgfsetlinewidth{0.803000pt}%
\definecolor{currentstroke}{rgb}{0.000000,0.000000,0.000000}%
\pgfsetstrokecolor{currentstroke}%
\pgfsetdash{}{0pt}%
\pgfpathmoveto{\pgfqpoint{0.644914in}{3.272483in}}%
\pgfpathlineto{\pgfqpoint{3.744914in}{3.272483in}}%
\pgfusepath{stroke}%
\end{pgfscope}%
\begin{pgfscope}%
\pgfsetbuttcap%
\pgfsetmiterjoin%
\pgfsetlinewidth{0.000000pt}%
\definecolor{currentstroke}{rgb}{0.800000,0.800000,0.800000}%
\pgfsetstrokecolor{currentstroke}%
\pgfsetstrokeopacity{0.000000}%
\pgfsetdash{}{0pt}%
\pgfpathmoveto{\pgfqpoint{1.863679in}{2.177697in}}%
\pgfpathlineto{\pgfqpoint{3.675469in}{2.177697in}}%
\pgfpathlineto{\pgfqpoint{3.675469in}{3.203038in}}%
\pgfpathlineto{\pgfqpoint{1.863679in}{3.203038in}}%
\pgfpathclose%
\pgfusepath{}%
\end{pgfscope}%
\begin{pgfscope}%
\pgfsetbuttcap%
\pgfsetroundjoin%
\definecolor{currentfill}{rgb}{0.000000,0.000000,0.000000}%
\pgfsetfillcolor{currentfill}%
\pgfsetlinewidth{1.003750pt}%
\definecolor{currentstroke}{rgb}{0.000000,0.000000,0.000000}%
\pgfsetstrokecolor{currentstroke}%
\pgfsetdash{}{0pt}%
\pgfsys@defobject{currentmarker}{\pgfqpoint{-0.041667in}{-0.041667in}}{\pgfqpoint{0.041667in}{0.041667in}}{%
\pgfpathmoveto{\pgfqpoint{0.000000in}{-0.041667in}}%
\pgfpathcurveto{\pgfqpoint{0.011050in}{-0.041667in}}{\pgfqpoint{0.021649in}{-0.037276in}}{\pgfqpoint{0.029463in}{-0.029463in}}%
\pgfpathcurveto{\pgfqpoint{0.037276in}{-0.021649in}}{\pgfqpoint{0.041667in}{-0.011050in}}{\pgfqpoint{0.041667in}{0.000000in}}%
\pgfpathcurveto{\pgfqpoint{0.041667in}{0.011050in}}{\pgfqpoint{0.037276in}{0.021649in}}{\pgfqpoint{0.029463in}{0.029463in}}%
\pgfpathcurveto{\pgfqpoint{0.021649in}{0.037276in}}{\pgfqpoint{0.011050in}{0.041667in}}{\pgfqpoint{0.000000in}{0.041667in}}%
\pgfpathcurveto{\pgfqpoint{-0.011050in}{0.041667in}}{\pgfqpoint{-0.021649in}{0.037276in}}{\pgfqpoint{-0.029463in}{0.029463in}}%
\pgfpathcurveto{\pgfqpoint{-0.037276in}{0.021649in}}{\pgfqpoint{-0.041667in}{0.011050in}}{\pgfqpoint{-0.041667in}{0.000000in}}%
\pgfpathcurveto{\pgfqpoint{-0.041667in}{-0.011050in}}{\pgfqpoint{-0.037276in}{-0.021649in}}{\pgfqpoint{-0.029463in}{-0.029463in}}%
\pgfpathcurveto{\pgfqpoint{-0.021649in}{-0.037276in}}{\pgfqpoint{-0.011050in}{-0.041667in}}{\pgfqpoint{0.000000in}{-0.041667in}}%
\pgfpathclose%
\pgfusepath{stroke,fill}%
}%
\begin{pgfscope}%
\pgfsys@transformshift{2.058123in}{3.098872in}%
\pgfsys@useobject{currentmarker}{}%
\end{pgfscope}%
\end{pgfscope}%
\begin{pgfscope}%
\pgftext[x=2.308123in,y=3.050261in,left,base]{\rmfamily\fontsize{10.000000}{12.000000}\selectfont \(\displaystyle P_C = 15\ \text{bar}\)}%
\end{pgfscope}%
\begin{pgfscope}%
\pgfsetbuttcap%
\pgfsetroundjoin%
\pgfsetlinewidth{1.003750pt}%
\definecolor{currentstroke}{rgb}{0.000000,0.000000,0.000000}%
\pgfsetstrokecolor{currentstroke}%
\pgfsetdash{}{0pt}%
\pgfsys@defobject{currentmarker}{\pgfqpoint{-0.041667in}{-0.041667in}}{\pgfqpoint{0.041667in}{0.041667in}}{%
\pgfpathmoveto{\pgfqpoint{0.000000in}{-0.041667in}}%
\pgfpathcurveto{\pgfqpoint{0.011050in}{-0.041667in}}{\pgfqpoint{0.021649in}{-0.037276in}}{\pgfqpoint{0.029463in}{-0.029463in}}%
\pgfpathcurveto{\pgfqpoint{0.037276in}{-0.021649in}}{\pgfqpoint{0.041667in}{-0.011050in}}{\pgfqpoint{0.041667in}{0.000000in}}%
\pgfpathcurveto{\pgfqpoint{0.041667in}{0.011050in}}{\pgfqpoint{0.037276in}{0.021649in}}{\pgfqpoint{0.029463in}{0.029463in}}%
\pgfpathcurveto{\pgfqpoint{0.021649in}{0.037276in}}{\pgfqpoint{0.011050in}{0.041667in}}{\pgfqpoint{0.000000in}{0.041667in}}%
\pgfpathcurveto{\pgfqpoint{-0.011050in}{0.041667in}}{\pgfqpoint{-0.021649in}{0.037276in}}{\pgfqpoint{-0.029463in}{0.029463in}}%
\pgfpathcurveto{\pgfqpoint{-0.037276in}{0.021649in}}{\pgfqpoint{-0.041667in}{0.011050in}}{\pgfqpoint{-0.041667in}{0.000000in}}%
\pgfpathcurveto{\pgfqpoint{-0.041667in}{-0.011050in}}{\pgfqpoint{-0.037276in}{-0.021649in}}{\pgfqpoint{-0.029463in}{-0.029463in}}%
\pgfpathcurveto{\pgfqpoint{-0.021649in}{-0.037276in}}{\pgfqpoint{-0.011050in}{-0.041667in}}{\pgfqpoint{0.000000in}{-0.041667in}}%
\pgfpathclose%
\pgfusepath{stroke}%
}%
\begin{pgfscope}%
\pgfsys@transformshift{2.058123in}{2.902137in}%
\pgfsys@useobject{currentmarker}{}%
\end{pgfscope}%
\end{pgfscope}%
\begin{pgfscope}%
\pgftext[x=2.308123in,y=2.853526in,left,base]{\rmfamily\fontsize{10.000000}{12.000000}\selectfont \(\displaystyle P_C = 30\ \text{bar}\)}%
\end{pgfscope}%
\begin{pgfscope}%
\pgfsetrectcap%
\pgfsetroundjoin%
\pgfsetlinewidth{1.505625pt}%
\definecolor{currentstroke}{rgb}{0.121569,0.466667,0.705882}%
\pgfsetstrokecolor{currentstroke}%
\pgfsetdash{}{0pt}%
\pgfpathmoveto{\pgfqpoint{1.919234in}{2.705402in}}%
\pgfpathlineto{\pgfqpoint{2.197012in}{2.705402in}}%
\pgfusepath{stroke}%
\end{pgfscope}%
\begin{pgfscope}%
\pgftext[x=2.308123in,y=2.656791in,left,base]{\rmfamily\fontsize{10.000000}{12.000000}\selectfont Experimental Data}%
\end{pgfscope}%
\begin{pgfscope}%
\pgfsetrectcap%
\pgfsetroundjoin%
\pgfsetlinewidth{1.505625pt}%
\definecolor{currentstroke}{rgb}{0.172549,0.627451,0.172549}%
\pgfsetstrokecolor{currentstroke}%
\pgfsetdash{}{0pt}%
\pgfpathmoveto{\pgfqpoint{1.919234in}{2.508667in}}%
\pgfpathlineto{\pgfqpoint{2.197012in}{2.508667in}}%
\pgfusepath{stroke}%
\end{pgfscope}%
\begin{pgfscope}%
\pgftext[x=2.308123in,y=2.460056in,left,base]{\rmfamily\fontsize{10.000000}{12.000000}\selectfont Di\'evart et al.\ \cite{Dievart2013} Model}%
\end{pgfscope}%
\begin{pgfscope}%
\pgfsetrectcap%
\pgfsetroundjoin%
\pgfsetlinewidth{1.505625pt}%
\definecolor{currentstroke}{rgb}{1.000000,0.498039,0.054902}%
\pgfsetstrokecolor{currentstroke}%
\pgfsetdash{}{0pt}%
\pgfpathmoveto{\pgfqpoint{1.919234in}{2.311932in}}%
\pgfpathlineto{\pgfqpoint{2.197012in}{2.311932in}}%
\pgfusepath{stroke}%
\end{pgfscope}%
\begin{pgfscope}%
\pgftext[x=2.308123in,y=2.263321in,left,base]{\rmfamily\fontsize{10.000000}{12.000000}\selectfont RMG Model}%
\end{pgfscope}%
\begin{pgfscope}%
\pgfsetbuttcap%
\pgfsetroundjoin%
\definecolor{currentfill}{rgb}{0.000000,0.000000,0.000000}%
\pgfsetfillcolor{currentfill}%
\pgfsetlinewidth{1.003750pt}%
\definecolor{currentstroke}{rgb}{0.000000,0.000000,0.000000}%
\pgfsetstrokecolor{currentstroke}%
\pgfsetdash{}{0pt}%
\pgfsys@defobject{currentmarker}{\pgfqpoint{0.000000in}{0.000000in}}{\pgfqpoint{0.000000in}{0.069444in}}{%
\pgfpathmoveto{\pgfqpoint{0.000000in}{0.000000in}}%
\pgfpathlineto{\pgfqpoint{0.000000in}{0.069444in}}%
\pgfusepath{stroke,fill}%
}%
\begin{pgfscope}%
\pgfsys@transformshift{1.113106in}{3.272483in}%
\pgfsys@useobject{currentmarker}{}%
\end{pgfscope}%
\end{pgfscope}%
\begin{pgfscope}%
\pgftext[x=1.113106in,y=3.390538in,,bottom]{\rmfamily\fontsize{10.000000}{12.000000}\selectfont 900 K}%
\end{pgfscope}%
\begin{pgfscope}%
\pgfsetbuttcap%
\pgfsetroundjoin%
\definecolor{currentfill}{rgb}{0.000000,0.000000,0.000000}%
\pgfsetfillcolor{currentfill}%
\pgfsetlinewidth{1.003750pt}%
\definecolor{currentstroke}{rgb}{0.000000,0.000000,0.000000}%
\pgfsetstrokecolor{currentstroke}%
\pgfsetdash{}{0pt}%
\pgfsys@defobject{currentmarker}{\pgfqpoint{0.000000in}{0.000000in}}{\pgfqpoint{0.000000in}{0.069444in}}{%
\pgfpathmoveto{\pgfqpoint{0.000000in}{0.000000in}}%
\pgfpathlineto{\pgfqpoint{0.000000in}{0.069444in}}%
\pgfusepath{stroke,fill}%
}%
\begin{pgfscope}%
\pgfsys@transformshift{2.090419in}{3.272483in}%
\pgfsys@useobject{currentmarker}{}%
\end{pgfscope}%
\end{pgfscope}%
\begin{pgfscope}%
\pgftext[x=2.090419in,y=3.390538in,,bottom]{\rmfamily\fontsize{10.000000}{12.000000}\selectfont 800 K}%
\end{pgfscope}%
\begin{pgfscope}%
\pgfsetbuttcap%
\pgfsetroundjoin%
\definecolor{currentfill}{rgb}{0.000000,0.000000,0.000000}%
\pgfsetfillcolor{currentfill}%
\pgfsetlinewidth{1.003750pt}%
\definecolor{currentstroke}{rgb}{0.000000,0.000000,0.000000}%
\pgfsetstrokecolor{currentstroke}%
\pgfsetdash{}{0pt}%
\pgfsys@defobject{currentmarker}{\pgfqpoint{0.000000in}{0.000000in}}{\pgfqpoint{0.000000in}{0.069444in}}{%
\pgfpathmoveto{\pgfqpoint{0.000000in}{0.000000in}}%
\pgfpathlineto{\pgfqpoint{0.000000in}{0.069444in}}%
\pgfusepath{stroke,fill}%
}%
\begin{pgfscope}%
\pgfsys@transformshift{3.346965in}{3.272483in}%
\pgfsys@useobject{currentmarker}{}%
\end{pgfscope}%
\end{pgfscope}%
\begin{pgfscope}%
\pgftext[x=3.346965in,y=3.390538in,,bottom]{\rmfamily\fontsize{10.000000}{12.000000}\selectfont 700 K}%
\end{pgfscope}%
\begin{pgfscope}%
\pgfsetbuttcap%
\pgfsetroundjoin%
\definecolor{currentfill}{rgb}{0.000000,0.000000,0.000000}%
\pgfsetfillcolor{currentfill}%
\pgfsetlinewidth{1.003750pt}%
\definecolor{currentstroke}{rgb}{0.000000,0.000000,0.000000}%
\pgfsetstrokecolor{currentstroke}%
\pgfsetdash{}{0pt}%
\pgfsys@defobject{currentmarker}{\pgfqpoint{0.000000in}{0.000000in}}{\pgfqpoint{0.000000in}{0.034722in}}{%
\pgfpathmoveto{\pgfqpoint{0.000000in}{0.000000in}}%
\pgfpathlineto{\pgfqpoint{0.000000in}{0.034722in}}%
\pgfusepath{stroke,fill}%
}%
\begin{pgfscope}%
\pgfsys@transformshift{0.780403in}{3.272483in}%
\pgfsys@useobject{currentmarker}{}%
\end{pgfscope}%
\end{pgfscope}%
\begin{pgfscope}%
\pgfsetbuttcap%
\pgfsetroundjoin%
\definecolor{currentfill}{rgb}{0.000000,0.000000,0.000000}%
\pgfsetfillcolor{currentfill}%
\pgfsetlinewidth{1.003750pt}%
\definecolor{currentstroke}{rgb}{0.000000,0.000000,0.000000}%
\pgfsetstrokecolor{currentstroke}%
\pgfsetdash{}{0pt}%
\pgfsys@defobject{currentmarker}{\pgfqpoint{0.000000in}{0.000000in}}{\pgfqpoint{0.000000in}{0.034722in}}{%
\pgfpathmoveto{\pgfqpoint{0.000000in}{0.000000in}}%
\pgfpathlineto{\pgfqpoint{0.000000in}{0.034722in}}%
\pgfusepath{stroke,fill}%
}%
\begin{pgfscope}%
\pgfsys@transformshift{0.943138in}{3.272483in}%
\pgfsys@useobject{currentmarker}{}%
\end{pgfscope}%
\end{pgfscope}%
\begin{pgfscope}%
\pgfsetbuttcap%
\pgfsetroundjoin%
\definecolor{currentfill}{rgb}{0.000000,0.000000,0.000000}%
\pgfsetfillcolor{currentfill}%
\pgfsetlinewidth{1.003750pt}%
\definecolor{currentstroke}{rgb}{0.000000,0.000000,0.000000}%
\pgfsetstrokecolor{currentstroke}%
\pgfsetdash{}{0pt}%
\pgfsys@defobject{currentmarker}{\pgfqpoint{0.000000in}{0.000000in}}{\pgfqpoint{0.000000in}{0.034722in}}{%
\pgfpathmoveto{\pgfqpoint{0.000000in}{0.000000in}}%
\pgfpathlineto{\pgfqpoint{0.000000in}{0.034722in}}%
\pgfusepath{stroke,fill}%
}%
\begin{pgfscope}%
\pgfsys@transformshift{1.113106in}{3.272483in}%
\pgfsys@useobject{currentmarker}{}%
\end{pgfscope}%
\end{pgfscope}%
\begin{pgfscope}%
\pgfsetbuttcap%
\pgfsetroundjoin%
\definecolor{currentfill}{rgb}{0.000000,0.000000,0.000000}%
\pgfsetfillcolor{currentfill}%
\pgfsetlinewidth{1.003750pt}%
\definecolor{currentstroke}{rgb}{0.000000,0.000000,0.000000}%
\pgfsetstrokecolor{currentstroke}%
\pgfsetdash{}{0pt}%
\pgfsys@defobject{currentmarker}{\pgfqpoint{0.000000in}{0.000000in}}{\pgfqpoint{0.000000in}{0.034722in}}{%
\pgfpathmoveto{\pgfqpoint{0.000000in}{0.000000in}}%
\pgfpathlineto{\pgfqpoint{0.000000in}{0.034722in}}%
\pgfusepath{stroke,fill}%
}%
\begin{pgfscope}%
\pgfsys@transformshift{1.290799in}{3.272483in}%
\pgfsys@useobject{currentmarker}{}%
\end{pgfscope}%
\end{pgfscope}%
\begin{pgfscope}%
\pgfsetbuttcap%
\pgfsetroundjoin%
\definecolor{currentfill}{rgb}{0.000000,0.000000,0.000000}%
\pgfsetfillcolor{currentfill}%
\pgfsetlinewidth{1.003750pt}%
\definecolor{currentstroke}{rgb}{0.000000,0.000000,0.000000}%
\pgfsetstrokecolor{currentstroke}%
\pgfsetdash{}{0pt}%
\pgfsys@defobject{currentmarker}{\pgfqpoint{0.000000in}{0.000000in}}{\pgfqpoint{0.000000in}{0.034722in}}{%
\pgfpathmoveto{\pgfqpoint{0.000000in}{0.000000in}}%
\pgfpathlineto{\pgfqpoint{0.000000in}{0.034722in}}%
\pgfusepath{stroke,fill}%
}%
\begin{pgfscope}%
\pgfsys@transformshift{1.476757in}{3.272483in}%
\pgfsys@useobject{currentmarker}{}%
\end{pgfscope}%
\end{pgfscope}%
\begin{pgfscope}%
\pgfsetbuttcap%
\pgfsetroundjoin%
\definecolor{currentfill}{rgb}{0.000000,0.000000,0.000000}%
\pgfsetfillcolor{currentfill}%
\pgfsetlinewidth{1.003750pt}%
\definecolor{currentstroke}{rgb}{0.000000,0.000000,0.000000}%
\pgfsetstrokecolor{currentstroke}%
\pgfsetdash{}{0pt}%
\pgfsys@defobject{currentmarker}{\pgfqpoint{0.000000in}{0.000000in}}{\pgfqpoint{0.000000in}{0.034722in}}{%
\pgfpathmoveto{\pgfqpoint{0.000000in}{0.000000in}}%
\pgfpathlineto{\pgfqpoint{0.000000in}{0.034722in}}%
\pgfusepath{stroke,fill}%
}%
\begin{pgfscope}%
\pgfsys@transformshift{1.671570in}{3.272483in}%
\pgfsys@useobject{currentmarker}{}%
\end{pgfscope}%
\end{pgfscope}%
\begin{pgfscope}%
\pgfsetbuttcap%
\pgfsetroundjoin%
\definecolor{currentfill}{rgb}{0.000000,0.000000,0.000000}%
\pgfsetfillcolor{currentfill}%
\pgfsetlinewidth{1.003750pt}%
\definecolor{currentstroke}{rgb}{0.000000,0.000000,0.000000}%
\pgfsetstrokecolor{currentstroke}%
\pgfsetdash{}{0pt}%
\pgfsys@defobject{currentmarker}{\pgfqpoint{0.000000in}{0.000000in}}{\pgfqpoint{0.000000in}{0.034722in}}{%
\pgfpathmoveto{\pgfqpoint{0.000000in}{0.000000in}}%
\pgfpathlineto{\pgfqpoint{0.000000in}{0.034722in}}%
\pgfusepath{stroke,fill}%
}%
\begin{pgfscope}%
\pgfsys@transformshift{1.875887in}{3.272483in}%
\pgfsys@useobject{currentmarker}{}%
\end{pgfscope}%
\end{pgfscope}%
\begin{pgfscope}%
\pgfsetbuttcap%
\pgfsetroundjoin%
\definecolor{currentfill}{rgb}{0.000000,0.000000,0.000000}%
\pgfsetfillcolor{currentfill}%
\pgfsetlinewidth{1.003750pt}%
\definecolor{currentstroke}{rgb}{0.000000,0.000000,0.000000}%
\pgfsetstrokecolor{currentstroke}%
\pgfsetdash{}{0pt}%
\pgfsys@defobject{currentmarker}{\pgfqpoint{0.000000in}{0.000000in}}{\pgfqpoint{0.000000in}{0.034722in}}{%
\pgfpathmoveto{\pgfqpoint{0.000000in}{0.000000in}}%
\pgfpathlineto{\pgfqpoint{0.000000in}{0.034722in}}%
\pgfusepath{stroke,fill}%
}%
\begin{pgfscope}%
\pgfsys@transformshift{2.090419in}{3.272483in}%
\pgfsys@useobject{currentmarker}{}%
\end{pgfscope}%
\end{pgfscope}%
\begin{pgfscope}%
\pgfsetbuttcap%
\pgfsetroundjoin%
\definecolor{currentfill}{rgb}{0.000000,0.000000,0.000000}%
\pgfsetfillcolor{currentfill}%
\pgfsetlinewidth{1.003750pt}%
\definecolor{currentstroke}{rgb}{0.000000,0.000000,0.000000}%
\pgfsetstrokecolor{currentstroke}%
\pgfsetdash{}{0pt}%
\pgfsys@defobject{currentmarker}{\pgfqpoint{0.000000in}{0.000000in}}{\pgfqpoint{0.000000in}{0.034722in}}{%
\pgfpathmoveto{\pgfqpoint{0.000000in}{0.000000in}}%
\pgfpathlineto{\pgfqpoint{0.000000in}{0.034722in}}%
\pgfusepath{stroke,fill}%
}%
\begin{pgfscope}%
\pgfsys@transformshift{2.315953in}{3.272483in}%
\pgfsys@useobject{currentmarker}{}%
\end{pgfscope}%
\end{pgfscope}%
\begin{pgfscope}%
\pgfsetbuttcap%
\pgfsetroundjoin%
\definecolor{currentfill}{rgb}{0.000000,0.000000,0.000000}%
\pgfsetfillcolor{currentfill}%
\pgfsetlinewidth{1.003750pt}%
\definecolor{currentstroke}{rgb}{0.000000,0.000000,0.000000}%
\pgfsetstrokecolor{currentstroke}%
\pgfsetdash{}{0pt}%
\pgfsys@defobject{currentmarker}{\pgfqpoint{0.000000in}{0.000000in}}{\pgfqpoint{0.000000in}{0.034722in}}{%
\pgfpathmoveto{\pgfqpoint{0.000000in}{0.000000in}}%
\pgfpathlineto{\pgfqpoint{0.000000in}{0.034722in}}%
\pgfusepath{stroke,fill}%
}%
\begin{pgfscope}%
\pgfsys@transformshift{2.553357in}{3.272483in}%
\pgfsys@useobject{currentmarker}{}%
\end{pgfscope}%
\end{pgfscope}%
\begin{pgfscope}%
\pgfsetbuttcap%
\pgfsetroundjoin%
\definecolor{currentfill}{rgb}{0.000000,0.000000,0.000000}%
\pgfsetfillcolor{currentfill}%
\pgfsetlinewidth{1.003750pt}%
\definecolor{currentstroke}{rgb}{0.000000,0.000000,0.000000}%
\pgfsetstrokecolor{currentstroke}%
\pgfsetdash{}{0pt}%
\pgfsys@defobject{currentmarker}{\pgfqpoint{0.000000in}{0.000000in}}{\pgfqpoint{0.000000in}{0.034722in}}{%
\pgfpathmoveto{\pgfqpoint{0.000000in}{0.000000in}}%
\pgfpathlineto{\pgfqpoint{0.000000in}{0.034722in}}%
\pgfusepath{stroke,fill}%
}%
\begin{pgfscope}%
\pgfsys@transformshift{2.803594in}{3.272483in}%
\pgfsys@useobject{currentmarker}{}%
\end{pgfscope}%
\end{pgfscope}%
\begin{pgfscope}%
\pgfsetbuttcap%
\pgfsetroundjoin%
\definecolor{currentfill}{rgb}{0.000000,0.000000,0.000000}%
\pgfsetfillcolor{currentfill}%
\pgfsetlinewidth{1.003750pt}%
\definecolor{currentstroke}{rgb}{0.000000,0.000000,0.000000}%
\pgfsetstrokecolor{currentstroke}%
\pgfsetdash{}{0pt}%
\pgfsys@defobject{currentmarker}{\pgfqpoint{0.000000in}{0.000000in}}{\pgfqpoint{0.000000in}{0.034722in}}{%
\pgfpathmoveto{\pgfqpoint{0.000000in}{0.000000in}}%
\pgfpathlineto{\pgfqpoint{0.000000in}{0.034722in}}%
\pgfusepath{stroke,fill}%
}%
\begin{pgfscope}%
\pgfsys@transformshift{3.067733in}{3.272483in}%
\pgfsys@useobject{currentmarker}{}%
\end{pgfscope}%
\end{pgfscope}%
\begin{pgfscope}%
\pgfsetbuttcap%
\pgfsetroundjoin%
\definecolor{currentfill}{rgb}{0.000000,0.000000,0.000000}%
\pgfsetfillcolor{currentfill}%
\pgfsetlinewidth{1.003750pt}%
\definecolor{currentstroke}{rgb}{0.000000,0.000000,0.000000}%
\pgfsetstrokecolor{currentstroke}%
\pgfsetdash{}{0pt}%
\pgfsys@defobject{currentmarker}{\pgfqpoint{0.000000in}{0.000000in}}{\pgfqpoint{0.000000in}{0.034722in}}{%
\pgfpathmoveto{\pgfqpoint{0.000000in}{0.000000in}}%
\pgfpathlineto{\pgfqpoint{0.000000in}{0.034722in}}%
\pgfusepath{stroke,fill}%
}%
\begin{pgfscope}%
\pgfsys@transformshift{3.346965in}{3.272483in}%
\pgfsys@useobject{currentmarker}{}%
\end{pgfscope}%
\end{pgfscope}%
\begin{pgfscope}%
\pgfsetbuttcap%
\pgfsetroundjoin%
\definecolor{currentfill}{rgb}{0.000000,0.000000,0.000000}%
\pgfsetfillcolor{currentfill}%
\pgfsetlinewidth{1.003750pt}%
\definecolor{currentstroke}{rgb}{0.000000,0.000000,0.000000}%
\pgfsetstrokecolor{currentstroke}%
\pgfsetdash{}{0pt}%
\pgfsys@defobject{currentmarker}{\pgfqpoint{0.000000in}{0.000000in}}{\pgfqpoint{0.000000in}{0.034722in}}{%
\pgfpathmoveto{\pgfqpoint{0.000000in}{0.000000in}}%
\pgfpathlineto{\pgfqpoint{0.000000in}{0.034722in}}%
\pgfusepath{stroke,fill}%
}%
\begin{pgfscope}%
\pgfsys@transformshift{3.642623in}{3.272483in}%
\pgfsys@useobject{currentmarker}{}%
\end{pgfscope}%
\end{pgfscope}%
\begin{pgfscope}%
\pgfsetrectcap%
\pgfsetmiterjoin%
\pgfsetlinewidth{0.803000pt}%
\definecolor{currentstroke}{rgb}{0.000000,0.000000,0.000000}%
\pgfsetstrokecolor{currentstroke}%
\pgfsetdash{}{0pt}%
\pgfpathmoveto{\pgfqpoint{0.644914in}{0.577483in}}%
\pgfpathlineto{\pgfqpoint{0.644914in}{3.272483in}}%
\pgfusepath{stroke}%
\end{pgfscope}%
\begin{pgfscope}%
\pgfsetrectcap%
\pgfsetmiterjoin%
\pgfsetlinewidth{0.803000pt}%
\definecolor{currentstroke}{rgb}{0.000000,0.000000,0.000000}%
\pgfsetstrokecolor{currentstroke}%
\pgfsetdash{}{0pt}%
\pgfpathmoveto{\pgfqpoint{3.744914in}{0.577483in}}%
\pgfpathlineto{\pgfqpoint{3.744914in}{3.272483in}}%
\pgfusepath{stroke}%
\end{pgfscope}%
\begin{pgfscope}%
\pgfsetrectcap%
\pgfsetmiterjoin%
\pgfsetlinewidth{0.803000pt}%
\definecolor{currentstroke}{rgb}{0.000000,0.000000,0.000000}%
\pgfsetstrokecolor{currentstroke}%
\pgfsetdash{}{0pt}%
\pgfpathmoveto{\pgfqpoint{0.644914in}{0.577483in}}%
\pgfpathlineto{\pgfqpoint{3.744914in}{0.577483in}}%
\pgfusepath{stroke}%
\end{pgfscope}%
\begin{pgfscope}%
\pgfsetrectcap%
\pgfsetmiterjoin%
\pgfsetlinewidth{0.803000pt}%
\definecolor{currentstroke}{rgb}{0.000000,0.000000,0.000000}%
\pgfsetstrokecolor{currentstroke}%
\pgfsetdash{}{0pt}%
\pgfpathmoveto{\pgfqpoint{0.644914in}{3.272483in}}%
\pgfpathlineto{\pgfqpoint{3.744914in}{3.272483in}}%
\pgfusepath{stroke}%
\end{pgfscope}%
\begin{pgfscope}%
\pgfsetbuttcap%
\pgfsetroundjoin%
\definecolor{currentfill}{rgb}{0.000000,0.000000,0.000000}%
\pgfsetfillcolor{currentfill}%
\pgfsetlinewidth{1.003750pt}%
\definecolor{currentstroke}{rgb}{0.000000,0.000000,0.000000}%
\pgfsetstrokecolor{currentstroke}%
\pgfsetdash{}{0pt}%
\pgfsys@defobject{currentmarker}{\pgfqpoint{0.000000in}{0.000000in}}{\pgfqpoint{0.000000in}{0.069444in}}{%
\pgfpathmoveto{\pgfqpoint{0.000000in}{0.000000in}}%
\pgfpathlineto{\pgfqpoint{0.000000in}{0.069444in}}%
\pgfusepath{stroke,fill}%
}%
\begin{pgfscope}%
\pgfsys@transformshift{1.113106in}{3.272483in}%
\pgfsys@useobject{currentmarker}{}%
\end{pgfscope}%
\end{pgfscope}%
\begin{pgfscope}%
\pgftext[x=1.113106in,y=3.390538in,,bottom]{\rmfamily\fontsize{10.000000}{12.000000}\selectfont 900 K}%
\end{pgfscope}%
\begin{pgfscope}%
\pgfsetbuttcap%
\pgfsetroundjoin%
\definecolor{currentfill}{rgb}{0.000000,0.000000,0.000000}%
\pgfsetfillcolor{currentfill}%
\pgfsetlinewidth{1.003750pt}%
\definecolor{currentstroke}{rgb}{0.000000,0.000000,0.000000}%
\pgfsetstrokecolor{currentstroke}%
\pgfsetdash{}{0pt}%
\pgfsys@defobject{currentmarker}{\pgfqpoint{0.000000in}{0.000000in}}{\pgfqpoint{0.000000in}{0.069444in}}{%
\pgfpathmoveto{\pgfqpoint{0.000000in}{0.000000in}}%
\pgfpathlineto{\pgfqpoint{0.000000in}{0.069444in}}%
\pgfusepath{stroke,fill}%
}%
\begin{pgfscope}%
\pgfsys@transformshift{2.090419in}{3.272483in}%
\pgfsys@useobject{currentmarker}{}%
\end{pgfscope}%
\end{pgfscope}%
\begin{pgfscope}%
\pgftext[x=2.090419in,y=3.390538in,,bottom]{\rmfamily\fontsize{10.000000}{12.000000}\selectfont 800 K}%
\end{pgfscope}%
\begin{pgfscope}%
\pgfsetbuttcap%
\pgfsetroundjoin%
\definecolor{currentfill}{rgb}{0.000000,0.000000,0.000000}%
\pgfsetfillcolor{currentfill}%
\pgfsetlinewidth{1.003750pt}%
\definecolor{currentstroke}{rgb}{0.000000,0.000000,0.000000}%
\pgfsetstrokecolor{currentstroke}%
\pgfsetdash{}{0pt}%
\pgfsys@defobject{currentmarker}{\pgfqpoint{0.000000in}{0.000000in}}{\pgfqpoint{0.000000in}{0.069444in}}{%
\pgfpathmoveto{\pgfqpoint{0.000000in}{0.000000in}}%
\pgfpathlineto{\pgfqpoint{0.000000in}{0.069444in}}%
\pgfusepath{stroke,fill}%
}%
\begin{pgfscope}%
\pgfsys@transformshift{3.346965in}{3.272483in}%
\pgfsys@useobject{currentmarker}{}%
\end{pgfscope}%
\end{pgfscope}%
\begin{pgfscope}%
\pgftext[x=3.346965in,y=3.390538in,,bottom]{\rmfamily\fontsize{10.000000}{12.000000}\selectfont 700 K}%
\end{pgfscope}%
\begin{pgfscope}%
\pgfsetbuttcap%
\pgfsetroundjoin%
\definecolor{currentfill}{rgb}{0.000000,0.000000,0.000000}%
\pgfsetfillcolor{currentfill}%
\pgfsetlinewidth{1.003750pt}%
\definecolor{currentstroke}{rgb}{0.000000,0.000000,0.000000}%
\pgfsetstrokecolor{currentstroke}%
\pgfsetdash{}{0pt}%
\pgfsys@defobject{currentmarker}{\pgfqpoint{0.000000in}{0.000000in}}{\pgfqpoint{0.000000in}{0.034722in}}{%
\pgfpathmoveto{\pgfqpoint{0.000000in}{0.000000in}}%
\pgfpathlineto{\pgfqpoint{0.000000in}{0.034722in}}%
\pgfusepath{stroke,fill}%
}%
\begin{pgfscope}%
\pgfsys@transformshift{0.780403in}{3.272483in}%
\pgfsys@useobject{currentmarker}{}%
\end{pgfscope}%
\end{pgfscope}%
\begin{pgfscope}%
\pgfsetbuttcap%
\pgfsetroundjoin%
\definecolor{currentfill}{rgb}{0.000000,0.000000,0.000000}%
\pgfsetfillcolor{currentfill}%
\pgfsetlinewidth{1.003750pt}%
\definecolor{currentstroke}{rgb}{0.000000,0.000000,0.000000}%
\pgfsetstrokecolor{currentstroke}%
\pgfsetdash{}{0pt}%
\pgfsys@defobject{currentmarker}{\pgfqpoint{0.000000in}{0.000000in}}{\pgfqpoint{0.000000in}{0.034722in}}{%
\pgfpathmoveto{\pgfqpoint{0.000000in}{0.000000in}}%
\pgfpathlineto{\pgfqpoint{0.000000in}{0.034722in}}%
\pgfusepath{stroke,fill}%
}%
\begin{pgfscope}%
\pgfsys@transformshift{0.943138in}{3.272483in}%
\pgfsys@useobject{currentmarker}{}%
\end{pgfscope}%
\end{pgfscope}%
\begin{pgfscope}%
\pgfsetbuttcap%
\pgfsetroundjoin%
\definecolor{currentfill}{rgb}{0.000000,0.000000,0.000000}%
\pgfsetfillcolor{currentfill}%
\pgfsetlinewidth{1.003750pt}%
\definecolor{currentstroke}{rgb}{0.000000,0.000000,0.000000}%
\pgfsetstrokecolor{currentstroke}%
\pgfsetdash{}{0pt}%
\pgfsys@defobject{currentmarker}{\pgfqpoint{0.000000in}{0.000000in}}{\pgfqpoint{0.000000in}{0.034722in}}{%
\pgfpathmoveto{\pgfqpoint{0.000000in}{0.000000in}}%
\pgfpathlineto{\pgfqpoint{0.000000in}{0.034722in}}%
\pgfusepath{stroke,fill}%
}%
\begin{pgfscope}%
\pgfsys@transformshift{1.113106in}{3.272483in}%
\pgfsys@useobject{currentmarker}{}%
\end{pgfscope}%
\end{pgfscope}%
\begin{pgfscope}%
\pgfsetbuttcap%
\pgfsetroundjoin%
\definecolor{currentfill}{rgb}{0.000000,0.000000,0.000000}%
\pgfsetfillcolor{currentfill}%
\pgfsetlinewidth{1.003750pt}%
\definecolor{currentstroke}{rgb}{0.000000,0.000000,0.000000}%
\pgfsetstrokecolor{currentstroke}%
\pgfsetdash{}{0pt}%
\pgfsys@defobject{currentmarker}{\pgfqpoint{0.000000in}{0.000000in}}{\pgfqpoint{0.000000in}{0.034722in}}{%
\pgfpathmoveto{\pgfqpoint{0.000000in}{0.000000in}}%
\pgfpathlineto{\pgfqpoint{0.000000in}{0.034722in}}%
\pgfusepath{stroke,fill}%
}%
\begin{pgfscope}%
\pgfsys@transformshift{1.290799in}{3.272483in}%
\pgfsys@useobject{currentmarker}{}%
\end{pgfscope}%
\end{pgfscope}%
\begin{pgfscope}%
\pgfsetbuttcap%
\pgfsetroundjoin%
\definecolor{currentfill}{rgb}{0.000000,0.000000,0.000000}%
\pgfsetfillcolor{currentfill}%
\pgfsetlinewidth{1.003750pt}%
\definecolor{currentstroke}{rgb}{0.000000,0.000000,0.000000}%
\pgfsetstrokecolor{currentstroke}%
\pgfsetdash{}{0pt}%
\pgfsys@defobject{currentmarker}{\pgfqpoint{0.000000in}{0.000000in}}{\pgfqpoint{0.000000in}{0.034722in}}{%
\pgfpathmoveto{\pgfqpoint{0.000000in}{0.000000in}}%
\pgfpathlineto{\pgfqpoint{0.000000in}{0.034722in}}%
\pgfusepath{stroke,fill}%
}%
\begin{pgfscope}%
\pgfsys@transformshift{1.476757in}{3.272483in}%
\pgfsys@useobject{currentmarker}{}%
\end{pgfscope}%
\end{pgfscope}%
\begin{pgfscope}%
\pgfsetbuttcap%
\pgfsetroundjoin%
\definecolor{currentfill}{rgb}{0.000000,0.000000,0.000000}%
\pgfsetfillcolor{currentfill}%
\pgfsetlinewidth{1.003750pt}%
\definecolor{currentstroke}{rgb}{0.000000,0.000000,0.000000}%
\pgfsetstrokecolor{currentstroke}%
\pgfsetdash{}{0pt}%
\pgfsys@defobject{currentmarker}{\pgfqpoint{0.000000in}{0.000000in}}{\pgfqpoint{0.000000in}{0.034722in}}{%
\pgfpathmoveto{\pgfqpoint{0.000000in}{0.000000in}}%
\pgfpathlineto{\pgfqpoint{0.000000in}{0.034722in}}%
\pgfusepath{stroke,fill}%
}%
\begin{pgfscope}%
\pgfsys@transformshift{1.671570in}{3.272483in}%
\pgfsys@useobject{currentmarker}{}%
\end{pgfscope}%
\end{pgfscope}%
\begin{pgfscope}%
\pgfsetbuttcap%
\pgfsetroundjoin%
\definecolor{currentfill}{rgb}{0.000000,0.000000,0.000000}%
\pgfsetfillcolor{currentfill}%
\pgfsetlinewidth{1.003750pt}%
\definecolor{currentstroke}{rgb}{0.000000,0.000000,0.000000}%
\pgfsetstrokecolor{currentstroke}%
\pgfsetdash{}{0pt}%
\pgfsys@defobject{currentmarker}{\pgfqpoint{0.000000in}{0.000000in}}{\pgfqpoint{0.000000in}{0.034722in}}{%
\pgfpathmoveto{\pgfqpoint{0.000000in}{0.000000in}}%
\pgfpathlineto{\pgfqpoint{0.000000in}{0.034722in}}%
\pgfusepath{stroke,fill}%
}%
\begin{pgfscope}%
\pgfsys@transformshift{1.875887in}{3.272483in}%
\pgfsys@useobject{currentmarker}{}%
\end{pgfscope}%
\end{pgfscope}%
\begin{pgfscope}%
\pgfsetbuttcap%
\pgfsetroundjoin%
\definecolor{currentfill}{rgb}{0.000000,0.000000,0.000000}%
\pgfsetfillcolor{currentfill}%
\pgfsetlinewidth{1.003750pt}%
\definecolor{currentstroke}{rgb}{0.000000,0.000000,0.000000}%
\pgfsetstrokecolor{currentstroke}%
\pgfsetdash{}{0pt}%
\pgfsys@defobject{currentmarker}{\pgfqpoint{0.000000in}{0.000000in}}{\pgfqpoint{0.000000in}{0.034722in}}{%
\pgfpathmoveto{\pgfqpoint{0.000000in}{0.000000in}}%
\pgfpathlineto{\pgfqpoint{0.000000in}{0.034722in}}%
\pgfusepath{stroke,fill}%
}%
\begin{pgfscope}%
\pgfsys@transformshift{2.090419in}{3.272483in}%
\pgfsys@useobject{currentmarker}{}%
\end{pgfscope}%
\end{pgfscope}%
\begin{pgfscope}%
\pgfsetbuttcap%
\pgfsetroundjoin%
\definecolor{currentfill}{rgb}{0.000000,0.000000,0.000000}%
\pgfsetfillcolor{currentfill}%
\pgfsetlinewidth{1.003750pt}%
\definecolor{currentstroke}{rgb}{0.000000,0.000000,0.000000}%
\pgfsetstrokecolor{currentstroke}%
\pgfsetdash{}{0pt}%
\pgfsys@defobject{currentmarker}{\pgfqpoint{0.000000in}{0.000000in}}{\pgfqpoint{0.000000in}{0.034722in}}{%
\pgfpathmoveto{\pgfqpoint{0.000000in}{0.000000in}}%
\pgfpathlineto{\pgfqpoint{0.000000in}{0.034722in}}%
\pgfusepath{stroke,fill}%
}%
\begin{pgfscope}%
\pgfsys@transformshift{2.315953in}{3.272483in}%
\pgfsys@useobject{currentmarker}{}%
\end{pgfscope}%
\end{pgfscope}%
\begin{pgfscope}%
\pgfsetbuttcap%
\pgfsetroundjoin%
\definecolor{currentfill}{rgb}{0.000000,0.000000,0.000000}%
\pgfsetfillcolor{currentfill}%
\pgfsetlinewidth{1.003750pt}%
\definecolor{currentstroke}{rgb}{0.000000,0.000000,0.000000}%
\pgfsetstrokecolor{currentstroke}%
\pgfsetdash{}{0pt}%
\pgfsys@defobject{currentmarker}{\pgfqpoint{0.000000in}{0.000000in}}{\pgfqpoint{0.000000in}{0.034722in}}{%
\pgfpathmoveto{\pgfqpoint{0.000000in}{0.000000in}}%
\pgfpathlineto{\pgfqpoint{0.000000in}{0.034722in}}%
\pgfusepath{stroke,fill}%
}%
\begin{pgfscope}%
\pgfsys@transformshift{2.553357in}{3.272483in}%
\pgfsys@useobject{currentmarker}{}%
\end{pgfscope}%
\end{pgfscope}%
\begin{pgfscope}%
\pgfsetbuttcap%
\pgfsetroundjoin%
\definecolor{currentfill}{rgb}{0.000000,0.000000,0.000000}%
\pgfsetfillcolor{currentfill}%
\pgfsetlinewidth{1.003750pt}%
\definecolor{currentstroke}{rgb}{0.000000,0.000000,0.000000}%
\pgfsetstrokecolor{currentstroke}%
\pgfsetdash{}{0pt}%
\pgfsys@defobject{currentmarker}{\pgfqpoint{0.000000in}{0.000000in}}{\pgfqpoint{0.000000in}{0.034722in}}{%
\pgfpathmoveto{\pgfqpoint{0.000000in}{0.000000in}}%
\pgfpathlineto{\pgfqpoint{0.000000in}{0.034722in}}%
\pgfusepath{stroke,fill}%
}%
\begin{pgfscope}%
\pgfsys@transformshift{2.803594in}{3.272483in}%
\pgfsys@useobject{currentmarker}{}%
\end{pgfscope}%
\end{pgfscope}%
\begin{pgfscope}%
\pgfsetbuttcap%
\pgfsetroundjoin%
\definecolor{currentfill}{rgb}{0.000000,0.000000,0.000000}%
\pgfsetfillcolor{currentfill}%
\pgfsetlinewidth{1.003750pt}%
\definecolor{currentstroke}{rgb}{0.000000,0.000000,0.000000}%
\pgfsetstrokecolor{currentstroke}%
\pgfsetdash{}{0pt}%
\pgfsys@defobject{currentmarker}{\pgfqpoint{0.000000in}{0.000000in}}{\pgfqpoint{0.000000in}{0.034722in}}{%
\pgfpathmoveto{\pgfqpoint{0.000000in}{0.000000in}}%
\pgfpathlineto{\pgfqpoint{0.000000in}{0.034722in}}%
\pgfusepath{stroke,fill}%
}%
\begin{pgfscope}%
\pgfsys@transformshift{3.067733in}{3.272483in}%
\pgfsys@useobject{currentmarker}{}%
\end{pgfscope}%
\end{pgfscope}%
\begin{pgfscope}%
\pgfsetbuttcap%
\pgfsetroundjoin%
\definecolor{currentfill}{rgb}{0.000000,0.000000,0.000000}%
\pgfsetfillcolor{currentfill}%
\pgfsetlinewidth{1.003750pt}%
\definecolor{currentstroke}{rgb}{0.000000,0.000000,0.000000}%
\pgfsetstrokecolor{currentstroke}%
\pgfsetdash{}{0pt}%
\pgfsys@defobject{currentmarker}{\pgfqpoint{0.000000in}{0.000000in}}{\pgfqpoint{0.000000in}{0.034722in}}{%
\pgfpathmoveto{\pgfqpoint{0.000000in}{0.000000in}}%
\pgfpathlineto{\pgfqpoint{0.000000in}{0.034722in}}%
\pgfusepath{stroke,fill}%
}%
\begin{pgfscope}%
\pgfsys@transformshift{3.346965in}{3.272483in}%
\pgfsys@useobject{currentmarker}{}%
\end{pgfscope}%
\end{pgfscope}%
\begin{pgfscope}%
\pgfsetbuttcap%
\pgfsetroundjoin%
\definecolor{currentfill}{rgb}{0.000000,0.000000,0.000000}%
\pgfsetfillcolor{currentfill}%
\pgfsetlinewidth{1.003750pt}%
\definecolor{currentstroke}{rgb}{0.000000,0.000000,0.000000}%
\pgfsetstrokecolor{currentstroke}%
\pgfsetdash{}{0pt}%
\pgfsys@defobject{currentmarker}{\pgfqpoint{0.000000in}{0.000000in}}{\pgfqpoint{0.000000in}{0.034722in}}{%
\pgfpathmoveto{\pgfqpoint{0.000000in}{0.000000in}}%
\pgfpathlineto{\pgfqpoint{0.000000in}{0.034722in}}%
\pgfusepath{stroke,fill}%
}%
\begin{pgfscope}%
\pgfsys@transformshift{3.642623in}{3.272483in}%
\pgfsys@useobject{currentmarker}{}%
\end{pgfscope}%
\end{pgfscope}%
\begin{pgfscope}%
\pgfsetrectcap%
\pgfsetmiterjoin%
\pgfsetlinewidth{0.803000pt}%
\definecolor{currentstroke}{rgb}{0.000000,0.000000,0.000000}%
\pgfsetstrokecolor{currentstroke}%
\pgfsetdash{}{0pt}%
\pgfpathmoveto{\pgfqpoint{0.644914in}{0.577483in}}%
\pgfpathlineto{\pgfqpoint{0.644914in}{3.272483in}}%
\pgfusepath{stroke}%
\end{pgfscope}%
\begin{pgfscope}%
\pgfsetrectcap%
\pgfsetmiterjoin%
\pgfsetlinewidth{0.803000pt}%
\definecolor{currentstroke}{rgb}{0.000000,0.000000,0.000000}%
\pgfsetstrokecolor{currentstroke}%
\pgfsetdash{}{0pt}%
\pgfpathmoveto{\pgfqpoint{3.744914in}{0.577483in}}%
\pgfpathlineto{\pgfqpoint{3.744914in}{3.272483in}}%
\pgfusepath{stroke}%
\end{pgfscope}%
\begin{pgfscope}%
\pgfsetrectcap%
\pgfsetmiterjoin%
\pgfsetlinewidth{0.803000pt}%
\definecolor{currentstroke}{rgb}{0.000000,0.000000,0.000000}%
\pgfsetstrokecolor{currentstroke}%
\pgfsetdash{}{0pt}%
\pgfpathmoveto{\pgfqpoint{0.644914in}{0.577483in}}%
\pgfpathlineto{\pgfqpoint{3.744914in}{0.577483in}}%
\pgfusepath{stroke}%
\end{pgfscope}%
\begin{pgfscope}%
\pgfsetrectcap%
\pgfsetmiterjoin%
\pgfsetlinewidth{0.803000pt}%
\definecolor{currentstroke}{rgb}{0.000000,0.000000,0.000000}%
\pgfsetstrokecolor{currentstroke}%
\pgfsetdash{}{0pt}%
\pgfpathmoveto{\pgfqpoint{0.644914in}{3.272483in}}%
\pgfpathlineto{\pgfqpoint{3.744914in}{3.272483in}}%
\pgfusepath{stroke}%
\end{pgfscope}%
\begin{pgfscope}%
\pgfsetbuttcap%
\pgfsetroundjoin%
\definecolor{currentfill}{rgb}{0.000000,0.000000,0.000000}%
\pgfsetfillcolor{currentfill}%
\pgfsetlinewidth{1.003750pt}%
\definecolor{currentstroke}{rgb}{0.000000,0.000000,0.000000}%
\pgfsetstrokecolor{currentstroke}%
\pgfsetdash{}{0pt}%
\pgfsys@defobject{currentmarker}{\pgfqpoint{0.000000in}{0.000000in}}{\pgfqpoint{0.000000in}{0.069444in}}{%
\pgfpathmoveto{\pgfqpoint{0.000000in}{0.000000in}}%
\pgfpathlineto{\pgfqpoint{0.000000in}{0.069444in}}%
\pgfusepath{stroke,fill}%
}%
\begin{pgfscope}%
\pgfsys@transformshift{1.113106in}{3.272483in}%
\pgfsys@useobject{currentmarker}{}%
\end{pgfscope}%
\end{pgfscope}%
\begin{pgfscope}%
\pgftext[x=1.113106in,y=3.390538in,,bottom]{\rmfamily\fontsize{10.000000}{12.000000}\selectfont 900 K}%
\end{pgfscope}%
\begin{pgfscope}%
\pgfsetbuttcap%
\pgfsetroundjoin%
\definecolor{currentfill}{rgb}{0.000000,0.000000,0.000000}%
\pgfsetfillcolor{currentfill}%
\pgfsetlinewidth{1.003750pt}%
\definecolor{currentstroke}{rgb}{0.000000,0.000000,0.000000}%
\pgfsetstrokecolor{currentstroke}%
\pgfsetdash{}{0pt}%
\pgfsys@defobject{currentmarker}{\pgfqpoint{0.000000in}{0.000000in}}{\pgfqpoint{0.000000in}{0.069444in}}{%
\pgfpathmoveto{\pgfqpoint{0.000000in}{0.000000in}}%
\pgfpathlineto{\pgfqpoint{0.000000in}{0.069444in}}%
\pgfusepath{stroke,fill}%
}%
\begin{pgfscope}%
\pgfsys@transformshift{2.090419in}{3.272483in}%
\pgfsys@useobject{currentmarker}{}%
\end{pgfscope}%
\end{pgfscope}%
\begin{pgfscope}%
\pgftext[x=2.090419in,y=3.390538in,,bottom]{\rmfamily\fontsize{10.000000}{12.000000}\selectfont 800 K}%
\end{pgfscope}%
\begin{pgfscope}%
\pgfsetbuttcap%
\pgfsetroundjoin%
\definecolor{currentfill}{rgb}{0.000000,0.000000,0.000000}%
\pgfsetfillcolor{currentfill}%
\pgfsetlinewidth{1.003750pt}%
\definecolor{currentstroke}{rgb}{0.000000,0.000000,0.000000}%
\pgfsetstrokecolor{currentstroke}%
\pgfsetdash{}{0pt}%
\pgfsys@defobject{currentmarker}{\pgfqpoint{0.000000in}{0.000000in}}{\pgfqpoint{0.000000in}{0.069444in}}{%
\pgfpathmoveto{\pgfqpoint{0.000000in}{0.000000in}}%
\pgfpathlineto{\pgfqpoint{0.000000in}{0.069444in}}%
\pgfusepath{stroke,fill}%
}%
\begin{pgfscope}%
\pgfsys@transformshift{3.346965in}{3.272483in}%
\pgfsys@useobject{currentmarker}{}%
\end{pgfscope}%
\end{pgfscope}%
\begin{pgfscope}%
\pgftext[x=3.346965in,y=3.390538in,,bottom]{\rmfamily\fontsize{10.000000}{12.000000}\selectfont 700 K}%
\end{pgfscope}%
\begin{pgfscope}%
\pgfsetbuttcap%
\pgfsetroundjoin%
\definecolor{currentfill}{rgb}{0.000000,0.000000,0.000000}%
\pgfsetfillcolor{currentfill}%
\pgfsetlinewidth{1.003750pt}%
\definecolor{currentstroke}{rgb}{0.000000,0.000000,0.000000}%
\pgfsetstrokecolor{currentstroke}%
\pgfsetdash{}{0pt}%
\pgfsys@defobject{currentmarker}{\pgfqpoint{0.000000in}{0.000000in}}{\pgfqpoint{0.000000in}{0.034722in}}{%
\pgfpathmoveto{\pgfqpoint{0.000000in}{0.000000in}}%
\pgfpathlineto{\pgfqpoint{0.000000in}{0.034722in}}%
\pgfusepath{stroke,fill}%
}%
\begin{pgfscope}%
\pgfsys@transformshift{0.780403in}{3.272483in}%
\pgfsys@useobject{currentmarker}{}%
\end{pgfscope}%
\end{pgfscope}%
\begin{pgfscope}%
\pgfsetbuttcap%
\pgfsetroundjoin%
\definecolor{currentfill}{rgb}{0.000000,0.000000,0.000000}%
\pgfsetfillcolor{currentfill}%
\pgfsetlinewidth{1.003750pt}%
\definecolor{currentstroke}{rgb}{0.000000,0.000000,0.000000}%
\pgfsetstrokecolor{currentstroke}%
\pgfsetdash{}{0pt}%
\pgfsys@defobject{currentmarker}{\pgfqpoint{0.000000in}{0.000000in}}{\pgfqpoint{0.000000in}{0.034722in}}{%
\pgfpathmoveto{\pgfqpoint{0.000000in}{0.000000in}}%
\pgfpathlineto{\pgfqpoint{0.000000in}{0.034722in}}%
\pgfusepath{stroke,fill}%
}%
\begin{pgfscope}%
\pgfsys@transformshift{0.943138in}{3.272483in}%
\pgfsys@useobject{currentmarker}{}%
\end{pgfscope}%
\end{pgfscope}%
\begin{pgfscope}%
\pgfsetbuttcap%
\pgfsetroundjoin%
\definecolor{currentfill}{rgb}{0.000000,0.000000,0.000000}%
\pgfsetfillcolor{currentfill}%
\pgfsetlinewidth{1.003750pt}%
\definecolor{currentstroke}{rgb}{0.000000,0.000000,0.000000}%
\pgfsetstrokecolor{currentstroke}%
\pgfsetdash{}{0pt}%
\pgfsys@defobject{currentmarker}{\pgfqpoint{0.000000in}{0.000000in}}{\pgfqpoint{0.000000in}{0.034722in}}{%
\pgfpathmoveto{\pgfqpoint{0.000000in}{0.000000in}}%
\pgfpathlineto{\pgfqpoint{0.000000in}{0.034722in}}%
\pgfusepath{stroke,fill}%
}%
\begin{pgfscope}%
\pgfsys@transformshift{1.113106in}{3.272483in}%
\pgfsys@useobject{currentmarker}{}%
\end{pgfscope}%
\end{pgfscope}%
\begin{pgfscope}%
\pgfsetbuttcap%
\pgfsetroundjoin%
\definecolor{currentfill}{rgb}{0.000000,0.000000,0.000000}%
\pgfsetfillcolor{currentfill}%
\pgfsetlinewidth{1.003750pt}%
\definecolor{currentstroke}{rgb}{0.000000,0.000000,0.000000}%
\pgfsetstrokecolor{currentstroke}%
\pgfsetdash{}{0pt}%
\pgfsys@defobject{currentmarker}{\pgfqpoint{0.000000in}{0.000000in}}{\pgfqpoint{0.000000in}{0.034722in}}{%
\pgfpathmoveto{\pgfqpoint{0.000000in}{0.000000in}}%
\pgfpathlineto{\pgfqpoint{0.000000in}{0.034722in}}%
\pgfusepath{stroke,fill}%
}%
\begin{pgfscope}%
\pgfsys@transformshift{1.290799in}{3.272483in}%
\pgfsys@useobject{currentmarker}{}%
\end{pgfscope}%
\end{pgfscope}%
\begin{pgfscope}%
\pgfsetbuttcap%
\pgfsetroundjoin%
\definecolor{currentfill}{rgb}{0.000000,0.000000,0.000000}%
\pgfsetfillcolor{currentfill}%
\pgfsetlinewidth{1.003750pt}%
\definecolor{currentstroke}{rgb}{0.000000,0.000000,0.000000}%
\pgfsetstrokecolor{currentstroke}%
\pgfsetdash{}{0pt}%
\pgfsys@defobject{currentmarker}{\pgfqpoint{0.000000in}{0.000000in}}{\pgfqpoint{0.000000in}{0.034722in}}{%
\pgfpathmoveto{\pgfqpoint{0.000000in}{0.000000in}}%
\pgfpathlineto{\pgfqpoint{0.000000in}{0.034722in}}%
\pgfusepath{stroke,fill}%
}%
\begin{pgfscope}%
\pgfsys@transformshift{1.476757in}{3.272483in}%
\pgfsys@useobject{currentmarker}{}%
\end{pgfscope}%
\end{pgfscope}%
\begin{pgfscope}%
\pgfsetbuttcap%
\pgfsetroundjoin%
\definecolor{currentfill}{rgb}{0.000000,0.000000,0.000000}%
\pgfsetfillcolor{currentfill}%
\pgfsetlinewidth{1.003750pt}%
\definecolor{currentstroke}{rgb}{0.000000,0.000000,0.000000}%
\pgfsetstrokecolor{currentstroke}%
\pgfsetdash{}{0pt}%
\pgfsys@defobject{currentmarker}{\pgfqpoint{0.000000in}{0.000000in}}{\pgfqpoint{0.000000in}{0.034722in}}{%
\pgfpathmoveto{\pgfqpoint{0.000000in}{0.000000in}}%
\pgfpathlineto{\pgfqpoint{0.000000in}{0.034722in}}%
\pgfusepath{stroke,fill}%
}%
\begin{pgfscope}%
\pgfsys@transformshift{1.671570in}{3.272483in}%
\pgfsys@useobject{currentmarker}{}%
\end{pgfscope}%
\end{pgfscope}%
\begin{pgfscope}%
\pgfsetbuttcap%
\pgfsetroundjoin%
\definecolor{currentfill}{rgb}{0.000000,0.000000,0.000000}%
\pgfsetfillcolor{currentfill}%
\pgfsetlinewidth{1.003750pt}%
\definecolor{currentstroke}{rgb}{0.000000,0.000000,0.000000}%
\pgfsetstrokecolor{currentstroke}%
\pgfsetdash{}{0pt}%
\pgfsys@defobject{currentmarker}{\pgfqpoint{0.000000in}{0.000000in}}{\pgfqpoint{0.000000in}{0.034722in}}{%
\pgfpathmoveto{\pgfqpoint{0.000000in}{0.000000in}}%
\pgfpathlineto{\pgfqpoint{0.000000in}{0.034722in}}%
\pgfusepath{stroke,fill}%
}%
\begin{pgfscope}%
\pgfsys@transformshift{1.875887in}{3.272483in}%
\pgfsys@useobject{currentmarker}{}%
\end{pgfscope}%
\end{pgfscope}%
\begin{pgfscope}%
\pgfsetbuttcap%
\pgfsetroundjoin%
\definecolor{currentfill}{rgb}{0.000000,0.000000,0.000000}%
\pgfsetfillcolor{currentfill}%
\pgfsetlinewidth{1.003750pt}%
\definecolor{currentstroke}{rgb}{0.000000,0.000000,0.000000}%
\pgfsetstrokecolor{currentstroke}%
\pgfsetdash{}{0pt}%
\pgfsys@defobject{currentmarker}{\pgfqpoint{0.000000in}{0.000000in}}{\pgfqpoint{0.000000in}{0.034722in}}{%
\pgfpathmoveto{\pgfqpoint{0.000000in}{0.000000in}}%
\pgfpathlineto{\pgfqpoint{0.000000in}{0.034722in}}%
\pgfusepath{stroke,fill}%
}%
\begin{pgfscope}%
\pgfsys@transformshift{2.090419in}{3.272483in}%
\pgfsys@useobject{currentmarker}{}%
\end{pgfscope}%
\end{pgfscope}%
\begin{pgfscope}%
\pgfsetbuttcap%
\pgfsetroundjoin%
\definecolor{currentfill}{rgb}{0.000000,0.000000,0.000000}%
\pgfsetfillcolor{currentfill}%
\pgfsetlinewidth{1.003750pt}%
\definecolor{currentstroke}{rgb}{0.000000,0.000000,0.000000}%
\pgfsetstrokecolor{currentstroke}%
\pgfsetdash{}{0pt}%
\pgfsys@defobject{currentmarker}{\pgfqpoint{0.000000in}{0.000000in}}{\pgfqpoint{0.000000in}{0.034722in}}{%
\pgfpathmoveto{\pgfqpoint{0.000000in}{0.000000in}}%
\pgfpathlineto{\pgfqpoint{0.000000in}{0.034722in}}%
\pgfusepath{stroke,fill}%
}%
\begin{pgfscope}%
\pgfsys@transformshift{2.315953in}{3.272483in}%
\pgfsys@useobject{currentmarker}{}%
\end{pgfscope}%
\end{pgfscope}%
\begin{pgfscope}%
\pgfsetbuttcap%
\pgfsetroundjoin%
\definecolor{currentfill}{rgb}{0.000000,0.000000,0.000000}%
\pgfsetfillcolor{currentfill}%
\pgfsetlinewidth{1.003750pt}%
\definecolor{currentstroke}{rgb}{0.000000,0.000000,0.000000}%
\pgfsetstrokecolor{currentstroke}%
\pgfsetdash{}{0pt}%
\pgfsys@defobject{currentmarker}{\pgfqpoint{0.000000in}{0.000000in}}{\pgfqpoint{0.000000in}{0.034722in}}{%
\pgfpathmoveto{\pgfqpoint{0.000000in}{0.000000in}}%
\pgfpathlineto{\pgfqpoint{0.000000in}{0.034722in}}%
\pgfusepath{stroke,fill}%
}%
\begin{pgfscope}%
\pgfsys@transformshift{2.553357in}{3.272483in}%
\pgfsys@useobject{currentmarker}{}%
\end{pgfscope}%
\end{pgfscope}%
\begin{pgfscope}%
\pgfsetbuttcap%
\pgfsetroundjoin%
\definecolor{currentfill}{rgb}{0.000000,0.000000,0.000000}%
\pgfsetfillcolor{currentfill}%
\pgfsetlinewidth{1.003750pt}%
\definecolor{currentstroke}{rgb}{0.000000,0.000000,0.000000}%
\pgfsetstrokecolor{currentstroke}%
\pgfsetdash{}{0pt}%
\pgfsys@defobject{currentmarker}{\pgfqpoint{0.000000in}{0.000000in}}{\pgfqpoint{0.000000in}{0.034722in}}{%
\pgfpathmoveto{\pgfqpoint{0.000000in}{0.000000in}}%
\pgfpathlineto{\pgfqpoint{0.000000in}{0.034722in}}%
\pgfusepath{stroke,fill}%
}%
\begin{pgfscope}%
\pgfsys@transformshift{2.803594in}{3.272483in}%
\pgfsys@useobject{currentmarker}{}%
\end{pgfscope}%
\end{pgfscope}%
\begin{pgfscope}%
\pgfsetbuttcap%
\pgfsetroundjoin%
\definecolor{currentfill}{rgb}{0.000000,0.000000,0.000000}%
\pgfsetfillcolor{currentfill}%
\pgfsetlinewidth{1.003750pt}%
\definecolor{currentstroke}{rgb}{0.000000,0.000000,0.000000}%
\pgfsetstrokecolor{currentstroke}%
\pgfsetdash{}{0pt}%
\pgfsys@defobject{currentmarker}{\pgfqpoint{0.000000in}{0.000000in}}{\pgfqpoint{0.000000in}{0.034722in}}{%
\pgfpathmoveto{\pgfqpoint{0.000000in}{0.000000in}}%
\pgfpathlineto{\pgfqpoint{0.000000in}{0.034722in}}%
\pgfusepath{stroke,fill}%
}%
\begin{pgfscope}%
\pgfsys@transformshift{3.067733in}{3.272483in}%
\pgfsys@useobject{currentmarker}{}%
\end{pgfscope}%
\end{pgfscope}%
\begin{pgfscope}%
\pgfsetbuttcap%
\pgfsetroundjoin%
\definecolor{currentfill}{rgb}{0.000000,0.000000,0.000000}%
\pgfsetfillcolor{currentfill}%
\pgfsetlinewidth{1.003750pt}%
\definecolor{currentstroke}{rgb}{0.000000,0.000000,0.000000}%
\pgfsetstrokecolor{currentstroke}%
\pgfsetdash{}{0pt}%
\pgfsys@defobject{currentmarker}{\pgfqpoint{0.000000in}{0.000000in}}{\pgfqpoint{0.000000in}{0.034722in}}{%
\pgfpathmoveto{\pgfqpoint{0.000000in}{0.000000in}}%
\pgfpathlineto{\pgfqpoint{0.000000in}{0.034722in}}%
\pgfusepath{stroke,fill}%
}%
\begin{pgfscope}%
\pgfsys@transformshift{3.346965in}{3.272483in}%
\pgfsys@useobject{currentmarker}{}%
\end{pgfscope}%
\end{pgfscope}%
\begin{pgfscope}%
\pgfsetbuttcap%
\pgfsetroundjoin%
\definecolor{currentfill}{rgb}{0.000000,0.000000,0.000000}%
\pgfsetfillcolor{currentfill}%
\pgfsetlinewidth{1.003750pt}%
\definecolor{currentstroke}{rgb}{0.000000,0.000000,0.000000}%
\pgfsetstrokecolor{currentstroke}%
\pgfsetdash{}{0pt}%
\pgfsys@defobject{currentmarker}{\pgfqpoint{0.000000in}{0.000000in}}{\pgfqpoint{0.000000in}{0.034722in}}{%
\pgfpathmoveto{\pgfqpoint{0.000000in}{0.000000in}}%
\pgfpathlineto{\pgfqpoint{0.000000in}{0.034722in}}%
\pgfusepath{stroke,fill}%
}%
\begin{pgfscope}%
\pgfsys@transformshift{3.642623in}{3.272483in}%
\pgfsys@useobject{currentmarker}{}%
\end{pgfscope}%
\end{pgfscope}%
\begin{pgfscope}%
\pgfsetrectcap%
\pgfsetmiterjoin%
\pgfsetlinewidth{0.803000pt}%
\definecolor{currentstroke}{rgb}{0.000000,0.000000,0.000000}%
\pgfsetstrokecolor{currentstroke}%
\pgfsetdash{}{0pt}%
\pgfpathmoveto{\pgfqpoint{0.644914in}{0.577483in}}%
\pgfpathlineto{\pgfqpoint{0.644914in}{3.272483in}}%
\pgfusepath{stroke}%
\end{pgfscope}%
\begin{pgfscope}%
\pgfsetrectcap%
\pgfsetmiterjoin%
\pgfsetlinewidth{0.803000pt}%
\definecolor{currentstroke}{rgb}{0.000000,0.000000,0.000000}%
\pgfsetstrokecolor{currentstroke}%
\pgfsetdash{}{0pt}%
\pgfpathmoveto{\pgfqpoint{3.744914in}{0.577483in}}%
\pgfpathlineto{\pgfqpoint{3.744914in}{3.272483in}}%
\pgfusepath{stroke}%
\end{pgfscope}%
\begin{pgfscope}%
\pgfsetrectcap%
\pgfsetmiterjoin%
\pgfsetlinewidth{0.803000pt}%
\definecolor{currentstroke}{rgb}{0.000000,0.000000,0.000000}%
\pgfsetstrokecolor{currentstroke}%
\pgfsetdash{}{0pt}%
\pgfpathmoveto{\pgfqpoint{0.644914in}{0.577483in}}%
\pgfpathlineto{\pgfqpoint{3.744914in}{0.577483in}}%
\pgfusepath{stroke}%
\end{pgfscope}%
\begin{pgfscope}%
\pgfsetrectcap%
\pgfsetmiterjoin%
\pgfsetlinewidth{0.803000pt}%
\definecolor{currentstroke}{rgb}{0.000000,0.000000,0.000000}%
\pgfsetstrokecolor{currentstroke}%
\pgfsetdash{}{0pt}%
\pgfpathmoveto{\pgfqpoint{0.644914in}{3.272483in}}%
\pgfpathlineto{\pgfqpoint{3.744914in}{3.272483in}}%
\pgfusepath{stroke}%
\end{pgfscope}%
\end{pgfpicture}%
\makeatother%
\endgroup%
}
        \caption{Ignition delays of blends of DME and MeOH as a function of
        inverse temperature, for an equivalence ratio of \(\phi = 1.0\) and
        \(P_C = \SI[number-unit-product={\ }]{30}{\bar}\). Constant volume,
        adiabatic simulations are shown as the solid lines.}
        \label{fig:ign-delays}
    \end{minipage}\hfill%
    \begin{minipage}[t]{0.48\textwidth}
        \centering
        \resizebox{\linewidth}{!}{%% Creator: Matplotlib, PGF backend
%%
%% To include the figure in your LaTeX document, write
%%   \input{<filename>.pgf}
%%
%% Make sure the required packages are loaded in your preamble
%%   \usepackage{pgf}
%%
%% Figures using additional raster images can only be included by \input if
%% they are in the same directory as the main LaTeX file. For loading figures
%% from other directories you can use the `import` package
%%   \usepackage{import}
%% and then include the figures with
%%   \import{<path to file>}{<filename>.pgf}
%%
%% Matplotlib used the following preamble
%%   \usepackage[utf8x]{inputenc}
%%   \usepackage[T1]{fontenc}
%%   \usepackage{mathptmx}
%%   \usepackage{mathtools}
%%
\begingroup%
\makeatletter%
\begin{pgfpicture}%
\pgfpathrectangle{\pgfpointorigin}{\pgfqpoint{3.881025in}{3.391925in}}%
\pgfusepath{use as bounding box, clip}%
\begin{pgfscope}%
\pgfsetbuttcap%
\pgfsetmiterjoin%
\definecolor{currentfill}{rgb}{1.000000,1.000000,1.000000}%
\pgfsetfillcolor{currentfill}%
\pgfsetlinewidth{0.000000pt}%
\definecolor{currentstroke}{rgb}{1.000000,1.000000,1.000000}%
\pgfsetstrokecolor{currentstroke}%
\pgfsetdash{}{0pt}%
\pgfpathmoveto{\pgfqpoint{0.000000in}{0.000000in}}%
\pgfpathlineto{\pgfqpoint{3.881025in}{0.000000in}}%
\pgfpathlineto{\pgfqpoint{3.881025in}{3.391925in}}%
\pgfpathlineto{\pgfqpoint{0.000000in}{3.391925in}}%
\pgfpathclose%
\pgfusepath{fill}%
\end{pgfscope}%
\begin{pgfscope}%
\pgfsetbuttcap%
\pgfsetmiterjoin%
\definecolor{currentfill}{rgb}{1.000000,1.000000,1.000000}%
\pgfsetfillcolor{currentfill}%
\pgfsetlinewidth{0.000000pt}%
\definecolor{currentstroke}{rgb}{0.000000,0.000000,0.000000}%
\pgfsetstrokecolor{currentstroke}%
\pgfsetstrokeopacity{0.000000}%
\pgfsetdash{}{0pt}%
\pgfpathmoveto{\pgfqpoint{0.631025in}{0.546925in}}%
\pgfpathlineto{\pgfqpoint{3.731025in}{0.546925in}}%
\pgfpathlineto{\pgfqpoint{3.731025in}{3.241925in}}%
\pgfpathlineto{\pgfqpoint{0.631025in}{3.241925in}}%
\pgfpathclose%
\pgfusepath{fill}%
\end{pgfscope}%
\begin{pgfscope}%
\pgfsetbuttcap%
\pgfsetroundjoin%
\definecolor{currentfill}{rgb}{0.000000,0.000000,0.000000}%
\pgfsetfillcolor{currentfill}%
\pgfsetlinewidth{1.003750pt}%
\definecolor{currentstroke}{rgb}{0.000000,0.000000,0.000000}%
\pgfsetstrokecolor{currentstroke}%
\pgfsetdash{}{0pt}%
\pgfsys@defobject{currentmarker}{\pgfqpoint{0.000000in}{-0.069444in}}{\pgfqpoint{0.000000in}{0.000000in}}{%
\pgfpathmoveto{\pgfqpoint{0.000000in}{0.000000in}}%
\pgfpathlineto{\pgfqpoint{0.000000in}{-0.069444in}}%
\pgfusepath{stroke,fill}%
}%
\begin{pgfscope}%
\pgfsys@transformshift{0.889358in}{0.546925in}%
\pgfsys@useobject{currentmarker}{}%
\end{pgfscope}%
\end{pgfscope}%
\begin{pgfscope}%
\pgftext[x=0.889358in,y=0.428869in,,top]{\rmfamily\fontsize{10.000000}{12.000000}\selectfont \(\displaystyle 0\)}%
\end{pgfscope}%
\begin{pgfscope}%
\pgfsetbuttcap%
\pgfsetroundjoin%
\definecolor{currentfill}{rgb}{0.000000,0.000000,0.000000}%
\pgfsetfillcolor{currentfill}%
\pgfsetlinewidth{1.003750pt}%
\definecolor{currentstroke}{rgb}{0.000000,0.000000,0.000000}%
\pgfsetstrokecolor{currentstroke}%
\pgfsetdash{}{0pt}%
\pgfsys@defobject{currentmarker}{\pgfqpoint{0.000000in}{-0.069444in}}{\pgfqpoint{0.000000in}{0.000000in}}{%
\pgfpathmoveto{\pgfqpoint{0.000000in}{0.000000in}}%
\pgfpathlineto{\pgfqpoint{0.000000in}{-0.069444in}}%
\pgfusepath{stroke,fill}%
}%
\begin{pgfscope}%
\pgfsys@transformshift{1.406025in}{0.546925in}%
\pgfsys@useobject{currentmarker}{}%
\end{pgfscope}%
\end{pgfscope}%
\begin{pgfscope}%
\pgftext[x=1.406025in,y=0.428869in,,top]{\rmfamily\fontsize{10.000000}{12.000000}\selectfont \(\displaystyle 20\)}%
\end{pgfscope}%
\begin{pgfscope}%
\pgfsetbuttcap%
\pgfsetroundjoin%
\definecolor{currentfill}{rgb}{0.000000,0.000000,0.000000}%
\pgfsetfillcolor{currentfill}%
\pgfsetlinewidth{1.003750pt}%
\definecolor{currentstroke}{rgb}{0.000000,0.000000,0.000000}%
\pgfsetstrokecolor{currentstroke}%
\pgfsetdash{}{0pt}%
\pgfsys@defobject{currentmarker}{\pgfqpoint{0.000000in}{-0.069444in}}{\pgfqpoint{0.000000in}{0.000000in}}{%
\pgfpathmoveto{\pgfqpoint{0.000000in}{0.000000in}}%
\pgfpathlineto{\pgfqpoint{0.000000in}{-0.069444in}}%
\pgfusepath{stroke,fill}%
}%
\begin{pgfscope}%
\pgfsys@transformshift{1.922691in}{0.546925in}%
\pgfsys@useobject{currentmarker}{}%
\end{pgfscope}%
\end{pgfscope}%
\begin{pgfscope}%
\pgftext[x=1.922691in,y=0.428869in,,top]{\rmfamily\fontsize{10.000000}{12.000000}\selectfont \(\displaystyle 40\)}%
\end{pgfscope}%
\begin{pgfscope}%
\pgfsetbuttcap%
\pgfsetroundjoin%
\definecolor{currentfill}{rgb}{0.000000,0.000000,0.000000}%
\pgfsetfillcolor{currentfill}%
\pgfsetlinewidth{1.003750pt}%
\definecolor{currentstroke}{rgb}{0.000000,0.000000,0.000000}%
\pgfsetstrokecolor{currentstroke}%
\pgfsetdash{}{0pt}%
\pgfsys@defobject{currentmarker}{\pgfqpoint{0.000000in}{-0.069444in}}{\pgfqpoint{0.000000in}{0.000000in}}{%
\pgfpathmoveto{\pgfqpoint{0.000000in}{0.000000in}}%
\pgfpathlineto{\pgfqpoint{0.000000in}{-0.069444in}}%
\pgfusepath{stroke,fill}%
}%
\begin{pgfscope}%
\pgfsys@transformshift{2.439358in}{0.546925in}%
\pgfsys@useobject{currentmarker}{}%
\end{pgfscope}%
\end{pgfscope}%
\begin{pgfscope}%
\pgftext[x=2.439358in,y=0.428869in,,top]{\rmfamily\fontsize{10.000000}{12.000000}\selectfont \(\displaystyle 60\)}%
\end{pgfscope}%
\begin{pgfscope}%
\pgfsetbuttcap%
\pgfsetroundjoin%
\definecolor{currentfill}{rgb}{0.000000,0.000000,0.000000}%
\pgfsetfillcolor{currentfill}%
\pgfsetlinewidth{1.003750pt}%
\definecolor{currentstroke}{rgb}{0.000000,0.000000,0.000000}%
\pgfsetstrokecolor{currentstroke}%
\pgfsetdash{}{0pt}%
\pgfsys@defobject{currentmarker}{\pgfqpoint{0.000000in}{-0.069444in}}{\pgfqpoint{0.000000in}{0.000000in}}{%
\pgfpathmoveto{\pgfqpoint{0.000000in}{0.000000in}}%
\pgfpathlineto{\pgfqpoint{0.000000in}{-0.069444in}}%
\pgfusepath{stroke,fill}%
}%
\begin{pgfscope}%
\pgfsys@transformshift{2.956025in}{0.546925in}%
\pgfsys@useobject{currentmarker}{}%
\end{pgfscope}%
\end{pgfscope}%
\begin{pgfscope}%
\pgftext[x=2.956025in,y=0.428869in,,top]{\rmfamily\fontsize{10.000000}{12.000000}\selectfont \(\displaystyle 80\)}%
\end{pgfscope}%
\begin{pgfscope}%
\pgfsetbuttcap%
\pgfsetroundjoin%
\definecolor{currentfill}{rgb}{0.000000,0.000000,0.000000}%
\pgfsetfillcolor{currentfill}%
\pgfsetlinewidth{1.003750pt}%
\definecolor{currentstroke}{rgb}{0.000000,0.000000,0.000000}%
\pgfsetstrokecolor{currentstroke}%
\pgfsetdash{}{0pt}%
\pgfsys@defobject{currentmarker}{\pgfqpoint{0.000000in}{-0.069444in}}{\pgfqpoint{0.000000in}{0.000000in}}{%
\pgfpathmoveto{\pgfqpoint{0.000000in}{0.000000in}}%
\pgfpathlineto{\pgfqpoint{0.000000in}{-0.069444in}}%
\pgfusepath{stroke,fill}%
}%
\begin{pgfscope}%
\pgfsys@transformshift{3.472691in}{0.546925in}%
\pgfsys@useobject{currentmarker}{}%
\end{pgfscope}%
\end{pgfscope}%
\begin{pgfscope}%
\pgftext[x=3.472691in,y=0.428869in,,top]{\rmfamily\fontsize{10.000000}{12.000000}\selectfont \(\displaystyle 100\)}%
\end{pgfscope}%
\begin{pgfscope}%
\pgfsetbuttcap%
\pgfsetroundjoin%
\definecolor{currentfill}{rgb}{0.000000,0.000000,0.000000}%
\pgfsetfillcolor{currentfill}%
\pgfsetlinewidth{1.003750pt}%
\definecolor{currentstroke}{rgb}{0.000000,0.000000,0.000000}%
\pgfsetstrokecolor{currentstroke}%
\pgfsetdash{}{0pt}%
\pgfsys@defobject{currentmarker}{\pgfqpoint{0.000000in}{-0.034722in}}{\pgfqpoint{0.000000in}{0.000000in}}{%
\pgfpathmoveto{\pgfqpoint{0.000000in}{0.000000in}}%
\pgfpathlineto{\pgfqpoint{0.000000in}{-0.034722in}}%
\pgfusepath{stroke,fill}%
}%
\begin{pgfscope}%
\pgfsys@transformshift{0.760191in}{0.546925in}%
\pgfsys@useobject{currentmarker}{}%
\end{pgfscope}%
\end{pgfscope}%
\begin{pgfscope}%
\pgfsetbuttcap%
\pgfsetroundjoin%
\definecolor{currentfill}{rgb}{0.000000,0.000000,0.000000}%
\pgfsetfillcolor{currentfill}%
\pgfsetlinewidth{1.003750pt}%
\definecolor{currentstroke}{rgb}{0.000000,0.000000,0.000000}%
\pgfsetstrokecolor{currentstroke}%
\pgfsetdash{}{0pt}%
\pgfsys@defobject{currentmarker}{\pgfqpoint{0.000000in}{-0.034722in}}{\pgfqpoint{0.000000in}{0.000000in}}{%
\pgfpathmoveto{\pgfqpoint{0.000000in}{0.000000in}}%
\pgfpathlineto{\pgfqpoint{0.000000in}{-0.034722in}}%
\pgfusepath{stroke,fill}%
}%
\begin{pgfscope}%
\pgfsys@transformshift{1.018525in}{0.546925in}%
\pgfsys@useobject{currentmarker}{}%
\end{pgfscope}%
\end{pgfscope}%
\begin{pgfscope}%
\pgfsetbuttcap%
\pgfsetroundjoin%
\definecolor{currentfill}{rgb}{0.000000,0.000000,0.000000}%
\pgfsetfillcolor{currentfill}%
\pgfsetlinewidth{1.003750pt}%
\definecolor{currentstroke}{rgb}{0.000000,0.000000,0.000000}%
\pgfsetstrokecolor{currentstroke}%
\pgfsetdash{}{0pt}%
\pgfsys@defobject{currentmarker}{\pgfqpoint{0.000000in}{-0.034722in}}{\pgfqpoint{0.000000in}{0.000000in}}{%
\pgfpathmoveto{\pgfqpoint{0.000000in}{0.000000in}}%
\pgfpathlineto{\pgfqpoint{0.000000in}{-0.034722in}}%
\pgfusepath{stroke,fill}%
}%
\begin{pgfscope}%
\pgfsys@transformshift{1.147691in}{0.546925in}%
\pgfsys@useobject{currentmarker}{}%
\end{pgfscope}%
\end{pgfscope}%
\begin{pgfscope}%
\pgfsetbuttcap%
\pgfsetroundjoin%
\definecolor{currentfill}{rgb}{0.000000,0.000000,0.000000}%
\pgfsetfillcolor{currentfill}%
\pgfsetlinewidth{1.003750pt}%
\definecolor{currentstroke}{rgb}{0.000000,0.000000,0.000000}%
\pgfsetstrokecolor{currentstroke}%
\pgfsetdash{}{0pt}%
\pgfsys@defobject{currentmarker}{\pgfqpoint{0.000000in}{-0.034722in}}{\pgfqpoint{0.000000in}{0.000000in}}{%
\pgfpathmoveto{\pgfqpoint{0.000000in}{0.000000in}}%
\pgfpathlineto{\pgfqpoint{0.000000in}{-0.034722in}}%
\pgfusepath{stroke,fill}%
}%
\begin{pgfscope}%
\pgfsys@transformshift{1.276858in}{0.546925in}%
\pgfsys@useobject{currentmarker}{}%
\end{pgfscope}%
\end{pgfscope}%
\begin{pgfscope}%
\pgfsetbuttcap%
\pgfsetroundjoin%
\definecolor{currentfill}{rgb}{0.000000,0.000000,0.000000}%
\pgfsetfillcolor{currentfill}%
\pgfsetlinewidth{1.003750pt}%
\definecolor{currentstroke}{rgb}{0.000000,0.000000,0.000000}%
\pgfsetstrokecolor{currentstroke}%
\pgfsetdash{}{0pt}%
\pgfsys@defobject{currentmarker}{\pgfqpoint{0.000000in}{-0.034722in}}{\pgfqpoint{0.000000in}{0.000000in}}{%
\pgfpathmoveto{\pgfqpoint{0.000000in}{0.000000in}}%
\pgfpathlineto{\pgfqpoint{0.000000in}{-0.034722in}}%
\pgfusepath{stroke,fill}%
}%
\begin{pgfscope}%
\pgfsys@transformshift{1.535191in}{0.546925in}%
\pgfsys@useobject{currentmarker}{}%
\end{pgfscope}%
\end{pgfscope}%
\begin{pgfscope}%
\pgfsetbuttcap%
\pgfsetroundjoin%
\definecolor{currentfill}{rgb}{0.000000,0.000000,0.000000}%
\pgfsetfillcolor{currentfill}%
\pgfsetlinewidth{1.003750pt}%
\definecolor{currentstroke}{rgb}{0.000000,0.000000,0.000000}%
\pgfsetstrokecolor{currentstroke}%
\pgfsetdash{}{0pt}%
\pgfsys@defobject{currentmarker}{\pgfqpoint{0.000000in}{-0.034722in}}{\pgfqpoint{0.000000in}{0.000000in}}{%
\pgfpathmoveto{\pgfqpoint{0.000000in}{0.000000in}}%
\pgfpathlineto{\pgfqpoint{0.000000in}{-0.034722in}}%
\pgfusepath{stroke,fill}%
}%
\begin{pgfscope}%
\pgfsys@transformshift{1.664358in}{0.546925in}%
\pgfsys@useobject{currentmarker}{}%
\end{pgfscope}%
\end{pgfscope}%
\begin{pgfscope}%
\pgfsetbuttcap%
\pgfsetroundjoin%
\definecolor{currentfill}{rgb}{0.000000,0.000000,0.000000}%
\pgfsetfillcolor{currentfill}%
\pgfsetlinewidth{1.003750pt}%
\definecolor{currentstroke}{rgb}{0.000000,0.000000,0.000000}%
\pgfsetstrokecolor{currentstroke}%
\pgfsetdash{}{0pt}%
\pgfsys@defobject{currentmarker}{\pgfqpoint{0.000000in}{-0.034722in}}{\pgfqpoint{0.000000in}{0.000000in}}{%
\pgfpathmoveto{\pgfqpoint{0.000000in}{0.000000in}}%
\pgfpathlineto{\pgfqpoint{0.000000in}{-0.034722in}}%
\pgfusepath{stroke,fill}%
}%
\begin{pgfscope}%
\pgfsys@transformshift{1.793525in}{0.546925in}%
\pgfsys@useobject{currentmarker}{}%
\end{pgfscope}%
\end{pgfscope}%
\begin{pgfscope}%
\pgfsetbuttcap%
\pgfsetroundjoin%
\definecolor{currentfill}{rgb}{0.000000,0.000000,0.000000}%
\pgfsetfillcolor{currentfill}%
\pgfsetlinewidth{1.003750pt}%
\definecolor{currentstroke}{rgb}{0.000000,0.000000,0.000000}%
\pgfsetstrokecolor{currentstroke}%
\pgfsetdash{}{0pt}%
\pgfsys@defobject{currentmarker}{\pgfqpoint{0.000000in}{-0.034722in}}{\pgfqpoint{0.000000in}{0.000000in}}{%
\pgfpathmoveto{\pgfqpoint{0.000000in}{0.000000in}}%
\pgfpathlineto{\pgfqpoint{0.000000in}{-0.034722in}}%
\pgfusepath{stroke,fill}%
}%
\begin{pgfscope}%
\pgfsys@transformshift{2.051858in}{0.546925in}%
\pgfsys@useobject{currentmarker}{}%
\end{pgfscope}%
\end{pgfscope}%
\begin{pgfscope}%
\pgfsetbuttcap%
\pgfsetroundjoin%
\definecolor{currentfill}{rgb}{0.000000,0.000000,0.000000}%
\pgfsetfillcolor{currentfill}%
\pgfsetlinewidth{1.003750pt}%
\definecolor{currentstroke}{rgb}{0.000000,0.000000,0.000000}%
\pgfsetstrokecolor{currentstroke}%
\pgfsetdash{}{0pt}%
\pgfsys@defobject{currentmarker}{\pgfqpoint{0.000000in}{-0.034722in}}{\pgfqpoint{0.000000in}{0.000000in}}{%
\pgfpathmoveto{\pgfqpoint{0.000000in}{0.000000in}}%
\pgfpathlineto{\pgfqpoint{0.000000in}{-0.034722in}}%
\pgfusepath{stroke,fill}%
}%
\begin{pgfscope}%
\pgfsys@transformshift{2.181025in}{0.546925in}%
\pgfsys@useobject{currentmarker}{}%
\end{pgfscope}%
\end{pgfscope}%
\begin{pgfscope}%
\pgfsetbuttcap%
\pgfsetroundjoin%
\definecolor{currentfill}{rgb}{0.000000,0.000000,0.000000}%
\pgfsetfillcolor{currentfill}%
\pgfsetlinewidth{1.003750pt}%
\definecolor{currentstroke}{rgb}{0.000000,0.000000,0.000000}%
\pgfsetstrokecolor{currentstroke}%
\pgfsetdash{}{0pt}%
\pgfsys@defobject{currentmarker}{\pgfqpoint{0.000000in}{-0.034722in}}{\pgfqpoint{0.000000in}{0.000000in}}{%
\pgfpathmoveto{\pgfqpoint{0.000000in}{0.000000in}}%
\pgfpathlineto{\pgfqpoint{0.000000in}{-0.034722in}}%
\pgfusepath{stroke,fill}%
}%
\begin{pgfscope}%
\pgfsys@transformshift{2.310191in}{0.546925in}%
\pgfsys@useobject{currentmarker}{}%
\end{pgfscope}%
\end{pgfscope}%
\begin{pgfscope}%
\pgfsetbuttcap%
\pgfsetroundjoin%
\definecolor{currentfill}{rgb}{0.000000,0.000000,0.000000}%
\pgfsetfillcolor{currentfill}%
\pgfsetlinewidth{1.003750pt}%
\definecolor{currentstroke}{rgb}{0.000000,0.000000,0.000000}%
\pgfsetstrokecolor{currentstroke}%
\pgfsetdash{}{0pt}%
\pgfsys@defobject{currentmarker}{\pgfqpoint{0.000000in}{-0.034722in}}{\pgfqpoint{0.000000in}{0.000000in}}{%
\pgfpathmoveto{\pgfqpoint{0.000000in}{0.000000in}}%
\pgfpathlineto{\pgfqpoint{0.000000in}{-0.034722in}}%
\pgfusepath{stroke,fill}%
}%
\begin{pgfscope}%
\pgfsys@transformshift{2.568525in}{0.546925in}%
\pgfsys@useobject{currentmarker}{}%
\end{pgfscope}%
\end{pgfscope}%
\begin{pgfscope}%
\pgfsetbuttcap%
\pgfsetroundjoin%
\definecolor{currentfill}{rgb}{0.000000,0.000000,0.000000}%
\pgfsetfillcolor{currentfill}%
\pgfsetlinewidth{1.003750pt}%
\definecolor{currentstroke}{rgb}{0.000000,0.000000,0.000000}%
\pgfsetstrokecolor{currentstroke}%
\pgfsetdash{}{0pt}%
\pgfsys@defobject{currentmarker}{\pgfqpoint{0.000000in}{-0.034722in}}{\pgfqpoint{0.000000in}{0.000000in}}{%
\pgfpathmoveto{\pgfqpoint{0.000000in}{0.000000in}}%
\pgfpathlineto{\pgfqpoint{0.000000in}{-0.034722in}}%
\pgfusepath{stroke,fill}%
}%
\begin{pgfscope}%
\pgfsys@transformshift{2.697691in}{0.546925in}%
\pgfsys@useobject{currentmarker}{}%
\end{pgfscope}%
\end{pgfscope}%
\begin{pgfscope}%
\pgfsetbuttcap%
\pgfsetroundjoin%
\definecolor{currentfill}{rgb}{0.000000,0.000000,0.000000}%
\pgfsetfillcolor{currentfill}%
\pgfsetlinewidth{1.003750pt}%
\definecolor{currentstroke}{rgb}{0.000000,0.000000,0.000000}%
\pgfsetstrokecolor{currentstroke}%
\pgfsetdash{}{0pt}%
\pgfsys@defobject{currentmarker}{\pgfqpoint{0.000000in}{-0.034722in}}{\pgfqpoint{0.000000in}{0.000000in}}{%
\pgfpathmoveto{\pgfqpoint{0.000000in}{0.000000in}}%
\pgfpathlineto{\pgfqpoint{0.000000in}{-0.034722in}}%
\pgfusepath{stroke,fill}%
}%
\begin{pgfscope}%
\pgfsys@transformshift{2.826858in}{0.546925in}%
\pgfsys@useobject{currentmarker}{}%
\end{pgfscope}%
\end{pgfscope}%
\begin{pgfscope}%
\pgfsetbuttcap%
\pgfsetroundjoin%
\definecolor{currentfill}{rgb}{0.000000,0.000000,0.000000}%
\pgfsetfillcolor{currentfill}%
\pgfsetlinewidth{1.003750pt}%
\definecolor{currentstroke}{rgb}{0.000000,0.000000,0.000000}%
\pgfsetstrokecolor{currentstroke}%
\pgfsetdash{}{0pt}%
\pgfsys@defobject{currentmarker}{\pgfqpoint{0.000000in}{-0.034722in}}{\pgfqpoint{0.000000in}{0.000000in}}{%
\pgfpathmoveto{\pgfqpoint{0.000000in}{0.000000in}}%
\pgfpathlineto{\pgfqpoint{0.000000in}{-0.034722in}}%
\pgfusepath{stroke,fill}%
}%
\begin{pgfscope}%
\pgfsys@transformshift{3.085191in}{0.546925in}%
\pgfsys@useobject{currentmarker}{}%
\end{pgfscope}%
\end{pgfscope}%
\begin{pgfscope}%
\pgfsetbuttcap%
\pgfsetroundjoin%
\definecolor{currentfill}{rgb}{0.000000,0.000000,0.000000}%
\pgfsetfillcolor{currentfill}%
\pgfsetlinewidth{1.003750pt}%
\definecolor{currentstroke}{rgb}{0.000000,0.000000,0.000000}%
\pgfsetstrokecolor{currentstroke}%
\pgfsetdash{}{0pt}%
\pgfsys@defobject{currentmarker}{\pgfqpoint{0.000000in}{-0.034722in}}{\pgfqpoint{0.000000in}{0.000000in}}{%
\pgfpathmoveto{\pgfqpoint{0.000000in}{0.000000in}}%
\pgfpathlineto{\pgfqpoint{0.000000in}{-0.034722in}}%
\pgfusepath{stroke,fill}%
}%
\begin{pgfscope}%
\pgfsys@transformshift{3.214358in}{0.546925in}%
\pgfsys@useobject{currentmarker}{}%
\end{pgfscope}%
\end{pgfscope}%
\begin{pgfscope}%
\pgfsetbuttcap%
\pgfsetroundjoin%
\definecolor{currentfill}{rgb}{0.000000,0.000000,0.000000}%
\pgfsetfillcolor{currentfill}%
\pgfsetlinewidth{1.003750pt}%
\definecolor{currentstroke}{rgb}{0.000000,0.000000,0.000000}%
\pgfsetstrokecolor{currentstroke}%
\pgfsetdash{}{0pt}%
\pgfsys@defobject{currentmarker}{\pgfqpoint{0.000000in}{-0.034722in}}{\pgfqpoint{0.000000in}{0.000000in}}{%
\pgfpathmoveto{\pgfqpoint{0.000000in}{0.000000in}}%
\pgfpathlineto{\pgfqpoint{0.000000in}{-0.034722in}}%
\pgfusepath{stroke,fill}%
}%
\begin{pgfscope}%
\pgfsys@transformshift{3.343525in}{0.546925in}%
\pgfsys@useobject{currentmarker}{}%
\end{pgfscope}%
\end{pgfscope}%
\begin{pgfscope}%
\pgfsetbuttcap%
\pgfsetroundjoin%
\definecolor{currentfill}{rgb}{0.000000,0.000000,0.000000}%
\pgfsetfillcolor{currentfill}%
\pgfsetlinewidth{1.003750pt}%
\definecolor{currentstroke}{rgb}{0.000000,0.000000,0.000000}%
\pgfsetstrokecolor{currentstroke}%
\pgfsetdash{}{0pt}%
\pgfsys@defobject{currentmarker}{\pgfqpoint{0.000000in}{-0.034722in}}{\pgfqpoint{0.000000in}{0.000000in}}{%
\pgfpathmoveto{\pgfqpoint{0.000000in}{0.000000in}}%
\pgfpathlineto{\pgfqpoint{0.000000in}{-0.034722in}}%
\pgfusepath{stroke,fill}%
}%
\begin{pgfscope}%
\pgfsys@transformshift{3.601858in}{0.546925in}%
\pgfsys@useobject{currentmarker}{}%
\end{pgfscope}%
\end{pgfscope}%
\begin{pgfscope}%
\pgfsetbuttcap%
\pgfsetroundjoin%
\definecolor{currentfill}{rgb}{0.000000,0.000000,0.000000}%
\pgfsetfillcolor{currentfill}%
\pgfsetlinewidth{1.003750pt}%
\definecolor{currentstroke}{rgb}{0.000000,0.000000,0.000000}%
\pgfsetstrokecolor{currentstroke}%
\pgfsetdash{}{0pt}%
\pgfsys@defobject{currentmarker}{\pgfqpoint{0.000000in}{-0.034722in}}{\pgfqpoint{0.000000in}{0.000000in}}{%
\pgfpathmoveto{\pgfqpoint{0.000000in}{0.000000in}}%
\pgfpathlineto{\pgfqpoint{0.000000in}{-0.034722in}}%
\pgfusepath{stroke,fill}%
}%
\begin{pgfscope}%
\pgfsys@transformshift{3.731025in}{0.546925in}%
\pgfsys@useobject{currentmarker}{}%
\end{pgfscope}%
\end{pgfscope}%
\begin{pgfscope}%
\pgftext[x=2.181025in,y=0.249080in,,top]{\rmfamily\fontsize{12.000000}{14.400000}\selectfont \% DME}%
\end{pgfscope}%
\begin{pgfscope}%
\pgfsetbuttcap%
\pgfsetroundjoin%
\definecolor{currentfill}{rgb}{0.000000,0.000000,0.000000}%
\pgfsetfillcolor{currentfill}%
\pgfsetlinewidth{1.003750pt}%
\definecolor{currentstroke}{rgb}{0.000000,0.000000,0.000000}%
\pgfsetstrokecolor{currentstroke}%
\pgfsetdash{}{0pt}%
\pgfsys@defobject{currentmarker}{\pgfqpoint{-0.069444in}{0.000000in}}{\pgfqpoint{0.000000in}{0.000000in}}{%
\pgfpathmoveto{\pgfqpoint{0.000000in}{0.000000in}}%
\pgfpathlineto{\pgfqpoint{-0.069444in}{0.000000in}}%
\pgfusepath{stroke,fill}%
}%
\begin{pgfscope}%
\pgfsys@transformshift{0.631025in}{0.546925in}%
\pgfsys@useobject{currentmarker}{}%
\end{pgfscope}%
\end{pgfscope}%
\begin{pgfscope}%
\pgftext[x=0.304636in,y=0.499842in,left,base]{\rmfamily\fontsize{10.000000}{12.000000}\selectfont \(\displaystyle 600\)}%
\end{pgfscope}%
\begin{pgfscope}%
\pgfsetbuttcap%
\pgfsetroundjoin%
\definecolor{currentfill}{rgb}{0.000000,0.000000,0.000000}%
\pgfsetfillcolor{currentfill}%
\pgfsetlinewidth{1.003750pt}%
\definecolor{currentstroke}{rgb}{0.000000,0.000000,0.000000}%
\pgfsetstrokecolor{currentstroke}%
\pgfsetdash{}{0pt}%
\pgfsys@defobject{currentmarker}{\pgfqpoint{-0.069444in}{0.000000in}}{\pgfqpoint{0.000000in}{0.000000in}}{%
\pgfpathmoveto{\pgfqpoint{0.000000in}{0.000000in}}%
\pgfpathlineto{\pgfqpoint{-0.069444in}{0.000000in}}%
\pgfusepath{stroke,fill}%
}%
\begin{pgfscope}%
\pgfsys@transformshift{0.631025in}{1.085925in}%
\pgfsys@useobject{currentmarker}{}%
\end{pgfscope}%
\end{pgfscope}%
\begin{pgfscope}%
\pgftext[x=0.304636in,y=1.038842in,left,base]{\rmfamily\fontsize{10.000000}{12.000000}\selectfont \(\displaystyle 650\)}%
\end{pgfscope}%
\begin{pgfscope}%
\pgfsetbuttcap%
\pgfsetroundjoin%
\definecolor{currentfill}{rgb}{0.000000,0.000000,0.000000}%
\pgfsetfillcolor{currentfill}%
\pgfsetlinewidth{1.003750pt}%
\definecolor{currentstroke}{rgb}{0.000000,0.000000,0.000000}%
\pgfsetstrokecolor{currentstroke}%
\pgfsetdash{}{0pt}%
\pgfsys@defobject{currentmarker}{\pgfqpoint{-0.069444in}{0.000000in}}{\pgfqpoint{0.000000in}{0.000000in}}{%
\pgfpathmoveto{\pgfqpoint{0.000000in}{0.000000in}}%
\pgfpathlineto{\pgfqpoint{-0.069444in}{0.000000in}}%
\pgfusepath{stroke,fill}%
}%
\begin{pgfscope}%
\pgfsys@transformshift{0.631025in}{1.624925in}%
\pgfsys@useobject{currentmarker}{}%
\end{pgfscope}%
\end{pgfscope}%
\begin{pgfscope}%
\pgftext[x=0.304636in,y=1.577842in,left,base]{\rmfamily\fontsize{10.000000}{12.000000}\selectfont \(\displaystyle 700\)}%
\end{pgfscope}%
\begin{pgfscope}%
\pgfsetbuttcap%
\pgfsetroundjoin%
\definecolor{currentfill}{rgb}{0.000000,0.000000,0.000000}%
\pgfsetfillcolor{currentfill}%
\pgfsetlinewidth{1.003750pt}%
\definecolor{currentstroke}{rgb}{0.000000,0.000000,0.000000}%
\pgfsetstrokecolor{currentstroke}%
\pgfsetdash{}{0pt}%
\pgfsys@defobject{currentmarker}{\pgfqpoint{-0.069444in}{0.000000in}}{\pgfqpoint{0.000000in}{0.000000in}}{%
\pgfpathmoveto{\pgfqpoint{0.000000in}{0.000000in}}%
\pgfpathlineto{\pgfqpoint{-0.069444in}{0.000000in}}%
\pgfusepath{stroke,fill}%
}%
\begin{pgfscope}%
\pgfsys@transformshift{0.631025in}{2.163925in}%
\pgfsys@useobject{currentmarker}{}%
\end{pgfscope}%
\end{pgfscope}%
\begin{pgfscope}%
\pgftext[x=0.304636in,y=2.116842in,left,base]{\rmfamily\fontsize{10.000000}{12.000000}\selectfont \(\displaystyle 750\)}%
\end{pgfscope}%
\begin{pgfscope}%
\pgfsetbuttcap%
\pgfsetroundjoin%
\definecolor{currentfill}{rgb}{0.000000,0.000000,0.000000}%
\pgfsetfillcolor{currentfill}%
\pgfsetlinewidth{1.003750pt}%
\definecolor{currentstroke}{rgb}{0.000000,0.000000,0.000000}%
\pgfsetstrokecolor{currentstroke}%
\pgfsetdash{}{0pt}%
\pgfsys@defobject{currentmarker}{\pgfqpoint{-0.069444in}{0.000000in}}{\pgfqpoint{0.000000in}{0.000000in}}{%
\pgfpathmoveto{\pgfqpoint{0.000000in}{0.000000in}}%
\pgfpathlineto{\pgfqpoint{-0.069444in}{0.000000in}}%
\pgfusepath{stroke,fill}%
}%
\begin{pgfscope}%
\pgfsys@transformshift{0.631025in}{2.702925in}%
\pgfsys@useobject{currentmarker}{}%
\end{pgfscope}%
\end{pgfscope}%
\begin{pgfscope}%
\pgftext[x=0.304636in,y=2.655842in,left,base]{\rmfamily\fontsize{10.000000}{12.000000}\selectfont \(\displaystyle 800\)}%
\end{pgfscope}%
\begin{pgfscope}%
\pgfsetbuttcap%
\pgfsetroundjoin%
\definecolor{currentfill}{rgb}{0.000000,0.000000,0.000000}%
\pgfsetfillcolor{currentfill}%
\pgfsetlinewidth{1.003750pt}%
\definecolor{currentstroke}{rgb}{0.000000,0.000000,0.000000}%
\pgfsetstrokecolor{currentstroke}%
\pgfsetdash{}{0pt}%
\pgfsys@defobject{currentmarker}{\pgfqpoint{-0.069444in}{0.000000in}}{\pgfqpoint{0.000000in}{0.000000in}}{%
\pgfpathmoveto{\pgfqpoint{0.000000in}{0.000000in}}%
\pgfpathlineto{\pgfqpoint{-0.069444in}{0.000000in}}%
\pgfusepath{stroke,fill}%
}%
\begin{pgfscope}%
\pgfsys@transformshift{0.631025in}{3.241925in}%
\pgfsys@useobject{currentmarker}{}%
\end{pgfscope}%
\end{pgfscope}%
\begin{pgfscope}%
\pgftext[x=0.304636in,y=3.194842in,left,base]{\rmfamily\fontsize{10.000000}{12.000000}\selectfont \(\displaystyle 850\)}%
\end{pgfscope}%
\begin{pgfscope}%
\pgfsetbuttcap%
\pgfsetroundjoin%
\definecolor{currentfill}{rgb}{0.000000,0.000000,0.000000}%
\pgfsetfillcolor{currentfill}%
\pgfsetlinewidth{1.003750pt}%
\definecolor{currentstroke}{rgb}{0.000000,0.000000,0.000000}%
\pgfsetstrokecolor{currentstroke}%
\pgfsetdash{}{0pt}%
\pgfsys@defobject{currentmarker}{\pgfqpoint{-0.034722in}{0.000000in}}{\pgfqpoint{0.000000in}{0.000000in}}{%
\pgfpathmoveto{\pgfqpoint{0.000000in}{0.000000in}}%
\pgfpathlineto{\pgfqpoint{-0.034722in}{0.000000in}}%
\pgfusepath{stroke,fill}%
}%
\begin{pgfscope}%
\pgfsys@transformshift{0.631025in}{0.681675in}%
\pgfsys@useobject{currentmarker}{}%
\end{pgfscope}%
\end{pgfscope}%
\begin{pgfscope}%
\pgfsetbuttcap%
\pgfsetroundjoin%
\definecolor{currentfill}{rgb}{0.000000,0.000000,0.000000}%
\pgfsetfillcolor{currentfill}%
\pgfsetlinewidth{1.003750pt}%
\definecolor{currentstroke}{rgb}{0.000000,0.000000,0.000000}%
\pgfsetstrokecolor{currentstroke}%
\pgfsetdash{}{0pt}%
\pgfsys@defobject{currentmarker}{\pgfqpoint{-0.034722in}{0.000000in}}{\pgfqpoint{0.000000in}{0.000000in}}{%
\pgfpathmoveto{\pgfqpoint{0.000000in}{0.000000in}}%
\pgfpathlineto{\pgfqpoint{-0.034722in}{0.000000in}}%
\pgfusepath{stroke,fill}%
}%
\begin{pgfscope}%
\pgfsys@transformshift{0.631025in}{0.816425in}%
\pgfsys@useobject{currentmarker}{}%
\end{pgfscope}%
\end{pgfscope}%
\begin{pgfscope}%
\pgfsetbuttcap%
\pgfsetroundjoin%
\definecolor{currentfill}{rgb}{0.000000,0.000000,0.000000}%
\pgfsetfillcolor{currentfill}%
\pgfsetlinewidth{1.003750pt}%
\definecolor{currentstroke}{rgb}{0.000000,0.000000,0.000000}%
\pgfsetstrokecolor{currentstroke}%
\pgfsetdash{}{0pt}%
\pgfsys@defobject{currentmarker}{\pgfqpoint{-0.034722in}{0.000000in}}{\pgfqpoint{0.000000in}{0.000000in}}{%
\pgfpathmoveto{\pgfqpoint{0.000000in}{0.000000in}}%
\pgfpathlineto{\pgfqpoint{-0.034722in}{0.000000in}}%
\pgfusepath{stroke,fill}%
}%
\begin{pgfscope}%
\pgfsys@transformshift{0.631025in}{0.951175in}%
\pgfsys@useobject{currentmarker}{}%
\end{pgfscope}%
\end{pgfscope}%
\begin{pgfscope}%
\pgfsetbuttcap%
\pgfsetroundjoin%
\definecolor{currentfill}{rgb}{0.000000,0.000000,0.000000}%
\pgfsetfillcolor{currentfill}%
\pgfsetlinewidth{1.003750pt}%
\definecolor{currentstroke}{rgb}{0.000000,0.000000,0.000000}%
\pgfsetstrokecolor{currentstroke}%
\pgfsetdash{}{0pt}%
\pgfsys@defobject{currentmarker}{\pgfqpoint{-0.034722in}{0.000000in}}{\pgfqpoint{0.000000in}{0.000000in}}{%
\pgfpathmoveto{\pgfqpoint{0.000000in}{0.000000in}}%
\pgfpathlineto{\pgfqpoint{-0.034722in}{0.000000in}}%
\pgfusepath{stroke,fill}%
}%
\begin{pgfscope}%
\pgfsys@transformshift{0.631025in}{1.220675in}%
\pgfsys@useobject{currentmarker}{}%
\end{pgfscope}%
\end{pgfscope}%
\begin{pgfscope}%
\pgfsetbuttcap%
\pgfsetroundjoin%
\definecolor{currentfill}{rgb}{0.000000,0.000000,0.000000}%
\pgfsetfillcolor{currentfill}%
\pgfsetlinewidth{1.003750pt}%
\definecolor{currentstroke}{rgb}{0.000000,0.000000,0.000000}%
\pgfsetstrokecolor{currentstroke}%
\pgfsetdash{}{0pt}%
\pgfsys@defobject{currentmarker}{\pgfqpoint{-0.034722in}{0.000000in}}{\pgfqpoint{0.000000in}{0.000000in}}{%
\pgfpathmoveto{\pgfqpoint{0.000000in}{0.000000in}}%
\pgfpathlineto{\pgfqpoint{-0.034722in}{0.000000in}}%
\pgfusepath{stroke,fill}%
}%
\begin{pgfscope}%
\pgfsys@transformshift{0.631025in}{1.355425in}%
\pgfsys@useobject{currentmarker}{}%
\end{pgfscope}%
\end{pgfscope}%
\begin{pgfscope}%
\pgfsetbuttcap%
\pgfsetroundjoin%
\definecolor{currentfill}{rgb}{0.000000,0.000000,0.000000}%
\pgfsetfillcolor{currentfill}%
\pgfsetlinewidth{1.003750pt}%
\definecolor{currentstroke}{rgb}{0.000000,0.000000,0.000000}%
\pgfsetstrokecolor{currentstroke}%
\pgfsetdash{}{0pt}%
\pgfsys@defobject{currentmarker}{\pgfqpoint{-0.034722in}{0.000000in}}{\pgfqpoint{0.000000in}{0.000000in}}{%
\pgfpathmoveto{\pgfqpoint{0.000000in}{0.000000in}}%
\pgfpathlineto{\pgfqpoint{-0.034722in}{0.000000in}}%
\pgfusepath{stroke,fill}%
}%
\begin{pgfscope}%
\pgfsys@transformshift{0.631025in}{1.490175in}%
\pgfsys@useobject{currentmarker}{}%
\end{pgfscope}%
\end{pgfscope}%
\begin{pgfscope}%
\pgfsetbuttcap%
\pgfsetroundjoin%
\definecolor{currentfill}{rgb}{0.000000,0.000000,0.000000}%
\pgfsetfillcolor{currentfill}%
\pgfsetlinewidth{1.003750pt}%
\definecolor{currentstroke}{rgb}{0.000000,0.000000,0.000000}%
\pgfsetstrokecolor{currentstroke}%
\pgfsetdash{}{0pt}%
\pgfsys@defobject{currentmarker}{\pgfqpoint{-0.034722in}{0.000000in}}{\pgfqpoint{0.000000in}{0.000000in}}{%
\pgfpathmoveto{\pgfqpoint{0.000000in}{0.000000in}}%
\pgfpathlineto{\pgfqpoint{-0.034722in}{0.000000in}}%
\pgfusepath{stroke,fill}%
}%
\begin{pgfscope}%
\pgfsys@transformshift{0.631025in}{1.759675in}%
\pgfsys@useobject{currentmarker}{}%
\end{pgfscope}%
\end{pgfscope}%
\begin{pgfscope}%
\pgfsetbuttcap%
\pgfsetroundjoin%
\definecolor{currentfill}{rgb}{0.000000,0.000000,0.000000}%
\pgfsetfillcolor{currentfill}%
\pgfsetlinewidth{1.003750pt}%
\definecolor{currentstroke}{rgb}{0.000000,0.000000,0.000000}%
\pgfsetstrokecolor{currentstroke}%
\pgfsetdash{}{0pt}%
\pgfsys@defobject{currentmarker}{\pgfqpoint{-0.034722in}{0.000000in}}{\pgfqpoint{0.000000in}{0.000000in}}{%
\pgfpathmoveto{\pgfqpoint{0.000000in}{0.000000in}}%
\pgfpathlineto{\pgfqpoint{-0.034722in}{0.000000in}}%
\pgfusepath{stroke,fill}%
}%
\begin{pgfscope}%
\pgfsys@transformshift{0.631025in}{1.894425in}%
\pgfsys@useobject{currentmarker}{}%
\end{pgfscope}%
\end{pgfscope}%
\begin{pgfscope}%
\pgfsetbuttcap%
\pgfsetroundjoin%
\definecolor{currentfill}{rgb}{0.000000,0.000000,0.000000}%
\pgfsetfillcolor{currentfill}%
\pgfsetlinewidth{1.003750pt}%
\definecolor{currentstroke}{rgb}{0.000000,0.000000,0.000000}%
\pgfsetstrokecolor{currentstroke}%
\pgfsetdash{}{0pt}%
\pgfsys@defobject{currentmarker}{\pgfqpoint{-0.034722in}{0.000000in}}{\pgfqpoint{0.000000in}{0.000000in}}{%
\pgfpathmoveto{\pgfqpoint{0.000000in}{0.000000in}}%
\pgfpathlineto{\pgfqpoint{-0.034722in}{0.000000in}}%
\pgfusepath{stroke,fill}%
}%
\begin{pgfscope}%
\pgfsys@transformshift{0.631025in}{2.029175in}%
\pgfsys@useobject{currentmarker}{}%
\end{pgfscope}%
\end{pgfscope}%
\begin{pgfscope}%
\pgfsetbuttcap%
\pgfsetroundjoin%
\definecolor{currentfill}{rgb}{0.000000,0.000000,0.000000}%
\pgfsetfillcolor{currentfill}%
\pgfsetlinewidth{1.003750pt}%
\definecolor{currentstroke}{rgb}{0.000000,0.000000,0.000000}%
\pgfsetstrokecolor{currentstroke}%
\pgfsetdash{}{0pt}%
\pgfsys@defobject{currentmarker}{\pgfqpoint{-0.034722in}{0.000000in}}{\pgfqpoint{0.000000in}{0.000000in}}{%
\pgfpathmoveto{\pgfqpoint{0.000000in}{0.000000in}}%
\pgfpathlineto{\pgfqpoint{-0.034722in}{0.000000in}}%
\pgfusepath{stroke,fill}%
}%
\begin{pgfscope}%
\pgfsys@transformshift{0.631025in}{2.298675in}%
\pgfsys@useobject{currentmarker}{}%
\end{pgfscope}%
\end{pgfscope}%
\begin{pgfscope}%
\pgfsetbuttcap%
\pgfsetroundjoin%
\definecolor{currentfill}{rgb}{0.000000,0.000000,0.000000}%
\pgfsetfillcolor{currentfill}%
\pgfsetlinewidth{1.003750pt}%
\definecolor{currentstroke}{rgb}{0.000000,0.000000,0.000000}%
\pgfsetstrokecolor{currentstroke}%
\pgfsetdash{}{0pt}%
\pgfsys@defobject{currentmarker}{\pgfqpoint{-0.034722in}{0.000000in}}{\pgfqpoint{0.000000in}{0.000000in}}{%
\pgfpathmoveto{\pgfqpoint{0.000000in}{0.000000in}}%
\pgfpathlineto{\pgfqpoint{-0.034722in}{0.000000in}}%
\pgfusepath{stroke,fill}%
}%
\begin{pgfscope}%
\pgfsys@transformshift{0.631025in}{2.433425in}%
\pgfsys@useobject{currentmarker}{}%
\end{pgfscope}%
\end{pgfscope}%
\begin{pgfscope}%
\pgfsetbuttcap%
\pgfsetroundjoin%
\definecolor{currentfill}{rgb}{0.000000,0.000000,0.000000}%
\pgfsetfillcolor{currentfill}%
\pgfsetlinewidth{1.003750pt}%
\definecolor{currentstroke}{rgb}{0.000000,0.000000,0.000000}%
\pgfsetstrokecolor{currentstroke}%
\pgfsetdash{}{0pt}%
\pgfsys@defobject{currentmarker}{\pgfqpoint{-0.034722in}{0.000000in}}{\pgfqpoint{0.000000in}{0.000000in}}{%
\pgfpathmoveto{\pgfqpoint{0.000000in}{0.000000in}}%
\pgfpathlineto{\pgfqpoint{-0.034722in}{0.000000in}}%
\pgfusepath{stroke,fill}%
}%
\begin{pgfscope}%
\pgfsys@transformshift{0.631025in}{2.568175in}%
\pgfsys@useobject{currentmarker}{}%
\end{pgfscope}%
\end{pgfscope}%
\begin{pgfscope}%
\pgfsetbuttcap%
\pgfsetroundjoin%
\definecolor{currentfill}{rgb}{0.000000,0.000000,0.000000}%
\pgfsetfillcolor{currentfill}%
\pgfsetlinewidth{1.003750pt}%
\definecolor{currentstroke}{rgb}{0.000000,0.000000,0.000000}%
\pgfsetstrokecolor{currentstroke}%
\pgfsetdash{}{0pt}%
\pgfsys@defobject{currentmarker}{\pgfqpoint{-0.034722in}{0.000000in}}{\pgfqpoint{0.000000in}{0.000000in}}{%
\pgfpathmoveto{\pgfqpoint{0.000000in}{0.000000in}}%
\pgfpathlineto{\pgfqpoint{-0.034722in}{0.000000in}}%
\pgfusepath{stroke,fill}%
}%
\begin{pgfscope}%
\pgfsys@transformshift{0.631025in}{2.837675in}%
\pgfsys@useobject{currentmarker}{}%
\end{pgfscope}%
\end{pgfscope}%
\begin{pgfscope}%
\pgfsetbuttcap%
\pgfsetroundjoin%
\definecolor{currentfill}{rgb}{0.000000,0.000000,0.000000}%
\pgfsetfillcolor{currentfill}%
\pgfsetlinewidth{1.003750pt}%
\definecolor{currentstroke}{rgb}{0.000000,0.000000,0.000000}%
\pgfsetstrokecolor{currentstroke}%
\pgfsetdash{}{0pt}%
\pgfsys@defobject{currentmarker}{\pgfqpoint{-0.034722in}{0.000000in}}{\pgfqpoint{0.000000in}{0.000000in}}{%
\pgfpathmoveto{\pgfqpoint{0.000000in}{0.000000in}}%
\pgfpathlineto{\pgfqpoint{-0.034722in}{0.000000in}}%
\pgfusepath{stroke,fill}%
}%
\begin{pgfscope}%
\pgfsys@transformshift{0.631025in}{2.972425in}%
\pgfsys@useobject{currentmarker}{}%
\end{pgfscope}%
\end{pgfscope}%
\begin{pgfscope}%
\pgfsetbuttcap%
\pgfsetroundjoin%
\definecolor{currentfill}{rgb}{0.000000,0.000000,0.000000}%
\pgfsetfillcolor{currentfill}%
\pgfsetlinewidth{1.003750pt}%
\definecolor{currentstroke}{rgb}{0.000000,0.000000,0.000000}%
\pgfsetstrokecolor{currentstroke}%
\pgfsetdash{}{0pt}%
\pgfsys@defobject{currentmarker}{\pgfqpoint{-0.034722in}{0.000000in}}{\pgfqpoint{0.000000in}{0.000000in}}{%
\pgfpathmoveto{\pgfqpoint{0.000000in}{0.000000in}}%
\pgfpathlineto{\pgfqpoint{-0.034722in}{0.000000in}}%
\pgfusepath{stroke,fill}%
}%
\begin{pgfscope}%
\pgfsys@transformshift{0.631025in}{3.107175in}%
\pgfsys@useobject{currentmarker}{}%
\end{pgfscope}%
\end{pgfscope}%
\begin{pgfscope}%
\pgftext[x=0.249080in,y=1.894425in,,bottom,rotate=90.000000]{\rmfamily\fontsize{12.000000}{14.400000}\selectfont \(\displaystyle T_C\), K}%
\end{pgfscope}%
\begin{pgfscope}%
\pgfpathrectangle{\pgfqpoint{0.631025in}{0.546925in}}{\pgfqpoint{3.100000in}{2.695000in}} %
\pgfusepath{clip}%
\pgfsetrectcap%
\pgfsetroundjoin%
\pgfsetlinewidth{1.505625pt}%
\definecolor{currentstroke}{rgb}{0.121569,0.466667,0.705882}%
\pgfsetstrokecolor{currentstroke}%
\pgfsetdash{}{0pt}%
\pgfpathmoveto{\pgfqpoint{3.472691in}{0.848765in}}%
\pgfpathlineto{\pgfqpoint{2.826858in}{0.950830in}}%
\pgfpathlineto{\pgfqpoint{2.181025in}{1.250117in}}%
\pgfpathlineto{\pgfqpoint{1.535191in}{2.354622in}}%
\pgfpathlineto{\pgfqpoint{0.889358in}{3.144889in}}%
\pgfusepath{stroke}%
\end{pgfscope}%
\begin{pgfscope}%
\pgfpathrectangle{\pgfqpoint{0.631025in}{0.546925in}}{\pgfqpoint{3.100000in}{2.695000in}} %
\pgfusepath{clip}%
\pgfsetbuttcap%
\pgfsetroundjoin%
\definecolor{currentfill}{rgb}{0.121569,0.466667,0.705882}%
\pgfsetfillcolor{currentfill}%
\pgfsetlinewidth{0.150562pt}%
\definecolor{currentstroke}{rgb}{0.121569,0.466667,0.705882}%
\pgfsetstrokecolor{currentstroke}%
\pgfsetdash{}{0pt}%
\pgfsys@defobject{currentmarker}{\pgfqpoint{-0.041667in}{-0.041667in}}{\pgfqpoint{0.041667in}{0.041667in}}{%
\pgfpathmoveto{\pgfqpoint{0.000000in}{-0.041667in}}%
\pgfpathcurveto{\pgfqpoint{0.011050in}{-0.041667in}}{\pgfqpoint{0.021649in}{-0.037276in}}{\pgfqpoint{0.029463in}{-0.029463in}}%
\pgfpathcurveto{\pgfqpoint{0.037276in}{-0.021649in}}{\pgfqpoint{0.041667in}{-0.011050in}}{\pgfqpoint{0.041667in}{0.000000in}}%
\pgfpathcurveto{\pgfqpoint{0.041667in}{0.011050in}}{\pgfqpoint{0.037276in}{0.021649in}}{\pgfqpoint{0.029463in}{0.029463in}}%
\pgfpathcurveto{\pgfqpoint{0.021649in}{0.037276in}}{\pgfqpoint{0.011050in}{0.041667in}}{\pgfqpoint{0.000000in}{0.041667in}}%
\pgfpathcurveto{\pgfqpoint{-0.011050in}{0.041667in}}{\pgfqpoint{-0.021649in}{0.037276in}}{\pgfqpoint{-0.029463in}{0.029463in}}%
\pgfpathcurveto{\pgfqpoint{-0.037276in}{0.021649in}}{\pgfqpoint{-0.041667in}{0.011050in}}{\pgfqpoint{-0.041667in}{0.000000in}}%
\pgfpathcurveto{\pgfqpoint{-0.041667in}{-0.011050in}}{\pgfqpoint{-0.037276in}{-0.021649in}}{\pgfqpoint{-0.029463in}{-0.029463in}}%
\pgfpathcurveto{\pgfqpoint{-0.021649in}{-0.037276in}}{\pgfqpoint{-0.011050in}{-0.041667in}}{\pgfqpoint{0.000000in}{-0.041667in}}%
\pgfpathclose%
\pgfusepath{stroke,fill}%
}%
\begin{pgfscope}%
\pgfsys@transformshift{3.472691in}{0.848765in}%
\pgfsys@useobject{currentmarker}{}%
\end{pgfscope}%
\begin{pgfscope}%
\pgfsys@transformshift{2.826858in}{0.950830in}%
\pgfsys@useobject{currentmarker}{}%
\end{pgfscope}%
\begin{pgfscope}%
\pgfsys@transformshift{2.181025in}{1.250117in}%
\pgfsys@useobject{currentmarker}{}%
\end{pgfscope}%
\begin{pgfscope}%
\pgfsys@transformshift{1.535191in}{2.354622in}%
\pgfsys@useobject{currentmarker}{}%
\end{pgfscope}%
\begin{pgfscope}%
\pgfsys@transformshift{0.889358in}{3.144889in}%
\pgfsys@useobject{currentmarker}{}%
\end{pgfscope}%
\end{pgfscope}%
\begin{pgfscope}%
\pgfsetrectcap%
\pgfsetmiterjoin%
\pgfsetlinewidth{0.803000pt}%
\definecolor{currentstroke}{rgb}{0.000000,0.000000,0.000000}%
\pgfsetstrokecolor{currentstroke}%
\pgfsetdash{}{0pt}%
\pgfpathmoveto{\pgfqpoint{0.631025in}{0.546925in}}%
\pgfpathlineto{\pgfqpoint{0.631025in}{3.241925in}}%
\pgfusepath{stroke}%
\end{pgfscope}%
\begin{pgfscope}%
\pgfsetrectcap%
\pgfsetmiterjoin%
\pgfsetlinewidth{0.803000pt}%
\definecolor{currentstroke}{rgb}{0.000000,0.000000,0.000000}%
\pgfsetstrokecolor{currentstroke}%
\pgfsetdash{}{0pt}%
\pgfpathmoveto{\pgfqpoint{3.731025in}{0.546925in}}%
\pgfpathlineto{\pgfqpoint{3.731025in}{3.241925in}}%
\pgfusepath{stroke}%
\end{pgfscope}%
\begin{pgfscope}%
\pgfsetrectcap%
\pgfsetmiterjoin%
\pgfsetlinewidth{0.803000pt}%
\definecolor{currentstroke}{rgb}{0.000000,0.000000,0.000000}%
\pgfsetstrokecolor{currentstroke}%
\pgfsetdash{}{0pt}%
\pgfpathmoveto{\pgfqpoint{0.631025in}{0.546925in}}%
\pgfpathlineto{\pgfqpoint{3.731025in}{0.546925in}}%
\pgfusepath{stroke}%
\end{pgfscope}%
\begin{pgfscope}%
\pgfsetrectcap%
\pgfsetmiterjoin%
\pgfsetlinewidth{0.803000pt}%
\definecolor{currentstroke}{rgb}{0.000000,0.000000,0.000000}%
\pgfsetstrokecolor{currentstroke}%
\pgfsetdash{}{0pt}%
\pgfpathmoveto{\pgfqpoint{0.631025in}{3.241925in}}%
\pgfpathlineto{\pgfqpoint{3.731025in}{3.241925in}}%
\pgfusepath{stroke}%
\end{pgfscope}%
\end{pgfpicture}%
\makeatother%
\endgroup%
}
        \caption{\(T_C\) values for ignition delays near \SI{20}{\ms} at the
        range of blends considered in this study}
        \label{fig:temp-comp}
    \end{minipage}\hfill%
\end{figure}

Also shown on \cref{fig:ign-delays} are constant volume, adiabatic simulations
computed according to the procedure laid out in \cref{sec:rcm-modeling}. In
general, the agreement between the model and the experiments is quite good over
the entire range of the experiments. It can be seen in \cref{fig:ign-delays}
that at low temperatures for a given mixture composition the ignition delay
tends to be under-predicted, while at the higher temperatures the ignition delay
is over-predicted.

As discussed by \textcite{Mittal2008}, this is likely due in part to the
modeling procedure used in this work. In general, we expect constant volume
simulations to have shorter ignition delays than the experiments for long
ignition delays because they do not include the effect of post-compression heat
loss; conducting simulations that include the post-compression heat loss are
very likely to improve agreement in this region. Furthermore, for short ignition
delays, we expect constant volume simulations to over-predict the experimental
ignition delay because they do not include the effect of radical pool buildup
during the compression stroke. Therefore, conducting simulations that include
the compression stroke are very likely to improve the agreement for short
ignition delays.

\section{Conclusions}\label{sec:conclusions}

In this study, we have measured ignition delays for binary blends of dimethyl
ether and methanol for engine-relevant pressure, temperature, and equivalence
ratio conditions using a heated rapid compression machine. The ignition delay
results show that pure DME is more reactive than pure MeOH, and that the
increase in ignition delay as DME is replaced by MeOH is non-linear as a
function of the blending fraction. The ignition delays are also compared to a
chemical kinetic model compiled by combining independent models for the two
fuels. This model does not consider cross reactions between DME and MeOH.
Nonetheless, the model gives quite good agreement with the data, supporting the
hypothesis that the fuels do not interact via cross reactions but instead
through common radicals such as \ce{OH}. In addition, this further demonstrates
that models for low-reactivity fuels such as methanol and high-reactivity fuels
such as DME can be constructed by simple concatenation and deduplication of
their respective independent models.

\section{Acknowledgements}\label{acknowledgements}

This work was supported by the National Science Foundation under Grant No.
CBET-1402231.

\printbibliography

\end{document}
