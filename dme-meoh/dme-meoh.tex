%%%%%%%%%%%%%%%%%%%%%%%%%%%%%%%%%%%%%%%%%%%%%%%%%%%%%%%%%%%%%%%%%%%%%%%
% This work is licensed under the Creative Commons Attribution 4.0
% International License. To view a copy of this license, visit
% http://creativecommons.org/licenses/by/4.0/.
%%%%%%%%%%%%%%%%%%%%%%%%%%%%%%%%%%%%%%%%%%%%%%%%%%%%%%%%%%%%%%%%%%%%%%%
\documentclass[12pt]{../ussci}

\usepackage{enumitem}
\setlist{noitemsep}

\usepackage[version=4]{mhchem}
\usepackage{siunitx}
\sisetup{separate-uncertainty=true}

\usepackage{graphicx}
\usepackage{pgf}
\usepgflibrary{fpu}

\usepackage[tableposition=top]{caption}
\usepackage{booktabs}
\usepackage{mathtools}
\usepackage{microtype}

\usepackage[capitalize]{cleveref}
%======================================================================
\addbibresource{dme-meoh.bib}
%======================================================================
\newcommand\papertopic{Reaction Kinetics}
%======================================================================
\DeclareSIUnit\torr{torr}

\title{ High-Pressure Autoignition of Binary Blends of Methanol and Dimethyl Ether }

\author[1]{Hongfu Wang}
\author[2*]{Bryan W.\ Weber}
\author[2]{Ruozhou Fang}
\author[2]{Chih-Jen Sung}
\affil[2]{Department of Mechanical Engineering, University of Connecticut, Storrs,
CT, USA}
\affil[*]{Corresponding Author: \email{bryan.weber@uconn.edu}}

\begin{document}
\maketitle

%====================================================================
\begin{abstract} % not to exceed 200 words

    Reactivity Controlled Compression Ignition (RCCI) is a new advanced engine
    concept that uses a dual fuel mode of operation to achieve significant
    improvements in fuel economy and emissions output. The fuels that are
    typically used in this mode include a low- and a high-reactivity fuel in
    varying proportions to control ignition timing. As such, understanding the
    interaction effects during auto-ignition of binary fuel blends is critical
    to optimizing these RCCI engines. In this work, we measure the auto-ignition
    delays of binary mixtures of dimethyl ether (\ce{C2H6O}, DME) and methanol
    (\ce{CH4O}, MeOH) in a rapid compression machine. In these experiments,
    dimethyl ether and methanol function as the high- and low-reactivity fuels,
    respectively. We considered five fuel blends at varying blending ratios (by
    mole), including \SI{100}{\percent} DME-\SI{0}{\percent} MeOH,
    \SI{75}{\percent} DME-\SI{25}{\percent} MeOH, \SI{50}{\percent}
    DME-\SI{50}{\percent} MeOH, \SI{25}{\percent} DME-\SI{75}{\percent} MeOH,
    and \SI{0}{\percent} DME-\SI{100}{\percent} MeOH. Experiments are conducted
    at an engine-relevant pressure of \SI{30}{\bar}, for the stoichiometric
    equivalence ratio. In addition, the experimental results are compared with
    simulations using a chemical kinetic model for DME/MeOH combustion generated
    by merging independent, well-validated models for DME and MeOH.

\end{abstract}

% (Provide 2-4 keywords describing your research. Only abbreviations firmly
% established in the field may be used. These keywords will be used for
% sessioning/indexing purposes.)
\begin{keyword}
    chemical kinetics\sep rapid compression machine\sep binary fuel blends\sep advanced engines
\end{keyword}

\section{Introduction}\label{introduction}

To reduce the environmental impact of combustion, future combustion processes
must feature substantially reduced pollutant emissions while maintaining high
efficiency. A promising concept in this respect is low-temperature combustion
(LTC). As an outstanding representative of LTC technologies, the dual-fuel RCCI
operation has great potential in terms of combustion controllability. The
general principle of Dual-fuel RCCI combustion requires two fuels with different
reactivities, that is using a high reactivity fuel (such as diesel or dimethyl
ether (DME)) to trigger the ignition and combustion of low-reactivity fuels
(such as gasoline, methanol, ethanol, or butanol).

DME is considered an efficient alternative fuel for use in diesel engines
because it has excellent auto-ignition characteristics. The low boiling point
(\SI{248}{\K}), low critical point (\SI{400}{\K}), and high cetane number (\(>
55\)) of DME \autocite{Arcoumanis2008,Teng2001} make it well suited for
compression ignition engines. In addition, the high oxygen content of DME
(\SI{34.8}{\percent} by mass) together with the absence of a C-C bond
contributes to ultra-low soot formation during DME combustion
\autocite{Arcoumanis2008}.

The promoting effect of DME on fuels with poor auto-ignition qualities such as
methane and propane is a promising feature with respect to the RCCI concept.
Therefore, some research has been conducted to reveal the promoting potential of
DME. Several researchers \autocite{Burke2015a,Tang2012a,Chen2007a} have studied
the ignition delay of DME/methane mixtures in a shock tube. The results have
shown that DME has a strong promoting effect on the auto-ignition of methane,
even when the concentration of methane is much higher than that of DME. Further,
\textcite{Dames2016} showed the promoting effect of DME in DME/propane mixtures
using a rapid compression machine. The results showed that propane combustion is
promoted due to the large amount of radicals produced by low-temperature DME
oxidation. Those studies indicate the potential of DME as a combustion promoter
together with fuels with poor auto-ignition qualities.

For dual-fuel RCCI operation, a lower reactivity fuel should be studied in
combination with the high reactivity fuel. Methanol (MeOH) is well known as a
widely used alcoholic alternative fuel, but when operated in a single fuel mode
it tends to have poor auto-ignition quality \autocite{Siebers1987}. Given the
great potential of DME as a combustion promoter, we expect that DME can
significantly improve the auto-ignition quality of methanol and the combination
will be effective for dual-fuel RCCI operation. In this work, we explore the
auto-ignition characteristics of DME/MeOH binary mixtures. A set of ignition
delay time data for DME/MeOH at different mixing ratios over a wide range of
temperature and pressure at engine relevant conditions is obtained in a rapid
compression machine (RCM). In addition, we construct a chemical kinetic model
for DME/MeOH combustion by merging independent models for the fuels. This
technique was successful for DME/propane mixtures \autocite{Dames2016}.
Experimental results are compared to simulation results to validate the model.

\section{Experimental Methods}\label{sec:experimental-methods}

The RCM used in this study is a single piston arrangement and is pneumatically
driven and hydraulically stopped. The device has been described in detail
previously \autocite{Mittal2007a} and will be described here briefly for
reference. The end of compression (EOC) temperature and pressure (\(T_C\) and
\(P_C\) respectively), are independently changed by varying the overall
compression ratio, initial pressure, and initial temperature of the experiments.
The primary diagnostic on the RCM is the in-cylinder pressure. The pressure data
is processed by a Python package called UConnRCMPy \autocite{uconnrcmpy}, which
calculates \(P_C\), \(T_C\), and the ignition delay(s). The definition of the
ignition delay is shown in \cref{fig:ign-delay-def}. The time of the EOC is
defined as the maximum of the pressure trace prior to the start of ignition and
the ignition delays are defined as the time from the EOC until local maxima in
the first time derivative of the pressure.

In addition to the reactive experiments, non-reactive experiments are conducted
to determine the influence of machine-specific behavior on the experimental
conditions and permit the calculation of the EOC temperature via the isentropic
relations between pressure and temperature \autocite{Lee1998}. The EOC
temperature is calculated by the procedure described in
\cref{sec:computational-methods}.

The RCM is equipped with heaters to control the initial temperature of the
mixture. After filling in the components to the mixing tanks, the heaters are
switched on and the system is allowed \SI{1.5}{\hour} to come to steady state.
The mixing tanks are also equipped with magnetic stir bars to the reactants are
well mixed for the duration of the experiments.

The mixtures considered in this study are shown in \cref{tab:mixtures}. Mixtures
are prepared in stainless steel mixing tanks. The proportions of reactants in
the mixture are determined by specifying the absolute mass of the methanol in
the mixture (if present), the equivalence ratio, and the ratio of DME/MeOH in
the fuel. Since MeOH is a liquid at room temperature and pressure, it is
injected into the mixing tank through a septum. Proportions of DME, \ce{O2}, and
\ce{N2} are added manometrically at room temperature.

\begin{table}[htb]
    \centering
    \caption{Mixtures considered in this work}
    \begin{tabular}{cccccc}
        \toprule
        \% DME & \% MeOH & \multicolumn{4}{c}{Mole Fraction (purity)} \\
        \cmidrule{3-6}
         & & DME (\SI{99.7}{\percent}) & MeOH (\SI{100.00}{\percent}) & \ce{O2} (\SI{99.999}{\percent}) & \ce{N2} (\SI{99.999}{\percent})  \\
        \midrule
        100 & 0 & 0.0654 & 0.0000 & 0.1963 & 0.7382 \\
        75 & 25 & 0.0556 & 0.0185 & 0.1945 & 0.7314 \\
        50 & 50 & 0.0427 & 0.0427 & 0.1921 & 0.7225 \\
        25 & 75 & 0.0252 & 0.0756 & 0.1889 & 0.7103 \\
        0 & 100 & 0.0000 & 0.1229 & 0.1843 & 0.6928 \\
        \bottomrule
    \end{tabular}
    \label{tab:mixtures}
\end{table}

\section{Computational Methods}\label{sec:computational-methods}
\subsection{Experimental Modeling}\label{sec:experimental-modeling}

The Python 3.5 interface of Cantera \autocite{cantera} version 2.3.0 is used for
all simulations in this work. Detailed descriptions of the use of Cantera for
these simulations can be found in the work of \textcite{Weber2016a} and
\textcite{Dames2016}; a brief overview is given here. As mentioned in
\cref{sec:experimental-methods}, non-reactive experiments are conducted to
characterize the machine-specific effects on the experimental conditions in the
RCM. This pressure trace is used to compute a volume trace by assuming that the
reactants undergo a reversible, adiabatic, constant composition (i.e.,
isentropic) compression during the compression stroke and an isentropic
expansion after the EOC. The volume trace is applied to a simulation conducted
in an \verb|IdealGasReactor| in Cantera \autocite{cantera} using the CVODES
solver from the SUNDIALS suite \autocite{Hindmarsh2005}. The ignition delay from
the simulations is defined in the same manner as in the experiments. The time
derivative of the pressure in the simulations was computed by second order
Lagrange polynomials, as discussed by \textcite{Chapra2010}.

\subsection{Chemical Kinetic Model}\label{sec:chemical-kinetic-model}

To the best of our knowledge, there are no chemical kinetic models for the
combustion of binary mixtures of DME and MeOH available in the literature.
Therefore, we construct a kinetic model in this work by combining two
independent models. The kinetics for DME are taken from the work of
\textcite{Dames2016} while the kinetics for MeOH are taken from the work of
\textcite{Burke2016}. In the work of \textcite{Dames2016}, the authors used the
DME chemistry from the work of \textcite{Burke2015a} and included updates to
many of the reaction rates important for low-temperature ignition
\autocite{Dames2016}. Therefore, we prefer the DME submechanism from
\textcite{Dames2016}.

To combine the two models, the propane submechanism was removed from the model
of \textcite{Dames2016} and duplicate reactions and species were taken from the
\textcite{Dames2016} model. No cross-reactions between DME and MeOH were
considered in this work. In the work of \textcite{Dames2016}, it was found that
for combined models of high-reactivity fuels such as DME and low-reactivity
fuels such as methanol, cross-reactions between the fuels do not strongly affect
the ignition delay and the fuels instead interact through radical species such
as \ce{OH}.

\section{Acknowledgements}\label{acknowledgements}

This paper is based on material supported by the National Science
Foundation under Grant No. CBET-1402231.

\printbibliography

\end{document}
